\chapternonumtoc{Remerciements}

Je tiens tout d'abord à remercier Pierre Agostini et Giovanni De Ninno d'avoir accepté d'être rapporteurs de cette thèse, ainsi que Sophie Kazamias, Laurent Nahon et Antonio Zelaquett Khoury d'avoir été membres de mon jury.\\ \\
C'est très sincèrement que je désire remercier Thierry Ruchon, qui n'a pas officiellement dirigé ma thèse mais l'a très certainement fait en pratique. J'ai eu la chance de travailler aux côtés d'un excellent chercheur, d'un très bon pédagogue et avant tout d'une personne remarquable. Thierry a été un modèle, une source d'idées, d'inspiration et d'encouragement perpétuel, m'a donné l'opportunité de travailler sur des sujets passionnants et de visiter de nombreux pays et laboratoires pour mener à bien mes travaux. Grâce à lui, ces trois années sont à ranger parmi les plus agréables que j'ai passées et je me félicite tous les jours d'avoir choisi de travailler avec lui - je ne saurais trop vous conseiller de faire de même.\\ \\
Mes remerciements s'adressent ensuite à Antoine Camper, mon prédécesseur au laboratoire. En plus de son savoir pratique, il m'a sans cesse imprégné de sa pensée scientifique, dont la rigueur et la créativité semblent infatigables quelle que soit l'heure du jour ou de la nuit. Je suis particulièrement heureux d'avoir pu continuer à travailler avec lui pendant toute ma thèse, à Saclay, Columbus ou Rehovot.\\ \\
L'atmosphère de travail stimulante et agréable du groupe étant à attribuer à tous ses membres, j'ai une pensée particulière pour Thierry Auguste, dont nous n'aurions pu nous passer tant il manie aussi bien le clavier que le calembour, pour Olivier Gobert, sans qui nombre d'expériences n'auraient pas été conclues, ainsi que pour Bertrand Carré et Pascal Salières, qui ont toujours été d'une grande aide et de bon conseil. \\Merci à Michel Perdrix, Delphine Guillaumet et Fabien Lepetit de s'être occupé de ce bon LUCA, qui malgré son grand âge n'a pas toujours pas cessé de faire de la belle physique. Merci à Véronique Gereczy, Caroline Lebe, Jacqueline Bandura, Didier Guyader, André Fillon, Sylvain Foucquart, Marc Billon, Sarah Kieffer, Gilles Le Chevallier, Sylvie Jubera, et Michel Bougeard dont les compétences m'ont chacunes été très précieuses.\\
Merci à l'ensemble de mes collègues thésards et post-doc qui ont fait de ces trois années un plaisir au laboratoire et en dehors. Une pensée particulière pour ceux qui sont devenus un peu plus que des collègues, Vincent Gruson, Gustave Pariente, Dominik Franz, Adrien Leblanc et je garde bien sûr le meilleur pour la fin avec Aura Inés Gonzalez Angarita, que je remercie pour de superbes moments ainsi que de prendre soin de mon vieil ami Fléchois. \\ \\
Je souhaite également remercier tous les gens avec qui j'ai eu la chance de collaborer, qui m'ont permis d'enrichir et d'élargir énormément mon sujet d'étude. Je pense principalement à mes collègues bordelais, Yann Mairesse, Valérie Blanchet, Baptiste Fabre, Bernard Pons, Samuel Beaulieu et Amélie Ferré, avec qui j'ai eu un immense plaisir à travailler pendant plusieurs mois. Un grand merci également à Jérémie Caillat et Richard Taïeb du LCPMR, qui en plus d'avoir la salle de réunion la plus classe de la région ont été d'une grande aide. Merci à Tim Gorman et Lou DiMauro de l'Ohio State University, à Laurent Nahon et Gustavo Garcia du synchrotron Soleil, à David Gauthier et Giovanni De Ninno de l'université de Nova Gorica, et à Oren Pedatzur et Nirit Dudovich du Weizmann Institute of Science. \\ \\
Un énorme merci à ma famille, qui m'ont fait le grand plaisir de leur présence à ma soutenance de thèse, et qui me soutiennent et m'aident depuis bien plus longtemps que trois ans et qui, j'en suis sûr, le feront toujours quand ma thèse ne sera qu'un lointain souvenir. Ma dernière et certainement pas moindre pensée est pour Lou, que je remercie de tout mon cœur; rien parmi ces trois dernières années ne m'aura rendu plus heureux que les moments passés avec toi.\\ \\
Merci à tous !\\
Romain