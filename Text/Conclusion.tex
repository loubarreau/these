\cleardoublepage
% we want the numbering to start back at A
\renewcommand{\thesection}{\Roman{section}}
\setcounter{section}{0}
%it is going to mess hyperref links because it will have the same name as previous sections -> we change it
\renewcommand{\theHsection}{CL.\the\value{section}}
% same for figures
\renewcommand{\thefigure}{\Roman{section}.\arabic{figure}}
\renewcommand{\theHfigure}{CL.\Roman{section}.\arabic{figure}}
\setcounter{figure}{0}

\chapternonumtoc{Conclusions et Perspectives}
\makeatletter
\def\toclevel@chapter{-1}
\def\toclevel@section{0}
\def\toclevel@subsection{1}
\makeatother

Le travail présenté dans ce manuscrit est centré sur la photoionisation produite par le rayonnement harmonique. Dans un premier temps, les harmoniques d'ordre élevé sont utilisées pour étudier la dynamique de l'autoionisation à l'échelle attoseconde. Ensuite, les propriétés de la photoionisation dissociative de molécules de NO sont exploitées pour déterminer l'état de polarisation des harmoniques générées dans des conditions particulières.

\subsection*{Observation des interférences électroniques entre ionisation directe et résonante lors de l'autoionisation dans les gaz rares}
Dans ce travail, nous avons dans un premier temps étudié les dynamiques électroniques lors de l'autoionisation dans plusieurs gaz rares grâce aux harmoniques d'ordre élevé et à l'interférométrie électronique résolue spectralement. Les principales caractéristiques du rayonnement harmonique et sa production expérimentale ont été exposées dans la partie \ref{part:GHOE}. En particulier, la cohérence des harmoniques d'ordre élevé permet de les utiliser pour l'interférométrie de paquets d'ondes électroniques produits par photoionisation à deux photons et deux couleurs (XUV + IR). Cette technique, appelée RABBIT, a été initialement développée pour mesurer la phase spectrale des harmoniques, puis a été étendue à la caractérisation du processus de photoionisation lui-même. En effet, dans la partie \ref{part:Delais} nous avons montré que la phase mesurée par RABBIT se décompose en une contribution reliée au délai de groupe du rayonnement ionisant, et une constribution reliée au délai de photoionisation (ou délai de Wigner). Ce dernier est défini par la théorie des collisions comme la dérivée spectrale de la phase de diffusion, qui caractérise le déphasage de la fonction d'onde électronique causé par la diffusion dans le potentiel ionique. Le délai de photoionisation correspond ainsi au délai entre l'absorption du photon XUV et la diffusion du paquet d'ondes électronique hors du potentiel ionique. Il est donc très sensible aux détails fins de ce potentiel.

Ici, nous avons introduit un état intermédiaire autoionisant dans l'interféromètre électronique. Lorsque le système est porté dans un de ces états excités au-dessus de son potentiel d'ionisation par l'absorption d'un photon XUV, deux voies d'ionisation sont possibles: l'ionisation directe et l'ionisation indirecte \textit{via} l'excitation transitoire. Les interférences entre ces deux chemins produisent un profil de raie asymétrique (dit profil de Fano), d'une largeur de quelques dizaines de milli-électron-volts. Au voisinage de ces transitions résonantes, on s'attend à une variation rapide de la phase de diffusion et la dynamique d'ionisation ne peut alors plus être simplement caractérisée par un délai du paquet d'ondes émis. La dynamique électronique complète au cours du processus d'autoionisation est contenue dans les variations spectrales du moment de transition \textit{complexe} (amplitude et phase) vers la résonance. Ces dernières sont accessibles expérimentalement par l'interférométrie électronique. Les expériences ont d'abord été effectuées au voisinage de l'état doublement excité $2s2p$ de l'hélium, très bien connu en spectroscopie statique. Afin de mesurer les variations spectrales de l'amplitude et de la phase du moment de transition vers la résonance, nous avons développé une nouvelle méthode d'interférométrie électronique résolue spectralement, variante du RABBIT dans des conditions particulières, que nous avons appelée  \textit{Rainbow} RABBIT. Cette méthode robuste et simple permet d'obtenir en un unique spectrogramme les variations complètes de l'amplitude et de la phase spectrales avec une haute résolution. La transformée de Fourier permet ensuite d'obtenir la dynamique complète de l'autoionisation dans le domaine temporel: aux temps courts après l'excitation (à l'échelle de quelques femtosecondes) on observe les interférences électroniques entre les processus d'excitation direct et résonant, puis la décroissance de la résonance d'une durée de vie typique de l'ordre de la dizaine de femtosecondes. Ces interférences temporelles n'étaient jusqu'alors pas accessibles en spectroscopie. Ces observations purement expérimentales sont en excellent accord avec les simulations des groupes de F. Mart\'{i}n à l'université autonome de Madrid et de R. Taïeb à l'université Pierre et Marie Curie.

Les mesures Rainbow RABBIT ont ensuite été appliquées à d'autres résonances lors d'expériences en collaboration avec le groupe du Prof. A. L'Huillier à l'université de Lund. Les résultats obtenus pour l'état $2s2p$ de l'hélium ont été reproduits et ont permis la description de la dynamique d'autoionisation grâce à différentes représentations temps-énergie: la transformée de Fourier inverse au temps courts qui montre la construction du profil spectral de la résonance au cours du temps, la transformée de Gabor qui indique le spectre "instantané" du paquet d'ondes électroniques émis, et la distribution de Wigner-Ville qui permet de séparer dans le domaine temporo-spectral les contributions des chemins d'ionisation direct et résonant , et de faire apparaître leur cohérence mutuelle. Grâce aux conditions expérimentales particulières du laboratoire de Lund (spectromètre de photoélectrons de haute résolution, longueur d'onde centrale du laser de génération autour de 800 nm), nous avons pu mesurer l'amplitude et la phase spectrales au voisinage d'une seconde résonance doublement excitée de l'hélium ($sp3^+$) et exciter de manière cohérente un paquet d'ondes électroniques à deux électrons dans les états $2s2p$ et $sp3^+$. La bonne résolution du spectromètre nous a également permis d'appliquer la méthode Rainbow RABBIT à la résonance $3s4p$ de l'argon dans les deux canaux spin-orbite de Ar$^+$ séparés de seulement 180 meV. Après avoir séparé numériquement les deux contributions, nous avons reconstruit la dynamique de l'autoionisation et la construction du profil de la résonance "fenêtre" au cours du temps dans un seul canal spin-orbite. 

Les variations spectrales de l'amplitude et de la phase au voisinage d'une résonance peuvent également être mesurées en combinant la technique RABBIT de base avec la variation de la longueur d'onde du fondamental, ce qui implique l'acquisition de plusieurs spectrogrammes et une résolution spectrale limitée par la largeur des harmoniques. Cette méthode a été appliquée aux résonances $3s4p$ de l'argon, et $2s3p$ du néon (expérience effectuée en collaboration avec le groupe du Prof. L. DiMauro en Ohio). Nos résultats expérimentaux, en très bon accord avec les simulations du groupe de F. Mart\'{i}n, ont mis en évidence l'influence des paramètres du rayonnement harmonique et du faisceau d'habillage sur la mesure de phase.

\subsection*{Caractérisation complète de l'état de polarisation des harmoniques générées par un champ à symétrie dynamique}
Dans un second temps, nous avons étudié l'état de polarisation des harmoniques d'ordre élevé produites lors de l'interaction d'un gaz rare avec un champ à deux couleurs (fondamental et seconde harmonique) polarisées circulairement en sens opposé. Dans cette configuration, la symétrie dynamique de l'interaction champ-atome est responsable de la production d'harmoniques circulaires d'hélicité opposée pour les ordres $3m-1$ et $3m+1$, et de la non-émission des harmoniques $3m$ ($m \in \mathbb{N}$). Grâce à la résolution numérique de l'équation de Schrödinger dépendante du temps, nous avons identifié plusieurs causes possibles de rupture de la symétrie dynamique qui produisent des déviations aux propriétés idéales de polarisation des harmoniques: l'enveloppe rapidement variable des impulsions, leur phase relative à l'échelle du cycle optique non contrôlée, et l'ionisation du milieu. Ces caractéristiques expérimentales intrinsèques peuvent réduire significativement la circularité des harmoniques produites et causer de la dépolarisation. Nous avons interprété ces résultats grâce aux variations temporo-spectrales des propriétés de polarisation des harmoniques. Ces observations nous ont permis de proposer des conditions expérimentales optimales pour générer des harmoniques circulaires et complètement polarisées en utilisant un champ à deux couleurs. Entre autres, nous proposons l'utilisation d'impulsions ultra-brèves mises en forme ("carrées") ainsi que la génération dans des gaz rares à potentiel d'ionisation élevé avec des longueurs d'onde dans l'IR moyen. L'interféromètre à deux couleurs doit être le plus stable possible, c'est-à-dire compact ou stabilisé activement. 

Les simulations ont été motivées par les résultats d'expériences de polarimétrie moléculaire effectuées en collaboration avec l'équipe de D. Dowek à l'université d'Orsay. \`{A} la différence de la polarimétrie optique incomplète utilisée jusqu'à présent pour caractériser la polarisation harmonique, la polarimétrie moléculaire permet la caractérisation complète de l'état de polarisation du rayonnement, c'est-à-dire la mesure simultanée des trois paramètres de Stokes normalisés $s_1$, $s_2$ et $s_3$. Cette méthode repose sur la mesure en coïncidence de l'ion et de l'électron produits lors de la photoionisation dissociative d'une molécule induite par une harmonique, qui permet d'obtenir la distribution angulaire des photoélectrons dans le référentiel moléculaire. Cette observable est sensible aux trois paramètres de Stokes du rayonnement incident, et permet notamment de séparer la partie non polarisée du rayonnement de celle polarisée circulairement grâce au dichroïsme circulaire dans les distributions angulaires de photoélectrons. Ainsi, nous avons présenté les premières mesures de caractérisation complète de l'état de polarisation des harmoniques produites par un champ à deux couleurs polarisées circulairement en sens opposé. Nous avons confirmé l'alternance de l'hélicité entre les harmoniques $3m-1$ et $3m+1$, et son contrôle par l'hélicité de chaque couleur. Cependant dans nos conditions expérimentales les harmoniques caractérisées ne sont ni parfaitement circulaires ni complètement polarisées. Ces écarts sont exaltés lorsque l'un des champs est elliptique. L'étude théorique suggère que les observations expérimentales seraient le résultat combiné d'une ionisation rapide du milieu, d'une circularité non parfaite des champs générateurs, d'une variation rapide de l'enveloppe temporelle et d'une instabilité de la phase relative des deux couleurs. Ces résultats démontrent l'importance des mesures de polarimétrie complète.

\begin{equation*}
\ast \ast \ast
\end{equation*}

\paragraph*{Vers l'étude résolue angulairement des dynamiques électroniques lors de la photoionisation ?} L'une des perspectives ouvertes par les deux aspects de ce travail est la mesure des délais de photoionisation résolue angulairement dans les atomes, et dans le référentiel moléculaire pour les molécules. En effet, les fonctions d'onde des photoélectrons émis peuvent être décomposés en différentes ondes partielles, et distribution angulaire d'émission est différente pour les différentes ondes partielles. En effectuant une expérience de type RABBIT dans un spectromètre imageur de vitesses ou un COLTRIMS, il est possible de résoudre angulairement les oscillations des pics satellites et ainsi de mesurer la variation angulaire du délai de photoionisation pour les différentes directions d'éjection de l'électron \mycite{HeuserPRA2016}\mycite{HockettJPhysB2017}. Ceci est particulièrement pertinent pour les molécules où une forte variation spatiale est attendue du fait de la nature anisotrope du potentiel moléculaire. Par ailleurs, la présence de résonances (atomiques ou moléculaires) peut largement influer sur la variation spectrale et spatiale des délais. Par exemple, au voisinage de résonances de Fano, le paramètre d'anisotropie $\beta$ qui caractérise la distribution angulaire des électrons varie fortement avec l'énergie. On s'attend alors à des variations de phase différentes pour les différentes ondes partielles mises en jeu. Récemment, l'équipe d'A. L'Huillier à Lund a effectué des mesures RABBIT au voisinage de la résonance $3s4p$ de l'argon dans un spectromètre imageur de vitesses en variant la longueur d'onde de génération \mycite{Zhong2017} dont l'interprétation et la comparaison avec des simulations sont en cours. 

Dans les molécules, le groupe de H. J. Wörner à Zurich a mesuré des délais de photoionisation dans H$_2$O et N$_2$O dans une bouteille magnétique, c'est-à-dire intégrés sur la distribution angulaire des électrons \mycite{HuppertPRL2016}. Résoudre complètement la dynamique d'ionisation moléculaire est un grand défi car on devient doublement différentiel: par rapport à l'angle entre la molécule et la polarisation incidente, et par rapport à l'angle d'émission de l'électron dans le référentiel moléculaire. Dans le référentiel moléculaire, les simulations de \mycite{HockettJPhysB2016} font apparaître de nombreuses structures, dépendant à la fois de l'énergie et de l'angle, avec des délais variant entre -200 et +200 as, en particulier dus à la présence de résonances de forme qui peuvent "piéger" l'électron lors de son émission. Ce type d'étude complète de la photoionisation dans le référentiel moléculaire est prévue sur la plateforme ATTOLAB, en collaboration avec l'équipe de D. Dowek, en mettant à profit la cadence relativement élevée (10 kHz) du laser FAB10 pour les mesures en coïncidence dans un COLTRIMS, ou bien la cadence plus faible (1 kHz) mais la plus forte énergie par impulsion (15 mJ) du laser FAB1 pour aligner les molécules dans un spectromètre imageur de vitesses. Les perspectives ouvertes pas les nouvelles sources harmoniques à très haute cadence (>100 kHz) sont très prometteuses pour la poursuite et la généralisation de ces études.

\paragraph*{\'{E}tude systématique des paramètres modifiant l'état de polarisation des harmoniques} Le plus haut taux de répétition du laser FAB10 de la plateforme ATTOLAB pourra également être utilisé pour de nouvelles mesures de polarimétrie moléculaire. \`{A} 10 kHz, la statistique de comptage sera meilleure que celle des expériences effectuées sur le laser PLFA à 1 kHz. La durée de mesure sera réduite, ce qui assurera une plus grande stabilité des conditions expérimentales. Il sera alors possible par exemple d'étudier expérimentalement l'influence des différents paramètres critiques dans le schéma à deux couleurs mis en évidence par nos simulations. L'étude d'autres dispositifs de production d'harmoniques polarisées circulairement pourra également être envisagé. En particulier, des simulations présentées dans la thèse de Vincent Gruson \mycite{TheseGruson} ont montré qu'une inhomogénéité spatiale ou temporelle d'alignement pouvait être responsable de dépolarisation du rayonnement harmonique produit dans des molécules de N$_2$ alignées. Ces expériences permettront de définir exactement les meilleurs conditions de génération d'un rayonnement harmonique de polarisation circulaire et de maximisation du paramètre $s_3$, essentiel pour étudier les phénomènes de dichroïsmes dans les molécules ou les matériaux qui donnent généralement de faibles signaux.

\paragraph*{Les applications du Rainbow RABBIT au-delà des résonances d'autoionisation} La méthode d'analyse résolue spectralement des oscillations des pics satellites que nous avons développée présente un énorme potentiel pour l'étude fine de tous types de résonances atomiques, moléculaires, voire en physique du solide. C'est ce qui sera développé en particulier sur le laser FAB1 de la plateforme ATTOLAB. Notons que le Rainbow RABBIT a d'ores et déjà été utilisée par d'autres groupes pour étudier des processus différents. L'équipe de Y. Mairesse à Bordeaux a effectué des expériences d'ionisation au-dessus du seuil (\textit{Above Threshold Ionization, ATI}) avec une impulsion polarisée circulairement dans une molécule chirale en présence d'un faisceau d'habillage\footnote{Il s'agit d'un analogue multiphotonique au RABBIT \mycite{ZippOptica2014}.}. L'ionisation multiphotonique d'une molécule chirale par une impulsion polarisée circulairement fait apparaître une asymétrie avant-arrière dans la distribution angulaire des photoélectrons (dichroïsme circulaire de photoélectrons) \mycite{BeaulieuNJP2016}. En utilisant le Rainbow RABBIT, le groupe de Bordeaux a pu reconstruire le paquet d'ondes électronique émis au voisinage d'une résonance "vers l'avant" et "vers l'arrière", et mis en évidence un délai attoseconde entre ces deux paquets d'ondes \mycite{Beaulieu2017}.

Dans le groupe d'A. L'Huillier à Lund, la mesure de la phase des pics satellites résolue spectralement a permis d'identifier des processus de shake-up dans la photoionisation du néon en couche $p$ contribuant aux oscillations des pics satellites de l'ionisation directe \mycite{IsingerArXiv2017}. Grâce à cette analyse, Isinger \textit{et al.} ont réinterprété les résultats expérimentaux de \mycite{SchultzeScience2010} (présentés au chapitre \ref{chap:DelaiPI}); ils ont attribué le désaccord entre la mesure et la simulation par la contribution de satellites de shake-up dans l'expérience de streaking de Schultze \textit{et al.} qui n'étaient ni résolus dans l'expérience ni pas pris en compte dans le calcul.

\paragraph*{Influence des résonances de Fano dans le milieu de génération d'harmoniques} Dans ce travail, nous avons étudié les résonances de Fano dans la photoionisation. \'{E}tant donné que la troisième étape du processus de génération d'harmoniques d'ordre élevé (dans le modèle simple en trois étapes) est la recombinaison radiative, c'est-à-dire l'inverse de la photoionisation, on s'attend à observer un effet des résonances de Fano dans la génération également. La GHOE au voisinage de ces structures permettrait la mise en forme de l'amplitude et de la phase des impulsions attosecondes \mycite{SchounPRL2014}.

De tels effets ont été observés pour la génération dans des plasma d'ablation de divers métaux: lorsque une harmonique est résonante avec un état autoionisant de l'ion métallique, une augmentation de l'efficacité de génération \mycite{GaneevOL2006} et une modification de la phase spectrale \mycite{HaesslerNJP2013} ont été mesurées. Ces observations ont été interprétées à l'aide d'un modèle en quatre étapes, où l'électron reste piégé transitoirement dans la résonance avant de recombiner sur l'ion parent en émettant un photon harmonique \mycite{StrelkovPRL2010}. Dans les gaz, \mycite{RothhardtPRL2014} ont constaté l'augmentation de l'efficacité de génération de l'harmonique résonante avec la résonance fenêtre $3s4p$ de l'argon à haute pression de génération, par rapport aux autres harmoniques. Cependant le mécanisme proposé ici est différent de celui invoqué dans les ions métalliques: les harmoniques non résonantes sont absorbées par le gaz à haute pression, mais ce n'est pas le cas de l'harmonique résonante qui est émise au voisinage d'un minimum de section efficace d'absorption. Il s'agit donc ici d'un effet collectif ou macroscopique plutôt qu'un effet de résonance dans la réponse de "l'atome unique".

Au cours de nos travaux sur les résonances d'autoionisation de l'argon, nous avons cherché à observer des modifications de l'amplitude et la phase lorsque l'argon est utilisé pour la GHOE. Aucun effet de résonance n'a été observé jusqu'à présent lorsque nous avons fait varier la longueur d'onde de génération. Une raison possible serait que, lors du processus de GHOE, le champ laser intense est présent au moment de la recombinaison. On s'attend alors à un déplacement pondéromoteur de l'énergie des états électroniques par rapport aux valeurs tabulées sans champ. En générant dans l'hélium, \mycite{PedatzurNatPhys2015} n'ont pas observé non plus d'influence de la résonance $2s2p$ sur l'intensité de l'émission harmonique. Les auteurs interprètent l'insensibilité de la GHOE aux résonances à la très brève émission de l'XUV dans le train d'impulsions attosecondes (voir \textit{Supplementary Information} de \mycite{PedatzurNatPhys2015}). La durée d'émission de seulement $\approx 300$ as des impulsions n'est pas suffisante pour peupler significativement l'état doublement excité d'une durée de vie de 17 fs. En revanche, des expériences de GHOE dans des molécules de N$_2$ alignées ont montré des déviations de la phase harmonique qui sont interprétées grâce à la présence de canaux autoionisants dans la génération \mycite{SchounPRL2017}.

De nouvelles expériences pour étudier l'influence des résonances d'autoionisation dans la génération d'harmoniques d'ordre élevé restent donc à effectuer. Cette question s'inscrit également dans la problématique plus large de la comparaison entre la spectroscopie harmonique (qui utilise l'émission harmonique comme sonde de la photo-recombinaison et des dynamiques en champ fort du milieu de génération) et la spectroscopie de photoionisation.

\paragraph*{Tomographie quantique de paquets d'ondes électroniques} La photoionisation d'un atome par un peigne d'harmoniques d'ordre élevé produit un paquet d'ondes électronique. Dans le formalisme de la matrice densité, la mesure du spectre de photoélectrons donne accès aux populations c'est-à-dire aux éléments diagonaux. Pour caractériser entièrement le paquet d'ondes électronique, il est nécessaire de mesurer également les éléments non-diagonaux. Il s'agit donc de faire interférer tous les pics de photoélectrons entre eux. Ceci est possible en effectuant une expérience de type RABBIT avec une forte intensité du laser d'habillage qui permet les transitions multiphotoniques couplant les pics les plus éloignés. C. Bourassin-Bouchet du Laboratoire Charles Fabry a proposé une telle approche afin d'effectuer une tomographie quantique du paquet d'ondes électroniques. Dans une expérience en collaboration, nous avons induit la photoionisation d'atomes de néon par un peigne de quatre harmoniques en présence d'un laser d'habillage intense de délai variable. En appliquant la méthode Mixed-FROG \mycite{BourassinNatComm2015} au spectrogramme, il a été possible de reconstruire le paquet d'ondes électronique émis. En particulier, nous avons mis en évidence plusieurs sources expérimentales de décohérence du paquet d'ondes telles que la réponse d'appareil du spectromètre et les fluctuations d'instant d'arrivée de l'XUV \mycite{Bourassin2017}. Cette méthode expérimentale et numérique est donc un très bon outil pour la métrologie attoseconde. En outre, si l'expérience est d'abord réalisée dans un gaz de référence pour identifier les sources expérimentales de décohérence, il serait possible d'étudier dans un second temps un autre système cible et d'ainsi caractériser complètement la photoionisation.

% Résolution angulaire des délais, Bordeaux + Lund

% Résonances de Fano dans la génération

% Tomographie quantique de paquets d'onde électroniques

% décroissance libre de l'induction

% Rabbit Auger ? 

% Exploration des paramètres de la PM à 10 kHz