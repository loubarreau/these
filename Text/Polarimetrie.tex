\part{Polarimétrie de l'émission harmonique générée par un champ à deux couleurs polarisées circulairement en sens opposé}
\label{part:Polarimétrie}
%% Biblio. Romain chap 13. 1. GHOE elliptiques. Budil Antoine modèle classique. 2. Autres méthodes: molécules alignées, résonances. 3. 2 couleurs
%% Calculs de TA
%% Expérience:voir VG et KV
% Principe de la PM. Résultats expérimentaux avec H19. Discussion sur le fond etc


% POurquoi faire de l'xuv elliptique, résolution temporelle
% Dans le vis IR, ok avec optiques, mais pas dans l'xuv (quand même Vodungbo sur des harmoniques mais efficacité de quelques %. Utilisé expériementalement pour du XMCD Willems PRB 2015)
% FEL (Allaria)

La question de la génération d'harmoniques d'ordre élevé polarisées elliptiquement est un problème actuel de la physique attoseconde. Cette partie est centrée sur le schéma de génération à deux couleurs polarisées circulairement en sens opposé. Pour un traitement plus large, le lecteur pourra se référer par exemple à \mycite{TheseHiguet}, \mycite{TheseGruson} ou \mycite{TheseGeneaux}.

\chapter{Génération d'harmoniques d'ordre élevé polarisées elliptiquement}
\label{chap:GHOE_elliptiques}
Dans ce chapitre, nous définirons d'abord les grandeurs utilisées pour caractériser la polarisation du rayonnement. Ensuite, nous présenterons les différentes méthodes développées pour produire des harmoniques d'ordre élevé polarisées elliptiquement. Nous montrerons d'abord que l'utilisation d'un faisceau fondamental elliptique est peu efficace et ne permet pas d'attendre des ellipticités élevées, puis nous présenterons brièvement les méthodes de GHOE dans les molécules alignées et au voisinage de résonances permettant d'obtenir un rayonnement harmonique elliptique. Enfin nous détaillerons le schéma de GHOE utilisant une impulsion laser à deux couleurs polarisées circulairement en sens opposé qui sera étudié numériquement et expérimentalement dans les chapitres suivants. 

\section{Polarisation de la lumière}
Considérons une onde plane monochromatique de pulsation $\omega_0$ $\vec{E}(z,t)$. D'après les équations de Maxwell, le champ électrique est transverse, c'est-à-dire orthogonal à son vecteur d'onde $\vec{k}$. Dans un repère cartésien $(Oxyz)$ tel que le champ se propage selon la direction $(Oz)$, on a:
\begin{equation}
\vec{E}(z,t) = \begin{pmatrix}
E_x \\
E_y\\
E_z
\end{pmatrix} =
\begin{pmatrix}
E_{0x} \: \cos(\omega_0 t - k z) \\
E_{0y} \: \cos(\omega_0 t - k z + \phi)  \\
0
\end{pmatrix}
\end{equation}
L'extrémité du vecteur champ électrique est l'ensemble des points tel que \mycite{BornWolf}: 
\begin{equation}
\left(\frac{E_x}{E_{0x}}\right)^2 + \left(\frac{E_y}{E_{0y}}\right)^2 - 2 \frac{E_x E_y}{E_{0x} E_{0y}} \cos \phi = \sin^2 \phi
\end{equation}
qui est l'équation d'une ellipse (voir figure \ref{fig:Polarellipse}). L'ellipse de polarisation est entièrement décrite dans $(Oxyz)$ par la donnée des amplitudes du champ selon $(Ox)$ et $(Oy)$, $E_{0x}$ et $E_{0y}$, et de la phase relative entre ces deux composantes $\phi$. Ces données sont équivalentes aux paramètres géométriques de l'ellipse: son demi grand axe $a$ et demi petit axe $b$ ainsi que l'angle $\psi$ entre le grand axe de l'ellipse et la direction $(Ox)$. L'ellipticité $\epsilon \in [-1; 1]$ est définie telle que 
\begin{equation}
\epsilon = \tan \chi = \pm \frac{b}{a}
\end{equation}
Si $\epsilon > 0$ (resp. $\epsilon <0$), l'ellipse est parcourue dans le sens horaire (resp. trigonométrique) lorsqu'elle est regardée par un observateur dans la direction opposée à la direction de propagation et la polarisation est appelée polarisation elliptique droite (resp. gauche). Si $E_{0x} = E_{0y}$, la polarisation est circulaire et $\epsilon = \pm 1$. Si $\epsilon = 0$, la polarisation est linéaire.

\begin{figure}
\centering
\def\svgwidth{0.7\textwidth}
\import{Figures/Polarimetrie/}{polarellipse.pdf_tex}
\caption{Ellipse de polarisation.}
\label{fig:Polarellipse}
\end{figure}

En pratique, le champ laser (ou harmonique) possède une durée finie, donc une certaine largeur spectrale autour de $\omega_0$, et un profil transverse non homogène. En chaque point $(x,y)$ du plan transverse et à l'instant $t$, le champ électrique est décrit localement par
\begin{equation}
\vec{E}(x,y,t,\omega) = E_x(x,y,t,\omega) \vec{u_x} + E_y(x,y,t,\omega) \vec{u_y}
\end{equation}
Pour décrire globalement un tel champ, Stokes introduit une représentation à quatre paramètres définis par \mycite{BornWolf}:
\begin{align}
S_0 & = <E_x E_x^* + E_y E_y^*> \\
S_1 & = <E_x E_x^* - E_y E_y^*> \\
S_2 & = <E_x E_y^* + E_y E_x^*>\\
S_3 & = i <E_x E_y^* - E_y E_x^*>
\end{align}
où la moyenne $< >$ est une moyenne temporelle, spatiale, et spectrale.
Il est possible que les différentes composantes temporelles et spatiales ne soient pas corrélées. Dans le cas extrême, toutes les orientations de $E_x$ et $E_y$ sont équiprobables et non corrélées durant la durée de l'observation et dans l'espace: la lumière est émise indépendamment par un grand nombre d'émetteurs. L'émission est alors dite non polarisée. C'est le cas de la lumière naturelle émise par rayonnement du corps noir, ou bien de la lumière émise par fluorescence. En général la lumière est partiellement polarisée, et on a la relation suivante entre les paramètres de Stokes:
\begin{equation}
S_0^2 \geqslant S_1^2 + S_2^2 + S_3^2
\end{equation}
avec égalité si et seulement si la lumière est complètement polarisée. Il est utile de définir les paramètres de Stokes normalisés par $S_0$: $s_1 = S_1 /S_0$, $s_2 = S_2 /S_0$ et $s_3 = S_3 /S_0$.
Le degré de polarisation de l'onde, $P$, est donné par 
\begin{equation}
P = \frac{I_{\text{pol}}}{I_{\text{tot}}} = \frac{\sqrt{S_1^2 + S_2^2 + S_3^2}}{S_0} = \sqrt{s_1^2 + s_2^2 + s_3^2}
\label{eq:degrépolarisation}
\end{equation}
Les paramètres de Stokes sont reliés aux paramètres géométriques de l'ellipse:
\begin{align}
\psi & = \frac{1}{2} \arctan \frac{s_2}{s_1} \label{eq:PsiStokes} \\
\epsilon & = \tan \left[ \frac{1}{2} \arcsin \frac{s_3}{\sqrt{s_1^2 + s_2^2 + s_3^2}} \right]
\label{eq:epsilon_Stokes}
\end{align}
La donnée des quatre paramètres de Stokes est suffisante pour décrire complètement la polarisation du champ, et est équivalente à la donnée de l'ellipticité $\epsilon$, de l'angle entre le grand axe de l'ellipse et $(Ox)$ $\psi$ et du degré de polarisation $P$. D'après l'équation \ref{eq:epsilon_Stokes}, si la lumière est complètement polarisée, il existe une relation directe entre $\epsilon$ et $s_3$. Cependant, si la lumière est partiellement polarisée, il est nécessaire de mesurer simultanément $s_1$, $s_2$ et $s_3$ pour déterminer l'ellipticité.

\section{Génération d'harmoniques d'ordre élevé à partir d'un laser polarisé elliptiquement}
L'idée la plus simple pour produire un rayonnement harmonique polarisé elliptiquement est de chercher à transférer à l'XUV les propriétés de polarisation du faisceau fondamental.

La première expérience de GHOE à partir d'un laser polarisé elliptiquement a été brillamment effectuée par \mycite{BudilPRA1993}. Un laser à colquiriite dopée au chrome Cr:LiSrAlF$_6$ produit des impulsions de 125 fs à 825 nm qui sont focalisées dans un gaz rare pour générer les harmoniques d'ordre élevé. La polarisation du laser est variée  de linéaire à circulaire gauche et droite grâce à une lame quart d'onde large bande. Le signal harmonique est mesuré en fonction de l'ellipticité du laser. Les résultats pour les harmoniques 15 à 63 dans le néon sont reportés figure \ref{fig:Budil}. On constate que l'efficacité de génération décroit exponentiellement avec l'ellipticité IR. Pour $|\epsilon| = 0.2$, le signal diminue de plus d'un ordre de grandeur par rapport au cas linéaire. La décroissance est plus rapide lorsque l'ordre de l'harmonique est grand.

\begin{figure}
\centering
\def\svgwidth{0.8\textwidth}
\import{Figures/Polarimetrie/}{Budil.pdf_tex}
\caption{Signal harmonique en fonction de l'ellipticité du laser IR, normalisé par rapport à la polarisation linéaire (échelle logarithmique). Extrait de \mycite{BudilPRA1993}.}
\label{fig:Budil}
\end{figure}

Quelques années plus tard, \mycite{AntoinePRA1997} ont mesuré optiquement l'ellipticité et l'angle de l'ellipse de polarisation d'harmoniques générées dans plusieurs gaz rares à partir d'un faisceau IR polarisé elliptiquement. Les résultats expérimentaux sont comparés à des calculs dans l'approximation du champ fort (\textit{Strong Field Approximation, SFA}), incluant la propagation, de l'ellipticité et du degré de polarisation de l'XUV. Les résultats pour les harmoniques 17 et 23, correspondant respectivement à la fin du plateau et à la coupure à l'intensité utilisée dans l'argon, sont reproduits figure \ref{fig:Antoine}. La mesure de l'ellipticité par une méthode optique avec une loi de Malus ne permet pas de déterminer le paramètre de Stokes $s_3$, mais seulement $s_1$ et $s_2$ \mycite{AntoinePRA1997}\mycite{TheseGruson}. Le calcul de l'ellipticité se fait alors en utilisant les relations \ref{eq:degrépolarisation} et \ref{eq:epsilon_Stokes}, en supposant le rayonnement complètement polarisé ($P = 1$). Ainsi, la grandeur mesurée est seulement une valeur apparente maximale de l'ellipticité harmonique, $\epsilon_q^{\text{app}}$. Si le rayonnement est seulement partiellement polarisé, l'ellipticité de la partie polarisée de la lumière ("vraie" ellipticité) est inférieure à $\epsilon_q^{\text{app}}$. D'une part, on constate que l'angle de rotation de l'ellipse est faible et augmente de manière quasi-linéaire avec l'ellipticité IR. D'autre part, l'ellipticité des harmoniques est une fonction croissante de l'ellipticité du fondamental, mais lui est inférieure, de manière plus significative pour l'harmonique de la coupure. Pour l'harmonique du plateau, une grande ellipticité mesurée est associée à un plus faible degré de polarisation. L'ellipticité "vraie" calculée est alors deux fois plus faible que l'ellipticité mesurée.

Les différentes observations s'interprètent facilement dans le cadre semi-classique du modèle en trois étapes, voir par exemple \mycite{TheseGeneaux} chap. 13. En particulier, la diminution de l'efficacité de génération avec l'ellipticité du champ est due au fait que la trajectoire électronique dans le continuum ne recombine pas sur l'ion parent.

Ces deux expériences montrent qu'il est possible de transférer l'ellipticité du fondamental vers l'XUV lors de la génération d'harmoniques. Cependant, le processus est très peu efficace et ne permet pas d'atteindre des ellipticités élevées. Il est donc nécessaire de développer d'autres méthodes de production d'harmoniques et d'impulsions attoseconde polarisées elliptiquement. Par ailleurs, \mycite{AntoinePRA1997} soulignent le fait que les mesures de type loi de Malus sans élément déphaseur sur le champ XUV ne peuvent déterminer ni le signe de l'ellipticité ni le paramètre de Stokes $s_3$. Ainsi, pour une caractérisation complète de la polarisation de l'émission harmonique, il est également nécessaire de développer d'autres méthodes polarimétriques. Une telle méthode sera présentée et utilisée au chapitre \ref{chap:MesurePolar}.


\begin{figure}
\centering
\def\svgwidth{\textwidth}
\import{Figures/Polarimetrie/}{antoinePRA.pdf_tex}
\caption{(a-b) Ellipticité harmonique apparente mesurée $\epsilon_q^{\text{app}}$ par une loi de Malus (points rouges) et calculée (pointillés rouges), "Vraie" ellipticité calculée (pointillés noirs) et degré de polarisation calculé (pointillés verts, échelle de droite) pour les harmoniques 17 (a) et 23 (b) de l'argon. Le signe de l'ellipticité mesurée n'est pas possible à déterminer ici, il est donc choisi identique à celui de l'ellipticité calculée. (c-d) Angle du grand axe de l'ellipse $\psi$ mesuré (points rouges) et calculé (pointillés rouges) pour les harmoniques 17 (c) et 23 (d) de l'argon. Extrait de \mycite{TheseGeneaux}, adapté de \mycite{AntoinePRA1997}.}
\label{fig:Antoine}
\end{figure}

\section{Autres méthodes de génération d'harmoniques d'ordre élevé polarisées elliptiquement}
\subsection{Génération dans des molécules alignées}
Par symétrie, les harmoniques d'ordre élevé générées par un laser polarisé linéairement dans un milieu isotrope sont polarisées linéairement. L'ellipticité ne peut provenir que d'une brisure de symétrie. Dans le paragraphe précédent, la brisure de symétrie provient du champ IR elliptique qui génère les harmoniques dans un milieu isotrope. Les rôles du champ et du milieu peuvent être inversés: les harmoniques peuvent être générées par un laser polarisé linéairement dans un milieu orienté.

\begin{figure}[ht]
\centering
\def\svgwidth{\textwidth}
\import{Figures/Polarimetrie/}{ZhouMairesse.pdf_tex}
\caption{(a) Ellipticité $\epsilon_q^{\text{app}}$ en fonction de l'ordre harmonique pour plusieurs angles d'alignement de la molécule N$_2$ par rapport à la polarisation linéaire du laser de génération ($I_{\text{gen}} = 2 \times 10^{14}$ W/cm$^2$). Adapté de \mycite{ZhouPRL2009}. (b) Ellipticité $\epsilon_q^{\text{app}}$ en fonction de l'ordre harmonique pour plusieurs angles d'alignement de la molécule N$_2$ par rapport à la polarisation linéaire du laser de génération ($I_{\text{gen}} = 8 \times 10^{13}$ W/cm$^2$). Adapté de \mycite{MairessePRL2010}.}
\label{fig:ZhouMairesse}
\end{figure}

Plusieurs méthodes d'alignement de molécules ont été développées pour la GHOE, en particulier dans le cadre de la spectroscopie harmonique. Dans le schéma le plus courant d'alignement impulsionnel, une pré-impulsion laser intense de quelques centaines de femtosecondes est focalisée dans le gaz et créée un paquet d'onde rotationnel dans l'état fondamental de la molécule. Ce paquet d'onde évolue librement après l'interaction, et se rephase périodiquement (toutes les périodes rotationnelles de la molécule $T_{\text{rot}} = 1/2cB$; pour N$_2$ $B = 1.998$ cm$^{-1}$ d'où $T_{\text{rot}} = 8.35$ ps) à des instants où les molécules sont alignées parallèlement ou perpendiculairement à la polarisation du laser de pompe \mycite{RoscaPrunaPRL2001}. L'impulsion laser de GHOE est retardée de manière à générer les harmoniques à l'instant où les molécules sont alignées. \`{A} cet instant, l'impulsion laser de pompe n'est plus présente dans le milieu et ne perturbe pas la GHOE.

En utilisant cette méthode, \mycite{ZhouPRL2009} puis \mycite{MairessePRL2010} ont mesuré optiquement l'ellipticité maximale des harmoniques générées dans des molécules de N$_2$ alignées, pour plusieurs angles d'alignement. Les résultats sont reproduits figure \ref{fig:ZhouMairesse}. Dans ces deux expériences, une ellipticité apparente $\epsilon_q^{\text{app}}$ significative ($\approx 0.35$ et $\approx 0.4 - 0.5$ resp.) a été mesurée pour les harmoniques 21 - 23 d'un laser à 800 nm lorsque les molécules sont alignées à 60° de la polarisation du laser. Soulignons ici que le signe de l'ellipticité ainsi que le degré de polarisation du rayonnement n'est pas mesurable dans ces expériences. En réalité, l'interprétation de ces résultats est plus complexe qu'une simple brisure de symétrie: la même expérience de GHOE dans CO$_2$ aligné produit des harmoniques polarisées linéairement \mycite{ZhouPRL2009}. L'ellipticité est due à l'interférence entre plusieurs canaux d'ionisation lors de la GHOE dans N$_2$, et est donc un résultat important pour la spectroscopie harmonique.

En revanche, la production d'harmoniques elliptiques à partir de molécules alignées s'avère peu pratique. Si le niveau de signal est cette fois-ci suffisant, la technique est complexe à mettre en \oe uvre. Une ellipticité harmonique élevée requiert un excellent taux d'alignement des molécules qui peut s'obtenir:
\begin{itemize}
\item En refroidissant la température rotationnelle du gaz grâce à l'utilisation d'un jet pulsé \mycite{TheseCamper}; cependant l'utilisation d'un jet pulsé diminue la densité du milieu et donc le flux de photons XUV.
\item En utilisant un schéma d'alignement à plusieurs impulsions \mycite{CryanPRA2009}: une nouvelle impulsion d'alignement interagit avec les molécules à chaque période rotationnelle, ce qui complexifie considérablement le dispositif expérimental et requiert une quantité d'énergie importante pour la pompe.
\end{itemize}
Par ailleurs, si l'intensité du faisceau d'alignement n'est pas homogène spatialement dans la zone de génération des harmoniques, ou bien s'il existe une gigue spatiale ou temporelle du faisceau d'alignement, des simulations avec le modèle QRS (\textit{Quantitative Rescattering Theory}, \mycite{LePRA2009}) ont montré que les harmoniques générées dans N$_2$ aligné n'étaient pas complètement polarisées ($P \approx 0.9$) \mycite{TheseGruson}. Toutes ces contraintes pour obtenir un rayonnement harmonique d'ellipticité significative mais inférieure à 0.5 ont contribué au développement d'autres méthodes de GHOE polarisées elliptiquement.

\subsection{Génération au voisinage d'une résonance}
La préparation du milieu préalable à la GHOE elliptiques requiert un relativement lourd équipement expérimental. Ainsi, il serait pratique d'utiliser une propriété intrinsèque au milieu de génération qui exalte la production de rayonnement elliptique. \mycite{FerreNatPhot2015} ont montré que la GHOE au voisinage de résonances possédait ce type de propriétés. Tout d'abord, les auteurs ont généré les harmoniques d'un laser à 400 nm polarisé elliptiquement ($\epsilon_{\text{fond}} = 0.4$) dans l'argon (figure \ref{fig:FerreAr}). Nous avons vu précédemment que l'ellipticité harmonique est inférieure à l'ellipticité du fondamental \mycite{AntoinePRA1997}. De manière surprenante, Ferré \textit{et al.} ont mesuré par polarimétrie optique $\epsilon_5^{\text{app}} = 0.77$, soit deux fois l'ellipticité du fondamental, pour l'harmonique 5. L'harmonique 5 à 400 nm ($\approx 5 \times 3.1 = 15.5$ eV) se trouve sous le seuil d'ionisation de l'argon, dans une région spectrale riche en états de Rydberg. La présence de ces résonances modifie donc les propriétés de polarisation de l'harmonique, ce qui est confirmé également par des calculs théoriques.

\begin{figure}
\centering
\def\svgwidth{\textwidth}
\import{Figures/Polarimetrie/}{FerreAr.pdf_tex}
\caption{Intensité du signal et ellipticité maximale $\epsilon_q^{\text{app}}$ mesurée par polarimétrie optique d'harmoniques générées dans l'argon par un laser à 400 nm d'ellipticité $\epsilon_{\text{fond}} = 0.4$. L'énergie de quelques états de Rydberg de l'argon est indiquée en vert. Adapté de \mycite{FerreNatPhot2015}.}
\label{fig:FerreAr}
\end{figure}

\begin{figure}
\centering
\def\svgwidth{\textwidth}
\import{Figures/Polarimetrie/}{FerreSF6.pdf_tex}
\caption{Intensité du signal (haut) et ellipticité apparente $\epsilon_q^{\text{app}}$ (bas) mesurée par polarimétrie optique d'harmoniques générées dans SF$_6$ par un laser à 400 nm ou 800 nm d'ellipticité variable. Adapté de \mycite{FerreNatPhot2015}.}
\label{fig:FerreSF6}
\end{figure}

Dans la suite de leur travail, \mycite{FerreNatPhot2015} montrent que cette propriété s'étend aux résonances de forme. L'ellipticité d'harmoniques générées au voisinage de la résonance de forme de SF$_6$ vers 25 eV \mycite{YangJElecSpec1998} \mycite{FerreNatComm2015} atteint $\epsilon_{15}^{\text{app}} \approx 0.8$ pour $\epsilon_{\text{fond}} = 0.2$ (figure \ref{fig:FerreSF6}).

Ces mesures de polarimétrie optique ont été confirmées par notre groupe lors de mesures de polarimétrie moléculaire \mycite{VeyrinasFaraday2016} \mycite{TheseGruson}. Cette méthode sera présentée en détail dans le chapitre \ref{chap:MesurePolar}. Elle permet, entre autres, de mesurer tous les paramètres de Stokes du rayonnement, donc de séparer la polarisation circulaire de la partie non polarisée. L'ellipticité correspond donc à la "vraie" ellipticité de la partie polarisée du rayonnement. Les résultats obtenus sont présentés figure \ref{fig:ResultatsFaradSF6}. Nous avons mesuré une ellipticité de $\epsilon_q \approx +0.6$ avec $\epsilon_{\text{fond}} = +0.23$ et $\epsilon_q \approx -0.6$ avec $\epsilon_{\text{fond}} = -0.17$ pour les harmoniques 15 et 17 générées dans SF$_6$ avec un laser à 800 nm. Les deux autres harmoniques mesurées possèdent une ellipticité modeste. Nous soulignons ici qu'il s'agit de la première mesure du signe de $\epsilon_q$ dans ces conditions. Par ailleurs, nous avons pu déterminer le degré de polarisation des harmoniques et mis en évidence l'existence d'une dépolarisation significative dans le rayonnement harmonique produit au voisinage de la résonance de forme de SF$_6$.

\begin{figure}
\centering
\def\svgwidth{\textwidth}
\import{Figures/Polarimetrie/}{BordeldeVeyrinas.pdf_tex}
\caption{Ellipticité "vraie" $\epsilon_q$ (a) et degré de polarisation $P$ (b) d'harmoniques générées dans SF$_6$ par un laser à 800 nm polarisé elliptiquement avec $\epsilon_{IR} = +0.23$ (bleu) et -0.17 (rouge) mesurés par polarimétrie moléculaire. Adapté de \mycite{VeyrinasFaraday2016}.}
\label{fig:ResultatsFaradSF6}
\end{figure}

Les harmoniques de SF$_6$ ont été utilisées dans une expérience de dichroïsme circulaire de photoélectrons \mycite{FerreNatPhot2015}, démontrant ainsi que ce schéma permet la génération d'un flux suffisant de photons XUV de polarisation quasi-circulaire. Cependant, les résonances sous le seuil comme les résonances de forme sont généralement situées au voisinage des potentiels d'ionisation atomiques ou moléculaires, soit à des énergies de l'ordre de quelques dizaines d'électron-volts. Les résonances de forme sont moins étroites spectralement que les résonances atomiques, mais restent localisées spectralement. Ce schéma de génération apparaît donc comme un bon outil de production de rayonnement XUV elliptique pour l'étude de dichroïsmes circulaires dans les molécules, mais ne permet pas d'atteindre les énergies et les largeurs spectrales nécessaires, par exemple, pour  l'étude de matériaux magnétiques \mycite{StohrJElecSpec1995}.

\section[GHOE à partir d'un champ à deux couleurs polarisées circulairement en sens opposé]{Génération d'harmoniques d'ordre élevé à partir d'un champ à deux couleurs polarisées circulairement en sens opposé}
\label{sec:Intro2couleurs}
Le schéma de GHOE à partir d'un mélange de deux impulsions à la fréquence fondamentale et sa seconde harmonique (généralement 800 nm et 400 nm) polarisées circulairement en sens opposé a été proposé expérimentalement par \mycite{EichmannPRA1995}. De manière surprenante, cette configuration à deux couleurs produit efficacement un spectre très étendu, composé de toutes les harmoniques exceptées les multiples de 3. En effet, dans une approche multiphotonique de la GHOE, le système absorbe un nombre $m$ de photons de fréquence $\omega$ et un nombre $n$ de photons de fréquence 2$\omega$: 
\begin{equation}
\Omega_{n,m} = n \times \omega + m \times 2 \omega
\end{equation}
L'émission du photon harmonique de fréquence $\Omega_{n,m}$ correspond à une variation de moment angulaire orbital du système $\Delta \ell = \pm 1$. Ainsi le système doit absorber un nombre total de photons impair, soit un photon supplémentaire d'une couleur, pour conserver le moment angulaire:
\begin{equation}
n = m \pm 1
\end{equation}
Les fréquences harmoniques autorisées sont donc
\begin{equation}
\Omega_{m \pm 1,m} = (3m \pm 1) \omega
\end{equation}
Par ailleurs, les deux impulsions à $\omega$ et 2$\omega$ sont polarisées circulairement en sens opposé. Donc l'un des faisceaux porte un moment angulaire de spin de +1 tandis que l'autre porte un moment angulaire de spin de -1. Par conservation du moment angulaire de spin, l'harmonique émise porte le moment angulaire du photon supplémentaire absorbé. Donc les harmoniques $3m+1$ (resp. $3m-1$) sont polarisées circulairement dans le même sens que le faisceau à la fréquence $\omega$ (resp. 2$\omega$).

Ces conclusions des lois de conservations ont été remarquées dès les travaux pionniers de \mycite{EichmannPRA1995}, mais sans que la mesure de l'état de polarisation des harmoniques ne soit effectuée. Ce schéma permet donc en principe la production d'harmoniques polarisées circulairement sur tout le spectre, dont l'hélicité alterne et est contrôlée par l'hélicité respective des champs à $\omega$ et 2$\omega$.

\begin{figure}[ht]
\centering
\def\svgwidth{\textwidth}
\import{Figures/Polarimetrie/}{Trajectoires_Milosevic.pdf_tex}
\caption{(a) Courbe décrite par le champ électrique à deux couleurs pendant une période du champ à $\omega$, avec $\hbar \omega = 1.6$ eV et $I_{\omega} = I_{2\omega} = 4 \times 10^{14}$ W/cm$^2$. (b) Trajectoires électroniques correspondant aux harmoniques 19 à 51. Les instants d'émission et de recombinaison pour les différentes trajectoires sont indiquées par des carrés blancs et des cercles noirs respectivement en (a). La position du c\oe ur ionique est matérialisée par un cercle. Extrait de \mycite{MilosevicPRA2000_Generation}}
\label{fig:Trajectoires_Milosevic}
\end{figure}

Bien qu'utilisant des champs polarisés circulairement ($\epsilon_{\omega, 2\omega} = \pm 1$), la GHOE avec ce schéma est très efficace. En réalité, le champ électrique total ressenti par l'électron ne décrit pas un cercle (figure \ref{fig:Polarellipse}) mais un trèfle à trois feuilles (figure \ref{fig:Trajectoires_Milosevic}(a)). Dans ce champ, \mycite{MilosevicPRA2000_Generation} ont calculé les trajectoires électroniques dans l'approximation du champ fort. Tout comme il existe des trajectoires "courtes" et "longues" dans la GHOE à une couleur polarisée linéairement (voir partie \textbf{ref}), plusieurs trajectoires électroniques sont susceptibles d'émettre des harmoniques. Cependant, les auteurs montrent qu'une famille de trajectoires domine l'émission; elles sont représentées figure \ref{fig:Trajectoires_Milosevic}(b) et correspondent à des instants d'ionisation et de recombinaison indiquées sur la figure \ref{fig:Trajectoires_Milosevic}(a). Les trajectoires commencent à $\approx$ 4 unités atomiques du c\oe ur (ce qui correspond à la "sortie du tunnel" lors de l'ionisation) et recombinent sur le c\oe ur en suivant une trajectoire très semblable au cas de la GHOE à une couleur linéaire, ce qui explique l'importante efficacité de ce dispositif de génération. 

Par ailleurs, l'émission harmonique correspondante est alors polarisée linéairement. Ce processus se répète trois fois pendant une période $T$ du champ à $\omega$ en tournant de 120°, c'est-à-dire à chaque "feuille" du trèfle. La combinaison des trois émissions attoseconde linéaires dont la polarisation tourne de 120° tous les $T/3$ est responsable de la circularité des harmoniques. Le rayonnement XUV possède donc des propriétés de polarisation inhabituelles: une harmonique donnée possède une polarisation circulaire, mais l'émission est constituée d'un train d'impulsion attoseconde émises tous les $T/3$ de polarisation linéaire tournant de 120° d'une impulsion à l'autre dans le train \mycite{MilosevicPRA2000_Unusual}.

\begin{figure}[ht]
\centering
\def\svgwidth{0.5\textwidth}
\import{Figures/Polarimetrie/}{CitationsEichmann.pdf_tex}
\caption{Nombre de citations par année de \mycite{EichmannPRA1995} analysé par Web of Knowledge en août 2017.}
\label{fig:CitationsEichmann}
\end{figure}

Comme le suggère la figure \ref{fig:CitationsEichmann}, l'expérience pionnière de \mycite{EichmannPRA1995} suscite un intérêt grandissant depuis 2014. Cette année-là, \mycite{FleischerNatPhot2014} déterminent l'ellipticité apparente $\epsilon_{q}^{\text{app}}$ par polarimétrie optique des harmoniques générées dans l'argon avec ce schéma à deux couleurs. Les auteurs mesurent $|\epsilon_{19}^{\text{app}}| \approx 1 $ et $|\epsilon_{20}^{\text{app}}| \approx 0.8$. Une autre mesure effectuée lorsque le champ à 800 nm n'est pas parfaitement circulaire indique une diminution de $|\epsilon_{20}^{\text{app}}|$, conformément aux prédictions théoriques de \mycite{MilosevicPRA2000_Generation}. Cependant, sans une mesure simultanée du paramètre de Stokes $s_3$ il est impossible d'attribuer l'ellipticité apparente à de la polarisation circulaire ou à une partie dépolarisée du rayonnement. Le signe de l'ellipticité n'est également pas déterminé ici. La première mesure du paramètre $s_3$ des harmoniques générées par un champ à deux couleurs polarisées circulairement en sens opposé est effectuée par \mycite{KfirNatPhot2015} grâce à une mesure de dichroïsme circulaire magnétique dans les rayons X (DCMX ou \textit{X-ray Magnetic Circular Dichroism}, XMCD) au seuil M du cobalt (figure \ref{fig:SM_Kfir}). Les auteurs mesurent des signes opposées pour les $s_3$ de deux harmoniques consécutives, indiquant l'alternance de l'hélicité, avec les harmoniques $3m+1$ ($3m-1$) possédant la même hélicité que le champ à $\omega$ (resp. 2$\omega$). Le paramètre $s_3$ des harmoniques $3m-1$ est significativement plus faible que pour les $3m+1$, indiquant une polarisation elliptique ou bien un faible degré de polarisation. Cette différence est attribuée à une différence d'accord de phase pour les deux hélicités, critique avec le dispositif de génération dans une fibre creuse longue de 2 cm utilisé par Kfir \textit{et al}. Notons ici que l'expérience de DCMX ne permet pas la mesure simultanée des trois paramètres de Stokes mais uniquement de $s_3$, les auteurs ne peuvent donc pas déterminer le degré de polarisation du rayonnement.

\begin{figure}
\centering
\def\svgwidth{\textwidth}
\import{Figures/Polarimetrie/}{SM_Kfir.pdf_tex}
\caption{(a) Spectre généré dans le néon par deux champs laser à 790 nm et 395 nm de polarisation respectivement circulaire gauche et circulaire droite (rouge). Les harmoniques $3m-1$ et $3m+1$ ne sont pas de même intensité. Extrait de \mycite{KfirNatPhot2015}. (b) Paramètre de Stokes normalisé $s_3$ des harmoniques (31, 32) et (34, 35)  mesurés par DCMX. Extrait du \textit{Supplementary Information} de \mycite{KfirNatPhot2015}.}
\label{fig:SM_Kfir}
\end{figure}

Avec les travaux de \mycite{FleischerNatPhot2014} et \mycite{KfirNatPhot2015}, la communauté de la physique ultra-rapide re-découvre l'immense potentiel de ce schéma de GHOE polarisées circulairement. En utilisant des longueurs d'onde plus élevées (800 nm et 1300 nm), \mycite{FanPNAS2015} montrent expérimentalement que le schéma de génération reste valide avec des longueurs d'onde non multiples. En utilisant l'hélium comme milieu de génération, Fan \textit{et al.} obtiennent un spectre d'harmoniques s'étendant jusqu'à 150 eV. Leurs paramètres $s_3$ sont déterminés par DCMX aux seuils M du fer ($\approx 53$ eV) et N du gadolinium ($\approx 145$ eV), mais n'atteint que $|s_3| \approx$ 0.6 pour les harmoniques les plus élevés (soit $\epsilon_{q}^{\text{app}} \approx$ 0.3 en appliquant la relation \ref{eq:epsilon_Stokes} et en supposant un rayonnement complètement polarisé). Cette déviation de la parfaite circularité est attribuée à une petite non-circularité des champs de génération \mycite{FleischerNatPhot2014}\mycite{MilosevicPRA2000_Generation}. Lorsque l'écart à la circularité des champs est important, les harmoniques $3m$ peuvent être générées et sont observées dans le spectre.

En spectroscopie harmonique, \mycite{BaykushevaPRL2016} utilisent ce schéma de génération comme sonde des symétries dynamiques de N$_2$ et SF$_6$, une utilisation également discutée théoriquement par \mycite{ReichPRL2016}. Plusieurs travaux ont étudié l'ionisation \mycite{MancusoPRA2016} et la double ionisation \mycite{MancusoPRL2016}\mycite{EckartPRL2016} en champ fort avec le dispositif à deux couleurs polarisées circulairement en sens opposé.

En particulier, ce schéma de génération pourrait permettre de produire des impulsions attoseconde polarisées cicrulairement. S'il est possible de contrôler la différence d'intensité observée entre les harmoniques d'hélicité différente (figure \ref{fig:SM_Kfir}(a)), et de générer un spectre ou l'une des familles (par exemple $3m+1$) domine, la polarisation sera circulaire également à l'échelle attoseconde. En étudiant les solutions de l'équation de Schrödinger dépendante du temps, \mycite{MedisauskasPRL2015} ont montré que cette différence d'intensité est due à la génération à partir d'électrons $p$, comme par exemple dans le néon ou l'argon, dont l'ionisation tunnel dépend de leur nombre quantique magnétique \mycite{AyusoNJP2017}. Les calculs dans l'approximation du champ fort de \mycite{MilosevicPRA2015} arrivent aux mêmes conclusions. Notons cependant que ces études se sont intéressées à l'état de polarisation de l'émission attoseconde dans le train et non à la polarisation des harmoniques individuellement. 

\`{A} notre connaissance, la littérature a jusqu'à présent supposé que les harmoniques générées avec ce dispositif étaient complètement polarisées, sans que tous leurs paramètres de Stokes n'aient été mesurés simultanément. De plus, lorsqu'il est caractérisé, l'écart à la circularité des harmoniques est attribué uniquement à une imparfaite circularité de l'un des deux champs de génération. En pratique, l'intensité des deux champs, leur délai relatif ou leur durée peuvent-ils avoir une influence sur l'ellipticité et le degré de polarisation des harmoniques? \`{A} quel point les règles de sélection dictant la circularité des harmoniques sont-elles strictes dans le cadre d'une expérience "réelle"? Compte-tenu de tous les effets expérimentaux, quel est l'état de polarisation complet ($s_1, s_2$ et $s_3$) des harmoniques générées avec un tel dispositif? D'une part, nous avons étudié numériquement l'influence de plusieurs paramètres expérimentaux sur les paramètres de Stokes des harmoniques en résolvant l'équation de Schrödinger dépendante du temps. Ces résultats seront présentés dans le chapitre \ref{chap:calculsTA}. D'autre part, nous avons mesuré l'état de polarisation complet des harmoniques générées avec deux couleurs polarisées circulairement en sens opposé grâce à la méthode de polarimétrie moléculaire \mycite{VeyrinasPRA2013}\mycite{VeyrinasFaraday2016}, en collaboration avec Kévin Veyrinas, Jean-Christophe Houver et Danielle Dowek de l'Institut des Sciences Moléculaires d'Orsay. Une version préliminaire de ces résultats a été présentée dans les thèses de Kévin Veyrinas \mycite{TheseVeyrinas} et de Vincent Gruson \mycite{TheseGruson}. Nous en présenterons dans le chapitre \ref{chap:MesurePolar} une version actualisée.

\chapter[Etude numérique de la GHOE par un champ à deux couleurs polarisées circulairement en sens opposé]{\'{E}tude numérique de la génération d'harmoniques d'ordre élevé par un champ à deux couleurs polarisées circulairement en sens opposé}
\label{chap:calculsTA}

Numériquement, il est possible d'étudier l'influence sur la polarisation des harmoniques de plusieurs paramètres du champ de manière indépendante. Dans ce chapitre, nous avons étudié la GHOE par un champ à deux couleurs polarisées elliptiquement en sens opposé dans les conditions suivantes:
\begin{enumerate}[label=\roman*)]
\item dans l'hélium avec des impulsions à $\omega$ et 2$\omega$ d'enveloppes $\sin^2$ ultra-brèves.
\item dans l'argon avec des impulsions trapézoïdales de plusieurs intensités.
\item dans le cas ii) à intensité donnée, nous avons fait varier l'ellipticité du champ à $\omega$.
\item dans le cas ii), à intensité donnée et pour une ellipticité du champ à $\omega \neq \pm 1$, nous avons fait varier la phase relative entre les champs à $\omega$ et 2$\omega$.
\end{enumerate}
Notons ici que les simulations ne considèrent que la réponse de l'atome unique et ne prennent pas en compte les effets de propagation susceptibles de modifier l'état de polarisation harmonique \mycite{AntoinePRA1997}\mycite{KfirNatPhot2015}\mycite{FanPNAS2015}.

\section{Méthodes numériques}
\label{sec:MethodesNum}
Tous les calculs de résolution de l'équation de Schrödinger dépendante du temps (ESDT) de ce chapitre ont été effectués par Thierry Auguste de l'équipe Attophysique du LIDYL. 

L'ESDT est résolue en jauge vitesse sur une grille cartésienne à deux dimensions de taille $819.2 \times 819.2$ unités atomiques avec un pas de 0.2 u.a. Les valeurs propres et vecteurs propres de l'Hamiltonien stationnaire sont calculés avec la méthode du temps de propagation imaginaire, pour un potentiel de Rochester ("\textit{soft-core}") reproduisant le potentiel d'ionisation de l'atome:
\begin{equation}
V(x,y) = - \frac{(Z-1) \: \rme^{-(x^2+y^2) + 1}}{\sqrt{x^2 + y^2 + \eta}}
\end{equation}
avec pour l'argon $Z = 18$ et $\eta = 1.04327$. Pour l'argon, les états $3p_{m=\pm 1}$ sont formés à partir des orbitales $3p_x$ et $3p_y$:
\begin{equation}
3p_{m=\pm 1} = 3p_x \pm i \: 3p_y
\end{equation}
Les fonctions d'onde initiales sont ensuite propagées dans le temps en utilisant la méthode de Fourier "\textit{split-step}" \mycite{FEITJCP1982} avec un pas de $6.1 \times 10^{-5}$ u.a. Pour éviter les artefacts lorsque le paquet d'onde atteint les limites du domaine numérique, la méthode d'absorption dépendante de la longueur d'onde est appliquée à chaque pas de temps \mycite{StrelkovPRA2012}. La profondeur de la couche absorbante est de 8 u.a. dans les deux dimensions et de chaque côté de la grille.

Le champ laser est la somme de deux ondes polarisées circulairement de longueurs d'onde centrales $\lambda_1 = 800$ nm et $\lambda_2 = 400$ nm:
\begin{equation}
\vec{E}(t) = f(t) \begin{pmatrix}
E_1 \cos (\omega t + \phi) + E_2 \cos (2\omega t) \\
E_1 \epsilon_1 \sin (\omega t + \phi) + E_2 \epsilon_2 \sin (2\omega t)
\end{pmatrix}
\label{eq:champTA}
\end{equation}
d'amplitude $E_i = \frac{E_0}{\sqrt{1 + \epsilon_i^2}}$ et d'ellipticité $\epsilon_i$ pour $i = 1,2$. $f(t)$ est l'enveloppe de l'impulsion; les influences de sa forme et de la phase porteuse-enveloppe sont étudiées dans la suite de ce chapitre. $\phi$ est la phase relative entre les champs à $\omega$ et 2$\omega$.

Les spectres harmoniques sont obtenus en effectuant la transformée de Fourier des composantes de l'accélération du dipôle calculées, intégrées sur tout l'espace pour chaque pas de temps. Les paramètres de Stokes sont calculés à partir de l'accélération du dipôle dans le domaine des fréquences. L'analyse est restreinte aux harmoniques au-dessus du potentiel d'ionisation du milieu considéré, les harmoniques sous le seuil possédant des propriétés particulières \mycite{MedisauskasPRL2015}. Dans un calcul à trois dimensions, la seule différence attendue est la diffusion du paquet d'onde électronique du continuum dans la dimension perpendiculaire au plan de polarisation du champ, identique pour les orbitales $p_{-1}$ et $p_{+1}$. Les résultats devraient donc être identiques dans un calcul à trois dimensions \mycite{MedisauskasPRL2015}.

\section{Champs de génération ultra-brefs}
\label{sec:ChampsBrefs}
Le trèfle à trois feuilles présenté en figure \ref{fig:Trajectoires_Milosevic} correspond à la courbe décrite par le champ électrique total lorsque l'intensité du champ à $\omega$ et du champ à 2$\omega$ sont égales, c'est-à-dire lorsque leur enveloppe est constante (figure \ref{fig:dessin_emission_t3}(a)). En réalité, et \textit{a fortiori} si l'on utilise des impulsions de quelques cycles pour générer une impulsion attoseconde unique, les impulsions utilisées expérimentalement possèdent une enveloppe variant de zéro à leur intensité maximale en quelques femtosecondes. Ainsi, l'intensité des champs à $\omega$ et 2$\omega$ varie très rapidement à l'échelle du cycle optique, comme le montre la figure \ref{fig:dessin_emission_t3}(b). Par conséquent, la symétrie d'ordre trois du champ électrique total est rompue: dans les fronts montant et descendant, les "feuilles" du trèfle ne sont pas de même taille et ne sont pas orientées à 120° les unes des autres (figure \ref{fig:dessin_emission_t3}(d)). On s'attend à ce que ceci modifie les trajectoires électroniques et donc la probabilité, l'instant et l'angle de recollision qui gouvernent respectivement l'intensité, la phase et la polarisation des harmoniques émises.

\begin{figure}
\centering
\def\svgwidth{\textwidth}
\import{Figures/Polarimetrie/}{dessin_emission_t3.pdf_tex}
\caption{Illustration de la rupture de symétrie du champ dans le front montant d'une impulsion ultra-brève. (a-b) Composantes du champ $\vec{E}(t)$ selon $x$ ($E_x$, bleu) et $y$ ($E_y$, rouge) pour une enveloppe  $f(t)$ constante (a) et $f(t) = \sin^2 (\pi t / 10 T)$ (b). (c-d) Courbes décrites par le champ dans le plan $(Oxy)$. Pour une enveloppe constante (c), le champ décrit un trèfle à trois feuilles régulier. Les trajectoires électroniques calculées par \mycite{MilosevicPRA2000_Generation} et présentées figure \ref{fig:Trajectoires_Milosevic} sont tracées en vert, orange et violet de manière illustrative. Elles se reproduisent de manière identique tous les $T/3$ et conduisent à l'émission d'impulsions attosecondes de même intensité polarisées linéairement, dont la direction de polarisation est donnée par l'angle de recollision de l'électron et tourne de 120° tous les $T/3$. Avec l'enveloppe d'une impulsion ultra-brève (d), le champ décrit un trèfle dont les feuilles sont de tailles différentes qui varie rapidement d'un tiers de période à l'autre. Les trajectoires électroniques (non représentées) sont modifiées. L'instant, la direction et la probabilité de recollision, qui déterminent respectivement la phase, la polarisation et l'intensité de l'émission attoseconde varient alors également d'un tiers de période à l'autre.}
\label{fig:dessin_emission_t3}
\end{figure}

Nous avons étudié l'influence de l'enveloppe ultra-brève des champs de génération en utilisant une enveloppe $f(t) = \sin^2 (\pi t / 10 T)$ (figure \ref{fig:dessin_emission_t3}(b)), de largeur à mi-hauteur $5 \: T$ (11.6 fs) et de durée totale $10 \: T$ (23.2 fs), où $T$ est la période du champ à $\omega$ : $T = 2\pi/\omega$. L'enveloppe est de forme $\sin^2$ qui est plus commode pour le calcul numérique que la forme gaussienne. Le champ $\omega$ possède une polarisation circulaire droite (PCD, $\epsilon_{\text{IR}} = +1$) et le champ à 2$\omega$ une polarisation circulaire gauche (PCG,  $\epsilon_{\text{UV}} = -1$). Le milieu de génération est un atome d'hélium. L'hélium possède un potentiel d'ionisation élevé et uniquement des électrons $s$, on s'affranchit donc d'éventuels effets d'ionisation ou de symétrie des orbitales mises en jeu.

Les composantes selon $x$ et $y$ du dipôle temporel, $a_x (t)$ et $a_y(t)$, sont calculées comme présenté au paragraphe \ref{sec:MethodesNum}. Leur transformée de Fourier donne le spectre présenté figure \ref{fig:ResultatsHe}(b). Le calcul des paramètres de Stokes de chaque harmonique, sommés sur la largeur spectrale décrite dans la légende de la figure \ref{fig:ResultatsHe}, permet de calculer l'ellipticité et le degré de polarisation de chaque pic harmonique (figure \ref{fig:ResultatsHe}(d)). On observe l'alternance de l'ellipticité avec l'ordre harmonique, ainsi que le signe de l'ellipticité correspondant à l'hélicité des champs de génération attendus d'après les lois de conservation (paragraphe \ref{sec:Intro2couleurs}). Cependant, l'ellipticité calculée est $|\epsilon_q| \approx 0.7$ pour tous les ordres considérés, et on observe une dépolarisation significative avec $P \approx 0.9$ dans le plateau et $P \approx 0.8$ dans la coupure.

\begin{figure}
\centering
\def\svgwidth{\textwidth}
\import{Figures/Polarimetrie/}{ReslutatsHe.pdf_tex}
\caption{Résultats des simulations avec des impulsions d'enveloppe $f(t) = \sin^2 (\pi t / 10 T)$ dans l'hélium. (a) Représentation temps-énergie de l'intensité harmonique, en échelle logarithmique, obtenue après transformation de Gabor du dipôle avec une fenêtre gaussienne de largeur à mi-hauteur $T$. (b) Spectre harmonique correspondant, en échelle logarithmique, avec les composantes possédant l'hélicité du champ à $\omega$ en rouge et l'hélicité du champ à 2$\omega$ en bleu. (c) Représentation temps-énergie de l'ellipticité harmonique. (d) Ellipticité (cercles rouges, échelle de gauche) et degré de polarisation (carrés bleus, échelle de droite) calculés sur la largeur spectrale des harmoniques ($\pm 0.25$ ordre jusqu'à H$_{32}$ et $\pm 0.5$ ordre au-delà).}
\label{fig:ResultatsHe}
\end{figure}


Une analyse des dipôles par transformée de Gabor permet d'obtenir le spectre harmonique instantané et d'étudier les propriétés de polarisation des harmoniques au cours de l'impulsion:
\begin{equation}
Ga_{x,y} (E,\tau) = \int_{- \infty}^{+\infty} g(t-\tau) a_{x,y}(t) \rme^{iEt/\hbar} \rmd t
\end{equation}
où $g$ est une gaussienne de largeur à mi-hauteur $T$, puis
\begin{equation}
\text{Spectre} \: (E, \tau) = |Ga_x (E,\tau)|^2 + |Ga_y (E,\tau)|^2
\end{equation}
La représentation temps-énergie de l'intensité harmonique calculée de cette façon se trouve figure \ref{fig:ResultatsHe}(a). On remarque l'émission des harmoniques $3m \pm 1$ uniquement. Les harmoniques les plus élevées ne sont émises qu'au maximum du champ.  Leur dérive de fréquence harmonique étant plus élevée que pour les harmoniques basses \mycite{SalieresPRL1995}\mycite{GaardePRA2002}, les harmoniques supérieures à H$_{34}$ sont décalées vers le bleu dans le front montant et vers le rouge dans le front descendant de l'impulsion. Cet élargissement spectral est responsable du recouvrement significatif des pics harmoniques au niveau de la coupure dans le spectre de la figure \ref{fig:ResultatsHe}(b). Ainsi lorsque l'on calcule l'état de polarisation d'une harmonique donnée en sommant les paramètres de Stokes sur une certaine largeur spectrale, les propriétés opposées de deux harmoniques voisines s'ajoutent. Les fortes variations spectrales de l'ellipticité sont responsables du plus faible degré de polarisation des harmoniques de la coupure.

Pour expliquer la non-circularité et la dépolarisation des harmoniques du plateau, on étudie les variations temporo-spectrales de l'ellipticité des harmoniques représentées figure \ref{fig:ResultatsHe}(c). On observe l'alternance de l'hélicité attendue, cependant, l'ellipticité d'une harmonique donnée n'est pas constante tout au long de l'impulsion. Pour les harmoniques de la coupure (H$_{19}$ à H$_{32}$), l'ellipticité dans les fronts montant et descendant est quasiment opposée à l'ellipticité au maximum de l'enveloppe de l'impulsion. Ainsi l'émission correspondant à toute l'impulsion n'est pas parfaitement circulaire et présente de la dépolarisation.

Avec des impulsions d'enveloppe rapidement variable, nous avons montré que les harmoniques émises dans le schéma à deux couleurs ne sont pas parfaitement circulaires bien que les champs de génération soient eux parfaitement circulaires. Cette cause de non-circularité et de dépolarisation est \textit{a priori} inévitable lorsque des impulsions ultra-brèves sont utilisées pour la GHOE, à moins s'utiliser des impulsions femtoseconde mises en forme temporellement de type "\textit{top-hat}" pour limiter les effets des fronts montant et descendant.

\section{Ionisation du milieu de génération}
\label{sec:Ionisation}
Une autre asymétrie temporelle peut provenir de l'ionisation du milieu, qui réduit le nombre d'émetteurs et conduit à une décroissance de l'émission au cours du temps. Pour étudier cet effet sans être perturbé par les effets d'impulsions brèves discutés précédemment, nous avons choisi désormais une impulsion d'enveloppe $f(t)$ trapézoïdale de durée totale $10 \: T$ avec deux cycles de montée, six cycles de plateau et deux cycles de descente (voir la zone grisée de la figure \ref{fig:Decroissance_exponentielle_ionisation}). Par ailleurs, le milieu de génération est un atome d'argon de potentiel d'ionisation plus faible que celui de l'hélium: les effets d'ionisation sont donc observables à des intensités "expérimentalement raisonnables" (chapitre \ref{chap:MesurePolar}). On considère toujours un champ à $\omega$ PCD ($\epsilon_{\text{IR}} = +1$) et un champ à 2$\omega$ PCG ($\epsilon_{\text{UV}} = -1$).

Dans le schéma à deux couleurs polarisées circulairement en sens opposé, le milieu de génération est sujet à la même intensité maximale que dans le cas d'un unique champ polarisé linéairement de  même énergie totale, mais trois fois par cycle au lieu de deux \mycite{MilosevicPRA2000_Generation}. Par conséquent, il est ionisé plus rapidement que dans la GHOE à une seule couleur en polarisation linéaire. La figure \ref{fig:Decroissance_exponentielle_ionisation} montre $a_x(t)$ la composante du dipôle selon $x$ calculée par la méthode décrite au paragraphe \ref{sec:MethodesNum} pour des intensités $I_\omega = I_{2\omega} = 8 \times 10^{13}$ W/cm$^2$ et $I_\omega = I_{2\omega} = 1.2 \times 10^{14}$ W/cm$^2$. $a_x(t)$ augmente pendant les deux premiers cycles du front montant de l'impulsion puis décroit exponentiellement lorsque l'enveloppe du champ est constante ($t > 2T$) avec un temps caractéristique de $2.5 \: T$ à $8 \times 10^{13}$ W/cm$^2$ et $1.3 \: T$ à $1.2 \times 10^{14}$ W/cm$^2$. En comparaison, les enveloppes des dipôles calculés de la même façon pour un champ à une couleur polarisé linéairement de même énergie totale sont représentés en pointillés. Les temps caractéristiques correspondants sont plus grands: $6.7 \: T$ à $1.6 \times 10^{14}$ W/cm$^2$ et $4 \: T$ à $2.4 \times 10^{14}$ W/cm$^2$. Le même comportement est observé pour $a_y (t)$. Remarquons ici que la haute intensité utilisée ici est susceptible de modifier l'énergie des orbitales $p$ dégénérées et de les mélanger \mycite{BarthJPB2014}. Nous ne séparons donc pas les contributions des orbitales $p_{+1}$ et $p_{-1}$ et ne discutons pas les effets observés en termes d'électrons tournant dans le même sens que le champ ou en sens opposé \mycite{MedisauskasPRL2015}.

\begin{figure}
\centering
\def\svgwidth{0.8\textwidth}
\import{Figures/Polarimetrie/}{Decroissance_exponentielle_ionisation.pdf_tex}
\caption{Accélération $a_x$ du dipôle selon $x$ calculée dans l'argon pour des impulsions à $\omega$ et 2$\omega$ polarisées circulairement en sens opposé d'enveloppe trapézoïdale (fond gris) et d'intensité $I_\omega = I_{2\omega} = 8 \times 10^{13}$ W/cm$^2$ (bleu) et $I_\omega = I_{2\omega} = 1.2 \times 10^{14}$ W/cm$^2$ (vert). Pour une meilleure visibilité, les enveloppes exponentielles des dipôles sont représentées en traits pleins bleu et vert. En comparaison, les enveloppes des dipôles calculées pour un unique champ à $\omega$ polarisé linéairement d'intensité totale équivalente ($I_\omega = 1.6 \times 10^{14}$ W/cm$^2$ pointillés bleus et $I_\omega = 2.4 \times 10^{14}$ W/cm$^2$ pointillés verts) sont également représentées.}
\label{fig:Decroissance_exponentielle_ionisation}
\end{figure}

Le spectre harmonique correspondant à la transformée de Fourier du dipôle total pour une intensité de $I_\omega = I_{2\omega} = 1.2 \times 10^{14}$ W/cm$^2$ est représenté figure \ref{fig:SpectresVSepsilon}(a). On remarque l'élargissement spectral des harmoniques dû à la réduction temporelle de l'émission par ionisation. L'ellipticité et le degré de polarisation des harmoniques sont calculés et correspondent aux cercles jaunes figure \ref{fig:ResultatsEpsilon}. On constate une comportement différent pour les harmoniques $3m+1$ et $3m-1$. L'ellipticité est de l'ordre de $\epsilon_{3m+1} \approx 0.8$ pour les harmoniques $3m+1$ et est significativement réduite en valeur absolue pour les harmoniques $3m-1$: $\epsilon_{3m-1} \approx -0.5$. Par ailleurs, le degré de polarisation des harmoniques $3m+1$ est proche de $P = 1$, mais il est plus faible pour les harmoniques $3m-1$ avec $P \approx 0.9$ (et jusqu'à $P = 0.7$ pour H$_{17}$).

\begin{figure}
\centering
\def\svgwidth{\textwidth}
\import{Figures/Polarimetrie/}{ResultatsArIonisation.pdf_tex}
\caption{Résultats des simulations avec des impulsions d'enveloppe trapézoïdale d'intensité $I_\omega = I_{2\omega} = 1.2 \times 10^{14}$ W/cm$^2$ dans l'argon. (a) Représentation temps-énergie de l'intensité harmonique, en échelle logarithmique, obtenue après transformation de Gabor du dipôle avec une fenêtre gaussienne de largeur à mi-hauteur $T$. (b) Spectre (noir, échelle de gauche) et ellipticité résolue spectralement (rouge, échelle de droite) de l'harmonique 16. (c) Représentation temps-énergie de l'ellipticité harmonique. (d) Spectre (noir, échelle de gauche) et ellipticité résolue spectralement (rouge, échelle de droite) de l'harmonique 17.}
\label{fig:ResultatsArIonisation}
\end{figure}

De la même manière qu'au paragraphe \ref{sec:ChampsBrefs}, nous tentons de rationaliser ces observations en représentant le spectre et l'ellipticité des harmoniques au cours du temps. La figure \ref{fig:ResultatsArIonisation}(a) montre la représentation temps-énergie de l'intensité harmonique. Deux domaines peuvent être distingués: le premier, pour $t < 2\: T$, correspond au front montant de l'impulsion dont les effets ont été discutés au paragraphe \ref{sec:ChampsBrefs}. Le second pour $t > 2 \: T$ correspond à la décroissance exponentielle de l'efficacité de génération par ionisation. Dans le premier domaine, l'ellipticité harmonique (figure \ref{fig:ResultatsArIonisation}(c)) varie rapidement au cours du temps et les harmoniques sont décalées vers le bleu. Ainsi, l'harmonique $3m+1$ est décalée vers les énergies supérieures, suffisamment pour atteindre l'énergie de l'harmonique $3m+2 = 3m'-1$ voisine. Lorsque l'on calcule la polarisation correspondant à l'émission harmonique pendant la totalité de l'impulsion, l'ellipticité des harmoniques $3m-1$ est réduite (en valeur absolue) à cause de la contribution de l'harmonique voisine, d'hélicité opposée, dans le front montant. Par ailleurs, cette variation temporelle de l'ellipticité est reliée aux variations spectrales de l'ellipticité (figure \ref{fig:ResultatsArIonisation}(b) et (d)) et est responsable de la forte dépolarisation observée pour les harmoniques $3m-1$. L'harmonique $3m$ susceptible d'être décalée à l'énergie de l'harmonique $3m+1$ n'est pas émise et ne modifie donc pas la polarisation de l'harmonique $3m-1$, qui possède alors une ellipticité plus grande (en valeur absolue) et un plus haut degré de polarisation. 

\begin{figure}[h]
\centering
\def\svgwidth{0.5\textwidth}
\import{Figures/Polarimetrie/}{Schema_petit_modele.pdf_tex}
\caption{Champ électrique d'amplitude décroissant exponentiellement.}
\label{fig:Schema_petit_modele}
\end{figure}

Par ailleurs, dans le second domaine ($t > 2 \: T$), la figure \ref{fig:ResultatsArIonisation}(c) montre une ellipticité constante au cours du temps, mais différente de $\pm 1$, avec $|\epsilon_q| \approx 0.85$. Nous proposons ici un modèle simple basé sur la décroissance exponentielle de l'émission pour expliquer cette observation. Le champ électrique total est la somme de trois champs polarisés linéairement, émis tous les $T/3$ avec un angle de polarisation augmentant de 120° chaque tiers de période. On considère que l'amplitude du champ est proportionnelle au nombre d'émetteurs dans le milieu, qui décroit exponentiellement en $\rme^{-t/\tau}$ avec les temps caractéristiques $\tau$ données par la figure \ref{fig:Decroissance_exponentielle_ionisation}. Selon les axes $x$ et $y$ définis sur la figure \ref{fig:Schema_petit_modele}, on a:
\begin{equation}
A_x (t) = \frac{\sqrt{3}}{2} \left[ E(t+2T/3) \delta(t+2T/3) - E(t+T/3) \delta(t+T/3) \right]
\end{equation}
et
\begin{equation}
A_y (t) = E(t) \delta(t) - \frac{1}{2} E(t+T/3) \delta(t+T/3) - \frac{1}{2} E(t+2T/3) \delta(t+2T/3) 
\end{equation}
où $\delta$ est la distribution de Dirac. Avec ce modèle, l'ellipticité est identique pour tous les ordres harmoniques et est égale à $|\epsilon_{\text{mod}}| = 0.86$ pour $\tau = 2.5 \: T$ et $|\epsilon_{\text{mod}}| = 0.74$ pour $\tau = 1.3 \: T$. Ces valeurs sont de l'ordre de l'ellipticité observée pour $t > 2 \: T$ dans la figure \ref{fig:ResultatsArIonisation}(c). Notons que dans ce modèle de décroissance exponentielle le rapport des intensités du champ tous les $T/3$ est constant, donc l'ellipticité ne varie pas au cours du temps et le champ est parfaitement polarisé ($P_{\text{mod}} = 1$).

Ces résultats montrent que l'ionisation du milieu, facilitée par l'intensité du champ total à deux couleurs, peut réduire l'ellipticité des harmoniques, conduire à de la dépolarisation et à des propriétés différentes pour les harmoniques $3m+1$ et $3m-1$. Afin d'éviter ces effets dans les expériences, nous suggérons l'utilisation de longueur d'onde plus grandes (dans le moyen IR, \mycite{FanPNAS2015}) et de gaz rares légers qui permettent d'étendre le spectre harmonique tout en gardant des taux d'ionisation faibles.

\section{Défaut de circularité du fondamental}

\begin{figure}
\centering
\def\svgwidth{0.75\textwidth}
\import{Figures/Polarimetrie/}{SpectresVSepsilon.pdf_tex}
\caption{Spectres harmoniques calculés dans l'argon à une intensité de $I_\omega = I_{2\omega} = 1.2 \times 10^{14}$ W/cm$^2$ pour $\epsilon_{\text{IR}} = +1$ (a); +0.9 (b) et +0.84 (c), le champ à 2$\omega$ restant parfaitement circulaire d'hélicité opposée ($\epsilon_{\text{UV}} = -1$). Les composantes possédant l'hélicité du champ à $\omega$ sont en rouge et celles d'hélicité du champ à 2$\omega$ en bleu.}
\label{fig:SpectresVSepsilon}
\end{figure}

Nous étudions maintenant l'influence de la non-circularité du champ à $\omega$ sur la polarisation des harmoniques. Cet effet a déjà été étudié par \mycite{MilosevicPRA2000_Generation} dans l'approximation du champ fort et \mycite{FleischerNatPhot2014} en résolvant l'ESDT, sans considérer le degré de polarisation des harmoniques. Les impulsions sont trapézoïdales d'intensité $I_\omega = I_{2\omega} = 1.2 \times 10^{14}$ W/cm$^2$, le milieu de génération est l'argon. Le champ à 2$\omega$ est parfaitement circulaire gauche ($\epsilon_{\text{UV}} = -1$) et l'on fait varier l'ellipticité du champ à $\omega$. La non-circularité du champ IR déforme le trèfle que décrit le champ électrique et brise la symétrie d'ordre trois du champ \mycite{FleischerNatPhot2014}.

La figure \ref{fig:SpectresVSepsilon} montre les spectres harmoniques calculés pour $\epsilon_{\text{IR}} = +1$ (a); +0.9 (b) et +0.84 (c). L'élargissement spectral des harmoniques est dû à l'ionisation, discutée au paragraphe \ref{sec:Ionisation}. Lorsque le champ à $\omega$ s'éloigne de la parfaite circularité (\textit{i.e.} $|\epsilon_{\text{IR}}|$ décroit), on observe l'émission des harmoniques $3m$ qui  n'est alors plus interdite par symétrie. L'ellipticité et le degré de polarisation des harmoniques, calculés sur une largeur spectrale de $\pm 0.25$ ordre autour du pic, sont représentés figure \ref{fig:ResultatsEpsilon}. Conformément aux calculs de \mycite{MilosevicPRA2000_Generation} et  \mycite{FleischerNatPhot2014}, l'ellipticité des harmoniques décroit (en valeur absolue) lorsque $|\epsilon_{\text{IR}}|$ décroit. Fleischer \textit{et al.} ont d'ailleurs proposé d'utiliser l'ellipticité du champ IR pour contrôler la polarisation des harmoniques de PCD à linéaire et PCG. Cependant, dans nos conditions de simulation, la diminution de l'ellipticité harmonique est corrélée à une diminution de leur degré de polarisation ($P \approx 0.9$, figure \ref{fig:ResultatsEpsilon}(b)). Remarquons également que lorsque le champ IR n'est pas parfaitement circulaire, le degré de polarisation est similaire pour les harmoniques $3m+1$ et $3m-1$. En effet, dans ce cas les harmoniques $3m$ sont émises et peuvent perturber la polarisation des harmoniques $3m+1$ en étant décalées vers le bleu dans le front montant de l'impulsion. Les cas des harmoniques $3m+1$ et $3m-1$ sont alors plus semblables que dans le cas parfaitement circulaire (paragraphe \ref{sec:Ionisation}).

\begin{figure}
\centering
\def\svgwidth{\textwidth}
\import{Figures/Polarimetrie/}{ResultatsEpsilon.pdf_tex}
\caption{Ellipticité (a) et degré de polarisation (b) calculés sur la largeur spectrale des harmoniques ($\pm 0.25$ ordre autour du pic) pour les différentes ellipticités IR étudiées, le champ à 2$\omega$ restant parfaitement circulaire d'hélicité opposée ($\epsilon_{\text{UV}} = -1$).}
\label{fig:ResultatsEpsilon}
\end{figure}

Enfin, la non-circularité de l'un des champs de génération peut avoir d'autres conséquences si la phase relative $\phi$ entre les deux champs n'est pas contrôlée (équation \ref{eq:champTA}). En effet, dans le cas parfaitement circulaire, une variation de la phase relative fait simplement tourner le trèfle dans le plan transverse, ce qui n'a aucune influence sur la polarisation des harmoniques. Mais lorsque la symétrie d'ordre trois est rompue par une petite non-circularité du champ IR, de l'ionisation ou dans le front montant ou descendant d'une impulsion brève, une variation de $\phi$ modifie le champ électrique total au cours du temps et par conséquent la polarisation des harmoniques. Pour prendre en compte cet effet supplémentaire, nous avons reproduit le calcul dans le cas $\epsilon_{\text{IR}} = +0.9 ; \: \epsilon_{\text{UV}} = -1$ à une intensité de $I_\omega = I_{2\omega} = 1.2 \times 10^{14}$ W/cm$^2$ dans l'argon pour 12 phases relatives $\phi$ différentes. L'ellipticité et le degré de polarisation des harmoniques 16 et 17 en fonction de $\phi$ sont représentés figure \ref{fig:MoyenneAngles}. L'intégration des paramètres de Stokes harmoniques pour tous ces angles conduit à une dépolarisation plus élevée avec $<P_{16}> = 0.93$ et $<P_{17}> = 0.61$.

\begin{figure}
\centering
\def\svgwidth{\textwidth}
\import{Figures/Polarimetrie/}{MoyenneAngles.pdf_tex}
\caption{\'{E}volution de l'ellipticité (a) et du degré de polarisation (b) calculés sur une largeur de $\pm$ 0.25 ordre autour du pic de H$_{16}$ (rouge) et H$_{17}$ (bleu) lorsque $\phi$ la phase relative entre les champs à $\omega$ et 2$\omega$ varie de 0 à 360°, dans un cas non parfaitement circulaire ($\epsilon_{\text{IR}} = +0.9 ; \: \epsilon_{\text{UV}} = -1$).}
\label{fig:MoyenneAngles}
\end{figure}

\section*{Conclusions}
Dans ce chapitre, nous avons mis en évidence plusieurs mécanismes de rupture de symétrie dynamique du champ électrique à deux couleurs. Dans les conditions étudiées, les harmoniques $3m \pm 1$ générées ne sont ni parfaitement circulaires ni complètement polarisées, bien que les champs de génération aient une ellipticité $|\epsilon_{\text{IR, UV}}| = 1$. Jusqu'à présent, les mécanismes de rupture de symétrie étudiés s'étaient limités à l'ellipticité de l'un des champs de génération \mycite{MilosevicPRA2000_Generation} \mycite{FanPNAS2015} et à l'anisotropie du milieu de génération (molécules alignées ou SF$_6$, \mycite{BaykushevaPRL2016}). Très récemment, \mycite{JimenezGalanArXiv2017} ont étudié la rupture de symétrie dynamique causée par un délai entre les deux impulsions (à l'échelle de l'impulsion et non sub-cycle comme sur la figure \ref{fig:MoyenneAngles}). Nous avons montré ici que des caractéristiques expérimentales intrinsèques telles que l'enveloppe rapidement variable des impulsions, l'ionisation du milieu ou la phase relative à l'échelle du cycle optique non contrôlée entre les deux couleurs peuvent réduire significativement la circularité des harmoniques produites et causer de la dépolarisation. Ces observations nous permettent de proposer des solutions expérimentales pour limiter ces effets: d'une part l'utilisation d'impulsions ultra-brèves mises en forme, et d'autre part la génération dans des gaz à potentiel d'ionisation élevé (hélium, néon) avec des longueurs d'onde dans l'IR moyen. L'utilisation d'un dispositif expérimental de production des impulsions à deux couleurs polarisées circulairement en sens opposé compact, comme proposé par \mycite{KfirAPL2016}, est conseillée pour garantir au maximum la stabilité de l'interféromètre à deux couleurs. L'interféromètre peut également être stabilisé activement \mycite{ChenScienceAdvances2016}. Par ailleurs, les résultats de nos simulations soulignent la nécessité de caractériser complètement la polarisation des harmoniques générées en mesurant simultanément les trois paramètres de Stokes du rayonnement, afin d'extraire le degré de polarisation et l'ellipticité de la partie polarisée du rayonnement en vue d'utilisation ultérieure.






\chapter{Mesure complète de l'état de polarisation de l'émission harmonique générée par un champ à deux couleurs polarisées circulairement en sens opposé} 
\label{chap:MesurePolar}
L'état de polarisation complet des harmoniques générées par un champ à deux couleurs polarisées circulairement en sens opposé a été caractérisé par la méthode de polarimétrie moléculaire développée par l'équipe de Danielle Dowek à l'Institut des Sciences Moléculaires d'Orsay. Dans ce chapitre, nous exposerons d'abord les principaux éléments de cette méthode. Nous présenterons ensuite les résultats des mesures effectuées à Saclay sur les harmoniques générées par un champ à deux couleurs polarisées circulairement en sens opposé. Enfin, nous commenterons les ellipticités et degrés de polarisation mesurés à la lumière de l'étude numérique du chapitre \ref{chap:calculsTA}.

\section{Principe de la polarimétrie moléculaire}
\label{sec:PrincipePM}
\subsection{Expression de la distribution angulaire des photoélectrons dans le référentiel moléculaire}
Nous décrivons ici brièvement la méthode de polarimétrie moléculaire. Pour une description plus détaillée, le lecteur pourra consulter \mycite{VeyrinasPRA2013}, \mycite{VeyrinasFaraday2016} et \mycite{TheseVeyrinas}.

La méthode de polarimétrie moléculaire repose sur la mesure de la distribution angulaire des photoélectrons dans le référentiel moléculaire (DAPRM, ou \textit{Molecular Frame Photoelectron Angular Distribution, MFPAD}), dont l'expression dépend des trois paramètres de Stokes normalisés du rayonnement ionisant $s_1$, $s_2$ et $s_3$ \mycite{LebechJCP2003}\mycite{DowekBook2012}. Lors de la photoionisation dissociative (PID) d'une molécule AB par un photon d'énergie $h \nu$,
\begin{equation}
\text{AB} + h \nu (s_1, s_2, s_3) \longrightarrow \text{AB}^+ + e^- \longrightarrow \text{A}^+ + \text{B} + e^- 
\end{equation}
la DAPRM est obtenue à partir de la mesure des corrélations vectorielles de la vitesse de l'ion fragment A$^+$ et de la vitesse de l'électron $(\vec{v}_{A^+}, \vec{v}_{e^-})$ pour chaque évènement de PID \mycite{LafossePRL2000}. Cette distribution angulaire dépend de l'orientation de la molécule dans le référentiel du laboratoire (RL) ainsi que de l'émission des photoélectrons dans le référentiel moléculaire (RM). On peut montrer que pour un rayonnement de polarisation elliptique quelconque, la DAPRM est une fonction des quatre angles $(\chi, \gamma, \theta_e, \phi_e)$ définis dans la figure \ref{fig:Referentiels} \mycite{DowekBook2012}\mycite{VeyrinasPRA2013}:
\begin{multline}
I(\chi, \gamma, \theta_e, \phi_e) = F_{00}(\theta_e) + F_{20}(\theta_e) \left[ - \frac{1}{2} P_2^0(\cos \chi) + \frac{1}{4} t_1(\gamma) P_2^2(\cos \chi) \right] \\
+ F_{21}(\theta_e) \left[ \left(-\frac{1}{2}-\frac{1}{2} t_1 (\gamma) \right) P_2^1(\cos \chi) \cos(\theta_e) - \frac{3}{2} t_2(\gamma) P_1^1(\cos \chi) \sin(\theta_e) \right] \\
+ F_{22}(\theta_e) \left[ \left( -\frac{1}{2} P_2^2(\cos \chi) + t_1(\gamma)(2 + P_2^0 (\cos \chi)) \right) \cos (2 \phi_e) + 3 t_2(\gamma) P_1^0 (\cos \chi) \sin(2 \phi_e) \right] \\
-s_3 F_{11}(\theta_e) P_1^1(\cos \chi) \sin(\phi_e)
\label{eq:MFPAD}
\end{multline}
où les fonctions $t$ s'expriment en fonction des paramètres de Stokes $s_1$ et $s_2$,
\begin{align}
t_1(\gamma) & = s_1 \cos(2\gamma) - s_2 \sin(2\gamma)\\
t_2(\gamma) & = s_1 \sin(2\gamma) + s_2 \cos(2\gamma)
\end{align}
les $P_L^N$ sont les polynômes associés de Legendre, et les $F_{ij}$ sont des fonctions à une dimension dépendant des éléments de matrice dipolaire de la transition \mycite{LucchesePRA2002}. Ces fonctions sont intrinsèques à la réaction de PID étudiée et sont donc indépendantes de l'état de polarisation du rayonnement.

\`{A} partir de cette expression, il est possible de déterminer les paramètres de Stokes du rayonnement incident.

\begin{figure}
\centering
\def\svgwidth{0.5\textwidth}
\import{Figures/Polarimetrie/}{Referentiels.pdf_tex}
\caption{Angles polaire $\chi$ et azimutal $\gamma$ d'émission de l'ion fragment dans le référentiel du laboratoire (RL), et angles polaire $\theta_e$ et azimutal $\phi_e$ d'émission de l'électron dans le référentiel moléculaire (RM). L'axe $z_{\text{RL}}$ est dans la direction de propagation du rayonnement $\vec{k}$, de polarisation elliptique quelconque. L'axe $z_{\text{RM}}$ est confondu avec l'axe de la molécule pour une molécule linéaire dans l'approximation du recul axial.}
\label{fig:Referentiels}
\end{figure}

\subsection{Détermination de $s_1$ et $s_2$ à partir de la distribution angulaire des ions dans le référentiel du laboratoire}
Tout d'abord, en intégrant l'expression \ref{eq:MFPAD} sur les angles d'émission de l'électron, on obtient $I(\chi, \gamma)$ la distribution angulaire des ions fragments dans le RL:
\begin{equation}
I(\chi, \gamma) = C \left( 1 + \beta_i \left[ - \frac{1}{2} P_2^0(\cos \chi) + \frac{1}{4} t_1(\gamma) P_2^2(\cos \chi) \right] \right)
\end{equation}
où $\beta_i$ est le paramètre d'asymétrie de la distribution des ions fragments. Sa valeur est comprise entre -1 et 2 et dépend de la symétrie des états initial et final associés à la PID étudiée. $\beta_i =2$ et $\beta_i = -1$ correspondent à des distributions anisotropes $I(\chi) \propto \cos^2 \chi$ et $\propto \sin^2 \chi$ respectivement, tandis que $\beta_i = 0$ correspond à une distribution isotrope. En intégrant d'une part sur l'angle $\gamma$ et d'autre part sur l'angle $\chi$, on obtient:
\begin{align}
I(\chi) & = C \left( 1 - \frac{\beta_i}{2} P_2^0(\cos \chi) \right) \\
I(\gamma) & = C \left( 1 + \frac{\beta_i}{2} \left[ s_1 \cos (2\gamma) - s_2 \sin (2\gamma) \right] \right)
\end{align}
Ces deux expressions permettent d'obtenir d'une part le paramètre d'asymétrie $\beta_i$, et si ce dernier est non nul, les paramètres de Stokes $s_1$ et $s_2$ d'autre part.

\subsection{Détermination de $s_3$ à partir du dichroïsme circulaire dans le référentiel moléculaire}
Le paramètre de Stokes $s_3$ apparaît dans le dernier terme de l'expression \ref{eq:MFPAD} à travers son produit par la fonction $F_{11}(\theta_e)$, qui caractérise le dichroïsme circulaire dans le référentiel moléculaire. En effet, la fonction $F_{11}$ est reliée dans le RM à l'asymétrie gauche-droite dans l'émission des photoélectrons dans le plan de polarisation ($\phi_e = 90$ ou 270°) lorsque l'axe internucléaire est perpendiculaire à l'axe de propagation de la lumière ($\chi =$90°). Ainsi, pour déterminer le paramètre $s_3$ d'un rayonnement inconnu, il est nécessaire de connaître la fonction $F_{11}(\theta_e)$ du processus de PID à l'énergie considérée. Celle-ci peut s'obtenir par un calcul utilisant la méthode \textit{multi channel Schwinger interaction configuration} \mycite{LebechJCP2003}, ou bien mesurée en utilisant un rayonnement de polarisation connue. Par exemple, la figure \ref{fig:F11_ref_H17} montre la fonction $F_{11}$ caractéristique de la PID de NO vers l'état $c ^3\Pi$ de NO$^+$ à l'énergie de l'harmonique 17 du 800 nm mesurée sur la ligne DESIRS du synchrotron SOLEIL avec une polarisation purement circulaire \mycite{NahonJSR2012}. La comparaison du produit $s_3 \times F_{11}(\theta_e)$ mesuré avec la fonction $F_{11}$ de référence permet de déterminer le paramètre de Stokes $s_3$. 

\begin{figure}
\centering
\def\svgwidth{0.6\textwidth}
\import{Figures/Polarimetrie/}{F11_ref_H17.pdf_tex}
\caption{Fonction $F_{11}$ caractéristique de la PID de NO vers l'état $c ^3\Pi$ de NO$^+$ mesurée avec un rayonnement synchrotron polarisé circulairement d'énergie 26.35 eV (points et trait plein), et calculé par le Prof. R. R. Lucchese avec la méthode \textit{multi channel Schwinger interaction configuration} (pointillés). Extrait de \mycite{TheseVeyrinas}.}
\label{fig:F11_ref_H17}
\end{figure}

\section{Dispositif expérimental}
\begin{figure}
\centering
\def\svgwidth{\textwidth}
\import{Figures/Polarimetrie/}{2couleurs_Setup.pdf_tex}
\caption{Dispositif expérimental et spectre harmonique typique mesuré sur le spectromètre de photons XUV. Les conventions choisies pour définir les polarisations circulaire gauche et droite sont indiquées sous le schéma.}
\label{fig:2couleurs_Setup}
\end{figure}

La méthode de polarimétrie moléculaire a été utilisée pour déterminer l'état de polarisation complet du rayonnement harmonique généré par un champ à deux couleurs polarisées circulairement en sens opposé à Saclay, en collaboration avec Kévin Veyrinas, Jean-Christophe Houver et Danielle Dowek de l'ISMO. Le dispositif expérimental est représenté sur la figure \ref{fig:2couleurs_Setup}. Le laser PLFA \mycite{WeberRSI2015} produit des impulsions polarisées horizontalement de 50 fs à 800 nm à une cadence de 1 kHz. Le faisceau est séparé en deux par une lame séparatrice 50/50. Une partie est utilisée pour générer la seconde harmonique à 400 nm dans un cristal de BBO de 200 $\mu$m. La polarisation circulaire de ce faisceau est contrôlée par une lame quart d'onde d'ordre zéro à 400 nm. La seconde partie du faisceau traverse une lame quart d'onde à 800 nm et est polarisée circulairement en sens opposé. Les deux faisceaux sont recombinés par un miroir dichroïque et focalisés par une lentille de focale $f = 80$ cm dans un jet effusif d'argon pour la GHOE. Le rayonnement XUV peut être envoyé sur un spectromètre XUV composé d'un réseau, d'une galette de micro-canaux et d'une caméra CCD pour mesurer le spectre de photons à l'aide d'un miroir en or motorisé. La transmission de ce miroir étant beaucoup plus efficace en polarisation verticale que horizontale, une lame demie onde à 800 nm est placée dans le second bras de l'interféromètre pour tourner la polarisation du fondamental de 90° afin d'observer un signal harmonique dans un cas linéaire lors de l'optimisation de l'expérience. L'intenisté des deux impulsions est déterminée grâce à la coupure du spectre harmonique généré avec chacune des impulsions polarisées linéairement, et est estimée à $I_\omega \approx I_{2\omega} \approx 1 \times 10^{14}$ W/cm$^2$. Le rayonnement XUV peut également être focalisé à l'aide d'un miroir torique en or dans le COLTRIMS CIEL \mycite{WeberRSI2015} pour effectuer la polarimétrie moléculaire. L'influence du miroir torique sur la polarisation du rayonnement est préalablement calibrée \mycite{TheseGruson}. Dans le spectromètre de moment 3D CIEL, le rayonnement XUV produit la photoionisation dissociative de molécules de NO produites dans un jet supersonique. La réaction étudiée est la photoionisation dissociative de NO \textit{via}
\begin{multline}
\text{NO}(^2\Pi, \: 4\sigma^2 5\sigma^2 1\pi^4 2\pi^1) + h\nu \longrightarrow \text{NO}^+ (c ^3\Pi, (4\sigma)^{-1})  + e^- \\\longrightarrow \text{N}^+ (^3P) + \text{O} (^3P) + e^-
\label{eq:PID_NO}
\end{multline}
qui se produit pour des énergies supérieures à 21.73 eV. La dissociation produit très majoritairement les ions fragments N$^+$ (à 90 \% \mycite{LafossePRL2000}). Cette réaction est tout à fait adaptée à la polarimétrie moléculaire grâce à la forte anisotropie de  l'émission des ions N$^+$ ($\beta_{N^+} \approx 1$ \mycite{LafossePRL2000}) et un dichroïsme circulaire de la distribution angulaire des électrons dans le RM important \mycite{LebechJCP2003}. Les électrons et les ions sont guidés vers des détecteurs sensibles en temps et en position à ligne à retard (DLD PSDs RoentDek) par des champs électrique et magnétique, assurant une détection des particules émises dans $4\pi$ sr. Notons ici que le taux de répétition de 1 kHz réduit grandement le taux de comptage des coïncidences et impose une durée de mesure de plusieurs heures. Des mesures complémentaires dans l'hélium peuvent être effectuées avec un plus haut taux de comptage mais permettent uniquement la détermination des paramètres $s_1$ et $s_2$.

\section{Résultats}
\subsection{Champs à $\omega$ et $2\omega$ polarisés circulairement en sens opposé}
Nous présentons d'abord les mesures de polarimétrie effectuées lorsque les champs IR et UV sont polarisés circulairement en sens opposé. Les deux configurations symétriques $\epsilon_{\text{IR}} = -1$; $\epsilon_{\text{UV}} = +1$ et $\epsilon_{\text{IR}} = +1$; $\epsilon_{\text{UV}} = -1$ sont étudiées.

La figure \ref{fig:SpectresPE_pm1} présente les spectres de photoélectrons issus de la PID de NO (réaction \ref{eq:PID_NO}) mesurés dans le COLTRIMS dans les deux configurations étudiées. Seules les harmoniques d'énergie supérieure à l'énergie nécessaire à la réaction \ref{eq:PID_NO} sont observées, à la différence du spectre de photons de la figure \ref{fig:2couleurs_Setup}. La suite de l'analyse est donc centrée sur les harmoniques 16 ($3m+1$) et 17 ($3m-1$). Dans les deux configurations, on observe le spectre caractéristique du schéma à deux couleurs où les harmoniques $3m$ (H$_{15}$ et H$_{18}$ ici) sont absentes. Cependant H$_{15}$ est plus intense dans le cas $\epsilon_{\text{IR}} = +1$; $\epsilon_{\text{UV}} = -1$ que dans le cas opposé, indiquant une asymétrie des lames $\lambda/4$ utilisées.

\begin{figure}[h]
\centering
\def\svgwidth{\textwidth}
\import{Figures/Polarimetrie/}{SpectresPE_pm1.pdf_tex}
\caption{Spectres de photoélectrons issus de la PID de NO produits par les harmoniques générées avec $\epsilon_{\text{IR}} = -1$; $\epsilon_{\text{UV}} = +1$ (a) et  $\epsilon_{\text{IR}} = +1$; $\epsilon_{\text{UV}} = -1$ (b). Le spectre vert correspond au spectre mesuré noir corrigé de la section efficace du processus \ref{eq:PID_NO} \mycite{IidaChemPhys1986}\mycite{StratmannJCP1996}.}
\label{fig:SpectresPE_pm1}
\end{figure}

Les DAPRM réduites après intégration sur l'angle $\gamma$, $I(\theta_e, \phi_e)$, lorsque la molécule est perpendiculaire à la direction de propagation du faisceau ($\chi = 90$°) pour les harmoniques 16 et 17 des deux configurations sont représentées sur la figure \ref{fig:MFPAD_pm1}. Les coupes de ces DAPRM dans le plan de polarisation $(\phi_e = 270$ ou 90°), dans lequel le dichroïsme circulaire est le plus important, sont représentées figure \ref{fig:CoupeMFPAD} avec les DAPRM calculées à partir des fonctions $F_{ij}$ de référence mesurées au synchrotron à l'énergie de chaque harmonique. Ces coupes permettent d'observer plus facilement l'asymétrie gauche-droite des DAPRM. Dans une configuration donnée, l'asymétrie gauche-droite est opposée pour H$_{16}$ et H$_{17}$, indiquant des hélicités opposées. Lorsque l'hélicité des deux couleurs est inversée, les hélicités de  H$_{16}$ et H$_{17}$ sont également inversées. Ces observations sont en accord avec l'image multiphotonique donnée au chapitre \ref{chap:GHOE_elliptiques} et les mesures de DCMX de \mycite{KfirNatPhot2015}. Elles sont par ailleurs confirmées par la mesure de l'harmonique 19 (figure \ref{fig:Resultatsbu2H19}) dans la configuration $\epsilon_{\text{IR}} = -1$; $\epsilon_{\text{UV}} = +1$, montrant un signe de $s_3$ identique à H$_{16}$.

\begin{figure}
\centering
\def\svgwidth{0.8\textwidth}
\import{Figures/Polarimetrie/}{MFPAD_pm1.pdf_tex}
\caption{DAPRM réduites $I(\theta_e,\phi_e)$ lorsque la molécule de NO est perpendiculaire à la direction de propagation du faisceau pour les harmoniques 16 (a-b et 17 (c-d) mesurées pour $\epsilon_{\text{IR}} = -1$; $\epsilon_{\text{UV}} = +1$ (a-c) et $\epsilon_{\text{IR}} = +1$; $\epsilon_{\text{UV}} = -1$ (b-d). Les définitions des angles utilisées et la direction de la molécule sont indiquées sur le schéma en haut de la figure.}
\label{fig:MFPAD_pm1}
\end{figure}

\begin{figure}
\centering
\def\svgwidth{0.7\textwidth}
\import{Figures/Polarimetrie/}{CoupeMFPAD.pdf_tex}
\caption{Coupe des DAPRM de la figure \ref{fig:MFPAD_pm1} dans le plan de polarisation du rayonnement XUV. Les points mesurés avec leurs barres d'erreur sont indiqués en rouge, avec leur ajustement en polynômes de Legendre en bleu, et la coupe de DAPRM de référence en noir.}
\label{fig:CoupeMFPAD}
\end{figure}

\begin{figure}
\centering
\def\svgwidth{\textwidth}
\import{Figures/Polarimetrie/}{F11_pm1.pdf_tex}
\caption{Produit $s_3 \times F_{11}(\theta_e)$ mesuré pour H$_{16}$ (rouge) et H$_{17}$ (bleu) dans les deux configurations symétriques $\epsilon_{\text{IR}} = -1$; $\epsilon_{\text{UV}} = +1$ (a) et  $\epsilon_{\text{IR}} = +1$; $\epsilon_{\text{UV}} = -1$ (b), et fonctions $F_{11}(\theta_e)$ de référence mesurées au synchrotron (brun). Dans cette gamme d'énergie, les $F_{11}$ sont positives, cependant pour mettre en évidence le signe de $s_3$ la fonction de référence $-F_{11}(\theta_e)$ est représentée dans les cas où $s_3 <0$.}
\label{fig:F11_pm1}
\end{figure}

\begin{figure}
\centering
\def\svgwidth{\textwidth}
\import{Figures/Polarimetrie/}{Resultatsbu2H19.pdf_tex}
\caption{DAPRM réduite $I(\theta_e,\phi_e)$ lorsque la molécule de NO est perpendiculaire à la direction de propagation du faisceau (a) et produit $s_3 \times F_{11}(\theta_e)$ (b) mesurés pour l'harmonique 19 avec $\epsilon_{\text{IR}} = -1$; $\epsilon_{\text{UV}} = +1$. En (b), la fonction de référence $+F_{11}(\theta_e)$ est en trait plein et $-F_{11}(\theta_e)$ en pointillés.}
\label{fig:Resultatsbu2H19}
\end{figure}

Plus quantitativement, l'asymétrie gauche-droite permet d'accéder au produit $s_3 \times F_{11}(\theta_3)$ pour le processus et l'harmonique considérée. Les $s_3 \times F_{11}(\theta_3)$ mesurés sont représentés sur la figure \ref{fig:F11_pm1} avec les fonctions $F_{11}$ de référence mesurées au synchrotron avec un rayonnement de même énergie polarisé circulairement. Dans les deux configurations, pour H$_{16}$ la mesure est très proche de la référence, indiquant un $|s_3|$ proche de 1. Les paramètres $s_1$ et $s_2$ sont déterminés indépendamment grâce à la mesure de la distribution angulaire des ions dans le référentiel du laboratoire (paragraphe \ref{sec:PrincipePM}). Ils indiquent ici un faible taux de polarisation linéaire $\approx 0.2 -0.3$.

\begin{table}[ht]
\centering
\begin{tabular}{|c||c|c|c|c|}
\hline
 & \multicolumn{2}{c|}{$\epsilon_{\text{IR}} = -1$; $\epsilon_{\text{UV}} = +1$} &  \multicolumn{2}{c|}{$\epsilon_{\text{IR}} = +1$; $\epsilon_{\text{UV}} = -1$} \\
\hline
Harmonique & H$_{16}$ & H$_{17}$ & H$_{16}$ & H$_{17}$ \\
\hline
$s_1$ & 0.27 (0.04) & 0.32 (0.05) & 0.24 (0.04) & 0.07 (0.07) \\
\hline
$s_2$ & 0.05 (0.04) & 0.03 (0.05) & -0.12 (0.04) & -0.20 (0.07) \\
\hline
$s_3$ & -0.80 (0.04) & 0.53 (0.04) & 0.77 (0.04) & -0.53 (0.04) \\
\hline
$P$ & 0.85 (0.03) & 0.62 (0.04) & 0.81 (0.04) & 0.57 (0.05) \\
\hline
$\psi$ (°) & 5 (4.5) & 2.5 (4.5) & 167 (5) & 144.5 (9.5) \\
\hline
$\epsilon$ & -0.72 (0.04) & 0.56 (0.05) & 0.71 (0.04) & -0.68 (0.08) \\
\hline
\end{tabular}
\caption{Paramètres de Stokes normalisés, degré de polarisation, angle de l'ellipse et ellipticité des harmoniques 16 et 17 générées dans le schéma à deux couleurs polarisées circulairement en sens opposé dans deux configurations symétriques. Les incertitudes sont indiquées entre parenthèses.}
\label{tab:resultatspm1}
\end{table}

\`{A} partir de cette mesure simultanée des trois paramètres de Stokes, on obtient la caractérisation complète de l'état de polarisation du rayonnement harmonique généré par un champ à deux couleurs polarisées circulairement en sens opposé. L'ellipticité de la partie polarisée du rayonnement $\epsilon$, l'angle de l'ellipse $\psi$ et le degré de polarisation $P$ sont calculés à partir des expressions \ref{eq:epsilon_Stokes}, \ref{eq:PsiStokes} et \ref{eq:degrépolarisation}. Les valeurs obtenues sont résumées dans le tableau \ref{tab:resultatspm1}. Les résultats sont symétriques pour les deux configurations de GHOE symétriques. Les mesures mettent en évidence une circularité imparfaite des harmoniques 16 et 17 avec $|\epsilon| \approx$ 0.7 et 0.6 respectivement. Le degré de polarisation mesuré est significativement inférieur à 1, avec $P \approx 0.8$ pour H$_{16}$ et $P \approx 0.6$ pour H$_{17}$. L'angle d'orientation de l'ellipse $\psi$ n'est pas discuté car le délai relatif entre les deux impulsions n'était pas stabilisé dans notre expérience.

\subsection{Champ à $\omega$ elliptique et champ à $2\omega$ circulaire d'hélicité opposée}
Le dispositif expérimental permet de mesurer l'état de polarisation des harmoniques lorsque l'un des champs possède un défaut de circularité. Nous avons tourné l'angle de la lame $\lambda/4$ à 800 nm de 5° pour obtenir une ellipticité IR de $\epsilon_{\text{IR}} = +0.84$. Le champ à 400 nm est lui polarisé circulairement avec $\epsilon_{\text{UV}} = -1$. Comme le montre la figure \ref{fig:Resultats_084}(a), les harmoniques $3m$ sont présentes dans le spectre harmonique. Il est donc possible d'effectuer la polarimétrie de H$_{15}$ en plus de H$_{16}$ et H$_{17}$. D'après les figures \ref{fig:Resultats_084}(c-d-f), on observe toujours une asymétrie gauche-droite dans la DAPRM, caractéristique de la présence de polarisation circulaire, mais plus faible que dans les configurations étudiées précédemment. La DAPRM de H$_{15}$ (figure \ref{fig:Resultats_084}(b-e) présente une légère asymétrie gauche-droite dans le même sens que H$_{16}$.

\begin{table}[h]
\centering
\begin{tabular}{|c||c|c|c|}
\hline
 & \multicolumn{3}{c|}{$\epsilon_{\text{IR}} = +0.84$; $\epsilon_{\text{UV}} = -1$} \\
\hline
Harmonique & H$_{15}$ & H$_{16}$ & H$_{17}$ \\
\hline
$s_1$ & 0.13 (0.07) & 0.26 (0.05) & -0.24 (0.08) \\
\hline
$s_2$ & -0.13 (0.07) & -0.34 (0.05) & -0.46 (0.07)  \\
\hline
$s_3$ & 0.33 (0.07) & 0.48 (0.04) & -0.22 (0.05) \\
\hline
$P$ & 0.38 (0.07) & 0.64 (0.04) & 0.57 (0.08) \\
\hline
$\psi$ (°) & 157.5 (13) & 154 (4) & 121.5 (4) \\
\hline
$\epsilon$ & 0.58 (0.12) & 0.45 (0.05) & -0.21 (0.04) \\
\hline
\end{tabular}
\caption{Paramètres de Stokes normalisés, degré de polarisation, angle de l'ellipse et ellipticité des harmoniques 15, 16 et 17 générées dans le schéma à deux couleurs avec le champ à 800 nm PED et le champ à 400 nm PCG. Les incertitudes sont indiquées entre parenthèses.}
\label{tab:resultats084}
\end{table}

Les résultats complets sont données dans le tableau \ref{tab:resultats084}. H$_{15}$ possède la même hélicité que H$_{16}$ avec une ellipticité $\epsilon \approx 0.6$ mais un faible degré de polarisation $P \approx 0.4$. L'ellipticité des harmoniques $3m+1$ et $3m-1$ possède toujours le même signe que celle des champs à $\omega$ et $2\omega$ respectivement, mais est fortement réduite par rapport au cas ou l'IR est PCD. On observe ici aussi une ellipticité plus faible (en valeur absolue) pour H$_{17}$ que pour H$_{16}$ avec $|\epsilon| \approx 0.2$ et 0.45 respectivement. Ces valeurs sont en accord avec les mesures précédentes de \mycite{FleischerNatPhot2014}. Cependant, on observe un faible degré de polarisation pour ces deux harmoniques, $P \approx 0.6$, qui n'était pas mesuré par Fleischer \textit{et al.}

\begin{figure}
\centering
\def\svgwidth{\textwidth}
\import{Figures/Polarimetrie/}{Resultats_084.pdf_tex}
\caption{Polarimétrie moléculaire dans le cas $\epsilon_{\text{IR}} = +0.84$; $\epsilon_{\text{UV}} = -1$. (a) Spectre de photoélectrons issus de la PID de NO mesuré (noir) et corrigé de la section efficace du processus (vert). (b-d) DAPRM réduites $I(\theta_e,\phi_e)$ lorsque la molécule de NO est perpendiculaire à la direction de propagation du faisceau pour les harmoniques 15, 16 et 17. (e-f) Produit $s_3 \times F_{11}(\theta_e)$ mesuré pour H$_{15}$ (orange), H$_{16}$ (rouge) et H$_{17}$ (bleu), et fonctions $F_{11}(\theta_e)$ de référence mesurées au synchrotron (brun). Dans cette gamme d'énergie, les $F_{11}$ sont positives, cependant pour mettre en évidence le signe de $s_3$ la fonction de référence $-F_{11}(\theta_e)$ est représentée dans les cas où $s_3 <0$.} 
\label{fig:Resultats_084}
\end{figure}

\section{Discussion}
Les résultats présentés ici constituent à notre connaissance la première mesure \textit{in situ} complète de l'état de polarisation des harmoniques générées par un champ à deux couleurs polarisées circulairement en sens opposé. De manière surprenante, les harmoniques mesurées ne sont pas parfaitement circulaires et présentent une partie dépolarisée significative. Au regard des simulations présentées dans le chapitre \ref{chap:calculsTA} et de nos conditions expérimentales, une forte ionisation du milieu peut être responsable de ces déviations. De plus, la présence de l'harmonique 15 dans les spectres de photoélectrons (figure \ref{fig:SpectresPE_pm1}) suggère un petit défaut de circularité de l'un des champs de génération, probablement le champ à 800 nm qui traverse deux lames d'onde dans le dispositif expérimental. Par ailleurs, le délai relatif entre les deux impulsions IR et UV n'était pas stabilisé activement. Les mesures en coïncidence requièrent l'acquisition de données pendant plusieurs heures, et donc la stabilité des conditions de génération pendant de longues durées. Des effets supplémentaires de propagation peuvent modifier l'état de polarisation des harmoniques \mycite{AntoinePRA1997}, bien que notre dispositif expérimental limite \textit{a priori} ces effets en utilisant la GHOE dans un jet effusif.

\section*{Conclusions de la partie \ref{part:Polarimétrie}}
Dans cette partie, nous avons présenté plusieurs méthodes de GHOE polarisées elliptiquement, en particulier le dispositif à deux couleurs polarisées circulairement en sens opposé. Ce schéma permet la génération efficace d'un spectre harmonique étendu composé de toutes les harmoniques sauf les $3m$. D'après les lois de conservation, la polarisation des harmoniques est circulaire, d'hélicité alternée et contrôlée par l'hélicité des deux champs générateurs. Grâce à ces propriétés, ce schéma est prometteur pour l'étude de dichroïsmes circulaires dans les matériaux et la génération d'impulsions attosecondes uniques polarisées circulairement.

Nous avons ensuite étudié numériquement, en résolvant l'ESDT, l'état de polarisation des harmoniques générées par le dispositif à deux couleurs polarisées circulairement en sens opposé lorsque l'enveloppe des champs est rapidement variable ou bien que le milieu est fortement ionisé. Dans ces conditions, bien que les deux couleurs soient polarisées circulairement, les harmoniques sont non circulaires et ne sont pas parfaitement polarisées. Nous avons interprété ces observations grâce aux variations temporo-spectrales des propriétés de polarisation des harmoniques. Ces observations nous ont permis de proposer des conditions expérimentales optimales pour générer des harmoniques circulaires et complètement polarisées.

Par ailleurs, les nombreuses études des harmoniques générées par ce dispositif supposaient jusqu'ici que le rayonnement était complètement polarisé, et aucune caractérisation complète de leur état de polarisation n'avait été effectuée. Nous avons effectué ces premières mesures grâce à la méthode de polarimétrie moléculaire, dans le cas où les deux couleurs étaient polarisées circulairement et également dans le cas où l'un des champ présentait un défaut de circularité. Nous avons confirmé l'alternance de l'hélicité et le contrôle de celle-ci par les hélicités des deux couleurs. Cependant dans nos conditions expérimentales les harmoniques caractérisées ne sont ni parfaitement circulaires ni complètement polarisées. Ces écarts sont exaltés par un défaut de circularité de l'un des champs. Nous avons pu en partie interpréter ces observations grâce à notre étude numérique. Ces résultats démontrent également l'importance des mesures de polarimétrie complète et le potentiel de la polarimétrie moléculaire à cet effet. Des mesures plus précises de polarimétrie moléculaire seront possible prochainement avec le développement des sources harmoniques à plus haute cadence (10 ou 100 kHz), permettant un plus haut taux de comptage et des durées de mesure réduites.


