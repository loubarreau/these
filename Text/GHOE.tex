% Chapitre GHOE

\part{La génération d'harmoniques d'ordre élevé}
\label{part:GHOE}
\chapter{Théorie de la génération d'harmoniques d'ordre élevé}
\label{chap:ThHHG}
% Une intro...
\section{Réponse de l'atome unique}
\subsection{Modèle semi-classique en trois étapes}
Le modèle semi-classique de la génération d'harmoniques d'ordre élevé est proposé en 1993 par \mycite{SchaferPRL1993} et \mycite{CorkumPRL1993}. Brièvement, l'interaction entre un champ laser intense et un atome ou une molécule déforme le potentiel ressenti par les électrons du système. Un paquet d'onde électronique peut alors être émis par ionisation tunnel (\textbf{1}). Le paquet d'onde électronique libéré est accéléré par le champ laser (\textbf{2}). Lorsque le champ change de signe, le paquet d'onde est ramené vers l'ion parent avec lequel il peut recombiner en émettant l'énergie cinétique accumulée dans le champ sous la forme d'un photon dans le domaine de l'extrême ultra-violet XUV (\textbf{3}). Dans la suite nous détaillons successivement ces trois étapes.

\paragraph{Ionisation tunnel} Considérons un électron de l'atome d'hydrogène\footnote{Par commodité nous présentons le cas de l'atome le plus simple, qui peut être généralisé aux autres atomes et aux molécules.} isolé dans son état fondamental, d'énergie $-Ip$ où $Ip$ est le potentiel d'ionisation. L'électron est soumis au potentiel coulombien du noyau, de la forme $V_0 (x) = - e /4\pi \epsilon_0 |x|$ où $x$ est la distance électron-noyau (figure \ref{fig:potentials}(a)), $e$ la charge de l'électron et $\epsilon_0$ la permittivité du vide. En présence d'un champ électrique polarisé linéairement, de pulsation $\omega$, $\vec{\mathcal{E}}(t) = \mathcal{E}_0 \cos(\omega t) \vec{u_x}$, l'énergie totale de l'électron devient
\begin{align}
V(x,t) & = e V_0 (x) - e x \mathcal{E}(t) \\
& = -\frac{e^2}{4\pi \epsilon_0 |x|} - e x \mathcal{E}_0 \cos(\omega t)
\label{eq:PotentielEffectif}
\end{align}

\begin{figure}
\centering
\def\svgwidth{\columnwidth}
\import{Figures/GHOE/}{potentials.pdf_tex}
\caption{Potentiel ressenti par un électron dans l'argon ($Ip$ = 15.76 eV = 0.58 ua) (a) en l'absence de champ électrique, (b) en présence d'un champ $\mathcal{E}_0 =$ 0.04 ua ($I = 5.5 \times 10^{13}$ W/cm$^2$), (c) en présence d'un champ à l'éclairement de saturation de l'argon $\mathcal{E}_0 = \mathcal{E}_{\text{sat}}$ = 0.084 ua ($I = 2.4 \times 10^{14}$ W/cm$^2$).}
\label{fig:potentials}
\end{figure}

Si le potentiel d'interaction avec le champ est négligeable devant le potentiel coulombien, l'influence du champ peut être traitée de manière perturbative. Le champ "intra-atomique" peut être estimé en prenant $x = a_0$ le rayon de l'orbite de Bohr, et est de l'ordre de $E_{\text{atom}} \backsim e/4\pi \epsilon_0 a_0^2 \backsim 6 \times 10^9$ V/m, soit une intensité $I_\text{atom} \backsim 4 \times 10^{16}$ W/cm$^2$. Ainsi, en présence d'un champ laser intense ($I \backsim 10^{14}$ W/cm$^2$), l'interaction avec le champ électrique n'est plus perturbative. Comme illustré sur la figure \ref{fig:potentials}(b), le champ laser abaisse la barrière de potentiel ressentie par l'électron. Une partie du paquet d'onde électronique (POE) peut alors la traverser par effet tunnel, avec une probabilité qui dépend de la hauteur et de l'épaisseur de la barrière ainsi que de la durée pendant laquelle elle est abaissée. Dans le cas extrême de la figure \ref{fig:potentials}(c), le champ laser est suffisamment intense pour supprimer complètement la barrière \mycite{AugstPRL1989}\mycite{BauerPRA1999}. Dans ces conditions, la valeur maximale de l'énergie est égale à $-Ip$. Cette valeur est atteinte en $x_0$ tel que $V'(x_0) = 0$. Considérons que ce maximum est atteint lorsque le champ électrique est maximal, $\mathcal{E} = \mathcal{E}_0$. En dérivant \ref{eq:PotentielEffectif}, il vient simplement:
\begin{equation}
|x_0| = \frac{e}{4\pi \epsilon_0 \sqrt{\mathcal{E}_0}}
\end{equation}
Avec $V(x_0) = - Ip$, l'intensité du champ laser nécessaire à la suppression de la barrière est 
\begin{equation}
I_{\text{sat}} = \frac{Ip^4}{16} \: \: ; \: \: I_{\text{sat}} [\text{W/cm$^2$}] = 4 \times 10^9 Ip^4 [\text{eV}]
\end{equation}
Les intensités de suppression de barrière, ou de saturation, pour différents gaz couramment utilisés pour la GHOE sont rassemblées dans le tableau \ref{tab:Isat}. Pour que la GHOE soit efficace, l'intensité laser doit être inférieure à l'intensité de saturation afin de ne pas complètement "vider" l'état fondamental. 

\begin{table}
\begin{center}
\begin{tabular}{|c|c|c|}
\hline
Gaz & $Ip$ (eV) & $I_\text{sat}$ (W/cm$^2$) \\
\hline
He & 24.58 & $1.5 \times 10^{15}$ \\
\hline
Ne & 21.56 & $8.7 \times 10^{14}$ \\
\hline
Ar & 15.76 & $2.5 \times 10^{14}$ \\
\hline
Xe & 12.12 & $9 \times 10^{13}$ \\
\hline
\end{tabular}
\end{center}
\caption{Potentiel d'ionisation $Ip$ et intensité de suppression de barrière $I_\text{sat}$ pour différents gaz couramment utilisés pour la GHOE.}
\label{tab:Isat}
\end{table}

Cette approche doit être complétée par un aspect dynamique dans le cas d'un champ laser oscillant: pour que l'ionisation tunnel ait lieu il faut que la barrière tunnel soit abaissée pendant une durée suffisante par rapport à la période d'oscillation du champ. Ce rapport est caractérisé par le paramètre de Keldysh \mycite{Keldysh1965} $\gamma = \sqrt{Ip/Up}$ où $Up$ est l'énergie pondéromotrice du champ:
\begin{equation}
Up = \frac{e^2 \mathcal{E}_0^2}{4 m \omega^2} \propto I \lambda^2
\label{eq:Up}
\end{equation}
Le régime d'ionisation tunnel correspond à $\gamma < 1$. Pour $\gamma > 1$, on parle de régime multiphotonique. Le régime de suppression de barrière correspond à $\gamma \ll 1$. Par exemple, pour l'argon avec un champ laser à 800 nm d'intensité $2 \times 10^{14}$ W/cm$^2$, $Up = 12$ eV soit $\gamma = 0.8$ et on se trouve dans le régime d'ionisation tunnel.

\paragraph{Accélération du paquet d'onde électronique dans le champ laser} La dynamique de l'électron libéré dans le continuum par ionisation tunnel, en présence du champ électrique, est traitée classiquement. Les effets à longue portée du potentiel atomique sont négligés. La seule force agissant sur l'électron étant la force de Lorentz, on a:
\begin{equation}
m \ddot{x} = -e \mathcal{E}_0 \cos(\omega t)
\label{eq:EDMelectron}
\end{equation} 
L'instant d'ionisation est noté $t_i$. On suppose que l'électron est émis en $x(t_i) = 0$, c'est-à-dire que l'on néglige le mouvement à travers la barrière tunnel, avec une vitesse initiale nulle $\dot{x}(t_i) = 0$. L'intégration de l'équation \ref{eq:EDMelectron} deux fois donne l'équation de la trajectoire de l'électron:
\begin{equation}
x(t) = \frac{e \mathcal{E}_0}{m \omega^2} \left[ \cos(\omega t) - \cos(\omega t_i) \right] + \frac{e \mathcal{E}_0}{m \omega}(t-t_i) \sin(\omega t_i)
\label{eq:TrajElec}
\end{equation}
L'électron oscille dans le champ selon la direction $\vec{u_x}$ et, pour certains instants d'ionisation, peut retourner en $x = 0$ c'est-à-dire recombiner sur son ion parent. L'excursion de l'électron dans le continuum est de l'ordre de 50 u. at., soit 2.6 nm.

\paragraph{Recombinaison} Lors de la recombinaison électron-ion, le système peut convertir l'énergie cinétique $Ec$ accumulée dans le continuum en un photon d'énergie $\hbar \omega = Ec + Ip$. L'équation \ref{eq:TrajElec} suggère que la recombinaison est possible à chaque oscillation de l'électron au voisinage de l'ion. Cependant, si la dynamique du POE dans la direction de polarisation du champ est bien décrite classiquement, elle l'est beaucoup moins dans la direction transverse à sa trajectoire. Le POE s'étale dans la direction transverse au cours de la propagation. La probabilité de recombinaison en $x = 0$ diminue donc à chaque période, ainsi on considère uniquement le premier retour en $x = 0$. La résolution numérique de l'équation \ref{eq:TrajElec} permet la détermination de cet instant de recombinaison $t_r$ tel que $x(t_r) = 0$. Les trajectoires électroniques correspondantes sont illustrées sur la figure \ref{fig:traj_class}(a). \`{A} la recombinaison, l'électron possède une énergie cinétique
\begin{equation}
Ec \: (t_r) = \frac{e^2 \mathcal{E}_0^2}{2m\omega^2} \left[ \sin(\omega t_r) - \sin(\omega t_i) \right]^2
\end{equation}
Le calcul de $Ec$ pour chaque couple $(t_i, t_r)$ est effectué pour les trajectoires de la figure \ref{fig:traj_class}(a), et est représenté figure \ref{fig:traj_class}(b). 

\begin{figure}
\centering
\def\svgwidth{0.6\columnwidth}
\import{Figures/GHOE/}{Trajectoires_Classique_Final.pdf_tex}
\caption{(a) Champ électrique (rouge) et trajectoires électroniques (vert) calculées classiquement pour un électron dans un champ d'intensité $I = 2.2 \times 10^{14}$  W/cm$^2$ à 800 nm. (b) Energie cinétique de l'électron à la recombinaison, instants d'ionisation (pointillés) et de recombinaison (trait continu) correspondant aux trajectoires classiques de (a).}
\label{fig:traj_class}
\end{figure}

On peut tirer plusieurs conlcusions importantes de ces calculs. Premièrement, on constate que l'énergie cinétique qui peut être accumulée par l'électron possède un maximum $Ec^{\text{max}} = 3.17 \: Up$ \mycite{KrausePRL1992}. Ainsi l'énergie de photon maximale atteinte, appelée énergie de coupure, est
\begin{equation}
(\hbar \omega)^{\text{max}} = Ip + 3.17 \: Up
\label{eq:LoiCoupure}
\end{equation}
Par exemple, dans l'argon avec un laser à 800 nm d'intensité $2 \times 10^{14}$ W/cm$^2$, $(\hbar \omega)^{\text{max}} \approx 15.8 + 3.17 \times 12 = 53.8$ eV, dans le domaine de l'extrême UV. Pour augmenter l'énergie de photon maximale atteignable, d'après l'équation \ref{eq:LoiCoupure} et la définition de l'énergie pondéromotrice \ref{eq:Up}, plusieurs solutions sont envisageables:
\begin{enumerate}[label=\roman*)]
\item utiliser un gaz de plus haut potentiel d'ionisation.
\item augmenter l'intensité du laser.
\item augmenter la longueur d'onde du laser.
\end{enumerate}
La gamme de potentiel d'ionisation des espèces neutres a une étendue limitée (12 eV entre le xénon et l'hélium, voir tableau \ref{tab:Isat}), ainsi la solution (i) appliquée toute seule ne permet pas d'atteindre des énergies de photon très élevées. Les ions ont des énergies d'ionisation bien plus élevées que les atomes neutres mais très peu d'expériences de GHOE dans un milieu ionisé ont été reportées à ce jour \mycite{PopmintchevScience2015}. En effet, la forte densité électronique correspondante est très défavorable à une émission macroscopique efficace (voir paragraphe \ref{sec:AccordDePhase}). Comme nous l'avons vu précédemment, l'intensité laser ne peut excéder l'intensité de suppression de barrière $I_{\text{sat}}$ pour que le processus soit efficace. Cependant, $I_{\text{sat}}$ varie très non-linéairement avec le potentiel d'ionisation (tableau \ref{tab:Isat}) si bien que la combinaison des solutions i) et ii) a permis de produire des énergies de photon s'étendant dans la fenêtre de l'eau (280 - 540 eV) et allant jusqu'au keV, grâce à des impulsions de seulement quelques cycles optiques qui minimisent les effets d'ionisation \mycite{SpielmannScience1997}\mycite{SeresNature2005}. Plus récemment, l'approche iii) a été privilégiée pour atteindre des énergies de photons dans le domaine des rayons X mous, jusqu'à 1.6 keV \mycite{ChenPRL2010}\mycite{PopmintchevScience2012} (voir aussi paragraphe \ref{sec:GHOE_MIR}).

Deuxièmement, la figure \ref{fig:traj_class}(b) montre qu'une énergie cinétique donnée peut être atteinte à deux instants de recombinaison différents, correspondant à des trajectoires électroniques distinctes. Les deux familles de trajectoires sont appelées "courtes" et "longues" selon la durée de l'excursion de l'électron dans le continuum. Pour la trajectoire courte (resp. longue), l'énergie augmente (resp. diminue) avec l'instant de recombinaison. Les harmoniques émises lors de ces différentes trajectoires possèdent des propriétés distinctes, qui seront détaillées par la suite. Elles convergent pour devenir indiscernables à l'énergie de coupure.

\paragraph{Structure spectrale de l'émission: les harmoniques d'ordre élevé} Ce processus en trois étapes se répète à chaque extremum du champ électrique, c'est-à-dire tous les demi-cycles (avec un changement de signe du dipôle induit). Il a donc une périodicité de $T/2$, où $T = 2 \pi /\omega$ est la période du laser de génération. Cette périodicité temporelle se traduit dans le domaine spectral par une périodicité à $2 \omega$. Le milieu de génération étant centro-symétrique, seules les harmoniques impaires sont émises. Ainsi, pour une impulsion laser suffisamment longue (plusieurs cycles), le spectre obtenu se compose d'un peigne d'harmoniques impaires séparées par $2 \hbar \omega$ (figure \ref{fig:SpectreSalieres}).

\begin{figure}
\centering
\def\svgwidth{0.6\columnwidth}
\import{Figures/GHOE/}{SpectreSalieres.pdf_tex}
\caption{Spectre harmonique typique. Adapté de \mycite{SalieresMST2001}.}
\label{fig:SpectreSalieres}
\end{figure}

Dans ce modèle, l'étape d'ionisation tunnel est traitée quantiquement tandis que la dynamique de l'électron libre dans le champ est traitée de manière classique, d'où  son appellation semi-classique. Il donne une image simple du processus et permet d'accéder à des quantités importantes telles que l'énergie de coupure et les instants d'ionisation et de recombinaison. Cependant, le processus de GHOE résulte de l'interférence entre la partie du paquet d'onde électronique ionisée et la partie restée dans l'état fondamental. Sa description complète fait donc appel à la mécanique quantique.

\subsection{Modèle quantique de Lewenstein}
Le traitement quantique de la GHOE est développé par Maciej Lewenstein en 1994 \mycite{LewensteinPRA1994}. Il donne une justification des hypothèses du modèle semi-classique et permet de prendre en compte les effets quantiques tels que l'ionisation tunnel, la diffusion du POE et les interférences entre chemins quantiques. Nous décrivons ici brièvement les bases de ce modèle. On considère un atome dans l'approximation d'un seul électron actif soumis au champ laser $\vec{\mathcal{E}}(t)$ polarisé linéairement selon $\vec{u_x}$. La dynamique électronique est décrite par l'équation de Schrödinger (en unités atomiques):
\begin{equation}
i \frac{\partial}{\partial t} \ket{\psi(\vec{x},t)} = \left( -\frac{1}{2}\nabla^2 + V_0(\vec{x}) - \mathcal{E}_0 x \cos(\omega t) \right) \ket{\psi(\vec{x},t)}
\end{equation} 
On fait alors les approximations suivantes pour calculer la fonction d'onde $\ket{\psi(\vec{x},t)}$:
\begin{enumerate}
\item En ce qui concerne les états liés, la contribution des états excités est négligeable, seul l'état fondamental est pris en compte. Ceci est valable dans le régime d'ionisation tunnel ($\gamma <1$) dans lequel le laser n'induit pas de transfert de population de l'état fondamental vers les états excités.
%\item La déplétion de l'état fondamental est négligée. Ceci est valable si l'intensité utilisée est inférieure à l'intensité de saturation définie précédemment.
\item L'influence du potentiel coulombien sur la dynamique de l'électron dans le continuum est négligée. Dans le continuum l'électron est uniquement soumis à un champ électrique intense; c'est l'approximation du champ fort (\textit{Strong Field Approximation}, SFA).
\end{enumerate}
Le spectre harmonique est calculé en effectuant la transformée de Fourier du moment dipolaire $x(t) =\bra{\psi(t)} x \ket{\psi(t)}$,
\begin{equation}
x(\Omega) = \int_{-\infty}^{+\infty} x(t) \: \rme^{i \Omega t}  \: \rmd t
\end{equation}
où $\Omega$ est la pulsation de l'harmonique émise, notée dans la suite indifféremment $\Omega = \omega_q = q \omega$. Le modèle de Lewenstein donne l'expression de $x(t_r)$ à l'instant de recombinaison:
\begin{equation}
x(t_r) = i \int_0^{t_r} \rmd t_i \int \rmd^3 \vec{p} \: \vec{d}^*_{\vec{p}+\vec{A}(t_r)} \: \text{exp}\left[i S(\vec{p}, t_i, t_r) \right] \: \vec{\mathcal{E}}(t_i) \: \vec{d}_{\vec{p}+\vec{A}(t_i)} (+ c.c. ??)
\label{eq:x_Lew}
\end{equation}
Dans cette expression, $\vec{p}$ est le moment canonique, $\vec{d}$ est le moment dipolaire de la transition entre l'état fondamental et le continuum, et $\vec{A}$ est le potentiel vecteur associé au champ électrique $\vec{\mathcal{E}}$. Le terme $S$ est appelée intégrale d'action le long de la trajectoire électronique. 

\paragraph{Interprétation} En lisant l'expression \ref{eq:x_Lew} de droite à gauche, on retrouve le modèle semi-classique en trois étapes:
\begin{enumerate}[label=(\arabic*)]
\item à l'instant d'ionisation $t_i$, une partie du POE passe de l'état fondamental au continuum \textit{via} une transition dipolaire électrique. $\vec{p}$ étant le moment canonique, l'impulsion à cet instant est égale à $\vec{p} + \vec{A}(t_i)$ et l'amplitude de la transition s'écrit $\vec{\mathcal{E}}(t_i) \: \vec{d}_{\vec{p}+\vec{A}(t_i)}$.
\item entre les instants $t_i$ et $t_r$, le POE se propage dans le continuum sous l'action du champ laser et acquiert une phase 
\begin{equation}
S(\vec{p}, t_i, t_r) = - \int_{t_i}^{t_r} \left( Ip + \frac{(\vec{p} + \vec{A}(t))^2}{2} \right) \rmd t
\end{equation}
\item à la recombinaison à l'instant $t_r$, l'impulsion est égale à $\vec{p} + \vec{A}(t_r)$. Le dipôle de recombinaison étant le complexe conjugué du dipôle de photoioinisation, l'amplitude de transition lors de la recombinaison est $\vec{d}^*_{\vec{p}+\vec{A}(t_r)}$.
\end{enumerate}

\paragraph{Calcul des trajectoires} La transformée de Fourier de l'expression \ref{eq:x_Lew} donne
\begin{equation}
x(\Omega) = \int \rmd t_r \int \rmd t_i \int \rmd^3 \vec{p} \: b(t_r,t_i, \vec{p}) \: \text{exp}\underbrace{\left[i S(\vec{p}, t_i, t_r) + i \Omega t_r \right]}_{i \phi_\Omega(\vec{p}, t_i, t_r)}
\label{eq:TFx_Lew}
\end{equation}
Dans cette expression, la somme est effectuée sur tous les instants d'ionisation, de recombinaison, et tous les moments canoniques, c'est-à-dire sur toutes les trajectoires électroniques. Cette infinité de chemins possibles rend le calcul difficile. Le calcul se simplifie si la somme n'est effectuée que sur les trajectoires contribuant significativement à l'émission. Afin de déterminer ces contributions majoritaires, on applique le principe de la phase stationnaire: la phase de l'intégrand dans l'expression \ref{eq:TFx_Lew} varie à une échelle de temps beaucoup plus rapide que son amplitude. Pour un chemin dont la phase varie très rapidement, les différentes contributions s'annulent dans la somme, rendant alors la contribution de ce chemin négligeable. Les principales trajectoires correspondent alors aux points où la phase ne varie pas le long des trois variables $\vec{p}$, $t_i$ et $t_r$. Cette condition se traduit par les équations de point selle suivantes:
\begin{align}
\frac{\partial \phi_\Omega(\vec{p}, t_i, t_r)}{\partial t_i} & = Ip + \frac{(\vec{p} + \vec{A}(t_i))^2}{2} = 0 \label{eq:Pointselle1}\\
\frac{\partial \phi_\Omega(\vec{p}, t_i, t_r)}{\partial t_r} & = -Ip - \frac{(\vec{p} + \vec{A}(t_r))^2}{2} + \Omega = 0 \label{eq:Pointselle2}\\
\nabla_{\vec{p}} \phi_\Omega(\vec{p}, t_i, t_r) & = - x(t_r) + x(t_i) = 0 \label{eq:Pointselle3}
\end{align}
\ref{eq:Pointselle1} correspond à la conservation de l'énergie à l'instant d'ionisation et indique que le POE possède initialement une énergie cinétique négative. Ceci correspond à un instant d'ionisation $t_i$ complexe et est un reliquat de l'ionisation tunnel. \ref{eq:Pointselle2} est simplement la conservation de l'énergie à l'instant de recombinaison, $\Omega = Ip + \frac{(\vec{p} + \vec{A}(t_r))^2}{2}$. Enfin, \ref{eq:Pointselle3} indique que le POE retourne à sa position initiale. La résolution de ce système d'équations permet de calculer $t_i$, $t_r$ et $\vec{p}$.

\begin{figure}[ht]
\centering
\def\svgwidth{\columnwidth}
\import{Figures/GHOE/}{SFA_vs_3stepmodel.pdf_tex}
\caption{Instants d'ionisation et de recombinaison en fonction de l'énergie du photon émis calculés par le modèle semi-classique (noir) et le modèle quantique (rouge) dans l'argon avec un champ laser à 800 nm d'intensité $2.5 \times 10^{14}$ W/cm$^2$.}
\label{fig:SFA_vs_3step}
\end{figure}

La figure \ref{fig:SFA_vs_3step} montre le calcul de la partie réelle de l'instant d'ionisation/de recombinaison en fonction de l'énergie du photon harmonique émis avec le modèle de Lewenstein. Comme dans le modèle semi-classique, il existe deux familles de trajectoires, "courtes" et "longues", conduisant à la même énergie de photon et convergeant dans la coupure. Le modèle quantique donne également une expression de l'énergie de coupure 
\begin{equation}
(\hbar \omega)^{\text{max}} = f(\frac{Ip}{Up}) + 3.17 \: Up
\end{equation}
où $f$ est un facteur dépendant du rapport $Ip/Up$ variant de 1.32 à 1.2 lorsque la rapport $Ip/Up$ varie de 1 à 4. Enfin, la comparaison avec les résultats du modèle semi-classique montre un bon accord. Bien que complexe dans le modèle de Lewenstein, l'instant d'ionisation $t_i$ possède une partie réelle comparable à l'instant d'ionisation calculé par le modèle semi-classique.

\section{Structure temporelle de l'émission harmonique}
Dans les paragraphes précédents, nous avons vu que l'émission harmonique se compose dans le domaine spectral d'un grand nombre de fréquences, multiples impaires de la fréquence laser fondamentale. La largeur spectrale émise supporte \textit{a priori} une durée d'impulsion attoseconde \mycite{FarkasPhysLettA1992}\mycite{HarrisOptComm1993}, si toutes les composantes spectrales émises possèdent la relation de phase adéquate. Considérons ici un spectre composé de $n$ harmoniques monochromatiques d'amplitude spectrale $A_q$ et de phase spectrale $\phi_q$. Ceci revient à considérer l'impulsion femtoseconde de génération comme infiniment longue. Le profil temporel de l'émission s'écrit alors:
\begin{equation}
\mathcal{I}(t) = \left| \sum_{q=1}^n A_q \rme^{-iq\omega t + i \phi_q} \right|^2
\end{equation}
Si $\phi_q$ est constante quel que soit $q$, l'impulsion est dite limitée par transformée de Fourier. Sa durée est alors minimale étant donnée sa largeur spectrale. Ce cas est illustré sur la figure \ref{fig:PeigneHH}: l'impulsion correspondant au spectre harmonique de largeur totale $N$ où chaque harmonique possède une largeur $\delta \omega$ est un train d'impulsions attosecondes. La largeur temporelle du train est $1/\delta \omega$, et chaque impulsion dans le train a une durée $1/N$. Si $\phi_q$ est linéaire avec $q$, le profil temporel est le même que précédemment, mais décalé temporellement de $t_e = \partial \phi_q / \partial \omega$, ce délai de groupe est également appelé temps d'émission. Si $\phi_q$ a un autre comportement, alors l'impulsion est plus longue que la durée donnée par la limite de Fourier. Les différentes composantes spectrales de l'impulsion ne sont pas émises au même moment. $t_e (\omega_q)$ est alors le retard de groupe associé à la fréquence $\omega_q$. Dans le cas extrême où la phase entre chaque harmonique est aléatoire, l'émission lumineuse devient continue. Il est donc important de connaître la phase spectrale des harmoniques pour l'étude et la mise en forme d'impulsions attosecondes.

\begin{figure}
\centering
\def\svgwidth{\columnwidth}
\import{Figures/GHOE/}{PeigneHH.pdf_tex}
\caption{Structure temporelle d'un peigne harmonique limité par transformée de Fourier. Extrait de \mycite{TheseMairesse}.}
\label{fig:PeigneHH}
\end{figure}

Comme l'illustre la figure \ref{fig:PeigneHH}, deux types de phases spectrales sont à considérer\footnote{\'{E}videmment, il n'y a qu'une seule phase spectrale, mais on peut l'étudier à différentes échelles.}: la relation de phase entre harmoniques consécutives qui a une influence sur la durée des impulsions dans le train, et la phase spectrale d'une harmonique donnée qui modifie globalement le train. Ces deux phases sont responsables respectivement de la dérive de fréquence attoseconde (ou \textit{chirp} atto) et de la dérive de fréquence harmonique (ou \textit{chirp} harmonique/femto).

\subsection{Structure attoseconde: le chirp atto} 
\label{subsec:ChirpAtto}
Les deux modèles présentés précédemment montrent que les différentes harmoniques ne sont pas émises au même instant (figures \ref{fig:traj_class} et \ref{fig:SFA_vs_3step}): pour les trajectoires courtes (resp. longues), les hautes énergies sont émises après (resp. avant) les basses énergies. \mycite{TheseMairesse} montre que $t_e(\omega_q)$ est directement relié à l'instant de recombinaison $t_r(\omega_q)$. Dans le cadre du modèle de Lewenstein, la variation de $t_e$ avec l'ordre harmonique est linéaire dans le plateau, avec une pente opposée pour les deux familles de trajectoires, et constante dans la coupure. Ainsi, sous réserve de sélectionner un type de trajectoires, la phase harmonique est quadratique dans le plateau et linéaire dans la coupure. Cette phase spectrale peut être mesurée (par exemple avec la méthode RABBIT exposée dans la suite), et la mesure pour les trajectoires courtes est en très bon accord avec le modèle théorique (figure \ref{fig:MairessePRL}) \mycite{MairesseScience2003}\mycite{MairessePRL2004}. Le chirp atto est intrinsèquement lié au processus de GHOE: dans un demi-cycle, les fréquences émises correspondent à différentes trajectoires électroniques et ne sont donc pas synchronisées.  

\begin{figure}
\centering
\def\svgwidth{0.7\columnwidth}
\import{Figures/GHOE/}{MairessePRL.pdf_tex}
\caption{(a) Intensité et (b) instant d'émission $t_e$ d'harmoniques générées dans le xénon à $3 \times 10^{13}$ W/cm$^2$ (rouge) et $6 \times 10^{13}$ W/cm$^2$ (vert), et dans l'argon à $9 \times 10^{13}$ W/cm$^2$ (bleu). Les symboles sont des valeurs mesurées par RABBIT. Les traits continus correspondent aux instants de recollision pour la trajectoire courte calculée avec le modèle de Lewenstein. Extrait de \mycite{MairessePRL2004}.}
\label{fig:MairessePRL}
\end{figure}

La dérive de fréquence linéaire dans le plateau correspond à une phase spectrale quadratique et donc à un élargissement temporel des impulsions attosecondes dans le train. Cependant, on remarque que la sélection spectrale des harmoniques de la coupure uniquement, dont la phase spectrale est linéaire, permet d'obtenir des impulsions limitées par Fourier mais dont la durée ne pourra pas être très courte du fait de la décroissance exponentielle de l'intensité harmonique. Notons que le délai de groupe linéaire peut être partiellement compensé si l'impulsion est propagée à travers un filtre métallique de délai de groupe opposé. Cette méthode peut être utilisée pour comprimer les impulsions attosecondes \mycite{LopezMartensPRL2005}\mycite{GustafssonOptLett2007}, et a permis de produire les impulsions les plus brèves à l'heure actuelle, de seulement 43 as \mycite{GaumnitzOE2017}. Une autre possibilité pour compenser ce délai de groupe intrinsèque à la GHOE est d'utiliser une réflexion sur un miroir à phase contrôlée \mycite{MorlensOL2005}\mycite{BourassinOE2011}.

\paragraph{Dépendance en intensité} Comme le montrent les résultats expérimentaux de \mycite{MairessePRL2004} reproduits figure \ref{fig:MairessePRL}, la pente de $t_e (\omega_q)$ diminue lorsque l'intensité de génération augmente (pour les trajectoires courtes). Ceci peut s'interpréter simplement grâce à la loi de la coupure: l'énergie de coupure est proportionnelle à $Up \propto I$. Ainsi lorsque l'intensité augmente, l'énergie de coupure augmente et la pente de $t_r (\omega_q)$ diminue en valeur absolue pour les deux familles de trajectoires: $t_e (\omega_q) \propto 1/I $. La productions d'impulsions attosecondes de plus courte durée est donc favorisée à haute intensité.

\subsection{Structure femtoseconde: le chirp harmonique}
Pour une harmonique donnée, la phase spectrale pour la trajectoire $j$ est donnée par la phase du dipôle (équation \ref{eq:TFx_Lew}):
\begin{equation}
\phi_q^j = \omega_q t_r^j - \int_{t_i^j}^{t_r^j} Ip + \frac{(\vec{p}^j + \vec{A}(t))^2}{2} \rmd t
\label{eq:PhaseHarmonique}
\end{equation}
Le second terme est l'intégrale d'action, qui représente la phase accumulée par le POE le long de la trajectoire considérée. Elle dépend de l'intensité laser \textit{via} le potentiel vecteur $\vec{A}(t)$.

\begin{figure}[ht]
\centering
\def\svgwidth{\columnwidth}
\import{Figures/GHOE/}{Varju.pdf_tex}
\caption{Calcul avec le modèle SFA. (a)Variations de la phase $\phi_q^j$ avec l'intensité laser pour l'harmonique 19 à 800 nm dans l'argon. (b) Variations de la dérivée $\partial \phi_q^j/\partial I = -\alpha_q^j$ avec l'ordre harmonique pour $I = 1.5 \times 10^{14}$ W/cm$^2$. La trajectoire courte est en trait continu et la trajectoire longue en pointillés. Extrait de \mycite{VarjuJMO2005}.}
\label{fig:Varju}
\end{figure}

La figure \ref{fig:Varju}(a) montre le calcul de la dépendance en intensité de la phase de l'harmonique 19 effectué avec le modèle SFA pour les deux trajectoires. \`{A} basse intensité, l'harmonique 19 se trouve dans la coupure et les deux trajectoires sont confondues. Pour les deux trajectoires, $\phi_q^j$ est approximativement linéaire avec l'intensité avec un coefficient de proportionnalité dépendant de la trajectoire considérée,
\begin{equation}
\phi_q^j = - \alpha_q^j \times I
\end{equation}
avec $\alpha_q^{\text{courte}} \ll \alpha_q^{\text{longue}}$. 

Lors de la GHOE avec une impulsion laser femtoseconde, l'intensité laser varie à l'échelle de l'enveloppe $I(t)$, ce qui implique une modification de la phase du dipole. L'émission femtoseconde harmonique n'est donc pas limitée par transformée de Fourier mais présente une dérive de fréquence proportionnelle à $\alpha_q$ \mycite{SalieresPRL1995}. Ce chirp harmonique est intrinsèquement lié à la variation de l'intensité laser à l'échelle de l'enveloppe femtoseconde. D'un demi-cycle à l'autre, les trajectoires électroniques conduisant à l'émission d'une énergie de photon donnée sont modifiées. Au sein du train d'impulsions attosecondes, on observe une modification de l'espacement des impulsions dans le train \mycite{VarjuJMO2005}. En pratique, le chirp atto et le chirp harmonique sont tous deux présents, comme illustré sur la figure \ref{fig:Varju_atto_harmo}.

\begin{figure}
\centering
\def\svgwidth{0.5\columnwidth}
\import{Figures/GHOE/}{Varju_atto_harmo.pdf_tex}
\caption{Illustration de la présence simultanée du chirp atto (rouge) et du chirp harmonique (bleu) dans le domaine spectral (a) et temporel (b). Extrait de \mycite{VarjuJMO2005}.}
\label{fig:Varju_atto_harmo}
\end{figure}

\paragraph{Trajectoires courtes et longues} $\alpha_q^j$ étant très différent pour les trajectoires courte et longue, la dérive de fréquence est très différente pour les deux trajectoires. Pour une harmonique donnée, la contribution de la trajectoire longue présentera un élargissement spectral beaucoup plus important que celle de la trajectoire courte, ce qui permet de différencier ces deux contributions avec un spectromètre \mycite{ZairPRL2008}. Par ailleurs, pour un faisceau gaussien, l'intensité présente également une modulation radiale $I(r)$. La dépendance en intensité de la phase du dipôle introduit ici une courbure de phase, différente pour les trajectoires courte et longue. Pour les harmoniques les plus basses, les trajectoires courtes possèdent une divergence bien plus faible que les longues \mycite{BelliniPRL1998}. Quand l'ordre harmonique augmente, la divergence des trajectoires courtes (resp. longues) augmente (resp. diminue) jusqu'à se confondre à la coupure. Ces deux effets sont bien visibles sur les spectres expérimentaux (voir par exemple la figure \ref{fig:SpectreAr800}).

\section{Réponse macroscopique}
\label{sec:AccordDePhase}
\subsection{Position du problème}
Jusqu'à présent nous avons considéré les propriétés de l'émission harmonique d'un unique émetteur. Dans le cas de l'émission d'un milieu macroscopique, les propriétés spatio-temporelles des harmoniques dépendent également des variations d'amplitude et de phase dans les trois dimensions transverses $(x,y)$ et longitudinale $z$ du milieu non-linéaire de génération. Comme l'illustre la figure \ref{fig:PhaseMatchingHeyl}, si les différentes contributions ne sont pas en phase, alors des interférences destructives empêcheront une émission efficace de rayonnement XUV.

\begin{figure}[ht]
\centering
\def\svgwidth{0.7\columnwidth}
\import{Figures/GHOE/}{PhaseMatching_Heyl.pdf_tex}
\caption{Illustration de l'émission harmonique avec et sans accord de phase, pour l'exemple de la seconde harmonique. Les parties rosées correspondent aux oscillations de la polarisation non linéaire. Extrait de \mycite{TheseHeyl}.}
\label{fig:PhaseMatchingHeyl}
\end{figure}

Considérons simplement l'équation de propagation des champs harmoniques $\mathcal{E}(\vec{r},t) = \sum_q A_q \rme^{i(\vec{k_q} \vec{r} - q\omega t)}$ dans le milieu de polarisation non linéaire $P_q^{\text{NL}}$. Dans les approximations paraxiale et de l'enveloppe lentement variable, on a:
\begin{equation}
\Delta_\bot A_q + 2 i k_q \frac{\partial A_q}{\partial z}= -\frac{q^2 \omega^2}{\epsilon_0 c^2} P_q^{\text{NL}} \rme^{i(q \vec{k_1} - \vec{k_q})z}
\end{equation} 
La quantité
\begin{equation}
\Delta \vec{k} (q) = q \vec{k_1} - \vec{k_q}
\end{equation}
est le désaccord de phase\footnote{La convention opposée est parfois rencontrée dans la littérature}. L'influence du désaccord de phase sur la GHOE peut être illustrée simplement par un modèle unidimensionnel. Pour $A_q(z)$, le signal harmonique $S_q$ est la somme cohérente sur tous les atomes du milieu de longueur $L$:
\begin{equation}
S_q \propto \left| \int_0^L x_q \rme^{i(\Delta k + i \eta_q)(L-z)} \: \rho \rmd z \right|^2
\end{equation}
où $x_q$ est l'amplitude du dipôle correspondant à l'harmonique $q$, $\rho$ la densité et $\eta_q$ le coefficient d'absorption du milieu à l'énergie de l'harmonique $q$. Si $x_q$, $\Delta k$, $\rho$ et $\eta_q$ ne dépendent pas de $z$, alors $S_q$ devient \mycite{ConstantPRL1999}\mycite{TheseHeyl}:
\begin{equation}
S_q \propto |d_q|^2 \rme^{-\eta_q L} \frac{\cosh(\eta_q L) - \cos(\Delta k L)}{\Delta k^2 + \eta_q^2}
\end{equation}
Finalement, pour $\eta_q \rightarrow 0$, on a
\begin{equation}
S_q \propto |d_q|^2 L^2 \left( \frac{\sin(\Delta k L /2\pi)}{\Delta k L /2\pi}\right)^2
\end{equation}
Pour $\Delta k = 0$ (accord de phase parfait), le signal harmonique augmente avec $L^2$. En revanche si $\Delta k \neq 0$, à longueur de milieu constante l'intensité harmonique est très sensible à $\Delta k$ à cause de la variation en sinus cardinal au carré. L'optimisation de la GHOE nécessite donc la réalisation de l'accord de phase, $\Delta \vec{k} = \vec{0}$.

A la lumière de la discussion ci-dessus, les grandeurs caractéristiques pertinentes pour déterminer l'efficacité de génération sont:
\begin{itemize}
\item la longueur du milieu $L$.
\item la longueur de cohérence $L_c = 1/\Delta k$.
\item la longueur d'absorption $L_{abs} = 1/\eta_q = 1/(\sigma \rho)$, où $\sigma$ est la section efficace d'absorption.
\item la longueur d'amplification $L_{amp}$ sur laquelle le dipôle $x_q$ a une amplitude significative.
\end{itemize}
La valeur relative de ces différentes longueurs (et, le cas échéant, la dimension tridimensionnelle du problème) déterminera l'émission macroscopique.

\subsection{Accord de phase pour la GHOE dans les gaz}
Dans les gaz, le désaccord de phase est la somme de quatre contributions \mycite{BalcouPRA1997}\mycite{TheseHergott}:
\begin{equation}
\Delta \vec{k} = \Delta \vec{k}_a + \Delta \vec{k}_{el} + \Delta \vec{k}_{foc} + \Delta \vec{k}_{dip}^{traj}
\end{equation}
Par commodité nous considérons dans la suite la seule composante de $\Delta \vec{k}$ selon la direction de propagation $z$ du champ.
\begin{itemize}
\item $\Delta k_a$ est le désaccord de phase dû à la dispersion atomique, et s'exprime en fonction des indices de réfraction du milieu à la fréquence fondamentale $n_1$ et harmonique $n_q$: \begin{equation}
\Delta k_a = q \frac{\omega}{c}(n_1 - n_q) 
\end{equation}
Généralement, $n_q < 1 < n_1$ \mycite{HenkeAtomicData1993}\mycite{CXRO}, d'où
\begin{equation}
\Delta k_a > 0
\end{equation}
Ce terme dépend de la densité d'atomes de gaz neutres dans le milieu, donc de la \textbf{pression} de génération.
\item $\Delta \vec{k}_{el}$ est le désaccord de phase dû à la dispersion par les électrons libres du milieu (produits par l'ionisation), de densité $N_e$. Son expression est analogue au terme précédent en remplaçant les indices de réfraction par les indices de réfraction du plasma: 
\begin{equation}
\Delta k_{el} = q \frac{\omega}{c}(n_1^e - n_q^e)
\label{eq:Deltakel}
\end{equation}
avec
\begin{equation}
n_q^e = \sqrt{1 - \frac{\omega_p^2}{\omega_q^2}} \approx 1 -  \frac{\omega_p^2}{2\omega_q^2}
\end{equation}
où $\omega_p$ est la fréquence plasma, $\omega_p^2 = \frac{e^2}{m \epsilon_0} N_e$. En remplaçant dans l'expression \ref{eq:Deltakel}, on obtient finalement
\begin{equation}
\Delta k_{el} \approx \frac{q \omega \omega_p^2}{2c}(\frac{1}{\omega_q^2} - \frac{1}{\omega^2}) < 0
\end{equation}
Ce terme dépend de la densité d'électrons libres dans le milieu, donc de la \textbf{pression} de génération et de l'\textbf{intensité}.
\item $\Delta k_{foc}$ est le désaccord de phase dû à la phase de Gouy. La focalisation du faisceau laser utilisé pour la GHOE induit une phase longitudinale, qui s'écrit pour un faisceau gaussien $- \arctan (z/z_R)$ où $z_R$ est la longueur de Rayleigh.
\begin{equation}
\Delta k_{foc} \approx -\frac{q}{z_R} < 0
\end{equation}
\item $\Delta k_{dip}^{traj}$ est le désaccord de phase dû à la phase du dipôle $\phi_q^j$ (équation \ref{eq:PhaseHarmonique}). Il provient de la réponse de l'atome unique exposée précédemment et dépend de la trajectoire considérée.
\begin{equation}
\Delta k_{dip}^{traj} = - \alpha^j \frac{\partial I}{\partial z}
\end{equation}
Cette contribution change de signe de part et d'autre du point focal du faisceau de génération.
\begin{align}
\Delta k_{dip}^{traj} \propto \text{sign}(z) & > 0 \: \: \text{Après le point focal} \\
& < 0 \: \: \text{Avant le point focal}
\end{align}
Ce terme dépend de la \textbf{position relative du milieu de génération et du point focal}, et est différent pour les deux familles de  \textbf{trajectoires}.
\end{itemize}

D'après ce qui précède, les paramètres expérimentaux pour optimiser l'accord de phase dans la GHOE sont la pression du gaz, l'intensité du laser et les conditions de focalisation \mycite{KazamiasPRA2011}. L'accord de phase est également un bon moyen pour isoler la contribution d'un type de trajectoires (en général les courtes): lorsque le jet de gaz est placé après le foyer les trajectoires courtes sont favorisées, tandis que les trajectoires longues sont favorisées lorsque le jet est placé avant le foyer \mycite{SalieresPRL1995}\mycite{AntoinePRL1996}\mycite{BalcouPRA1997}.

\section{Génération d'harmoniques d'ordre élevé dans l'infrarouge moyen}
\label{sec:GHOE_MIR}
Une grande partie des expériences présentées dans cette thèse étant effectuée avec un laser de génération dans l'infrarouge moyen ($\lambda = 1.3 - 2 \: \mu$m ici), nous discutons ici les variations avec la longueur d'onde des différentes grandeurs définies précédemment.

\begin{figure}[ht]
\centering
\def\svgwidth{0.7\columnwidth}
\import{Figures/GHOE/}{TempsRecomb_800nm_vs_1300nm.pdf_tex}
\caption{Calcul classique des instants d'ionisation (pointillés) et de recollision (trait continu) en fonction de l'énergie cinétique de l'électron pour un champ électrique à 800 nm (bleu) et 1300 nm (orange) de même intensité ($I = 2.2 \times 10^{14}$ W/cm$^2$).}
\label{fig:Recomb_vs_Lambda}
\end{figure}

\paragraph{Energie de coupure} D'après la loi de coupure (équation \ref{eq:LoiCoupure}) et l'expression de l'énergie pondéromotrice (équation \ref{eq:Up}),
\begin{equation}
(\hbar \omega)^{\text{max}} \propto \lambda^2
\end{equation}
Pour un même milieu de génération et à intensité égale, la GHOE à partir d'un laser dans l'IR moyen permet d'atteindre des énergies de photon plus élevées qu'à 800 nm \mycite{ChenPRL2010}\mycite{PopmintchevScience2012}.  Cette observation est illustrée par le calcul semi-classique de la figure \ref{fig:Recomb_vs_Lambda}. La GHOE dans l'IR moyen permet également d'atteindre des énergies de photons comparables au cas à 800 nm avec une intensité plus faible, donc en réduisant l'ionisation du milieu.

\paragraph{Chirp atto} Nous avons vu précédemment que le chirp atto est donné par la variation de l'instant de recombinaison avec l'énergie de photon. Comme le montre la figure \ref{fig:Recomb_vs_Lambda}, la pente de $t_e(\omega_q)$ est plus faible à grande longueur d'onde. Plus quantitativement: l'instant de recombinaison $t_r$ augmente linéairement avec la période du laser $T = \lambda/c$. L'énergie de coupure varie elle quadratiquement avec $\lambda$. On a alors
\begin{equation}
t_e (\omega_q) \propto \frac{t_r}{(\hbar \omega)^{\text{max}}} \propto \frac{1}{\lambda}
\end{equation}

\begin{figure}
\centering
\def\svgwidth{0.7\columnwidth}
\import{Figures/GHOE/}{Alpha_MIR.pdf_tex}
\caption{Calcul classique du chirp harmonique $\alpha_q^{\text{class}}$ en fonction de l'énergie cinétique de l'électron pour un champ électrique à 800 nm en bleu et 1300 nm en orange d'intensité égale ($I = 2.2 \times 10^{14}$ W/cm$^2$). Les trajectoires courtes sont indiquées en trait continu et les longues en pointillés.}
\label{fig:AlphaMIR}
\end{figure}

\paragraph{Chirp harmonique} La phase harmonique est donnée par l'expression \ref{eq:PhaseHarmonique}. L'intégrale d'action peut être interprétée de manière classique comme l'intégrale de l'énergie cinétique de l'électron (au signe près) \mycite{TheseMairesse}
\begin{equation}
S' = - \int_{t_i}^{t_r} \frac{(\vec{p}+\vec{A})^2}{2} \rmd t = \int_{t_i}^{t_r} \frac{\vec{v}^2}{2} \rmd t = \int_{t_i}^{t_r} Ec \: \rmd t
\end{equation}
Nous considérons que les variations avec l'intensité des deux autres termes de la phase, $\omega_q t_r$ et $\int_{t_i}^{t_r} Ip \rmd t$ sont négligeables devant celles de $S'$. Ainsi, 
\begin{equation}
\frac{\partial \phi_q}{\partial I} \approx \frac{\partial S'}{\partial I}
\end{equation}
La linéarité de la phase $\phi_q$ en fonction de l'intensité illustrée par la figure \ref{fig:Varju}(a) permet de faire l'approximation
\begin{equation}
\frac{\partial \phi_q}{\partial I} \approx \frac{S'}{I}
\end{equation}
La quantité $S'/I$ peut être calculée classiquement en calculant l'intégrale $S'$ pour chaque trajectoire électronique. On en déduit $\alpha_q^{\text{class}} = - S'/I$. Les résultats de ce calcul pour deux champs laser d'intensité égale à 800 nm et 1300 nm sont présentés figure \ref{fig:AlphaMIR}. Les valeurs obtenues à 800 nm sont en bon accord avec les valeurs de $\alpha_q^j$ calculées de manière quantique \mycite{GaardePRA2002}. Le chirp harmonique est donc plus important dans l'IR moyen qu'à 800 nm. Plus quantitativement, pour les trajectoires longues la durée de l'excursion de l'électron est de l'ordre de la période $T \propto \lambda$, et l'énergie est de l'ordre de $Up$, d'où
\begin{equation}
S' \propto Up \times T
\end{equation}
et
\begin{equation}
\alpha_q \propto \lambda^3
\end{equation}
Cette relation se retrouve dans le calcul de la figure \ref{fig:AlphaMIR} au voisinage de la coupure et pour les trajectoires longues.

% + gamma plus petit -> régime tunnel (LIED blablabla)
% alpha. modèle classique mairesse + explication qualitative durée de l'électron dans le continuum (mairesse).

%\section*{Conclusion}
%Dans ce chapitre, nous avons étudié le principe de la génération d'harmoniques d'ordre élevé dans les gaz. Nous avons mis en évidence les effets de la phase spectrale des harmoniques sur la structure temporelle du train d'impulsions attosecondes, à la fois à l'échelle de l'impulsion et du train. Nous avons montré que l'efficacité de la GHOE dans les gaz dépend de l'accord de phase, qui est modifié en fonction des conditions expérimentales. Dans la suite, nous présentons le dispositif expérimental de GHOE utilisé dans les expériences de cette thèse, ainsi que la technique de mesure de phase spectrale RABBIT.


\chapter{Aspects expérimentaux de la génération d'harmoniques d'ordre élevé}
\label{chap:ExpHHG}
\section{Dispositif expérimental de génération d'harmoniques d'ordre élevé}
\label{sec:HHG}
Les expériences de cette thèse ont été effectuées avec trois systèmes laser différents, au CEA-Saclay, à l'Université de Lund (Suède) et à l'Université de l'état de l'Ohio (\'{E}tats Unis). Ces trois lasers ont en commun d'être basés sur un milieu à gain composé de saphir dopé au titane et sur la technologie d'amplification à dérive de fréquence (\textit{Chirped Pulse Amplification, CPA}). \`{A} partir d'un oscillateur femtoseconde oscillant autour de 800 nm, le faisceau est étalé temporellement puis est amplifié avant d'être recomprimé dans un compresseur à réseaux \mycite{StricklandOptComm1985}. Les caractéristiques des trois systèmes laser sont résumées dans le tableau \ref{tab:Lasers}. Le laser PLFA (Plateforme Laser Femtoseconde Accordable) et la ligne de GHOE utilisés au CEA-Saclay sont décrits en détail par Weber \textit{et al.} \mycite{WeberRSI2015}. La longueur d'onde centrale peut être modifiée en utilisant différents dispositifs présentés au paragraphe \ref{sec:Accordabilité}.

\begin{table}[ht]
\begin{center}
\begin{tabular}{|c||c|c|c|}
\hline
Laser & Cadence & Puissance & Durée d'impulsion \\
\hline
Saclay & 1 kHz & 13 W & 50 fs \\
\hline
Lund & 1 kHz & 5 W & 20 fs  \\
\hline
Ohio State & 1 kHz & 13 W & 40 fs\\
\hline
\end{tabular}
\end{center}
\caption{Principales caractéristiques des différents systèmes laser utilisés dans les expériences de cette thèse. La durée indiquée correspond à la largeur à mi-hauteur de l'impulsion.}
\label{tab:Lasers}
\end{table}

Les conditions expérimentales précises sont détaillées avec chacune des expériences décrites par la suite. Nous présentons ici brièvement les principes d'une expérience de GHOE dans un gaz. Un schéma expérimental typique est visible sur la figure \ref{fig:SetupRabbit}.

Le faisceau laser est d'abord mis en forme: son diamètre est contôlé à l'aide d'un iris et son énergie peut être ajustée grâce à un atténuateur composé d'une lame demi-onde et de deux polariseurs. Si nécessaire, son état de polarisation est modifié (voir partie \ref{part:Polarimétrie}). Le faisceau est ensuite focalisé grâce à une lentille (si la largeur spectrale est étroite) ou un miroir sphérique (si la largeur spectrale est large, voir tableau \ref{tab:Lasers}) dans le gaz. Les ordres de grandeurs typiquement utilisés sont: diamètre de focalisation $\approx 10 - 15$ mm; énergie par impulsion $\approx 0.5 - 1$ mJ; distance focale $\approx 0.5 - 1$ m. Le gaz est délivré par un jet continu ou bien une cellule de gaz. Le jet, dont le diamètre de l'orifice est de 200 $\mu$m, permet de créer une extension supersonique du gaz qui garantit une courte longueur d'interaction avec le laser dans la dimension longitudinale. On limite ainsi l'importance des effets d'accord de phase discutés au paragraphe \ref{sec:AccordDePhase}. La cellule permet quant à elle d'augmenter la pression de gaz au niveau de l'interaction avec le laser tout en limitant la pression résiduelle dans la chambre à vide, ainsi que d'augmenter la longueur d'interaction $L$. Le choix du gaz dépend de l'expérience réalisée: on peut par exemple utiliser un gaz moléculaire dont on étudie la réponse non-linéaire, c'est le principe de la spectroscopie harmonique \mycite{TheseCamper}. Pour l'utilisation du rayonnement XUV en spectroscopie de photoionisation, on choisit un système simple, facile à se procurer et générant efficacement les harmoniques; généralement un gaz rare. Le gaz choisi dépend également de l'énergie de photon que l'on souhaite atteindre (équation \ref{eq:LoiCoupure}) et de l'efficacité de génération: les gaz rares légers peuvent générer des harmoniques plus élevées mais avec une efficacité typique dans le plateau bien inférieure aux gaz plus lourds, qui sont plus polarisables.

\begin{figure}[ht]
\centering
\def\svgwidth{\columnwidth}
\import{Figures/GHOE/}{SpectreHHGAr800.pdf_tex}
\caption{Spectre d'harmoniques générées dans l'argon à 800 nm mesuré par le spectromètre XUV à Saclay. Image mesurée sur la caméra CCD (haut) et somme du signal dans la direction spatiale (bas). Un signal de fond, mesuré dans les mêmes conditions expérimentales mais sans jet de gaz, est soustrait. Les pics de part et d'autre des harmoniques 9 et 11 sont les harmoniques plus élevées diffractées par le second ordre du réseau de diffraction. Les trajectoires longues contribuent au faible signal de grande divergence visible sur les harmoniques 17 à 23. Le faisceau harmonique se propage à travers un filtre d'aluminium de 200 nm qui absorbe significativement les harmoniques 9 et 11. Le spectre ne présente donc pas l'allure typique "plateau-coupure" présenté au chapitre \ref{chap:ThHHG}.}
\label{fig:SpectreAr800}
\end{figure}
% Spectre du 10 avril 2014

Après génération, le rayonnement XUV peut être dispersé par un spectromètre à réseau (réseau focalisant à pas variable Hitachi 0266 en incidence rasante) et imagé avec des galettes de micro-canaux couplées à un écran de phosphore. Le signal de phosphorescence est mesuré par une caméra CCD située à l'extérieur de l'enceinte à vide. L'imagerie du spectre XUV permet l'optimisation des paramètres expérimentaux pour atteindre l'énergie désirée, un flux de photons important et une bonne sélection des trajectoires courtes. La figure \ref{fig:SpectreAr800} montre un spectre harmonique typique généré dans l'argon mesuré avec ce dispositif. Un spectromètre XUV est disponible sur les lignes harmoniques de Saclay et de Lund. La caractérisation des harmoniques peut également être effectuée en focalisant le rayonnement XUV au point source d'un spectromètre de photoélectrons. Cette méthode est l'objet du paragraphe \ref{sec:CaracHHG}.




\section{Accordabilité du laser}
\label{sec:Accordabilité}
Dans les expériences présentées dans cette thèse, deux méthodes ont été utilisées pour accorder la longueur d'onde du laser tout en conservant une durée d'impulsion femtoseconde et une énergie par impulsion élevée:
\begin{itemize}[label=--,leftmargin=*]
\item à Saclay et en Ohio, un amplificateur paramétrique optique permet de convertir la longueur d'onde de 800 nm du système laser vers l'infrarouge moyen. L'accordabilité est totale entre 1.2 et 2.6 $\mu$m et l'impulsion est légèrement allongée par rapport au laser fondamental. \`{A} partir d'une énergie de $\approx$ 8 mJ à 800 nm, on produit $\approx 1 - 1.5$ mJ à la longueur d'onde désirée.
\item à Lund, une modulation acousto-optique met en forme le spectre du laser fondamental et permet de modifier la longueur d'onde centrale entre 780 et 810 nm. Pour permettre cette accordabilité, la largeur de bande spectrale est réduite, ce qui augmente la durée de l'impulsion.
\end{itemize}
Ces deux méthodes sont brièvement décrites dans la suite de ce paragraphe.
\subsection{Amplificateur paramétrique optique}
\label{subsec:OPA}
\paragraph{Principe} L'amplification paramétrique optique (figure \ref{fig:PrincipeOPA}) est un cas particulier de mélange à trois ondes dans un milieu à susceptibilité non linéaire d'ordre deux ($\chi^{(2)}$). Dans les conditions d'accord de phase pour le mélange à trois ondes, lorsque la pompe ($\omega_3$) intense et le signal ($\omega_2$) sont injectés dans le cristal, on montre que les amplitudes du signal et du complémentaire (\textit{idler} en anglais, $\omega_1 = \omega_3 - \omega_2$) augmentent exponentiellement avec la distance de propagation dans le milieu $z$ \mycite{Boyd} (jusqu'à atteindre une saturation lorsque la pompe est significativement déplétée):
\begin{align}
A_1(z) & \propto A_2^*(0) \sinh(gz) \backsim \rme^{gz} \: (z \: \text{grand})\\
A_2(z) & \propto A_2(0) \cosh(gz) \backsim \rme^{gz} \: (z \: \text{grand}) 
\end{align}
où $g$ est une constante positive dépendant des fréquences des trois ondes, des indices du milieu à ces trois fréquences, de la susceptibilité du milieu  et de l'amplitude de la pompe (considérée comme constante). Ainsi, on constate à la fois la production de la nouvelle fréquence $\omega_1$ et l'amplification du signal à la fréquence $\omega_2$.

\begin{figure}
\centering
\def\svgwidth{0.6\columnwidth}
\import{Figures/GHOE/}{PrincipeOPA.pdf_tex}
\caption{Principe de l'amplification paramétrique optique dans un milieu de susceptibilité non linéaire d'ordre deux.}
\label{fig:PrincipeOPA}
\end{figure}

\paragraph{Dispositif optique de conversion de fréquence} \`{A} Saclay et en Ohio, l'amplification paramétrique optique est réalisée grâce à un dispositif commercialisé par \textit{Light Conversion}: HE-TOPAS (\textit{High Energy Tunable Optical Parametric Amplifier System}). Le schéma de l'intérieur du HE-TOPAS est représenté sur la figure \ref{fig:HETOPAS}. Nous décrivons ici brièvement les différentes étapes d'amplification.

Le faisceau de pompe (H1), préalablement mis en forme pour avoir le diamètre, la polarisation et l'énergie requis ($\approx$ 8 mJ), est séparé en deux par une lame séparatrice. 90 à 98\% de l'intensité est réfléchie (H2) et sera utilisée pour la dernière étape d'amplification à haute puissance du signal. 2 à 10 \% est transmise (H3). Une partie de (H3) est transmise par deux autres lames séparatrices pour former le faisceau (3). Le reste de (H3) forme les faisceaux (2) et (4). (3) est focalisé dans une lame de saphir pour produire un continuum de lumière blanche par filamentation (5). Le continuum (5) est superposé à la fraction de pompe (4) en géométrie non-colinéaire dans un cristal non linéaire pour pré-amplifier la longueur d'onde choisie. Cette longueur d'onde est le \textbf{signal} à la fréquence $\omega_2$ que l'on souhaite amplifier au cours du processus ($\lambda_2 = 1.2 - 1.6 \: \mu$m). L'énergie de signal à l'issue de cette étape est de l'ordre de 1 à 3 $\mu$J. Le signal (6) est superposé à l'autre fraction de pompe (2) dans un deuxième cristal non linéaire. Un premier étage d'amplification paramétrique optique se produit dans ce cristal, et l'\textbf{idler} à la fréquence $\omega_1$ est créé ($\lambda_1 = 1.6 - 2.6 \: \mu$m). Après cette étape l'énergie du signal est de l'ordre de 10 à 50 $\mu$J. Le signal de pompe restant est séparé par un miroir dichroïque et est bloqué en bas à gauche du schéma. Le signal et l'idler sont collimatés pour former le faisceau (H4). (H4) et la pompe de haute puissance (H2) sont focalisés dans un dernier cristal non linéaire de grande taille et un second étage d'amplification paramétrique optique a lieu. La pompe (plusieurs mJ restants) est séparée du signal et de l'idler par un miroir dichroïque et est bloquée. Le signal et l'idler sont colinéaires et de polarisation différentes: le signal est polarisé verticalement et l'idler horizontalement. Ils sont ensuite séparés par un miroir dichroïque (non représenté). L'énergie totale signal + idler à l'issue du processus complet est de l'ordre de 2 à 2.5 mJ. La durée de l'impulsion est 1 à 1.5 fois supérieure à la durée du fondamental pour le signal et $< 2$ fois supérieure pour l'idler. La longueur d'onde du signal est choisie en changeant le délai entre la pompe (4) et le supercontinuum (5). Tous les délais entre les impulsions, ainsi que les angles des cristaux (nécessaires à un bon accord de phase dans les processus non linéaires mis en jeu) sont contrôlés par un ordinateur.

\begin{figure}
\centering
\def\svgwidth{\columnwidth}
\import{Figures/GHOE/}{HETOPAS_2.pdf_tex}
\caption{Schéma de l'intérieur du HE-TOPAS. Les différents étages sont décrits dans le texte. Les milieux non linéaires sont entourés en blanc. Il s'agit de saphir pour la génération de continuum de lumière blanche (bleu) et de $\beta$-borate de baryum (BBO) pour les étapes d'amplification (vert). Par souci de clarté tous les miroirs ne sont pas représentés. Seuls sont indiqués les lames séparatrices (tirets) et les miroirs dichroïques (pointillés). Adapté du manuel utilisateur du HE-TOPAS.}
\label{fig:HETOPAS}
\end{figure}


\subsection{Filtre dispersif acousto-optique programmable}
\`{A} Lund, le laser titane saphir est accordable grâce à la présence de filtres dispersifs acousto-optiques programmables en sortie d'oscillateur et à l'intérieur de la cavité de l'amplificateur régénératif. Il s'agit de dispostifs commerciaux développés par \textit{Fastlite} et \textit{Amplitude Technologies}, vendus sous le nom de Dazzler\footnote{Dazzler est le nom d'un personnage de l'univers des comics X-men dont le pouvoir est de transformer le son en lumière.} et Mazzler respectivement. Le principe de fonctionnement détaillé de ces dispositifs est exposé par \mycite{TournoisOptComm1997}\mycite{VerluiseOL2000} (Dazzler) et
\mycite{OksenhendlerAPB2006} (Mazzler). Très brièvement, ces dispositifs utilisent le couplage entre une onde acoustique et l'impulsion laser dans un cristal biréfringeant (TeO$_2$) pour transférer les propriétés de phase et d'amplitude acoustiques à l'onde lumineuse. Dans le cristal, la propagation dépend de la fréquence de l'onde. Il est ainsi possible de façonner le spectre et la phase spectrale (ou bien le profil temporel) de l'impulsion laser. Pour permettre l'accordabilité, il faut réduire la largeur de bande spectrale disponible. Les impulsions sont donc plus longues, comme nous le verrons au chapitre \ref{chap:He_Lund}. Plusieurs exemples de spectres et les impulsions correspondantes sont donnés sur la figure \ref{fig:Mazzler}. En pratique ces dispositifs sont entièrement programmés par ordinateur et il est très facile pour l'expérimentateur de choisir, par exemple, la longueur d'onde centrale désirée.

\begin{figure}
\centering
\def\svgwidth{0.5\columnwidth}
\import{Figures/GHOE/}{Mazzler.pdf_tex}
\caption{Spectre (haut) et intensité temporelle (bas) d'impulsions femtosecondes pour différentes modulations en amplitude et en phase induites par le Mazzler et le Dazzler. Extrait de \mycite{OksenhendlerAPB2006}.}
\label{fig:Mazzler}
\end{figure}

\section{Caractérisation des harmoniques d'ordre élevé}
\label{sec:CaracHHG}
Au paragraphe \ref{sec:HHG}, nous avons indiqué que les harmoniques peuvent être caractérisées en amplitude grâce à un spectromètre de photons. L'amplitude mais également la phase du rayonnement XUV peuvent être mesurées en photoionisant un gaz dans un spectromètre de photoélectrons. Nous présentons ici le principe de fonctionnement d'un spectromètre de photoélectrons à bouteille magnétique ainsi que la technique de mesure de la phase spectrale des harmoniques RABBIT. Ces deux éléments sont au c\oe ur du travail présenté dans ce manuscrit.

\subsection{Spectromètre à temps de vol d'électrons à bouteille magnétique}
\label{subsec:TOF}
Le principe du spectromètre de photoélectrons à bouteille magnétique a été développé par \mycite{KruitJPhysE1983}. Il permet de collecter le signal de photoélectrons avec une grande efficacité et une bonne résolution en énergie. Pour simplifier la discussion nous considérons que le spectre XUV est composé d'une seule énergie. Le rayonnement est focalisé dans un jet de gaz. Si l'énergie des photons est supérieure au potentiel d'ionisation du système, alors un électron est émis. La conservation de l'énergie implique
\begin{equation}
Ec = \hbar \omega - Ip 
\end{equation}
En supposant que les électrons parcourent la distance $L$ entre la zone d'interaction et le détecteur à vitesse constante, la mesure de la durée $t$ entre l'ionisation du système et la détection de l'électron (appelée \textbf{temps de vol}) permet de déterminer leur énergie.
\begin{equation}
Ec = \frac {1}{2}m \frac{L^2}{t^2}
\end{equation}
Ce principe est à la base de la spectroscopie à temps de vol d'électrons. La résolution en énergie relative est en général constante (sauf à faible énergie) 
\begin{equation}
\frac{\Delta Ec}{Ec} \approx \text{cste}
\end{equation}
La zone de meilleure résolution spectrale absolue $\Delta E$ correspond aux électrons de faible énergie cinétique, donc de temps de vol long.

La figure \ref{fig:TOF} représente le schéma du spectromètre de photoélectrons à bouteille magnétique de Saclay, mise à notre disposition par D. Cubaynes et M. Meyer. Pour une description détaillée du principe de ce type de spectromètre, le lecteur pourra consulter \mycite{TheseRoedig} ou \mycite{ChristinaYoutube}. Brièvement, un aimant permanent situé au voisinage du jet dans la zone d'interaction ainsi qu'un solénoïde entourant le tube de temps de vol guident les trajectoires électroniques et permettent d'augmenter l'efficacité de collection\footnote{L'aimant permanent courbe les trajectoires des électrons émis dans la direction opposée au tube à temps de vol pour les ramener vers le détecteur, ce qui permet la détection des électrons émis dans $4 \pi$ sr. Cependant des électrons de même énergie émis dans des directions différentes parcourent une distance différente donc n'arrivent pas en même temps sur le détecteur, ce qui élargit les pics et diminue la résolution spectrale. La résolution spectrale est améliorée si l'aimant est éloigné, mais dans ce cas les électrons émis dans la "mauvaise" direction ne sont pas collectés et l'intensité du signal diminue.}. Pour que ces champs magnétiques soient efficaces, le spectromètre doit être isolé des champs magnétiques extérieurs (en particulier du champ magnétique terrestre): le tube de temps de vol est donc entouré de mu-métal, un isolant magnétique. Les électrons sont détectés par des galettes de micro-canaux reliées à un convertisseur analogique-numérique synchronisé sur l'impulsion laser d'ionisation. L'intensité du signal de photoélectrons mesurée est le produit de l'intensité du rayonnement par la section efficace d'absorption du gaz à l'énergie considérée. Elle permet donc de caractériser l'amplitude des harmoniques si la section efficace est connue, ou bien de déterminer une section efficace si l'intensité harmonique est mesurée avec un photomultiplicateur calibré \mycite{BalcouZPD1995}. Lors de la photoionisation d'un gaz cible par un peigne d'harmoniques, on obtient un peigne de pics de photoélectrons reproduisant le spectre harmonique (multiplié par la section efficace de photoionisation).

\begin{figure}
\centering
\def\svgwidth{\columnwidth}
\import{Figures/GHOE/}{TOF.pdf_tex}
\caption{Schéma de la bouteille magnétique de Saclay et d'une partie de son bâti dessiné par Michel Bougeard au LIDYL. Le gaz est injecté par le dessus du spectromètre par une aiguille de 500 $\mu$m de diamètre. L'aiguille est entourée d'une résistance chauffante et d'un thermocouple destinés à l'utilisation de liquides. Le rayonnement est focalisé à l'extrémité de l'aiguille, où la densité de gaz est la plus importante. Un aimant permanent conique produit un champ magnétique permanent de 1 T dans la zone d'interaction, ce qui permet une détection des électrons émis dans un angle solide de $4 \pi$ sr. La position de l'aimant dans la direction de vol des électrons peut être variée pour améliorer l'efficacité de détection ou la résolution. Les électrons sont guidés dans le tube de temps de vol par le champ magnétique créé par un solénoïde entourant tout le tube (non représenté en entier par souci de clarté). Le courant à l'intérieur du solénoïde est ajustable pour optimiser la détection. Un champ électrique peut être appliqué au tube afin de ralentir ou accélérer les électrons, et ainsi placer une gamme d'énergie donnée dans la zone de meilleure résolution. Le détecteur est composé de deux galettes de micro-canaux et d'un écran de phosphore couplés à un convertisseur analogique-numérique Agilent DP1400 d'une résolution de 500 ps. L'écran de phosphore n'est pas nécessaire à la détection mais permet l'imagerie du faisceau d'électrons.}
\label{fig:TOF}
\end{figure}

La figure \ref{fig:TOF} représente le schéma de la bouteille magnétique de Saclay, mise à notre disposition par D. Cubaynes et M. Meyer. Pour une description détaillée du principe du spectromètre d'électrons à bouteille magnétique, le lecteur pourra consulter \mycite{TheseRoedig} ou \mycite{ChristinaYoutube}.

\subsection{Mesure de la phase spectrale par RABBIT}
\label{subsec:RABBIT}
\paragraph{Principe} Si l'on superpose dans la zone d'interaction du spectromètre un faible champ IR (dit "d'habillage") au rayonnement harmonique, de nouvelles transitions de photoionisation sont possibles. Il s'agit de transitions à deux photons et deux couleurs: absorption d'un photon harmonique et absorption ou émission stimulée d'un photon IR (figure \ref{fig:PrincipeRabbit2}). Les énergies des harmoniques étant séparées de deux fois l'énergie du photon IR, il apparaît de nouvelles composantes dans le spectre de photoélectrons situées entre les harmoniques (figure \ref{fig:JoliRabbit}(b)), appelées pics satellites (ou \textit{sidebands} en anglais). Chaque pic satellite (ordre $q+1$) est formé par l'interférence entre deux chemins quantiques: absorption de l'harmonique $q$ et abosorption d'un photon IR, et absorption de l'harmonique $q+2$ et émission stimulée d'un photon IR. Ainsi la présence du champ IR d'habillage permet de faire "interférer" deux harmoniques voisines.

\begin{figure}[ht]
\centering
\def\svgwidth{0.5\columnwidth}
\import{Figures/GHOE/}{Schema_Principe_Rabbit_simple.pdf_tex}
\caption{Principe du RABBIT: interférences entre transitions à deux photons et deux couleurs. L'absorption de l'harmonique $q$ et d'un photon IR (chemin $a$) conduit à la même énergie que l'absorption de l'harmonique $q+2$ et l'émission stimulée d'un photon IR (chemin $e$).}
\label{fig:PrincipeRabbit2}
\end{figure}

 \mycite{VeniardPRA1996} ont montré que  pour un champ harmonique
\begin{equation}
\mathcal{E}_q(t) = A_q \: \rme^{-iq\omega t + i\phi_q}
\end{equation}
l'intensité du pic satellite $q+1$ est modulée en fonction du délai $\tau$ entre les impulsions XUV et IR selon:
\begin{equation}
S_{q+1}(\tau) = A + B \cos(2\omega \tau + \phi_{q+2} - \phi_q + \Delta \theta_{at})
\label{eq:Rabbit_simple}
\end{equation}
Le terme $\Delta \theta_{at}$ est discuté dans les lignes qui suivent. Si on le suppose connu, la mesure de la phase des oscillations permet de déterminer la différence de phase entre deux harmoniques consécutives. La quantité $t_e (q\omega) = [\phi_{q+2} - \phi_q] / 2\omega$ est le délai de groupe des harmoniques défini au paragraphe \ref{subsec:ChirpAtto}. Cette méthode, baptisée RABBIT (\textit{Reconstruction of Attosecond Beatings by Interference of Two-photon transitions}) \mycite{MullerAPB2002}, a été utilisée expérimentalement pour la première fois par \mycite{PaulScience2001}. L'expérience de Paul \textit{et al.} a permis de démontrer la synthèse de trains d'impulsions attosecondes. Par la suite, les mesures de \mycite{MairesseScience2003} ont démontré le lien entre le délai de groupe des harmoniques et l'instant de recollision (voir paragraphe \ref{subsec:ChirpAtto}). 

Le terme de phase supplémentaire $\Delta \theta_{at}$ est dû au gaz photoionisé et est par conséquent appelé phase atomique. Il s'agit de la différence de phase entre les deux éléments de transition à deux photons mis en jeu. Dans les travaux pionniers cités précédemment, l'objectif était de caractériser la structure temporelle de l'émission harmonique. La phase $\Delta \theta_{at}$ était considérée comme "petite" et calculable "par des théories établies avec une grande précision" \mycite{PaulScience2001}. Si l'on photoionise le gaz loin de toutes résonances, dans un continuum lisse, cette affirmation est vraie. Cependant cette phase se révéla plus tard une observable de grand intérêt pour étudier la photoionisation à l'échelle attoseconde. En particulier, elle peut prendre des valeurs significatives lorsque la photoionisation a lieu au voisinage de résonances. C'est l'objet d'étude des expériences des parties \ref{part:Helium} et \ref{part:Argon}, et l'expression de $\Delta \varphi_{at}$ sera détaillée dans la partie \ref{part:Delais}.

\begin{figure}[ht]
\centering
\def\svgwidth{\columnwidth}
\import{Figures/GHOE/}{SetupRABBIT.pdf_tex}
\caption{Dispositif expérimental typiquement utilisé dans une expérience RABBIT.}
\label{fig:SetupRabbit}
\end{figure}

\paragraph{Dispositif expérimental} Du point de vue expérimental, le dispositif utilisé pour réaliser une expérience de type RABBIT est présenté sur la figure \ref{fig:SetupRabbit}. Le faisceau laser est séparé en deux avec une lame séparatrice. La majeure partie est focalisée par une lentille L$_{gen}$ dans un gaz pour produire des harmoniques d'ordre élevé (paragraphe \ref{sec:HHG}). L'IR de génération est absorbé par un filtre métallique de $\approx$ 200 nm d'épaisseur, généralement en aluminium qui a une bande passante entre 20 et 70 eV \mycite{CXRO}. L'XUV passe au centre d'un miroir troué puis est focalisé par un miroir torique en or dans la zone d'interaction d'un spectromètre de photoélectrons (paragraphe \ref{subsec:TOF}). L'IR transmis par la lame séparatrice parcourt une ligne à retard puis est focalisé par une lentille L$_{hab}$ de manière à ré-imager le foyer de L$_{hab}$ par le miroir torique dans la bouteille magnétique. L'intensité du faisceau d'habillage est ajustée par un iris et/ou un atténuateur (non représenté) de manière à produire uniquement des transitions à deux photons avec l'XUV ($I_{hab} \approx 10^{11}$ W/cm$^2$). Si l'intensité ou la phase de l'IR d'habillage n'est pas homogène sur toute la zone d'interaction du spectromètre, les pics satellites produits à deux endroits distincts du gaz n'oscilleront pas en phase. Ceci brouille les oscillations des pics satellites mesurées. Le diamètre et la position du foyer d'habillage peuvent être ajustés grâce à l'iris et à la translation longitudinale de la lentille L$_{hab}$ pour s'assurer que les conditions d'homogénéité soient remplies. Un spectre de photoélectrons est mesuré pour chaque délai entre l'IR et l'XUV, moyenné sur plusieurs milliers de tirs laser (plusieurs secondes), pour former un spectrogramme.

\begin{figure}
\centering
\def\svgwidth{\columnwidth}
\import{Figures/GHOE/}{JoliRabbit_Ar_800.pdf_tex}
\caption{(a) Spectrogramme RABBIT d'harmoniques générées à 800 nm dans l'argon et détectées dans l'argon. (b) Somme du spectrogramme (a) sur tous les délais, correspond au spectre de photoélectrons lors de la photoionsiation à deux photons et deux couleurs. (c) Somme sur leur largeur spectrale de deux pics satellites (indiqués en (a)) montrant des oscillations à la fréquence 2 $\omega$ déphasées. (d) Phase des oscillations des pics satellites 14 à 26 déterminée par transformée de Fourier.}
\label{fig:JoliRabbit}
\end{figure}

\begin{figure}
\centering
\def\svgwidth{0.6\columnwidth}
\import{Figures/GHOE/}{Chen.pdf_tex}
\caption{Section efficace d'absorption de l'argon entre 10 et 60 eV. Extrait de \mycite{ChenPRA1992}.}
\label{fig:ChenAr}
\end{figure}

La figure \ref{fig:JoliRabbit}(a) montre un spectrogramme typique mesuré avec des harmoniques générées dans l'argon à 800 nm et détectées dans l'argon. Les harmoniques et les pics satellites sont clairement distinguables jusqu'à $q = 26$ ($\approx 40$ eV). Les ordres supérieurs ne sont pas visibles à cause de la faible section efficace de photoionisation de l'argon au-dessus de 40 eV (figure \ref{fig:ChenAr}), qui diminue à la fois le flux d'harmoniques générées et la quantité de photoélectrons détectés. Les harmoniques sont déplétées pour former les pics satellites et oscillent donc en opposition de phase. Le signal des pics satellites est sommé sur leur largeur spectrale ($\approx 400$ meV). L'exemple des pics satellites 16 et 20 est donné sur la figure \ref{fig:JoliRabbit}(c) et montre des oscillations à la fréquence $2 \omega$ (période 1.33 fs) déphasées. La phase des oscillations est extraite pour tous les pics satellites par transformée de Fourier. Le résultat de cet analyse est visible sur la figure \ref{fig:JoliRabbit}(d). Dans cette expérience, l'argon est photoionisé loin de toute résonance donc le terme de phase atomique est faible et varie lentement, et la pente de cette courbe peut directement être identifiée au délai de groupe des harmoniques en négligeant l'influence de la transmission par le filtre en aluminium \mycite{LopezMartensPRL2005}. Dans le cas présenté on trouve $t_e (q\omega) = \Delta \phi / 2\omega = 90 \pm 6$ as. Cette valeur est typique de la GHOE dans l'argon à 800 nm aux intensités de l'ordre de $1 \times 10^{14}$ W/cm$^2$ \mycite{MairesseScience2003}\mycite{TheseMairesse}. \`{A} partir des intensités harmoniques mesurées, de la section efficace de photoionisation (figure \ref{fig:ChenAr}) et des phases mesurées par RABBIT (figure \ref{fig:JoliRabbit}(d)), on reconstruit le train d'impulsions attosecondes présenté en figure \ref{fig:TrainAtto}.

\begin{figure}
\centering
\def\svgwidth{0.5\columnwidth}
\import{Figures/GHOE/}{TrainAtto.pdf_tex}
\caption{Reconstruction du train d'impulsions attosecondes à partir du spectrogramme de la figure \ref{fig:JoliRabbit}. L'instant d'arrivée absolu des impulsions du train, c'est-à-dire la synchronisation des impulsions par rapport aux oscillations du laser générateur est inconnu ici par manque de référence (une telle référence peut être obtenue dans un autre dispositif où le faisceau d'habillage passe dans le milieu non-linéaire et interfère avec le faisceau de génération \mycite{DinuPRL2003}). L'intensité spectrale des harmoniques est ici supposée infiniment fine, ce qui revient à négliger l'enveloppe femtoseconde de l'impulsion.}
\label{fig:TrainAtto}
\end{figure}

\section*{Conclusions de la partie \ref{part:GHOE}}
Dans cette partie, nous avons présenté les principales caractéristiques de la génération d'harmoniques d'ordre élevé et de leur mise en \oe uvre expérimentale. 
\begin{itemize}
\item Lors de l'interaction entre un gaz et une impulsion laser femtoseconde d'intensité $I \approx 10^{14}$ W/cm$^2$, on produit un spectre très large dans le domaine de l'XUV composé des fréquences multiples impaires de la fréquence fondamentale. L'énergie de photon maximale atteinte dépend du potentiel d'ionisation du gaz, de l'intensité et de la longueur d'onde du laser.
\item Au niveau microscopique, le laser intense perturbe le potentiel ressenti par les électrons du système qui sont alors ionisés par effet tunnel et accélérés par le champ laser avant de recombiner sur l'ion parent en émettant l'énergie accumulée sous forme de lumière. Ce processus a lieu tous les demi-cycles laser.
\item Dans le domaine temporel, ceci correspond à l'émission d'un train d'impulsions attosecondes, caractérisé par un chirp attoseconde (différence de phase spectrale linéaire entre harmoniques) et un chirp femtoseconde (variation de phase spectrale à l'intérieur de chaque harmonique). Le chirp attoseconde est intrinsèquement lié au processus d'émission: les différentes fréquences harmoniques ne sont pas émises au même instant, et est responsable de la durée des impulsions dans le train. Le chirp femtoseconde est dû à la variation d'intensité du laser à l'échelle de l'enveloppe, et est responsable de la durée totale du train ainsi que de l'espacement des impulsions dans le train.
\item Il existe deux familles de trajectoires dont on peut isoler expérimentalement les contributions en modifiant l'accord de phase ou bien en les sélectionnant spatio-spectralement.
\end{itemize}

Dans la suite, nous utiliserons les harmoniques d'ordre élevé pour réaliser des expériences de photoionisation résolues temporellement. Dans les parties \ref{part:Helium} et \ref{part:Argon}, la technique RABBIT sera détournée de son utilisation première (mesurer la phase spectrale des harmoniques) afin de déterminer les phases des éléments de matrice de transition lors de la photoionisation résonante de gaz rares. Dans la partie \ref{part:Polarimétrie}, nous utiliserons la photoionisation de molécules de NO par les harmoniques pour mesurer leur état de polarisation.