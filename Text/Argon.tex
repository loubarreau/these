\part{Dynamiques d'autoionisation dans le néon et l'argon}
\label{part:Argon}
%% Mesures RABBIT Argon Saclay et OSU - Simuls Madrid - Discussion de l'influence du gaz de génération
%% Mesures RABBIT Néon + Simuls Madrid
%% Mesures Rainbow Argon. MZurch.m séparation des composantes SO
%% Reconstruction?

% Spectroscopie : \mycite{MaddenPR1969} \mycite{CodlingJphysB1980} différence SO, \mycite{SvenssonJPhysB1987},\mycite{SorensenPRA1994}, \mycite{BerrahJPB1996}, \mycite{ZhangJPhysB2009}
% citer Kotur
% citer Rothardt

\chapter[Mesure de la phase de la transition au voisinage de la résonance $2s2p^63p$ du néon par RABBIT]{Mesure de la phase de la transition au voisinage de la résonance \MakeLowercase{$2s2p^63p$} du néon par RABBIT}
\label{chap:Neon}

\section{Section efficace}
\begin{figure}[ht]
\centering
\def\svgwidth{0.6\textwidth}
\import{Figures/Argon/}{SchulzNe.pdf_tex}
\caption{(a) Rendement de photoionisation mesuré et (b) section efficace de photoionisation du néon calculée par la méthode R-matrix en jauge longueur (trait continu) et en jauge vitesse (pointillés). Extrait de \mycite{SchulzPRA1996}.}
\label{fig:SchulzNe}
\end{figure}

Le néon est le gaz noble le plus léger pour lequel le cation a un caractère polyélectronique. Les dynamiques de photoionisation sont donc plus riches que dans l'hélium, mais les effets relativistes ne sont pas à prendre en compte pour décrire les états autoionisants simplement excités \mycite{SchulzPRA1996}. La figure \ref{fig:SchulzNe} montre la section efficace de photoionisation du néon pour des énergies de photon de 44 à 53 eV. Cette région spectrale est très riche: on observe une série d'états simplement excités du néon $2s2p^{6}np$ ainsi que des séries d'états doublement excités $2s^{2}2p^{4}3snp$, $2s^{2}2p^{4}3pns$ et $2s^{2}2p^{4}3pnd$. 

Dans la suite nous nous intéressons à l'état simplement excité $2s2p^{6}3p$, pour lequel la section efficace est la plus importante dans cette région spectrale. Cet état est couplé par interaction de configuration aux continua $2p^{-1}\epsilon s$ et $2p^{-1}\epsilon d$. Les paramètres de cette résonance de Fano sont données dans le tableau \ref{tab:ParamètresFano}: $E_R = 45.54$ eV; $q = -1.58$; $\Gamma =  16$ meV; $\tau = 41$ fs; $\rho^2 = 0.75$.

\section{Mesure de la phase par interférométrie RABBIT}
Les mesures RABBIT présentées ici ont été effectuées en collaboration avec Antoine Camper, Timothy Gorman et Dietrich Kiesewetter dans le laboratoire de Louis DiMauro et Pierre Agostini à l'université de l'état de l'Ohio. Le spectromètre de photoélectrons à bouteille magnétique utilisé, d'une longueur de 1 m, ne possède pas une résolution suffisante pour le Rainbow RABBIT. Aucune structure dans l'amplitude ou la phase des pics satellites n'a été mesurée. Les mesures de phases ont donc été effectuées en appliquant la technique RABBIT pour différentes longueurs d'onde de génération.

Le dispositif expérimental est similaire à celui présenté sur la figure \ref{fig:SetupRabbit}. La longueur d'onde de génération est modifiée grâce à un amplificateur paramétrique optique (paragraphe \ref{subsec:OPA}). Pour échantillonner suffisamment la résonance, de faible largeur spectrale, on choisit une longueur d'onde de génération au voisinage de 1700 nm ($\hbar \omega \approx 0.73$ eV). Dans ces conditions, l'harmonique 63 est résonante avec la transition vers l'état $2s3p$. La largeur spectrale des harmoniques est $\approx 500$ meV. Les impulsions de génération ont une durée de $\approx 50$ fs, soit une largeur spectrale $\approx 36$ meV et les effets d'impulsions brèves sont négligeables. L'impulsion de l'idler est séparée en deux avec un miroir troué de diamètre 8 mm. Le faisceau réfléchi est focalisé avec une lentille $f = 50$ cm dans un jet de CO$_2$ émis par une vanne pulsée Even-Lavie pour la GHOE. Le faisceau de génération, annulaire en champ lointain, est ensuite bloqué par un iris. L'XUV est refocalisé grâce à un miroir torique en or dans la zone d'interaction du spectromètre de photoélectrons à bouteille magnétique. Le faisceau transmis par le miroir troué est recombiné avec l'XUV sur un second miroir troué de diamètre 6 mm. On mesure les spectres de photoionisation à deux photons et deux couleurs du néon en fonction du délai MIR-XUV. L'intensité de chaque pic satellite est sommée spectralement et on extrait la phase des oscillations par transformée de Fourier (équation \ref{eq:PhaseMoyenneHe}):
\begin{equation}
\bar{\Theta} = \arg_{2 \omega} \left[ \int \rmd \tau \rme^{i \omega \tau} \left( \int_{\text{largeur SB}} S_{\text{SB}}(\tau,E) \rmd E \right) \right]
\label{eq:PhaseMoyenneNe}
\end{equation} 

De manière similaire à la procédure employée au chapitre \ref{chap:HeSaclay_res}, on soustrait le délai de groupe linéaire (relié au chirp atto) du rayonnement harmonique ($\approx 0.04$ rad/eV ou 18 as/eV) pour obtenir la phase due uniquement au processus de photoionisation à deux photons \textit{via} la résonance de Fano. Cette procédure est appliquée pour chaque longueur d'onde de génération accordée autour de 1700 nm afin d'obtenir la variation complète de la phase au voisinage de la résonance. La figure \ref{fig:ResultatsNe} montre les résultats obtenus pour les pics satellites SB$_{62}$ et SB$_{64}$ de part et d'autre de l'harmonique résonante. On observe une variation de $\approx$ 0.2 rad quand l'énergie de l'harmonique varie de 0.6 eV autour de la résonance. On remarque que, contrairement au cas de l'hélium, les deux pics satellites n'ont pas une variation de phase opposée. Cette observation est discutée dans le paragraphe suivant. Pour calibrer ces courbes en énergie, on utilise l'intensité du pic de photoélectrons associé à l'harmonique résonante H$_{63}$ (figure \ref{fig:ResultatsNe}(c)). Pour chaque longueur d'onde, un spectre de photoélectrons sans habillage est mesuré dans les conditions de génération et de détection identiques au RABBIT. Le rapport entre l'intensité de H$_{63}$ et le signal total de photoélectrons est calculé et est tracé en fonction de $63 \times \hbar \omega$. Le maximum de cette courbe permet, par comparaison avec le maximum de la section efficace au voisinage de la résonance en figure \ref{fig:SchulzNe}, une calibration en énergie plus précise que celle donnée par la longueur d'onde centrale de l'amplificateur paramétrique optique. En effet, le faisceau laser se propage dans un milieu partiellement ionisé et la présence d'électrons libres modifie l'indice de réfraction (voir paragraphe \ref{sec:AccordDePhase}) et décale "vers le bleu" (c'est-à-dire vers les courtes longueurs d'onde) le spectre du fondamental. Cette calibration a été appliquée à tous les points expérimentaux de la figure \ref{fig:ResultatsNe}.

\begin{figure}
\centering
\def\svgwidth{\textwidth}
\import{Figures/Argon/}{ResultatsNeon_These.pdf_tex}
\caption{Différence de phase atomique $\Delta \bar{\theta}^{\text{at}}$ mesurée dans les pics satellites SB$_{62}$ (a) et SB$_{64}$ (b), et rapport de l'intensité de photoélectrons dans l'harmonique résonante et dans le spectre complet (c) pour les longueurs d'onde de génération indiquées dans la légende. Les phases sont comparées avec un calcul effectué par Carlos Marante à l'université autonome de Madrid (en noir). Les barres d'erreur sont calculées à partir du rapport signal sur bruit de la transformée de Fourier des oscillations des pics satellites. En (a) et (b), la courbe grise relie simplement les points expérimentaux. En (c), il s'agit d'un ajustement polynomial utilisé pour déterminer le maximum.}
\label{fig:ResultatsNe}
\end{figure}

\section{Discussion}
Les phases $\Delta \bar{\theta}^{\text{at}}$ mesurées sont comparées au résultat d'un calcul effectué dans le groupe de Fernando Mart\'{i}n à Madrid. Le modèle utilisé est identique à celui développé dans l'hélium (partie \ref{part:Helium}, \mycite{JimenezGalanPRL2014}\mycite{JimenezGalanPRA2016}). Les éléments de matrice de transition sont calculés avec le code XCHEM \mycite{MaranteJCTC2017}. Ce code a été préalablement testé en calculant les sections efficaces de photoionisation et les paramètres d'asymétrie $\beta$ au voisinage de la résonance de Fano \mycite{MarantePRA2017}. Les résultats des calculs sont en excellent accord avec les données spectroscopiques. Les amplitudes de transition sont calculées en utilisant les largeurs spectrales de l'XUV et de l'habillage issues de l'expérience. Les intensités des pics satellites sont calculées par interférence des deux chemins à deux photons menant au même état final, et convoluées spectralement avec la fonction d'appareil du spectromètre de photoélectrons (résolution $\approx 200$ meV). Comme il existe plusieurs continua finaux, les intensités sont ensuite sommées de manière incohérente sur tous les canaux d'ionisation. Enfin, la phase des oscillations à $2 \omega$ est déterminée par transformée de Fourier, et ceci est effectué pour toutes les longueurs d'onde d'excitation.

Pour cette résonance et avec cette énergie de photon d'habillage, le paramètre de couplage direct entre la partie liée de la résonance et le continuum final (introduit au chapitre \ref{chap:2photons_et_Fano}) est $\gamma = 0.27$. Dans ces conditions, et contrairement au cas de l'hélium, l'absorption et l'émission du photon MIR dans les transitions à deux photons ne sont pas équivalentes (voir figure \ref{fig:FanoComplexeEffectif}(b)). Ainsi les phases des pics satellites SB$_{62}$ et SB$_{64}$ calculées par le modèle et tracées figure \ref{fig:ResultatsNe}(a-b) ne sont pas opposées l'une de l'autre.

Pour le pic satellite SB$_{62}$, les résultats expérimentaux sont très bien reproduits par le calcul. En revanche, ce n'est pas le cas pour SB$_{64}$: l'accord est bon pour la partie à basse énergie, mais l'expérience est significativement différente de la simulation pour $45.5 < 63 \times \hbar \omega < 46$ eV. Dans ce domaine, l'harmonique H$_{65}$ possède une énergie $46.9 < 65 \times \hbar \omega < 47.4$ eV, elle est donc résonante avec l'état autoionisant $2s2p^{5}4p$ situé à $E_R = 47.12$ eV qui introduit un terme supplémentaire dans la phase du pic satellite SB$_{64}$. La résonance $2s4p$ n'est pas incluse dans la simulation, ce qui expliquerait le désaccord avec les mesures.

Nos mesures de phase fournissent des données expérimentales supplémentaires à comparer aux codes de chimie quantique \mycite{MarantePRA2017}. L'excellent accord entre l'expérience et le calcul pour SB$_{62}$ montre que la méthode XCHEM décrit les corrélations électroniques pour la résonance d'autoionisation $2s3p$ dans le système polyélectronique du néon de manière satisfaisante. Ces résultats donnent confiance dans la capacité de ce code à simuler la photoionisation résonante dans le cas de molécules.

Par ailleurs, comme nous l'avons vu en figure \ref{fig:SchulzNe}, la région autour de 50 eV dans le néon est très riche en états simplement et doublement excités. Par exemple, la résonance $2s2p^{5}5p$ ($E_R = 47.69$ eV) est perturbée par les résonances doublement excitées $2p^{4}3s4p$. Les états $2p^{4}3s4p$ ne sont en outre pas décrits par le couplage LS mais par le couplage $jj$. Les calculs avec la méthode R-matrix de \mycite{SchulzPRA1996} ne reproduisent pas la section efficace expérimentale dans cette région. Une mesure de phase par la méthode Rainbow RABBIT, avec une impulsion d'habillage monochromatique, pourrait donner de nouvelles informations sur ces résonances impliquant des interactions entre de nombreuses configurations. 





\chapter[Mesure de la phase de la transition au voisinage de la résonance $3s3p^64p$ de l'argon par RABBIT et Rainbow RABBIT]{Mesure de la phase de la transition au voisinage de la résonance \MakeLowercase{$3s3p^64p$} de l'argon par RABBIT et Rainbow RABBIT}

\section{Section efficace}
\begin{figure}
\centering
\def\svgwidth{0.85\textwidth}
\import{Figures/Argon/}{Zhang_SpectreAr.pdf_tex}
\caption{Rendement de photoélectrons résolu en spin-orbite au voisinage des résonances $3s^2 3p^6 \rightarrow 3s 3p^6 np \: (n = 4 - 9) $ de l'argon. La ligne pointillée noire est une mesure non résolue en spin-orbite mais avec une meilleure résolution \mycite{BerrahJPB1996}. Extrait de \mycite{ZhangJPhysB2009}.}
\label{fig:Zhang_SpectreAr}
\end{figure}

L'argon est un gaz noble plus lourd que le néon, le couplage spin-orbite y joue donc un rôle plus important. La photoionisation d'un électron de valence $p$ de l'argon produit un ion dans deux états spin-orbite différents $3p^{-1}$ $(^{2}P_{1/2})$ et $3p^{-1}$ $(^{2}P_{3/2})$ séparés de $\Delta E_{SO} = 178$ meV \mycite{Moore1949}. Si la résolution expérimentale est meilleure que cet écart énergétique, l'absorption d'un photon d'énergie $\hbar \Omega > Ip$ produit deux pics de photoélectrons séparés de $\Delta E_{SO}$. Les travaux de \mycite{JordanPRA2017}, publié pendant ma thèse et décrits au chapitre \ref{chap:DelaiPI}, ont montré qu'il existe un délai de photoionisation entre les électrons laissant l'ion dans deux états spin-orbite différents, au moins pour le krypton et le xénon. La phase de la transition à deux photons depuis l'état fondamental vers le continuum n'est donc pas identique pour les deux canaux. La phase de la transition vers une résonance de Fano est elle également différente ? La figure \ref{fig:Zhang_SpectreAr} montre le rendement de photoélectrons en fonction de l'énergie du photon incident pour les deux composantes spin-orbite de Ar$^+$ mesuré grâce au rayonnement synchrotron par Zhang \textit{et al.}. On remarque une série de résonances de Fano simplement excitées $3s3p^{6}np$ entre 26.5 et 29 eV couplées aux deux états de l'ion. De la même manière que pour le néon ces états autoionisants sont couplés aux continua $3p^{-1} \epsilon s$ et $3p^{-1} \epsilon d$. Remarquons ici que la section efficace de photoionisation diminue à l'énergie de la résonance, ce qui est caractéristique d'un paramètre de Fano $|q| < 1$. Ce type de résonance est parfois appelé "résonance fenêtre" ou "anti-résonance".

Dans la suite de ce chapitre nous nous intéresserons à la mesure de la phase de la transition vers la résonance $3s4p$ dont les paramètres sont donnés par le tableau \ref{tab:ParamètresFano}: $E_R = 26.606$ eV; $q = -0.28$; $\Gamma =  80$ meV; $\tau = 8$ fs; $\rho^2 = 0.84$. Deux types d'expériences ont été effectués:
\begin{enumerate}
\item des mesures RABBIT à Saclay et dans l'Ohio en générant les harmoniques à partir d'un laser dans l'infrarouge moyen.
\item des mesures Rainbow RABBIT à Lund en générant les harmoniques à 800 nm et avec un spectromètre de photoélectrons de meilleure résolution. Dans ces conditions les résonances dans les deux états de l'ion ont pu être isolées.
\end{enumerate}

\section{Mesure de la phase par interférométrie RABBIT}
\label{sec:ArRABBIT}
\subsection{Données de la littérature avec la GHOE à 800 nm}
\begin{figure}[ht]
\centering
\def\svgwidth{0.5\textwidth}
\import{Figures/Argon/}{Kotur.pdf_tex}
\caption{(a) Signal de photoélectrons de l'harmonique 17. Les points sont les mesures et la ligne rouge les simulations. (b-c) Variations de phases mesurées (points noirs) par RABBIT dans les pics satellites 16 (b) et 18 (c) lorsque l'énergie de H$_{17}$ est variée au voisinage de la résonance $3s3p^64p$ de l'argon. Les lignes pointillée verte et rouge correspondent à deux simulations prenant en comte ou non l'effet de $\gamma$ (défini au chapitre \ref{chap:2photonsFano}) respectivement. La ligne pointillée rouge en (c) est le symétrique de la ligne rouge de (b), qui est proche de la ligne rouge correspondante en (c) mis à part un décalage en énergie que l'on peut attribuer à l'influence du milieu partiellement ionisé sue la longueur d'onde du laser de génération au cours de sa propagation. Extrait de \mycite{KoturNatComm2016}.}
\label{fig:Kotur}
\end{figure}

La première mesure de phase au voisinage de cette résonance a été effectuée par le groupe d'A. L'Huillier à Lund \mycite{KoturNatComm2016}. Cette expérience est la première à avoir mesuré la phase de la transition au voisinage d'une résonance de Fano en utilisant l'interférométrie RABBIT avec des harmoniques accordables. La longueur d'onde centrale du laser de génération est modifiée avec des filtres acousto-optiques programmables (paragraphe \ref{sec:Accordabilité}). Les harmoniques sont générées dans l'argon. L'analyse des spectrogrammes RABBIT pour chaque longueur d'onde est identique à la méthode présentée aux chapitres \ref{chap:HeSaclay_res} et \ref{chap:Neon}. La variation de phase (que nous notons $\Delta \bar{\theta}^{\text{at}}$) en fonction de l'énergie de l'harmonique résonante mesurée par Kotur \textit{et al.} est présentée en figure \ref{fig:Kotur}. Les auteurs mesurent une variation de phase de $\sim 0.6$ rad pour les deux pics satellites SB$_{16}$ et SB$_{18}$ situés de part et d'autre de la résonance. Les mesures sont en assez bon accord avec les résultats de simulations (lignes rouge et pointillée verte sur la figure \ref{fig:Kotur}). Dans ces simulations, basées sur le modèle de \mycite{JimenezGalanPRA2016}, la largeur spectrale de l'impulsion IR d'habillage (durée $\approx 25$ fs; largeur spectrale $\approx 95$ meV) est prise en compte: les effets d'impulsions brèves ne sont plus ici négligeables.

\subsection{Résultats expérimentaux avec la GHOE dans l'infrarouge moyen}
\begin{figure}[ht]
\centering
\def\svgwidth{1\textwidth}
\import{Figures/Argon/}{Spectres_Argon_Saclay.pdf_tex}
\caption{Spectres de photoélectrons de l'argon ionisé par les harmoniques d'ordre élevé générées dans le krypton pour plusieurs longueurs d'onde de 1245 à 1285 nm (mesures effectuées à Saclay). Les spectres sont décalés verticalement pour une meilleure visibilité. Les pointillés noirs matérialisent la position de la résonance $3s4p$. La résonance fenêtre est à peine visible dans l'intensité du pic de photoélectrons correspondant à H$_{27}$, et disparait complètement dans les pics satellites (amplitude et phase). La largeur des pics est due à: la largeur spectrale des harmoniques, la résolution du spectromètre, et les deux composantes spin-orbite qui ne sont pas résolues dans cette expérience.}
\label{fig:Spectres_Argon_Saclay}
\end{figure}

Sans avoir eu connaissance des résultats de Kotur \textit{et al.}, nous avons effectué la même expérience à Saclay en 2015. Le dispositif expérimental utilisé est similaire à celui présenté sur la figure \ref{fig:SetupRabbit} et décrit au paragraphe \ref{sec:RabbitHeSaclay}. Brièvement, la longueur d'onde de génération est modifiée grâce à un amplificateur paramétrique optique afin que l'harmonique 27 soit résonante avec la transition vers l'état $3s4p$ ($\lambda \approx 1260$ nm). L'impulsion MIR est focalisée dans une cellule de krypton pour la GHOE. Nous avons choisi un gaz de génération qui ne possède pas de résonances dans la région spectrale qui nous intéresse.  La durée de l'impulsion MIR est $\approx 70$ fs, soit une largeur spectrale $\approx 26$ meV. Dans ces conditions, les effets d'impulsions brèves sont négligeables. Un spectrogramme RABBIT est mesuré pour chaque longueur d'onde de génération. Les pics satellites sont intégrés sur leur largeur spectrale et la phase des oscillations est extraite par transformée de Fourier (équation \ref{eq:PhaseMoyenneNe}). En dehors des résonances de Fano, le continuum de l'argon n'est pas "plat", comme le montre la variation de la section efficace (figure \ref{fig:ChenAr},\mycite{ChenPRA1992}). En particulier, la présence du minimum de Cooper \mycite{CooperPR1962} à 48 eV modifie la phase des transitions vers le continuum dans cette région spectrale, comme le montre le calcul de \mycite{MauritssonPRA2005}. Ainsi, nous n'utilisons pas la totalité des pics satellites mesurés pour déterminer le chirp atto (figure \ref{fig:ExtractionPhaseInter}) mais seulement les pics satellites voisins: ici SB$_{24}$ et SB$_{30}$. La différence entre la phase des oscillations des pics satellites et la variation linéaire due au chirp atto est la phase atomique moyennée $\Delta \bar{\theta}^{\text{at}}$.

La figure \ref{fig:Phases_Argon_Saclay} montre les résultats obtenus pour les pics satellites SB$_{26}$ et SB$_{28}$ de part et d'autre de l'harmonique résonante. On observe une variation de $\approx$ 0.5 rad quand l'énergie de l'harmonique varie de 0.4 eV autour de la résonance. Les deux pics satellites ont des variations de phase très symétriques, à la différence des résultats de la figure \ref{fig:Kotur}. Cette observation est discutée dans le paragraphe suivant. Pour calibrer ces courbes en énergie, on utilise l'intensité de l'harmonique résonante H$_{27}$ (figure \ref{fig:Phases_Argon_Saclay}(c)). Pour chaque longueur d'onde, un spectre de photoélectrons sans habillage est mesuré dans les conditions de génération et de détection identiques au RABBIT (figure \ref{fig:Spectres_Argon_Saclay}). Le rapport entre l'intensité de H$_{27}$ et le signal total de photoélectrons est calculé et tracé en fonction de $27 \times \hbar \omega$. La différence entre la position du minimum de cette courbe et du minimum de la section efficace de photoionisation de l'argon au voisinage de la résonance (figure \ref{fig:Zhang_SpectreAr}) donne un offset qui est appliqué à tous les points expérimentaux.

\begin{figure}[ht]
\centering
\def\svgwidth{1\textwidth}
\import{Figures/Argon/}{Fano_Argon_Saclay.pdf_tex}
\caption{Différence de phase atomique $\bar{\Delta \theta}^{\text{at}}$ mesurée dans les pics satellites SB$_{26}$ (a) et SB$_{28}$ (b), et rapport de l'intensité de photoélectrons dans l'harmonique résonante et dans le spectre complet (c) pour les longueurs d'onde de génération indiquées dans la légende. Les phases sont comparées avec un calcul effectué par \'{A}lvaro Jiménez-Gal\'{a}n à l'université autonome de Madrid (en noir). Pour certaines longueurs d'onde, l'expérience a été effectuée plusieurs fois et les différents résultats sont représentés, indiquant la variabilité statistique de nos mesures. Les barres d'erreur sont calculées à partir du rapport signal sur bruit de la transformée de Fourier des oscillations des pics satellites.}
\label{fig:Phases_Argon_Saclay}
\end{figure}

\begin{figure}[ht]
\centering
\def\svgwidth{1\textwidth}
\import{Figures/Argon/}{Fano_Argon_OSU_These.pdf_tex}
\caption{Différence de phase atomique $\bar{\Delta \theta}^{\text{at}}$ mesurée dans les pics satellites SB$_{36}$ (a) et SB$_{38}$ (b), et rapport de l'intensité de photoélectrons dans l'harmonique résonante et dans le spectre complet (c) pour les longueurs d'onde de génération indiquées dans la légende (mesures effectuées en Ohio). Les phases sont comparées avec un calcul effectué par \'{A}lvaro Jiménez-Gal\'{a}n à l'université autonome de Madrid (en noir, identique à la figure \ref{fig:Phases_Argon_Saclay}). Les barres d'erreur sont calculées à partir du rapport signal sur bruit de la transformée de Fourier des oscillations des pics satellites.}
\label{fig:Phases_Argon_OSU}
\end{figure}

La même expérience a été reproduite dans le laboratoire de Louis DiMauro et Pierre Agostini à l'université d'état de l'Ohio, en collaboration avec Antoine Camper, Timothy Gorman et Dietrich Kiesewetter. Les différences avec l'expérience de Saclay sont simplement:
\begin{itemize}
\item la longueur d'onde de génération se situe autour de 1700 nm.
\item la durée de l'impulsion d'habillage est légèrement inférieure ($\approx 50$ fs), mais les effets d'impulsions brèves sont toujours négligeables.
\item la GHOE est effectuée dans le CO$_2$, avec une vanne pulsée Even-Lavie.
\end{itemize}
Les résultats, présentés sur la figure \ref{fig:Phases_Argon_OSU}, sont qualitativement en accord avec les résultats de Saclay. En générant avec une vanne pulsée, le flux de photons XUV est plus faible qu'en générant dans une cellule. Les barres d'erreur sont donc plus importantes que sur la figure \ref{fig:Phases_Argon_Saclay}.

Les phases $\bar{\Delta \theta}^{\text{at}}$ mesurées dans les deux expériences sont comparées à une simulation effectuée par \'{A}lvaro Jiménez-Gal\'{a}n dans le groupe de Fernando Mart\'{i}n à l'université autonome de Madrid. Le modèle utilisé est identique à celui utilisé dans l'hélium et le néon \mycite{JimenezGalanPRA2016}.

\subsection{Discussion}
\begin{figure}[ht]
\centering
\def\svgwidth{1\textwidth}
\import{Figures/Argon/}{Compare_Ar_Saclay_Lund_These2.pdf_tex}
\caption{Comparaison des mesures de $\bar{\Delta \theta}^{\text{at}}$ effectuées à Saclay (points noirs) avec les résultats de Kotur \textit{et al.} (pointillés verts), ainsi que des simulations pour les paramètres expérimentaux correspondants (trait continu noir et vert respectivement). Les deux simualtions prennent en compte le $\gamma$, la courbe continue verte est donc la courbe pointillée verte de la figure \ref{fig:Kotur}.}
\label{fig:Compare_Ar_Saclay_Lund_These}
\end{figure}

La comparaison des résultats obtenus à Saclay avec les données de \mycite{KoturNatComm2016} sur la figure \ref{fig:Compare_Ar_Saclay_Lund_These} fait apparaître plusieurs différences, dues principalement aux conditions expérimentales distinctes. Premièrement, dans notre cas la variation de phase a lieu lorsque l'énergie de l'harmonique résonante varie de 26.4 à 26.8 eV, alors que le saut de phase se produit en $\approx 0.2$ eV dans les résultats de Kotur \textit{et al.}. Comme nous l'avons vu dans la partie \ref{part:GHOE}, le chirp harmonique est plus important dans l'infrarouge moyen qu'à 800 nm. Nos harmoniques sont alors plus larges spectralement que celles de l'expérience de l'équipe de Lund ($\approx$ 500 meV contre 150 meV). L'harmonique recouvre donc partiellement la résonance sur une plus grande gamme spectrale d'excitation, ce qui contribue à étaler la variation de phase spectrale mesurée dans nos conditions. L'évolution de phase est très bien reproduite par les simulations du groupe de Fernando Mart\'{i}n prenant en compte la largeur spectrale de nos harmoniques (trait continu noir figures 
\ref{fig:Phases_Argon_Saclay}(a-b) et \ref{fig:Compare_Ar_Saclay_Lund_These}).

Deuxièmement, les phases que nous avons mesurées sont quasiment opposées pour les pics satellites de part et d'autre de la résonance, contrairement aux mesures de Kotur \textit{et al.}. D'après le chapitre \ref{chap:2photons_et_Fano}, le paramètre $\gamma$ est proportionnel à $\omega$ c'est-à-dire à l'énergie du photon d'habillage. Les phases des deux pics satellites sont donc plus symétriques lorsque le photon d'habillage se situe dans l'infrarouge moyen. Cet effet est bien visible lorsque l'on compare les simulations des deux expériences (traits continus de la figure \ref{fig:Compare_Ar_Saclay_Lund_These}). Notons également que les effets d'impulsions brèves sont négligeables dans nos expériences. 

Enfin, remarquons que dans l'expérience de Saclay, l'énergie de l'harmonique 29 est suffisante pour exciter les résonances $3s5p$ et $3s6p$ de l'argon entre 28 et 28.5 eV. Ceci devrait donner une contribution supplémentaire à la phase du pic satellite SB$_{28}$. Ces résonances ne sont pas incluses dans les simulations, qui sont pourtant en accord remarquable avec nos résultats expérimentaux. Ainsi à la différence des mesures effectuées dans le néon et présentées au chapitre précédent, la présence d'une résonance dans le chemin "supérieur" (H$_{29} - \hbar \omega$) conduisant au pic satellite SB$_{28}$ ne semble pas être visible dans les phases mesurées.

\section{Mesure de la phase par Rainbow RABBIT}
\label{sec:ArRainbow}
\subsection{Dispositif expérimental}
Les résultats présentés dans ce paragraphe ont été obtenus lors de deux campagnes expérimentales menées à Lund en collaboration avec l'équipe du professeur Anne L'Huillier (David Busto, Mathieu Gisselbrecht, Anne Harth, Marcus Isinger, David Kroon, Shiyang Zhong) et en utilisant le spectromètre de photoélectrons prêté par le groupe de Raimund Feifel à l'Université de Gothenburg. Le dispositif expérimental utilisé est identique à celui décrit dans le chapitre \ref{chap:He_Lund}. Brièvement, on utilise un laser titane:saphir dont on modifie la longueur d'onde centrale grâce à des filtres acousto-optiques programmables. Les spectrogrammes RABBIT sont mesurés dans un spectromètre de photoélectrons à bouteille magnétique d'une longueur de 2 m, en appliquant un potentiel retardateur aux électrons des pics satellites d'intérêt pour les amener dans la zone de meilleure résolution spectrale. La largeur spectrale des impulsions d'habillage est de l'ordre de 130 à 190 meV. 

Comme le montre la figure \ref{fig:SpectreAr_Lund_786}(a), ces conditions expérimentales permettent de distinguer les contributions des photoélectrons couplés aux deux états spin-orbite de Ar$^+$. Les pics de photoélectrons correspondant aux harmoniques non résonantes 19 et 21 présentent un épaulement, et celui produit par l'harmonique résonante 17 a une forme non triviale avec un minimum local de chaque côté du maximum. 

\begin{figure}
\centering
\def\svgwidth{1\textwidth}
\import{Figures/Argon/}{SpectreAr_Lund_786.pdf_tex}
\caption{(a) Spectre de photoélectrons mesuré sans habillage lorsque l'harmonique 17 est résonante avec la transition vers l'état autoionisant $3s4p$. (b) Spectre séparé en deux contributions correspondant aux deux états spin-orbite de Ar$^+$ selon la procédure décrite dans le texte. La somme de ces deux contributions (selon l'équation \ref{eq:Rabbit_reconstruit}) est indiquée en jaune en (a) et est quasiment identique au spectre mesuré.}
\label{fig:SpectreAr_Lund_786}
\end{figure}

\subsection{Séparation des deux composantes spin-orbite}
\label{subsec:SeparationSO}
Dans l'argon, $\Delta E_{SO}$ n'est pas suffisamment important pour séparer complètement les deux "familles" de pics de photoélectrons comme dans \mycite{JordanPRA2017}. Seul un épaulement est visible. Cependant, il est possible d'extraire le signal provenant d'une état de l'ion en appliquant la procédure suivante \mycite{ZurchNatComm2017}: le signal total est considéré comme la somme incohérente des intensités de photoélectrons des deux canaux (le spectromètre d'électrons à bouteille magnétique intègre le signal sur toute la distribution angulaire)
\begin{equation}
S_{\text{tot}} (E) = S_{1/2}(E) + S_{3/2}(E)
\end{equation} 
On considère que les spectres des deux canaux sont \textbf{identiques}, séparés de $\Delta E_{SO}$ et que leur intensité relative est égale à la \textbf{dégénérescence} des états de Ar$^+$ $3p^{-1}$ $(^{2}P_{1/2})$ et $3p^{-1}$ $(^{2}P_{3/2})$, soit 1 et 2 respectivement. L'équation précédente devient
\begin{equation}
S_{\text{tot}} (E) = S_{1/2}(E) + 2 S_{1/2}(E - \Delta E_{SO})
\label{eq:SeparationSO_energie}
\end{equation}
Un décalage en énergie s'exprime simplement dans le domaine de Fourier. Soit $\xi$ la variable conjuguée de E. La transformée de Fourier de l'expression \ref{eq:SeparationSO_energie} s'écrit:
\begin{equation}
\tilde{S}_{\text{tot}}(\xi) = (1 + 2 \rme^{-i\xi \Delta E_{SO}} ) \tilde{S}_{1/2}(\xi)
\end{equation}
Ainsi, en divisant la transformée de Fourier du signal total par le facteur de phase $(1 + 2 \rme^{-i\xi \Delta E_{SO}})$, on obtient la transformée de Fourier de la contribution d'un seul canal. Par transformée de Fourier inverse on obtient le spectre de photoélectrons pour un seul canal. Le spectre de l'autre canal s'obtient simplement en multipliant le résultat par la dégénérescence et en appliquant le décalage en énergie. La calibration en énergie du spectre de photoélectrons pouvant être imparfaite, la valeur de $\Delta E_{SO}$ est optimisée par un algorithme minimisant la différence entre le spectre expérimental et le spectre calculé à partir de $S_{1/2, rec}(E) = TF^{-1}[\tilde{S}_{1/2}(\xi)]$:
\begin{equation}
S_{\text{tot, rec}} (E) = S_{1/2, rec}(E) + 2 S_{1/2, rec}(E - \Delta E_{SO,\text{opt}})
\label{eq:Rabbit_reconstruit}
\end{equation}
La valeur trouvée, $\Delta E_{SO,\text{opt}} = 180$ meV, est très proche de la valeur de la littérature. Le résultat de cette procédure appliquée au spectre de la figure \ref{fig:SpectreAr_Lund_786}(a) est visible sur la figure \ref{fig:SpectreAr_Lund_786}(b). Dans les spectres rouge et bleu de la figure \ref{fig:SpectreAr_Lund_786}, les intensités correspondant aux harmoniques non résonantes n'ont plus d'épaulement, et on distingue bien la résonance fenêtre dans le signal de l'harmonique résonante 17.

Remarquons que \mycite{CantonRoganPRL2000} ont mesuré le rapport des sections efficaces entre les deux canaux spin-orbite et déterminent $\sigma_{3/2}/\sigma_{1/2} \approx 1.9$ entre 26.4 et 29 eV, différent du rapport des dégénérescences égal à 2. De plus ce rapport varie beaucoup au voisinage des résonances de Fano \mycite{CaldwellMolPhys2000}: autour de la résonance $3s4p$ ses valeurs sont comprises entre 1.7 et 2.2. Lorsque ce paramètre est choisi comme ajustable, la convergence de l'algorithme et l'accord avec les spectres mesurés sont moins bons. Cet effet n'est donc pas pris en compte dans notre procédure de séparation des composantes spin-orbite, qui suppose $\sigma_{3/2}/\sigma_{1/2} = 2$. 

\subsection{Résultats}
On mesure un spectrogramme lorsque H$_{17}$ est résonante avec la transition vers l'état autoionisant $3s4p$ et on analyse l'amplitude et la phase à l'intérieur des pics satellites par la méthode Rainbow RABBIT décrite au chapitre \ref{chap:HeSaclay_res}. Chaque pic satellite est la somme incohérente de pics satellites produits par l'habillage de photoélectrons issus de Ar$^+$ $3p^{-1}$ $(^{2}P_{1/2})$ et $3p^{-1}$ $(^{2}P_{3/2})$. Ainsi dans l'analyse Rainbow RABBIT des pics satellites on retrouve une amplitude "à trois bosses" similaire à celle de H$_{17}$ figure \ref{fig:SpectreAr_Lund_786}(a), et on distingue deux contributions à la phase. Ces résultats sont visibles en violet sur la figure \ref{fig:Pha_Amp_SO_Argon}.

\begin{figure}[ht]
\centering
\def\svgwidth{1\textwidth}
\import{Figures/Argon/}{Pha_Amp_SO_Argon.pdf_tex}
\caption{Amplitude (a-b) et phase (c-d) des oscillations à $2\omega$ des pics satellites 16 et 18 extraites par Rainbow RABBIT du spectrogramme mesuré (violet) et du spectrogramme reconstruit par l'équation \ref{eq:Rabbit_reconstruit} après la séparation des deux canaux spin-orbite (jaune). L'analyse du spectrogramme réduit à la seule composante $(^{2}P_{1/2})$ est représentée en rouge.}
\label{fig:Pha_Amp_SO_Argon}
\end{figure}

Afin d'isoler la contribution d'un seul canal, la procédure décrite au paragraphe \ref{subsec:SeparationSO} est appliquée à chaque spectre composant le spectrogramme. On reconstruit ensuite le spectrogramme correspondant à un seul canal. L'amplitude et la phase des pics satellites sont analysées par la méthode Rainbow RABBIT. Comme le montrent les courbes rouges de la figure \ref{fig:Pha_Amp_SO_Argon} correspondant à la seule composante $(^{2}P_{1/2})$, l'amplitude présente désormais la résonance fenêtre et un unique saut de phase est mesuré. On observe un saut de phase de $\approx 0.75$ rad au voisinage du minimum d'amplitude, autour de 25.1 et 28.3 eV dans les pics satellites 16 et 18 respectivement. De manière surprenante, les variations de phase sont très similaires (et non symétriques) dans les deux pics satellites. Les phases sont également différentes des phases mesurées par la technique RABBIT en changeant la longueur d'onde de génération mesurées à Lund \mycite{KoturNatComm2016} et à Saclay. Nous avons déjà observé cet effet dans l'hélium au chapitre \ref{chap:HeSaclay_res}: la phase mesurée à chaque longueur d'onde en sommant spectralement le signal d'un pic satellite s'apparente à une moyenne, pondérée par l'intensité du signal, de la phase à l'intérieur du pic satellite. De plus dans le cas de l'argon, les mesures présentées au paragraphe \ref{sec:ArRABBIT} sont moyennées sur les deux états spin-orbite. Des simulations sont en cours dans le groupe de Fernando Mart\'{i}n pour calculer la phase au voisinage de la résonance fenêtre $3s4p$ mesurée par Rainbow RABBIT dans un seul canal afin de comparer aux résultats expérimentaux de la figure \ref{fig:Pha_Amp_SO_Argon}.

Un spectrogramme est recalculé en appliquant la relation \ref{eq:Rabbit_reconstruit} à chaque spectre. L'analyse Rainbow RABBIT de ce spectrogramme est visible en jaune sur la figure \ref{fig:Pha_Amp_SO_Argon}. De la même manière que pour les spectres sans habillage, l'accord avec l'analyse du spectrogramme mesuré est très bon, en particulier sur la partie à haute énergie du pic satellite. L'accord est meilleur sur le pic satellite 18. En effet le pic satellite 16 est détecté sur le spectromètre à basse énergie cinétique, ce qui assure une meilleure résolution spectrale mais également une plus grande sensibilité au bruit électronique du détecteur (visible sur la partie entre 1 et 1.5 eV du spectre mesuré représenté figure \ref{fig:SpectreAr_Lund_786}(a) par exemple). Dans la suite, on utilise les données séparées en spin-orbite du pic satellite 18, noté SB$_{18,1/2}$.

De manière similaire aux études présentées dans la partie \ref{part:Helium}, il est possible d'obtenir le paquet d'onde électronique dans le domaine temporel par transformée de Fourier. On applique la procédure décrite au chapitre \ref{chap:HeSaclay_reconstruction} avec 
\begin{align}
\left| M^{17+1,1/2}(E)\right| & = \frac{A_{18,1/2}(E)}{\sqrt{2 \: A_{20,1/2}(E + 2 \hbar \omega)}} \\ 
\Theta_{17+1,1/2} (E) & \approx - \theta^{\text{at}}_{17+1,1/2}(E)
\end{align}
et
\begin{equation}
\tilde{M}^{\text{res},(2)}(t) = \frac{1}{2 \pi} \int_{- \infty}^{+ \infty} \left| M^{17+1,1/2}(E) \right| \rme^{i \: \Theta_{17+1,1/2} (E)} \times \rme^{-i E t / \hbar} \: \rmd E
\end{equation}
Remarquons ici qu'à cause des effets d'impulsions brèves (dus à la largeur spectrale de l'impulsion d'habillage de Lund et discutés au chapitre \ref{chap:He_Lund}), le paquet d'ondes reconstruit correspond au paquet d'onde produit par l'ionisation à deux photons (qui n'est pas une simple réplique du paquet à un photon). L'intensité et la phase temporelles du paquet d'ondes à deux photons sont représentées sur la figure \ref{fig:Reconstruction_Ar_SO}(a). On observe une composante gaussienne centrée en $t=0$ puis une oscillation de l'intensité associée à un saut de phase de $\approx 3$ rad caractéristique de l'interférence entre l'ionisation directe et l'ionisation \textit{via} l'état autoionisant.

\begin{figure}[ht]
\centering
\def\svgwidth{0.7\textwidth}
\import{Figures/Argon/}{Reconstruction_Ar_SO.pdf_tex}
\caption{(a) Profil temporel (trait continu) et phase (pointillés) du paquet d'onde électronique obtenu à partir de la transformée de Fourier du pic satellite SB$_{18}$ après la séparation (i.e. SB$_{18}$ $(^{2}P_{1/2})$; données en rouge figure \ref{fig:Pha_Amp_SO_Argon}). Pour l'amplitude, la normalisation par le pic satellite non résonant décrite au chapitre \ref{chap:HeSaclay_reconstruction} est utilisée. (b) Spectre de photoélectrons tracé en fonction de la borne supérieure d'intégration de la transformée de Fourier inverse.}
\label{fig:Reconstruction_Ar_SO}
\end{figure}

La transformée de Fourier locale $W(E,t_{acc})$, définie par l'équation \ref{eq:Wicken}, est également calculée. La construction de la résonance au cours du temps, $|W(E,t_{acc})|^2$, est visible sur la figure \ref{fig:Reconstruction_Ar_SO}(b). Le spectre reproduit d'abord le spectre de l'excitation, puis on observe l'apparition de la résonance fenêtre pour une énergie de photon $\approx 28.33$ eV à partir de $t_{acc} \approx 15$ fs.

\subsection{Conclusions}
En conclusion, nous avons ici montré que la méthode Rainbow RABBIT s'applique également aux résonances fenêtres. Nous avons pu distinguer, grâce à la bonne résolution spectrale, deux contributions à la phase des pics satellites résonants, provenant de la somme incohérente des contributions de deux états de spin-orbite différent séparés de seulement 180 meV. Nous avons séparé numériquement les deux contributions et obtenu l'amplitude et la phase spectrales du paquet d'ondes électronique résonant d'un seul canal, puis nous avons reconstruit la dynamique du paquet d'ondes résonant et la construction du profil de la résonance au cours du temps. Intrinsèquement, la méthode numérique de séparation des canaux conduit à des dynamiques identiques pour les deux états spin-orbite. Le bon accord entre les amplitudes et phases mesurées et reconstruites par cette technique semble indiquer que les différences de dynamiques entre les deux états sont faibles. Une analyse attentive des données spectroscopiques de la figure \ref{fig:Zhang_SpectreAr} indique par exemple des valeurs de $q$ légèrement différentes dans les deux voies. Pour accéder à ces différences, plusieurs directions sont possibles: d'une part, théoriquement, développer une analyse plus précise des deux composantes qui se mélangent dans le spectrogramme, voire comparer directement avec des simulations avancées prenant les deux en compte; d'autre part, expérimentalement, utiliser des harmoniques spectralement plus étroites que $\Delta E_{SO}$ pour bien séparer les photoélectrons des deux canaux. Une différence de phase entre les deux voies pourrait également apparaître comme un changement du rapport des intensités des deux canaux ($\sigma_{3/2}/\sigma_{1/2}$) dans le spectrogramme. La procédure actuelle de séparation des contributions n'est pas sensible à un changement de ce rapport, mais pourrait être améliorée pour tenir compte de cet effet.

\section*{Conclusions de la partie \ref{part:Argon}}
Dans cette partie, nous avons étudié la phase des transitions vers des états autoionisants dans le néon et l'argon. \`{A} la différence des états de l'hélium étudiés dans la partie \ref{part:Helium}, les résonances étudiées sont couplées à deux continua de symétrie différente $s$ et $d$, et l'ion résultant peut être produit dans deux états de spin-orbite différent. Ce deuxième effet est plus visible dans l'argon, de numéro atomique plus élevé.

Dans le néon, nous avons effectué la première mesure de phase de la transition vers l'état $2s3p$ en accordant le laser de génération. Nous avons pu comparer les résultats avec des calculs théoriques utilisant des outils de chimie quantique récemment développées. Dans l'argon, nous avons d'abord reproduit les mesures de la phase de la transition vers la résonance $3s4p$ en accordant le laser de génération effectuées par \mycite{KoturNatComm2016}. Nous avons mis en évidence des différences entre nos résultats et ceux de la littérature que nous avons attribué aux conditions expérimentales distinctes. Par la suite, grâce à une meilleure résolution spectrale, nous avons mesuré cette phase grâce à la méthode Rainbow RABBIT. Avec cette méthode, nous avons distingué deux contributions provenant d'états de spin-orbite différent. Numériquement, il a été possible d'isoler une unique contribution afin de reconstruire la dynamique du paquet d'ondes électronique résonant d'un seul canal et la construction du profil de la résonance au cours du temps.

Les expériences que nous avons effectuées n'ont pas encore permis d'observer les différences de dynamiques d'autoionisation vers les deux continua, ni dans les deux canaux spin-orbite. La distribution angulaire des photoélectrons est différente selon la symétrie des états impliqués. Ainsi pour observer des différences entre les transitions vers les continua $s$ et $d$, il faudrait résoudre angulairement les oscillations des pics satellites \mycite{HockettJPhysB2017}. C'est-à-dire faire une expérience de type RABBIT en détectant les électrons dans un spectromètre imageur de vitesse \mycite{VilleneuveScience2017} ou un COLTRIMS.

Pour observer des différences entre canaux spin-orbite, il faudrait étudier une résonance d'autoionisation dans un gaz de numéro atomique plus élevé pour lequel le couplage spin-orbite est plus important (Kr $\Delta E_{SO} = 0.67$ eV ou Xe $\Delta E_{SO} = 1.31$ eV). L'analyse RABBIT est plus délicate lorsque les harmoniques et les pics satellites se recouvrent partiellement \mycite{JordanPRA2017}, mais l'utilisation de la méthode Rainbow RABBIT permettrait de résoudre spectralement les contributions des harmoniques et des pics satellites (comme utilisé par \mycite{IsingerArXiv2017}). Par ailleurs, les différences attendues sont plus importantes si le couplage spin-orbite joue un plus grand rôle. En spectroscopie statique, \mycite{CodlingJphysB1980} ont mesuré des paramètres d'asymétrie différents pour les deux composantes spin-orbite du xénon au voisinage de la résonance $5s6p$. Enfin, remarquons que les effets du couplage spin-orbite sont plus importants dans les états doublement excités proches des seuils, comme l'ont remarqué \mycite{SchulzPRA1996}\mycite{CantonRoganPRL2000} et \mycite{CaldwellMolPhys2000}.


%% Conclusion, perspectives: Résolution angulaire, beta qui varie autour de résonances.  Doughty, B., Haber, L. H., Hackett, C. & Leone, S. R. Photoelectron angular distributions from autoionizing 4s14p66p1 states in atomic krypton probed with femtosecond time resolution. J. Chem. Phys. 134, 094307 (2011). Angularly resolved Rabbit Hockett JPhysB + Science Villeneuve

% Différence entre les SO plus visible dans les atomes plus lourds (Xe Codling JPhysB 1980, sans doute d'autres données depuis)
% "Mirroring resonances", high level of excitation of the residual ion, doubly excited states near highly excited thresholds (Liu Starace PRA 1999). Exp CantonRogan PRL 2000 et Caldwell Mol Phys 2000, comportement différent pour les deux SO dans les mirroring resonances, breakdown of the LS coupling. 