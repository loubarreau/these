\part{Dynamiques d'autoionisation dans l'argon et le néon}
\label{part:Argon}
%% Mesures RABBIT Argon Saclay et OSU - Simuls Madrid - Discussion de l'influence du gaz de génération
%% Mesures RABBIT Néon + Simuls Madrid
%% Mesures Rainbow Argon. MZurch.m séparation des composantes SO
%% Reconstruction?
\chapter*{Introduction}
% Spectroscopie : \mycite{MaddenPR1969} \mycite{CodlingJphysB1980} différence SO, \mycite{SvenssonJPhysB1987},\mycite{SorensenPRA1994}, \mycite{BerrahJPB1996}, \mycite{ZhangJPhysB2009}
% citer Kotur
% citer Rothardt

\begin{figure}
\centering
\def\svgwidth{0.85\textwidth}
\import{Figures/Argon/}{Zhang_SpectreAr.pdf_tex}
\caption{Rendement de photoélectron résolu en spin-orbite au voisinage des résonances $3s^2 3p^6 \rightarrow 3s 3p^6 np \: (n = 4 - 9) $ de l'argon. La ligne pointillée noire est une mesure non résolue en spin-orbite mais avec une meilleure résolution \mycite{BerrahJPB1996}. Extrait de \mycite{ZhangJPhysB2009}.}
\label{fig:Zhang_SpectreAr}
\end{figure}

\chapter[Mesure de la phase de la transition au voisinage de la résonance $3s3p^64p$ de l'argon par RABBIT]{Mesure de la phase de la transition au voisinage de la résonance \MakeLowercase{$3s3p^64p$} de l'argon par RABBIT}

\begin{figure}
\centering
\def\svgwidth{0.5\textwidth}
\import{Figures/Argon/}{Kotur.pdf_tex}
\caption{Signal de photoélectrons de l'harmonique 17 (a), variations de phases mesurées par RABBIT dans les pics satellites 16 (b) et 18 (c) lorsque l'énergie de H$_{17}$ est variée au voisinage de la résonance $3s3p^64p$ de l'argon. Extrait de \mycite{KoturNatComm2016}.}
\label{fig:Kotur}
\end{figure}

\begin{figure}
\centering
\def\svgwidth{1\textwidth}
\import{Figures/Argon/}{Spectres_Argon_Saclay.pdf_tex}
\caption{fez}
\label{fig:Spectres_Argon_Saclay}
\end{figure}

\begin{figure}
\centering
\def\svgwidth{1\textwidth}
\import{Figures/Argon/}{Fano_Argon_Saclay.pdf_tex}
\caption{fez}
\label{fig:Phases_Argon_Saclay}
\end{figure}

\begin{figure}
\centering
\def\svgwidth{1\textwidth}
\import{Figures/Argon/}{Fano_Argon_OSU_These.pdf_tex}
\caption{fez}
\label{fig:Phases_Argon_OSU}
\end{figure}


%\chapter[Mesure de la phase de la transition au voisinage de la résonance $2s2p^63p$ du néon par RABBIT]{Mesure de la phase de la transition au voisinage de la résonance \MakeLowercase{$2s2p^63p$} du néon par RABBIT}

%\chapter[Mesure résolue en spin-orbite de la phase de la transition au voisinage de la résonance $3s3p^64p$ de l'argon par Rainbow RABBIT]{Mesure résolue en spin-orbite de la phase de la transition au voisinage de la résonance \MakeLowercase{$3s3p^64p$} de l'argon par Rainbow RABBIT}


%% Conclusion, perspectives: Résolution angulaire, beta qui varie autour de résonances.  Doughty, B., Haber, L. H., Hackett, C. & Leone, S. R. Photoelectron angular distributions from autoionizing 4s14p66p1 states in atomic krypton probed with femtosecond time resolution. J. Chem. Phys. 134, 094307 (2011). Angularly resolved Rabbit Hockett JPhysB + Science Villeneuve

% Différence entre les SO plus visible dans les atomes plus lourds (Xe Codling JPhysB 1980, sans doute d'autres données depuis)
% "Mirroring resonances", high level of excitation of the residual ion, doubly excited states near highly excited thresholds (Liu Starace PRA 1999). Exp CantonRogan PRL 2000 et Caldwell Mol Phys 2000, comportement différent pour les deux SO dans les mirroring resonances, breakdown of the LS coupling. 