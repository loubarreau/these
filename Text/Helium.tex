\part{Dynamiques d'autoionisation dans l'hélium}
%% Accordabilité de l'OPA permet d'accéder aux résonances... RABBIT normal, résultats, comparaison avec les calculs de Madrid. Ok mais variation de phase petite qui correspond plus à une "moyenne" de la phase (thèse de Vincent)
% Le Rainbow RABBIT. Méthode, résultats, paragraphe 6 du SM. 
%% Reconstruction du paquet d'onde dans le domaine temporel. Approximations, résultats, comparaison avec le calcul de Madrid et avec le POE non résonant.
% Construction de la résonance au cours du temps. Wickenhauserisation
%% Lund. Résultats expérimentaux pour la 2s2p et la sp3+ (meilleure résolution spectromètre = on résoud la sp3+ youpi), comparaison avec Alvaro, déconvolution et effets de l'IR d'habillage.
% Reconstruction de sp2+ toute seule et des deux résonances, autres représentations, POE à deux résonances
% Influence de l'intensité d'habillage sur q
%% Comparaison avec l'absorption transitoire. Gruson vs Kaldun.
Cette partie s'intéresse à la dynamique d'autoionisation de l'hélium \textit{via} les états se trouvant sous le seuil $N=2$ de He$^+$. Comme nous l'avons vu précédemment, ces états sont étudiés depuis les débuts de la spectroscopie(\mycite{FanoPR1961},\mycite{MaddenCodling1965}, figure \ref{fig:Beutler}(b)). Ils possèdent la particularité d'être des états doublement excités, ce qui fait de l'hélium le système le plus simple pour étudier les corrélations électroniques. Par ailleurs, l'hélium est modélisable \textit{ab initio}, ce qui rend l'expérience facilement confrontable à la théorie.

Dans cette partie, nous présenterons d'abord les résultats des mesures de phase de la transition vers la résonance $2s2p$ effectuées à Saclay par interférométrie RABBIT puis par une nouvelle méthode de RABBIT résolue spectralement ("Rainbow RABBIT"). Ensuite, nous montrerons comment les mesures de Rainbow RABBIT permettent d'obtenir toute la dynamique d'autoionisation ainsi que la construction du profil de la résonance au cours du temps. Dans un troisième chapitre, nous présenterons les résultats que nous avons obtenus en exportant la technique Rainbow RABBIT dans l'équipe d'Anne L'Huillier à Lund, en particulier l'étude d'une seconde résonance d'autoionisation ($sp3^+$) et de l'influence de l'intensité d'habillage sur le profil de raie. Enfin, nous comparerons le Rainbow RABBIT et l'absorption transitoire attoseconde pour l'étude des dynamiques d'autoionisation.

\begin{figure}[h]
\centering
\def\svgwidth{0.5\textwidth}
\import{Figures/Helium/}{Domke.pdf_tex}
\caption{Section efficace d'absorption de l'hélium entre 60 et 65 eV correspondant aux résonances doublement excitées $^1P^o$ convergeant vers le seuil d'ionisation $N=2$ de He$^+$ à 65.40 eV. Cette partie s'intéresse à la dynamique d'ionisation au voisinage des résonances $2s2p$ et $sp3^+$, dont les caractéristiques spectroscopiques sont rappelées dans le tableau \ref{tab:ParamètresFano}. Adapté de \mycite{DomkePRA1996}.}
\label{fig:Domke}
\end{figure}


\chapter[Mesure de la phase de la transition au voisinage de la résonance 2s2p par RABBIT et Rainbow RABBIT]{Mesure de la phase de la transition au voisinage de la résonance \MakeLowercase{2s2p} par RABBIT et Rainbow RABBIT}
\label{chap:HeSaclay_res}
\section{Mesures RABBIT}
L'amplificateur paramétrique optique (\textit{Optical Parametric Amplifier}, OPA) installé sur la ligne PLFA \mycite{WeberRSI2015} permet d'accorder la longueur d'onde de génération entre 1200 et 2000 nm (moyen infra-rouge, MIR), et de choisir ainsi l'énergie des harmoniques d'ordre élevé. Ainsi, en variant la longueur d'onde du laser entre 1285 et 1305 nm, l'harmonique 63 balaye le voisinage de la résonance autoionisante $2s2p$ de l'hélium. 

\begin{figure} [h]
\centering
\def\svgwidth{\textwidth}
\import{Figures/Helium/}{SpectresSansHabillage.pdf_tex}
\caption{Spectres de photoélectrons de l'hélium ionisé par les harmoniques 61 à 67 générées dans l'argon par un amplificateur paramétrique optique (OPA) de longueur d'onde variant de 1285 à 1305 nm. Les spectres sont décalés verticalement pour une meilleure visibilité. Les pointillés noirs matérialisent la position de la résonance $2s2p$.}
\label{fig:SpectresSansHabillageHe}
\end{figure}

L'impulsion MIR, d'une durée de $\approx$ 70 fs, est focalisée avec une lentille $f$ = 400 mm dans une cellule d'argon pour générer les harmoniques d'ordre élevé. L'XUV est refocalisé dans un spectromètre à temps de vol d'électrons à bouteille magnétique grâce à un miroir torique en or pour photoioniser un gaz d'hélium. Les spectres de photoélectrons obtenus pour $\lambda_{\text{OPA}}$ variant de 1285 à 1305 nm sont représentés sur la figure \ref{fig:SpectresSansHabillageHe}. Le signal de photoélectrons reproduit le spectre harmonique. Un signal plus intense, dû à la section efficace de photoionisation plus importante au voisinage de la résonance \mycite{DomkePRA1996}, est identifié vers 60 eV. En changeant la longueur d'onde de génération on modifie la position de l'harmonique 63 par rapport à la résonance. 

\begin{figure}[h]
\centering
\def\svgwidth{0.45\textwidth}
\import{Figures/Helium/}{Schema_2s2p.pdf_tex}
\caption{Principe de l'interférométrie RABBIT résonante. Au voisinage de la résonance $2s2p$ de l'hélium, les pics satellites sont formés par l'interférence entre un chemin quantique résonant et un chemin non résonant servant alors de référence.}
\label{fig:Schema_2s2p}
\end{figure}

Une partie du faisceau MIR initial est superposée à l'XUV au point source de la bouteille magnétique. Un spectrogramme RABBIT est enregistré pour chaque longueur d'onde du fondamental (figure \ref{fig:TracesExp_Sim_He}). De part et d'autre de la résonance, les pics satellites sont formés par l'interférence entre un chemin résonant impliquant l'harmonique 63 $\pm$ 1 photon MIR, et un chemin non résonant qui sert alors de référence (figure \ref{fig:Schema_2s2p}). On rappelle ici l'expression du signal du pic satellite en fonction du délai entre les impulsions de génération et d'habillage $\tau$, en présence d'une résonance de Fano intermédiaire (paragraphe \ref{subsec:PhaseRabbit} et chapitre \ref{chap:2photons_et_Fano}):
\begin{multline}
S_{\text{SB}}(\tau,q) \propto |M^{a}|^2 + |M^{e}|^2 + 2 |M^{a}||M^{e}| \\
\times \cos[2 \omega \tau + \phi_{\Omega_{q+2}} - \phi_{\Omega_{q}} + \underbrace{\eta_{\lambda}(\kappa_{q+2}) - \eta_{\lambda}(\kappa_{q}) + \phi_{cc}(\kappa_{q+2}) - \phi_{cc}(\kappa_q) \pm \arg \mathcal{R}_{\text{eff}}}_{\Delta \theta^{\text{at}}_q}]
\label{eq:SB_He}
\end{multline}
avec $+ \arg \mathcal{R}_{\text{eff}}$ si la résonance de Fano se trouve dans le chemin correspondant à l'émission du photon MIR ou $- \arg \mathcal{R}_{\text{eff}}$ si elle se trouve dans le chemin absorbant un photon MIR. $\Delta \theta^{\text{at}}_q$ correspond à la différence de phase entre les deux éléments de transition à deux photons ("phase atomique"). En sommant spectralement l'intensité de chaque pic satellite, on obtient un signal oscillant à deux fois la fréquence fondamentale ($2 \omega$), dont on extrait la phase par transformée de Fourier. La quantité mesurée est donc
\begin{equation}
\bar{\Theta} = \arg_{2 \omega} \left[ \int \rmd \tau \rme^{i \omega \tau} \left( \int_{\text{largeur SB}} S_{\text{SB}}(\tau,E) \rmd E \right) \right]
\label{eq:PhaseMoyenneHe}
\end{equation} 

\begin{figure}[!ht]
\centering
\def\svgwidth{0.70\textwidth}
\import{Figures/Helium/}{TracesRabbitHe.pdf_tex}
\caption{Spectrogrammes RABBIT mesurés dans l'hélium pour plusieurs longueurs d'onde de génération. \`{A} chaque pas de délai le spectre est normalisé par le signal total d'électrons. La résonance de Fano $2s2p$ se situe à 60.15 - 24.56 = 35.59 eV et est de plus en plus visible à mesure que l'énergie de l'harmonique 63 se rapproche de l'énergie de la résonance. Elle est élargie spectralement par la résolution du spectromètre ($\approx$ 200 meV à cette énergie).}
\label{fig:TracesRabbitHe}
\end{figure}

Pour chaque longueur d'onde de génération, on obtient une courbe similaire à la figure \ref{fig:ExtractionPhaseInter}. Hors résonance, la phase du pic satellite augmente linéairement avec l'ordre à cause du terme $\phi_{\Omega_{q+2}} - \phi_{\Omega_{q}}$ en suivant la dispersion de délai de groupe intrinsèque au processus de génération, l'\textit{attochirp} (chapitre \textbf{ref}, \mycite{MairesseScience2003}). Lorsque l'harmonique est résonante, on observe des déviations du comportement linéaire symétriques pour les pics satellites du dessous et du dessus (dû au $\pm$ dans l'expression \ref{eq:SB_He}) \mycite{TheseChirla}. L'\textit{attochirp} étant dû au processus de génération et dépendant peu de la longueur d'onde, on mesure une pente linéaire quasiment identique pour toutes les longueurs d'onde ($\approx$ 20 as/eV ou 0.06 rad/eV). En soustrayant un ajustement de cette pente (pointillés figure \ref{fig:ExtractionPhaseInter}), on obtient la phase due uniquement au processus de photoionisation à deux photons \textit{via} la résonance.

\begin{figure}
\centering
\def\svgwidth{\textwidth}
\import{Figures/Helium/}{ExtractionPhaseInter.pdf_tex}
\caption{Phase des pics satellites intégrés spectralement pour deux longueurs d'onde de génération. Quand l'harmonique 63 n'est pas résonante ($\lambda$ = 1275 nm), la phase est une fonction linéaire croissante de l'énergie, signature de la dispersion de délai de groupe intrinsèque aux harmoniques ("\textit{attochirp}") \mycite{MairesseScience2003}. Cette phase linéaire dépend peu de l'énergie dans la gamme de longueurs d'onde explorées. Lorsque $\text{H}_{63}$ est résonante, des déviations du comportement linéaire apparaissent de manière symétrique pour les pics satellites au-dessous et au-dessus de l'harmonique résonante.}
\label{fig:ExtractionPhaseInter}
\end{figure}

\begin{figure}
\centering
\def\svgwidth{\textwidth}
\import{Figures/Helium/}{VariationPhaseInter.pdf_tex}
\caption{RABBIT. Différence de phase atomique $\bar{\Delta \theta^{\text{at}}}$ expérimentale (points colorés, les couleurs correspondent à la légende de la figure \ref{fig:SpectresSansHabillageHe}) des pics satellites $\text{SB}_{62}$ et $\text{SB}_{64}$ intégrés spectralement extraite de spectrogrammes RABBIT effectués à différentes longueurs d'onde, et calculée théoriquement (pointillés gris). L'axe d'énergie correspond à l'énergie centrale du pic satellite dans chaque spectrogramme.}
\label{fig:VariationPhaseInter}
\end{figure}

En appliquant cette procédure à toutes les longueurs d'onde de génération \mycite{KoturNatComm2016}, on obtient la variation complète de la phase au voisinage de la résonance (figure  \ref{fig:VariationPhaseInter}). La phase des pics satellites varie de près de 0.5 rad lorsque l'énergie de l'harmonique varie de 0.8 eV autour de la résonance d'autoionisation. Les évolutions sont symétriques dans les pics satellites $\text{SB}_{62}$ et $\text{SB}_{64}$ en raison du signe + ou - dans l'équation \ref{eq:SB_He} évoqué précédemment. Les résultats obtenus sont en très bon accord avec les calculs du groupe de F. Mart\'{i}n, dont le principe est basé sur la théorie présentée au chapitre \ref{chap:2photons_et_Fano}, plus la prise en compte des durées des impulsions XUV et MIR et de la résolution du spectromètre d'électrons \mycite{JimenezGalanPRA2016}.

\section{Mesures Rainbow RABBIT}
L'intégration spectrale du signal de photoélectrons sur toute la largeur du pic satellite est généralement effectuée pour obtenir un meilleur rapport signal sur bruit. Cependant, cette intégration, apparaissant dans l'expression de la phase extraite \ref{eq:PhaseMoyenneHe}, n'est pas satisfaisante lorsque l'on cherche des variations de phase autour de résonances de largeur très inférieure à la largeur d'un pic satellite. En effet, la quantité \ref{eq:PhaseMoyenneHe} se rapproche plutôt d'une phase moyennée sur la largeur spectrale \mycite{TheseGruson} que de la phase de la transition résonante. 

De manière surprenante, on remarque que le signal des pics satellites résonants présente une structure spectrale lorsque l'harmonique 63 est résonante (figures \ref{fig:TracesRabbitHe} et \ref{fig:TracesExp_Sim_He}, zoom). Le profil de la résonance est transféré sur le pic satellite voisin et donne lieu à deux composantes spectrales qui oscillent à la fréquence $2 \omega$ mais ne sont pas en phase. Au lieu de sommer spectralement le signal du pic satellite, nous avons analysé les oscillations à $2 \omega$ à chaque énergie de photoélectrons à l'intérieur du pic satellite. Ceci correspond à effectuer une analyse RABBIT résolue spectralement, méthode que nous avons appelée \textit{Rainbow} RABBIT. On obtient ainsi toute la variation spectrale de la phase dans un seul spectrogramme. 
\begin{equation}
\Theta (E) = \arg_{2 \omega} \left[ \int \rmd \tau \rme^{i \omega \tau} S_{\text{SB}}(\tau,E) \right]
\label{eq:PhaseRainbow}
\end{equation} 
La même analyse permet d'obtenir également les variations de l'amplitude à $2 \omega$ résolues spectralement.

\begin{figure}
\centering
\def\svgwidth{\textwidth}
\import{Figures/Helium/}{TracesExp_Sim_He.pdf_tex}
\caption{Spectrogramme expérimental (a) et théorique (b) pour une longueur d'onde de génération de 1295 nm. L'harmonique 63 est résonante avec l'état autoionisant $2s2p$. Un zoom sur une oscillation du pic satellite $\text{SB}_{62}$ montre la structure due à la résonance et le déphasage des deux composantes spectrales observées. Le pic satellite $\text{SB}_{66}$, non résonant, ne présente pas cette structure.}
\label{fig:TracesExp_Sim_He}
\end{figure}

Les résultats obtenus en appliquant l'analyse Rainbow RABBIT au spectrogramme enregistré avec une longueur d'onde de génération de 1295 nm sont présentés figure \ref{fig:DataRainbowHe}. L'évolution de la phase est toujours symétrique pour les deux pics satellites résonants mais cette fois la phase varie de $\approx$ 1 rad sur 200 meV. Cet élargissement spectral de la résonance (de largeur naturelle $\Gamma$ = 17 meV, voir tableau \ref{tab:ParamètresFano}) est dû au spectromètre de photoélectrons dont la résolution est de $\approx$ 200 meV à cette énergie. On observe un saut de phase à la position de la résonance décalée de 1 photon MIR et à la position du minimum dans l'amplitude, conformément aux résultats du modèle de Fano présentés au chapitre \ref{chap:ResonancesFano}. En comparaison, l'amplitude du pic satellite non résonant $\text{SB}_{66}$ reproduit le spectre gaussien des harmoniques, et sa phase est plate.

\begin{figure}
\centering
\def\svgwidth{\textwidth}
\import{Figures/Helium/}{DataRainbowHe.pdf_tex}
\caption{Rainbow RABBIT. Amplitude (haut) et phase (bas) spectrales de la composante à 2$\omega$ des pics satellites, issues de l'expérience (traits pleins violets) et simulées (pointillés noirs) pour les deux pics satellites résonants $\text{SB}_{62}$ et $\text{SB}_{64}$ et un pic satellite non résonant $\text{SB}_{66}$. L'origine des phases est à zéro après soustraction de la composante linéaire due au délai de groupe des harmoniques (voir figure \ref{fig:ExtractionPhaseInter}). La position de la résonance $\pm$ un photon est matérialisée par le trait vertical gris.} 
\label{fig:DataRainbowHe}
\end{figure}

\paragraph*{Comparaison du RABBIT et du Rainbow RABBIT} Si la largeur spectrale de l'harmonique est suffisante pour couvrir toute la largeur de la résonance, la variation complète de la phase est ici obtenue en un unique spectrogramme enregistré à une longueur d'onde résonante. La quantité mesurée correspond aux réelles variations spectrales de la phase et non à une valeur moyennée sur la largeur du pic satellite. De plus, si la phase varie très rapidement à l'intérieur du pic satellite, les oscillations à 2 $\omega$ peuvent être brouillées, ce qui rend la procédure d'extraction de phase plus difficile dans le signal intégré (RABBIT). Enfin, dans le Rainbow RABBIT, la résolution spectrale est déterminée par la fonction d'appareil du spectromètre à électrons utilisé. Dans les expériences présentées ici, un potentiel retard de 26 V était appliqué aux électrons, soit une énergie cinétique de l'ordre de 10 eV pour les photoélectrons au voisinage de la résonance. Les pics satellites de plus basse énergie que la résonance étaient nécessaire pour déterminer précisément l'\textit{attochirp} dans l'analyse RABBIT. Cependant, en ajoutant plus de potentiel retard il aurait été possible de décaler les pics satellites résonants à plus basse énergie cinétique et ainsi augmenter la résolution spectrale.

\chapter{Reconstruction de la dynamique d'autoionisation}
\label{chap:HeSaclay_reconstruction}
\section{Paquet d'onde électronique dans le domaine spectral}
\subsection{Paquet d'onde à deux photons}
\label{subsec:POE_2phot_spec}
Pour simplifier la discussion, considérons uniquement le pic satellite $\text{SB}_{64}$. On réécrit l'équation \ref{eq:SB_He} en explicitant le cas de $\text{SB}_{64}$ et la dépendance en énergie (cas "Rainbow"):
\begin{multline}
S_{\text{SB}_{64}}(\tau,E) \propto \left| M^{63+1}(E)\right|^2 + \left| M^{65-1}(E)\right|^2 + 2 \left| M^{63+1}(E)\right| \left| M^{65-1}(E)\right| \\ \times \cos[2 \omega \tau + \phi_{\Omega_{65}}(E) - \phi_{\Omega_{63}}(E) + \theta^{\text{at}}_{65-1}(E) - \theta^{\text{at}}_{63+1}(E)]
\end{multline}
L'absorption des deux photons XUV et MIR (chemin "63+1") créé le paquet d'onde électronique résonant. Dans le pic satellite, il interfère avec un paquet d'onde électronique de référence créé par le chemin non résonant "65-1". On cherche ici à déterminer, à partir des mesures, l'amplitude $\left| M^{63+1}(E)\right|$ et la phase $\theta^{\text{at}}_{63+1}(E)$ du paquet d'onde électronique à deux photons qui serait créé par une excitation limitée par transformée de Fourier:
\begin{equation}
\setlength\fboxrule{0.5pt}
\boxed{
M^{63+1}(E) = \left| M^{63+1}(E)\right| \times \mathrm{exp} \left( \theta^{\text{at}}_{63+1}(E) \right)
}
\end{equation}

\paragraph*{Phase} Dans l'expérience, les harmoniques sont générées dans l'argon. Autour de 60 eV (au-delà du minimum de Cooper de l'argon \mycite{CooperPR1962}), \mycite{SchounPRL2014} ont mesuré de très faibles variations pour la phase spectrale des harmoniques. De plus, la mesure de phase du pic satellite non résonant $\text{SB}_{66}$ (figure \ref{fig:DataRainbowHe}) montre que la différence $\phi_{\Omega_{67}}(E) - \phi_{\Omega_{65}}(E)$ varie peu sur la largeur d'un pic satellite. Par conséquent nous pouvons considérer que pour le pic satellite résonant les variations spectrales de la différence de phase harmonique sont négligeables devant les variations de phase atomique, mettant en jeu la résonance. Nous pouvons alors approximer la phase mesurée par Rainbow RABBIT (équation \ref{eq:PhaseRainbow})
\begin{align}
\Theta_{64} (E) & = \phi_{\Omega_{65}}(E) - \phi_{\Omega_{63}}(E) + \theta^{\text{at}}_{65-1}(E) - \theta^{\text{at}}_{63+1}(E) \\
& \approx \theta^{\text{at}}_{65-1}(E) - \theta^{\text{at}}_{63+1}(E)
\end{align}
Notons que d'une manière générale la différence de phase harmonique peut être mesurée en photoionisant un autre gaz dans les mêmes conditions, puis soustraite pour obtenir uniquement la différence de phase atomique. 

Par ailleurs, la transition à deux photons non résonante fait intervenir un continnum "lisse", ne présentant aucune résonance. On considère donc que les variations spectrales de phase atomique $\theta^{\text{at}}_{65-1}(E)$ sont négligeables devant les fortes variations dans la transition impliquant la résonance $\theta^{\text{at}}_{63+1}(E)$. Finalement, la phase atomique résonante est approximée à la quantité mesurée par Rainbow RABBIT:
\begin{equation}
\setlength\fboxrule{0.25pt}
\boxed{
\Theta_{64} (E) \approx - \theta^{\text{at}}_{63+1}(E)
}
\end{equation}
L'excellent accord entre la théorie (calculant uniquement $\theta^{\text{at}}_{63+1}(E)$) et l'expérience montré figure \ref{fig:DataRainbowHe} prouve la validité des approximations utilisées.

\paragraph*{Amplitude} Le dispositif interférométrique permet d'accéder à l'intensité du pic satellite moyennée sur le temps: 
\begin{equation}
I_{64}(E) = \left| M^{63+1}(E)\right|^2 + \left| M^{65-1}(E)\right|^2
\label{eq:I}
\end{equation}
ainsi qu'à l'amplitude de l'oscillation à 2 $\omega$:
\begin{equation}
A_{64}(E) = 2 \left| M^{63+1}(E)\right| \left| M^{65-1}(E)\right|
\label{eq:A}
\end{equation}
En principe, ces deux équations donnent accès aux modules des deux paquets d'onde qui interfèrent $\left| M^{63+1}(E)\right|$ et $\left| M^{65-1}(E)\right|$. Cependant la présence d'un fond dans les spectres de photoélectrons ne nous permet pas d'utiliser la composante continue $I_{64}(E)$, et nous avons donc uniquement utilisé la composante à 2 $\omega$ $A_{64}(E)$ pour déterminer l'amplitude résonante $\left| M^{63+1}(E)\right|$.

En première approximation, et au regard des variations spectrales de l'intensité à 2 $\omega$ du pic satellite non résonant $\text{SB}_{66}$ présentées en figure \ref{fig:DataRainbowHe}, on peut considérer que l'amplitude du paquet d'onde non résonant varie peu sur la largeur du pic satellite. $\left| M^{65-1}(E)\right|$ est alors une constante et on a:
\begin{equation}
\setlength\fboxrule{0.25pt}
\boxed{
A_{64}(E) \propto \left| M^{63+1}(E)\right|
}
\label{eq:M_A64}
\end{equation}
Plus rigoureusement, les variations spectrales de $\left| M^{65-1}(E)\right|$ peuvent être évaluées à partir du pic satellite non résonant $\text{SB}_{66}$ en utilisant des hypothèses de l'approximation \textit{soft photon} \mycite{MaquetJMO2007}. Pour le pic satellite $\text{SB}_{66}$, les équations \ref{eq:I} et \ref{eq:A} s'écrivent:
\begin{equation}
I_{66}(E + 2 \hbar \omega) = \left| M^{65+1}(E + 2 \hbar \omega)\right|^2 + \left| M^{67-1}(E + 2 \hbar \omega)\right|^2
\end{equation}
et
\begin{equation}
A_{66}(E + 2 \hbar \omega) = 2 \left| M^{65+1}(E + 2 \hbar \omega)\right| \left| M^{67-1}(E + 2 \hbar \omega)\right|
\end{equation}
les deux chemins "65+1" et "67-1" étant non résonants. Loin du seuil d'ionisation et si les harmoniques 65 et 67 ont des profils similaires, on peut considérer les amplitudes des deux chemins impliquant la même harmonique égales et simplement décalées en énergie de deux photons MIR
\begin{equation}
\left| M^{65-1}(E)\right| \approx \left| M^{65+1}(E + 2 \hbar \omega)\right|
\label{eq:AsoftPhoton}
\end{equation}
et également égales à la transition non résonante impliquant l'harmonique voisine
\begin{equation}
\left| M^{65+1}(E + 2 \hbar \omega)\right| \approx \left| M^{67-1}(E + 2 \hbar \omega)\right|
\end{equation}
Ainsi, l'amplitude de l'oscillation à 2 $\omega$ de $\text{SB}_{66}$ s'écrit
\begin{equation}
A_{66}(E + 2 \hbar \omega) \approx  2 \left| M^{65+1}(E + 2 \hbar \omega)\right|^2
\end{equation}
Et en insérant \ref{eq:AsoftPhoton} dans \ref{eq:A}, il vient:
\begin{equation}
\left| M^{63+1}(E)\right| = \frac{A_{64}(E)}{2 \left| M^{65+1}(E + 2 \hbar \omega)\right|} 
\end{equation}
\begin{equation}
\setlength\fboxrule{0.25pt}
\boxed{
\left| M^{63+1}(E)\right| = \frac{A_{64}(E)}{\sqrt{2 \: A_{66}(E + 2 \hbar \omega)}}
}
\label{eq:M_A66}
\end{equation}
Cette approche est plus exacte en principe, mais l'on s'attend à ce que le module calculé dépende de l'amplitude de l'oscillation du pic satellite non résonant de référence, et donc soit plus sensible au bruit expérimental et la la variation spectrale de résolution du spectromètre d'électrons.

\subsection{Paquet d'onde à un photon}
Dans le chapitre \ref{chap:2photons_et_Fano}, nous avons exprimé la phase de l'élément de transition à deux photons \textit{via} une résonance de Fano (équation \ref{eq:Arg2photonsFano})
\begin{equation}
\arg M_{\vec{k}, \text{Fano}}^{(2)} \approx \arg M_{\vec{k}}^{(2)} + \arg \: [\mathcal{R}_{\text{eff}}(\epsilon)]
\end{equation}
qui diffère de la phase de la transition résonante à un photon (équation \ref{eq:Arg1photonFano})
\begin{equation}
\arg M^{(1)}_E = \arg \mathcal{R}(\epsilon)
\end{equation}
par le terme "continuum-continuum" (chapitre \ref{chap:DelaiPI} paragraphe \ref{sec:Matrice2photons}), et la présence du facteur résonant effectif.

Pour $\gamma \ll 1$, $\mathcal{R}_{\text{eff}}(\epsilon) \approx \mathcal{R}(\epsilon)$. Dans nos conditions expérimentales, on calcule 
\begin{equation}
\gamma = 0.0154
\end{equation}
indiquant un faible couplage de la résonance au continuum final, par rapport au couplage entre les continua intermédiaire et final. Cette petite valeur de $\gamma$ nous permet de considérer, dans nos conditions expérimentales
\begin{equation}
\mathcal{R}_{\text{eff}}(\epsilon) \approx \mathcal{R}(\epsilon)
\end{equation}

\begin{figure}
\centering
\def\svgwidth{0.5\textwidth}
\import{Figures/Helium/}{Phases1photon2photons.pdf_tex}
\caption{Phase simulée pour la transition à un photon résonante (gris), la transition à deux photons avec une impulsion MIR monochromatique et $\gamma = 0.0154$ (bleu foncé), la transition à deux photons prenant en compte la largeur spectrale de l'impulsion MIR (bleu clair), et la transition à deux photons  prenant en compte la largeur spectrale de l'impulsion MIR et la résolution du spectromètre de photoélectrons (pointillés noirs). La courbe pointillée est identique à celle présentée figure \ref{fig:DataRainbowHe}.} 
\label{fig:Phases1photon2photons}
\end{figure}

La figure \ref{fig:Phases1photon2photons} montre la phase calculée pour la transition à un photon et la transition à deux photons avec $\gamma = 0.0154$. La similarité des deux courbes montre que l'on peut ici approximer la transition à deux photons par la transition à un photon
\begin{equation}
\arg M_{\vec{k}, \text{Fano}}^{(2)} \approx \arg M^{(1)}_E
\end{equation}
et indique une contribution négligeable du terme "continuum-continuum".

Par ailleurs, la phase de la transition à deux photons calculée en prenant en compte l'effet de la bande spectrale de l'impulsion d'habillage (\mycite{JimenezGalanPRA2016},courbe bleu clair de la figure \ref{fig:Phases1photon2photons}) diffère peu de la phase de la transition à deux photons dans le cas monochromatique. Ceci montre que dans nos conditions expérimentales avec une durée de l'impulsion MIR de 70 fs soit une largeur spectrale de 26 meV, les effets d'impulsion finie sont négligeables. La principale distorsion de la phase observée dans nos mesures provient de l'élargissement spectral dû à la résolution du spectromètre ($\approx$ 200 meV). 

En conclusion, nous pouvons estimer que dans les conditions expérimentales utilisées le paquet d'onde électronique à deux photons caractérisé par Rainbow RABBIT est une réplique fidèle du paquet d'onde électronique résonant qui aurait été créé par une excitation harmonique à un photon. Son amplitude et sa phase peuvent être déterminées à partir des observables du Rainbow RABBIT:
\begin{equation}
\setlength\fboxrule{0.5pt}
\boxed{
M^{63+1}(E) \approx M^{\text{res}}(E) \approx A_{64}(E) \times \mathrm{exp} \left( - \Theta_{64}(E) \right)
}
\label{eq:POE_1ph_sp}
\end{equation}
Notons qu'un raisonnement similaire s'applique au pic satellite $\text{SB}_{62}$ en utilisant les chemins "63-1" et "62+1".

\section{Paquet d'onde électronique dans le domaine temporel}
\label{sec:POE_temporel}
\`{A} partir de l'expression du paquet d'onde électronique résonant dans le domaine spectral (équation \ref{eq:POE_1ph_sp}), on exprime le paquet d'onde dans le domaine temporel par transformée de Fourier:
\begin{equation}
\setlength\fboxrule{0.25pt}
\boxed{
\tilde{M}^{\text{res}}(t) = \frac{1}{2 \pi} \int_{- \infty}^{+ \infty} \left| M^{\text{res}}(E) \right| \rme^{i \: \theta^{\text{at}}_{63+1}(E)} \times \rme^{-i E t / \hbar} \: \rmd E
}
\label{eq:POE_1ph_temp}
\end{equation}

\begin{figure}[!ht]
\centering
\def\svgwidth{0.7\textwidth}
\import{Figures/Helium/}{ProfilTemporel.pdf_tex}
\caption{Profil temporel du paquet d'onde électronique obtenu à partir de la transformée de Fourier du pic satellite $\text{SB}_{64}$ (équations \ref{eq:POE_1ph_sp} et \ref{eq:POE_1ph_temp}; trait plein violet) et phase correspondante (pointillés violets). La zone coloriée correspond au profil temporel obtenu à partir du pic satellite non résonant $\text{SB}_{66}$.}
\label{fig:ProfilTemporel}
\end{figure}

Le profil temporel calculé pour le pic satellite $\text{SB}_{64}$ est présenté figure \ref{fig:ProfilTemporel}. Il est comparé au profil temporel obtenu à partir du pic satellite non résonant $\text{SB}_{66}$, qui est une gaussienne répliquant l'impulsion d'excitation et sert à déterminer le temps $t = 0$. Le profil temporel résonant présente un maximum à l'origine des temps, puis un minimum vers $\approx$ 4 fs et une oscillation avant une décroissance rapide. La présence d'un saut de phase de $\approx$ 2 rad associé au minimum d'intensité indique qu'il résulte d'une interférence entre deux composantes du paquet d'onde: la partie correspondant à l'ionisation directe et celle correspondant à l'ionisation \textit{via} la résonance $2s2p$. 

\begin{figure}
\centering
\def\svgwidth{0.7\textwidth}
\import{Figures/Helium/}{ProfilTemporel_th.pdf_tex}
\caption{Profil temporel du paquet d'onde électronique obtenu à partir de la transformée de Fourier du pic satellite $\text{SB}_{64}$ (figure \ref{fig:ProfilTemporel}, trait plein violet) comparé à différents paquets d'onde électroniques calculés: à partir de spectrogrammes simulés prenant en compté (pointillés orange) ou non (pointillés noirs) la résolution du spectromètre, et à partir de la détermination analytique du profil temporel du paquet d'onde électronique résonant à un photon (trait plein gris). Les interférences entre l'ionisation directe et résonante observées dans le paquet d'onde électronique expérimental sont également observées dans les paquets d'onde simulés. Lorsque la simulation prend en compte l'élargissement spectral dû au spectromètre, la durée de vie effective de la décroissance du paquet d'onde aux temps longs est réduite, et correspond à la décroissance observée expérimentalement.} 
\label{fig:ProfilTemporel_th}
\end{figure}

Dans le domaine spectral, le paquet d'onde électronique correspond au produit de l'amplitude de l'harmonique par le facteur résonant de Fano:
\begin{equation}
M^{\text{res}}(E) = \mathcal{R}(\epsilon) \times \mathcal{H} (E)
\end{equation}
Dans le domaine temporel, il s'agit donc du produit de convolution de l'amplitude temporelle de l'harmonique par la transformée de Fourier de $\mathcal{R}(\epsilon)$
\begin{equation}
\tilde{M}^{\text{res}}(t) = \left[ \tilde{\mathcal{R}} \ast \tilde{\mathcal{H}} \right] (t)
\end{equation} 
D'après le chapitre \ref{chap:2photons_et_Fano}, la transformée de Fourier de $\mathcal{R}(\epsilon)$ s'écrit
\begin{equation}
\tilde{\mathcal{R}}(t) = \delta(t) - i \frac{\Gamma}{2 \hbar} (q - i) \times \mathrm{exp} \left(\frac{i E_R}{\hbar} + \frac{\Gamma}{2 \hbar} \right) \vartheta (t)
\end{equation}
où l'on utilise les notations du chapitre \ref{chap:ResonancesFano}, et $\delta$ et $\vartheta$ représentent respectivement la distribution de Dirac et la fonction de Heaviside. Ainsi, $\tilde{M}^{\text{res}}(t)$ possède deux composantes: une gaussienne centrée à l'origine qui reproduit l'impulsion d'excitation, et une parie résonante, comme nos données expérimentales. Cette interprétation est appuyée par le calcul analytique du paquet d'onde électronique résonant, représenté en gris sur la figure \ref{fig:ProfilTemporel_th}, qui reproduit fidèlement l'interférence destructive observée. Les données expérimentales ne contiennent pas la décroissance exponentielle du paquet d'onde aux temps longs caractéristique de la durée de vie de la résonance, mais ceci est dû à l'élargissement spectral de la résonance par le spectromètre de photoélectrons (figure \ref{fig:ProfilTemporel_th}).

\section{Construction du profil spectral de la résonance au cours du temps}
\label{sec:ConstructionResonance}
\`{A} partir de l'amplitude et de la phase temporelles du paquet d'onde électronique résonant (figure \ref{fig:ProfilTemporel}), il est possible d'obtenir la construction du profil spectral au cours du temps, de manière similaire aux travaux de \mycite{WickenhauserPRL2005} présentés au chapitre \ref{chap:ResonancesFano}, figure \ref{fig:Wickenhauser}. Pour cela, on introduit une analyse temps-énergie basée sur une transformée de Fourier limitée
\begin{equation}
\setlength\fboxrule{0.5pt}
\boxed{
W(E, \: t_{\text{acc}}) = \int_{- \infty}^{t_{\text{acc}}} \tilde{M}^{\text{res}}(t) \times \rme^{i E t / \hbar} \: \rmd t
}
\label{eq:Wicken}
\end{equation}
qui montre comment le profil spectral se construit jusqu'à un temps d'accumulation $t_{\text{acc}}$. 

\begin{figure}[!ht]
\centering
\def\svgwidth{1.2\textwidth}
\import{Figures/Helium/}{Wickenhauserisation.pdf_tex}
\caption{(a) Reconstruction du profil spectral de la résonance au cours du temps calculée en utilisant l'analyse temps-énergie \ref{eq:Wicken}. Le spectre de photoélectrons est trancé en fonction de la limite supérieure de l'intégration utilisée dans la transformée de Fourier inverse, le temps d'accumulation $t_{\text{acc}}$. Les lignes grise, bleue et rouge indiquent des temps d'accumulation de 0, 3 et 20 fs respectivement. (b) Profil spectral de (a) toutes les 1 fs. On distingue d'abord la construction du profil de l'ionisation directe jusqu'à un maximum vers 3 fs (courbes bleues), puis l'apparition d'interférences spectrales convergeant vers le profil de raie de Fano (courbes rouges). Le cercle noir indique un point "isobestique" où toutes les courbes se croisent à partir de $t_{\text{acc}}$ = 3 fs, indiquant une énergie du spectre final à laquelle seule l'ionisation directe contribue.} 
\label{fig:Wickenhauserisation}
\end{figure}

La figure \ref{fig:Wickenhauserisation} montre l'évolution du profil de la résonance $\left| W(E, \: t_{\text{acc}}) \right|^2$ en fontion du temps d'accumulation $t_{\text{acc}}$. Jusqu'à $t_{\text{acc}}$ = 3 fs, le spectre est quasi-Gaussien et reproduit le spectre de l'impulsion d'ionisation. \`{A} ces temps courts, seule l'ionisation directe contribue au spectre de photoélectrons. Lorsque $t_{\text{acc}}$ augmente, la contribution de la résonance d'autoionisation est de plus en plus importante et on observe l'apparition progressive d'interférences spectrales. Après 20 fs, le spectre converge vers l'intensité spectrale mesurée par l'expérience (figure \ref{fig:DataRainbowHe}), conformément au profil temporel de la figure \ref{fig:ProfilTemporel} qui montre une intensité nulle pour $t > $20 fs. Remarquons ici que le temps de 3 fs n'est pas "intrinsèque" au processus mais dépend des caractéristiques de l'impulsion d'excitation et de la détection. Cependant, cette représentation temporo-spectrale permet de visualiser les deux processus d'ionisation directe et résonante du modèle de Fano, qui ont lieu à des échelles de temps différentes. 

Enfin, remarquons un point particulier dans la figure \ref{fig:Wickenhauserisation} (b) où tous les spectres se croisent \footnote{Par analogie avec ce qui est observé dans le spectre d'absorption lorsqu'une espèce chimique est transformée en une autre de même coefficient d'absorption à une certaine longueur d'onde, en gardant la somme des concentrations constantes \mycite{IUPAC}, nous avons qualifié ce point de "quasi-isobestique".} à partir de $t_{\text{acc}}$ = 3 fs, c'est-à-dire à partir du moment où l'on observe la contribution de l'ionisation résonante. Ce point est identifié en utilisant l'expression de la section efficace de Fano  \ref{eq:SectionEfficaceFano3termes} établie au chapitre \ref{chap:ResonancesFano}. On remarque que pour 
\begin{equation}
\epsilon_{\text{iso}} = \frac{1}{2} \left( \frac{1}{q} - 1 \right)
\end{equation}
les contributions de l'état discret et du couplage s'annule, ne laissant que la contribution du continuum, constante quelle que soit la population de la résonance.

\paragraph*{Construction du profil spectral de la résonance en utilisant la normalisation par le pic satellite non résonant} Au paragraphe \ref{subsec:POE_2phot_spec}, nous avons présenté deux méthodes permettant d'extraire le module du paquet d'onde résonant à partir des données de Rainbow RABBIT. Tous les résultats présentés aux paragraphes \ref{sec:POE_temporel} et \ref{sec:ConstructionResonance} sont obtenus en utilisant l'équation \ref{eq:M_A64}. La figure \ref{fig:Spectre_norm} compare la construction du profil spectral en utilisant les expressions \ref{eq:M_A64} ou bien \ref{eq:M_A66} pour le calcul du module résonant. On observe que la division par l'amplitude du pic satellite voisin (équation \ref{eq:M_A66}, figure \ref{fig:Spectre_norm}(b)) élargit spectralement les profils de raie, qui possèdent des structures additionnelles dues au bruit sur le pic satellite  $\text{SB}_{66}$. Cependant, la dynamique observée est très similaire avec les deux méthodes, ce qui valide les approximations utilisées précédemment. En particulier, l'énergie $\epsilon_{\text{iso}}$ est identique dans les deux familles de spectres \ref{fig:Spectre_norm}(a) et \ref{fig:Spectre_norm}(b).

\begin{figure}
\centering
\def\svgwidth{1\textwidth}
\import{Figures/Helium/}{Spectre_norm.pdf_tex}
\caption{Construction du profil spectral de la résonance calculée par l'équation \ref{eq:Wicken} avec les deux méthodes d'extraction de l'amplitude spectrale: sans normaliser par le pic satellite voisin (a) (figure identitque à \ref{fig:Wickenhauserisation}(b)), ou en normalisant par l'amplitude de $\text{SB}_{66}$ (b). Les dynamiques observées sont similaires dans les deux approches.} 
\label{fig:Spectre_norm}
\end{figure}

\chapter[Etude de la résonance sp3+ et de l'influence de paramètres expérimentaux]{Etude de la résonance \MakeLowercase{sp3+} et de l'influence de paramètres expérimentaux}
Les résultats présentés dans ce chapitre ont été obtenus lors de deux campagnes expérimentales menées à Lund en collaboration avec l'équipe du professeur Anne L'Huillier (David Busto, Mathieu Gisselbrecht, Anne Harth, Marcus Isinger, David Kroon, Shiyang Zhong) et le spectromètre de temps de vol d'électrons à bouteille magnétique de l'Université de Gothenburg (Raimund Feifel, Richard Squibb).

% Spectromètre qui résoud mieux. on voit la sp3+ (youpi), la 2s2p et les deux dans la même SB. Mais dommage on peut pas les coupler à 2 photons.
% Déconvolution du spectromètre. Mince on avait plein d'effets de finite pulse... Le POE n'est donc pas exactement une réplique du POE à 1 photon, c'est le POE à deux photons.
% On fait quand même les recontstructions temporo spectrales. Wicken + Gabor + Wigner.
% Le poe à deux électrons (les SB ne sont pas vraiment des répliques mais bon). + calculs Richard
% Influence de l'intensité d'habillage sur q

Dans les expériences présentées aux chapitres \ref{chap:HeSaclay_res} et \ref{chap:HeSaclay_reconstruction}, la phase et l'amplitude spectrales au voisinage de la résonance sont significativement élargies par convolution avec la résolution du spectromètre de photoélectrons. Ainsi, nous avons approfondi l'étude des résonances d'autoionisation de l'hélium avec une bouteille magnétique de meilleure résolution et en utilisant plus de potentiel retard pour diminuer l'énergie cinétique des photoélectrons étudiés. Ces conditions expérimentales différentes nous ont également permis de mesurer la phase de la transition vers la résonance $sp3+$\footnote{La nomenclature des états doublement excités de l'hélium est proposée par Cooper, Fano et Prats en 1963 \mycite{CooperPRL1963}. Les états autoionisants dans la région 60 - 65 eV convergent vers le niveau $n = 2$ de He$^+$. Les états $2s$ et $2p$ de He$^+$ étant dégénérés, deux séries d'états doublement excités convergent vers cette limite: $2snp$ et $2pns$, quasi dégénérées, qui interagissent alors pour former des états où les électrons ne sont plus indépendants et que l'on note $\ket{spn \pm} = \frac{1}{\sqrt{2}} \left( \ket{2snp} \pm \ket{2pns} \right)$. L'état $2s2p$ appartient à la série "+", et est parfois noté $sp2+$. La transition dipolaire depuis l'état fondamental de l'hélium vers les états "-" est quasi-interdite, ils n'avaient donc pas encore observés en 1963 mais sont visibles sur la figure \ref{fig:Domke}.}, et d'exciter simultanément les deux résonances avec deux harmoniques consécutives. Nous avons également développé plusieurs méthodes de représentation temps-énergie pour observer la dynamique des résonances d'autoionisation. Enfin, nous avons étudié l'influence de l'intensité du faisceau d'habillage sur le profil de raie.

\section{Mesures Rainbow RABBIT à Lund}
Les expériences ont été effectuées avec un laser titane:saphir à 1 kHz délivrant des impulsions de 20 fs à 800 nm avec une énergie de 5 mJ par impulsion. Le faisceau, stabilisé activement en position, est séparé en deux dans un interféromètre de type Mach-Zehnder. Une partie est focalisée par un miroir sphérique dans un jet de néon pulsé pour générer des harmoniques. L'infrarouge résiduel est filtré par 200 nm d'aluminium puis les impulsions XUV sont focalisées par un miroir torique dans un spectromètre à temps de vol d'électrons à bouteille magnétique d'une longueur de 2 m. L'autre partie du faisceau, dont l'intensité est réglée avec une lame $\lambda /2$ et un polariseur, est superposée à l'XUV au foyer du spectromètre après un trajet dans une ligne à retard. Les spectres de photoélectrons sont mesurées en fonction du délai entre l'IR et l'XUV, stabilisé activement. La longueur d'onde du laser peut-être variée grâce à des filtres acousto-optiques (MAZZLER, DAZZLER), au détriment de la bande spectrale et donc de la durée de l'impulsion.

Les résonances de Fano $2s2p$ et $sp3+$ sont séparées de 3.51 eV, soit plus que l'écart énergétique entre deux harmoniques consécutives dans la gamme spectrale accessible ici (il faudrait une longueur d'onde autour de 720 nm pour coupler les deux résonances avec deux photons IR). Cependant grâce à la largeur spectrale des harmoniques, autour de 200 meV, il est possible d'exciter simultanément les deux résonances avec les harmoniques 39 et 41 dans certaines conditions (figures \ref{fig:Schema_2res} et \ref{fig:HeLund_WLscan}).

\begin{figure}
\centering
\def\svgwidth{0.7\textwidth}
\import{Figures/Helium/}{Schema_2s2p_sp3.pdf_tex}
\caption{Schéma des états de l'hélium et des interférences à deux photons mises en jeu dans le RABBIT. Lorsque la longueur d'onde du laser est variée, les harmoniques 39 et 41 peuvent exciter séparément ou simultanément les deux résonances $2s2p$ et $sp3+$. L'écart entre ces deux niveaux est de 3.51 eV, il est donc impossible de les coupler avec deux photons IR. Cependant, la largeur spectrale des harmoniques ($\approx$ 200 meV) et de l'infrarouge ($\approx$ 130 meV) rend possible l'excitation simultanée des deux résonances par deux harmoniques consécutives.} 
\label{fig:Schema_2res}
\end{figure}

\begin{figure} [h]
\centering
\def\svgwidth{\textwidth}
\import{Figures/Helium/}{HeLund_WLscan.pdf_tex}
\caption{Spectres de photoélectrons de l'hélium ionisé par les harmoniques 39 à 43 générées dans l'argon par un laser titane:saphir accordable de longueur d'onde variant de 790 à 800 nm, de largeur 70 nm. Les spectres sont décalés verticalement pour une meilleure visibilité. Les pointillés noirs matérialisent la position des résonances $2s2p$ et $sp3+$.}
\label{fig:HeLund_WLscan}
\end{figure}

\subsection{Excitation de la résonance 2s2p}
D'après la figure \ref{fig:HeLund_WLscan}, pour une longueur d'onde de 799 nm seule la résonance $2s2p$ est excitée. Les mesures de phase au voisinage de cette résonance avec la méthode Rainbow RABBIT sont donc à nouveau effectuées avec ces conditions expérimentales. 

\begin{figure} [h]
\centering
\def\svgwidth{0.7\textwidth}
\import{Figures/Helium/}{Axes_Rabbit_HeLund_NeHe2.pdf_tex}
\caption{Spectrogramme RABBIT mesuré dans l'hélium à 799 nm. \`{A} chaque pas de délai le spectre est normalisé par le signal total d'électrons. Pour plus de visibilité des pics satellites, l'échelle de couleurs utilisée sature l'harmonique résonante.}
\label{fig:Rabbit_HeLund_2s2p}
\end{figure}

\begin{figure} [h]
\centering
\def\svgwidth{\textwidth}
\import{Figures/Helium/}{Axes_Rabbit_HeLund_NeHe2_deconv.pdf_tex}
\caption{Bla.}
\label{fig:Rabbit_HeLund_2s2p_deconv}
\end{figure}

\begin{figure} [h]
\centering
\def\svgwidth{\textwidth}
\import{Figures/Helium/}{Resultats_NeHe2_deconv.pdf_tex}
\caption{Bla.}
\label{fig:Resultats_NeHe2_deconv}
\end{figure}

% Déconvolution: chaque délai smoothé sur 10 points puis deconvblind. Fonction d'appareil Lorentzienne de largeur ~85 meV pour SB 38 et ~120 meV pour SB 40. Pb: Pas la même fonction d'appareil à chaque pas de délai + le deltaE/E n'est pas constant + pas ok avec Alvaro.
% Deconvbiggs de Lund: donne des valeurs complexes ?? Ne déconvolue de rien. Pour le mode "fonction constante", il faut que la fonction soit centrée sur l'intervalle d'énergie sinon pb. Très dépendant de la valeur de T.

\subsection{Excitation simultanée des résonances 2s2p et sp3+}

\begin{figure} [h]
\centering
\def\svgwidth{\textwidth}
\import{Figures/Helium/}{Resultats_NeHe6_deconv.pdf_tex}
\caption{Bla.}
\label{fig:Resultats_NeHe6_deconv}
\end{figure}


\section{Dynamiques d'autoionisation}
\subsection{Représentations temps-énergie}

\begin{figure} [h]
\centering
\def\svgwidth{0.5\textwidth}
\import{Figures/Helium/}{ProfilTemporel_Lund.pdf_tex}
\caption{Bla.}
\label{fig:ProfilTemporel_Lund}
\end{figure}

\begin{figure} [h]
\centering
\def\svgwidth{\textwidth}
\import{Figures/Helium/}{Wicken_Gabor.pdf_tex}
\caption{Bla.}
\label{fig:Wicken_Gabor}
\end{figure}

\begin{figure} [h]
\centering
\def\svgwidth{\textwidth}
\import{Figures/Helium/}{Wigner_Th.pdf_tex}
\caption{Bla.}
\label{fig:Wigner_Th}
\end{figure}

%\subsection{Paquet d'onde à deux électrons}
%
%\section{Influence de l'intensité d'habillage sur le profil de raie}

\chapter{Comparaison entre la spectroscopie de photoionisation et l'absorption transitoire attoseconde pour l'étude de la dynamique d'autoionisation de l'hélium}
Dans le même numéro de la revue $Science$, deux articles étudiant la construction du profil de Fano de la résonance$2s2p$ de l'hélium au cours du temps par deux méthodes différentes ont été publiés simultanément. Le premier, \mycite{GrusonScience2016}, utilise la photoionisation et la technique Rainbow RABBIT, et ses résultats ont été présentés aux chapitres \ref{chap:HeSaclay_res} et \ref{chap:HeSaclay_reconstruction}. 

Une seconde méthode proposée par \mycite{KaldunScience2016} utilise l'absorption transitoire attoseconde. La génération d'harmoniques d'ordre élevé dans un premier gaz à partir d'une impulsion visible-IR de 7 fs produit un spectre XUV large autour de 60 eV. L'impulsion XUV attoseconde est focalisée dans une cellule d'hélium et excite la transition vers l'état autoionisant. Une seconde impulsion IR de 7 fs intense ($\approx$ $10^{13}$ W/cm$^2$) modifie le dipôle induit en interrompant brutalement l'autoionisation par ionisation multiphotonique. L'absorbance, reliée à la partie imaginaire du dipôle, est mesurée grâce à un spectromètre à réseau en fonction du délai XUV - IR (figure \ref{fig:Kaldun}).

\begin{figure} [ht!]
\centering
\def\svgwidth{0.5\textwidth}
\import{Figures/Helium/}{Kaldun.pdf_tex}
\caption{\'{E}volution du profil d'absorption au voisinage de la résonance $2s2p$ de l'hélium en fonction du délai entre une impulsion XUV attoseconde qui excite la résonance et une impulsion IR brève et intense qui modifie le processus d'autoionisation. Adapté de \mycite{KaldunScience2016}.}
\label{fig:Kaldun}
\end{figure}

Si ces deux expériences s'intéressent au même processus: l'autoionisation de l'hélium \textit{via} la résonance $2s2p$, les observables mesurées sont en revanche différentes. La première mesure l'amplitude et la phase du paquet d'onde électronique qui nait dans le continuum, et reconstruit la modification de son spectre au cours du temps (figure \ref{fig:Wickenhauserisation}), soit l'autoionisation \textit{de l'extérieur}. La seconde observe l'évolution temporelle de la réponse dipolaire (figure \ref{fig:Kaldun}), soit l'autoionisation \textit{de l'intérieur}. Dans ce chapitre, on souhaite déterminer analytiquement et numériquement si ces deux dynamiques sont identiques à l'échelle attoseconde.

\section{Absorption transitoire attoseconde}
\subsection{Section efficace et dipôle induit}
La spectroscopie d'absorption mesure la section efficace d'absorption $\sigma (\omega)$. D'après la loi de Beer-Lambert, la section efficace es proportionnelle à la partie imaginaire de l'indice de réfraction $n(\omega) = \sqrt{1+ \chi(\omega)}$. La polarisation macroscopique $\vec{P}$ d'un milieu de $\rho$ émetteurs par unité de volume émettant un dipôle $d(\omega)$ est reliée au champ électrique incident $F$ par 
\begin{equation}
\vec{P} = \rho\: <d(\omega)> = \epsilon_0 \: \chi(\omega) \: \vec{F}(\omega)
\end{equation}
Après un développement limité au premier ordre de l'indice de réfraction \mycite{TheseKaldun}, on montre que
\begin{equation}
\sigma(\omega) \approx \frac{\rho \omega}{\epsilon_0 c} \: \mathcal{I} \text{\textit{m}} \left[ \frac{<d(\omega)>}{F(\omega)} \right]
\end{equation}
soit
\begin{equation}
\setlength\fboxrule{0.5pt}
\boxed{
\sigma (\omega) \propto \mathcal{I} \text{\textit{m}} <d(\omega)> 
}
\end{equation}
où $<d(\omega)>$ est la valeur moyenne de l'opérateur dipolaire dans l'état considéré $\ket{\Phi}$, et est relié par transformée de Fourier au dipôle temporel induit, dans l'approximation dipolaire
\begin{align}
<d(\omega)> & = \int_{- \infty}^{+ \infty} <\tilde{d}(t)> \rme^{i \omega t} \rmd t \\
& =  \int_{- \infty}^{+ \infty} \bra{\Phi(t)} F_z \: \rme^{i \Omega t} z \ket{\Phi(t)} \rme^{i \omega t} \rmd t 
\end{align}
en supposant le champ polarisé selon l'axe $z$. % \Omega ou \omega pour le champ F ???

\subsection{Dipôle dépendant du temps au voisinage d'une résonance de Fano}
\paragraph{Fonction d'onde dans la base des configurations} D'après le chapitre \ref{chap:ResonancesFano}, les fonctions propres de l'hamiltonien décrit par les équations \ref{eq:Hamiltonien_Fano} s'écrivent, en fonction des configurations de l'état lié $\ket{\varphi}$ et du continuum $ \ket{\psi_{E'}}$
\begin{equation}
\ket{\Psi_E} = a_E \ket{\varphi} + \int b_{E'} \ket{\psi_{E'}} \rmd E'
\end{equation}
avec les expressions des coefficients $a_E$ et $b_{E'}$ données au chapitre \ref{chap:ResonancesFano}. Dans la suite on suppose que le couplage entre l'état lié et le continuum noté $V_{E'}$ au chapitre  \ref{chap:ResonancesFano} est indépendant de l'énergie et égal à $V$.

Considérons que le système est ionisé depuis son état fondamental $\ket{g}$ par une impulsion XUV ultra-brève. Après le passage de cette impulsion, la fonction d'onde dépendante du temps caractérisant le système est donnée par
\begin{equation}
\ket{\Phi (t)} = c_g \: \rme^{-i E_g t / \hbar} \ket{g} + \int c_E \: \rme^{-iEt/ \hbar} \ket{\Psi_E} \rmd E
\end{equation}
Les états $\ket{\Psi_E}$ étant fonctions propres du système, les coefficients $c_E$ sont indépendants du temps en l'absence de l'impulsion de pompe. La dynamique de l'autoionisation n'est donc pas contenue dans $|c_E|^2$ et l'information temporelle est cachée, bien que l'autoionisation ait lieu.

Pour faire apparaître les aspects dépendants du temps, \mycite{ChuPRA2010} proposent de décomposer la fonction d'onde $\ket{\Phi (t)}$ non pas sur la base des fonctions propres de l'hamiltonien mais sur la base des configurations $\ket{\varphi}$ et $ \ket{\psi_{E'}}$. On a alors
\begin{equation}
\ket{\Phi (t)} = c_g \: \rme^{-i E_g t / \hbar} \ket{g} + c_{\varphi} (t) \ket{\varphi} + \int c_{E'} (t) \ket{\psi_{E'}} \rmd E'
\label{eq:ExpressionDePhi}
\end{equation}
L'évolution temporelle des coefficients $c_{\varphi} (t)$ et $c_{E'} (t)$ est gouvernée par l'équation de Schrödinger dépendante du temps $i \hbar \frac{\partial}{\partial t} \ket{\Phi (t)} = H \ket{\Phi (t)}$, ce qui conduit aux équations différentielles couplées
\begin{align}
i \hbar \: \frac{\rmd c_{\varphi}}{\rmd t} & = E_r \: c_{\varphi} (t) + \int V \: c_{E'}(t) \: \rmd E' \\
i \hbar \: \frac{\rmd c_{E'}}{\rmd t} & = V \: c_{\varphi}(t) + E'\: c_{E'}(t)
\end{align}
En supposant connues les conditions initiales $c_{\varphi}^{(0)}=\braket{\varphi|\Phi (t = 0)}$ et $c_{E'}^{(0)}=\braket{\psi_{E'}|\Phi (t = 0)}$, la solution est donnée par
\begin{align}
c_{\varphi} (t) & = \left( c_{\varphi}^{(0)} \rme^{-\frac{\Gamma}{2 \hbar}t} + \int c_{E'}^{(0)} g_{E'} (t) \rmd E' \right) \rme^{-i E_r t / \hbar} \\
c_{E'}(t) & = \left( c_{\varphi}^{(0)} g_{E'} (t) + \int c_{E}^{(0)} f_{EE'} (t) \rmd E \right) \rme^{-i E_r t / \hbar} + c_{E'}^{(0)} \rme^{-i E' t / \hbar}
\end{align}
Avec les fonctions $g_E$ et $f_{EE'}$
\begin{align}
g_E (t) & = \frac{V}{E - E_r + i \frac{\Gamma}{2} } \left( \rme^{-i(E-E_r)t / \hbar} - \rme^{-\frac{\Gamma}{2 \hbar} t} \right) \\
f_{EE'} (t) & = \frac{V}{E'-E} \left(g_{E'}(t) - g_{E}(t) \right)
\end{align}
Si l'on considère que le continuum initial est plat, c'est-à-dire que $c_{E'}^{(0)}$ est une constante et ne dépend pas de l'énergie au voisinage de la résonance, les expressions précédentes se simplifient. En utilisant l'énergie réduite $\epsilon$ définie au chapitre \ref{chap:ResonancesFano}, et le paramètre $q$ dans le cas d'un couplage $V$ constant
\begin{equation}
q = \frac{\bra{\varphi} z \ket{g}}{\pi V \bra{\psi_{E'}} z \ket{g}} = \frac{c_{\varphi}^{(0)}}{\pi V c_{E'}^{(0)}}
\label{eq:q_en_fonction_des_c0}
\end{equation}
on a \footnote{On remarque ici que les limites de $c_{\varphi} (t)$ et $c_{\epsilon}(t)$ pour $t \rightarrow 0$ ne sont pas égales à $c_{\varphi}^{(0)}$ et $c_{\epsilon}^{(0)}$, $c_{\varphi}$ et $c_{\epsilon}$ sont donc discontinues en $t = 0$. Ceci est dû à l'intégration des fonctions $g_E$ et $f_{EE'}$ sur une gamme spectrale infinie alors qu'en pratique seule une gamme d'énergie de largeur de l'ordre de $\Gamma$ doit être pris en compte.} 
\begin{align}
c_{\varphi} (t) & = c_{\varphi}^{(0)} \left( 1 - \frac{i}{q} \right) \rme^{-\frac{\Gamma}{2 \hbar}t} \: \rme^{-i E_r t} \label{eq:c_varphi} \\
c_{\epsilon}(t) & = \frac{c_{\epsilon}^{(0)}}{\epsilon + i} \left[ (q+\epsilon ) \rme^{- i \frac{\Gamma}{2 \hbar} \epsilon t } - (q - i) \rme^{-\frac{\Gamma}{2 \hbar}t} \right] \: \rme^{-i E_r t} \label{eq:c_eprime}
\end{align}
$|c_{\varphi}(t)|^2 $ décroit exponentiellement avec la durée de vie $1/\Gamma$, reflétant la décroissance de l'état lié. Les variations temporelles de $|c_{\epsilon}(t)|^2$ sont plus complexes et contiennent la dynamique complète de l'autoionisation, incluant les couplages entre l'état lié et le continuum ainsi qu'entre différentes énergies du continuum \textit{via} l'état lié. Le comportement aux temps longs de $|c_{\epsilon}(t)|^2$ converge vers le profil de raie de Fano:
\begin{equation}
\lim\limits_{t \rightarrow + \infty} |c_{\epsilon}(t)|^2 = |c_{\epsilon}^{(0)}|^2 \frac{\left( \epsilon + q \right) ^2 }{1+ \epsilon^2}
\end{equation}

\paragraph{Calcul du dipôle} \`{A} partir des équations \ref{eq:c_varphi} et \ref{eq:c_eprime}, on peut calculer le dipôle temporel induit pour la fonction d'onde $\ket{\Phi (t)}$ (équation \ref{eq:ExpressionDePhi})
\begin{equation}
<\tilde{d}(t)> = \bra{\Phi(t)} F_z \: \rme^{i \Omega t} z \ket{\Phi(t)}
\end{equation} % \Omega ou \omega pour le champ F ??
Seuls $\ket{g}$ et $\ket{\varphi}$, et $\ket{g}$ et $\ket{\psi_E}$ sont couplés radiativement. On note $ \Omega_r = E_r - E_g$. D'où
\begin{multline}
<\tilde{d}(t)> = F_z \: c_g^{*} \: c_{\varphi}^{(0)} \left( 1-\frac{i}{q} \right) \rme^{-\frac{\Gamma}{2 \hbar} t} \: \rme^{i (\Omega - \Omega_r) t}  \bra{g}  z \ket{\varphi} \\
 + F_z \: c_g^{*} \: c_{\epsilon}^{(0)} \: \rme^{i (\Omega - \Omega_r) t} \int \frac{1}{\epsilon + i} \left[ (q+\epsilon ) \rme^{- i \frac{\Gamma}{2 \hbar} \epsilon t } - (q - i) \rme^{-\frac{\Gamma}{2 \hbar}t} \right] \bra{g} z \ket{\psi_E} \rmd E \\
+ F_z \: c_g \: c_{\varphi}^{(0)*} \left( 1+\frac{i}{q} \right) \rme^{-\frac{\Gamma}{2 \hbar} t} \: \rme^{i (\Omega + \Omega_r) t}  \bra{g}  z \ket{\varphi}^{*} \\
+ F_z \: c_g \: c_{\epsilon}^{(0)*} \: \rme^{i (\Omega + \Omega_r) t} \int \frac{1}{\epsilon - i} \left[ (q+\epsilon ) \rme^{i \frac{\Gamma}{2 \hbar} \epsilon t } - (q + i) \rme^{-\frac{\Gamma}{2 \hbar}t} \right] \bra{g} z \ket{\psi_E}^{*} \rmd E
\label{eq:DipoleRayonneTotal}
\end{multline}
On s'intéresse au champ rayonné au voisinage de la résonance, soit $\Omega \approx \Omega_r$. Les deux derniers termes de l'expression \ref{eq:DipoleRayonneTotal} oscillent à la fréquence $\Omega + \Omega_r \gg \Omega - \Omega_r$ et se moyennent rapidement à zéro. En appliquant l'approximation de l'onde tournante, on peut négliger ces deux derniers termes devant les deux premiers.
\begin{multline}
<\tilde{d}(t)> = F_z \: c_g^{*} \: c_{\varphi}^{(0)} \bra{g}  z \ket{\varphi} \rme^{i (\Omega - \Omega_r) t} \\ \times \left[ \left( 1-\frac{i}{q} \right) \rme^{-\frac{\Gamma}{2 \hbar} t} + \frac{c_{\epsilon}^{(0)}}{c_{\varphi}^{(0)}} \int \frac{(q+\epsilon ) \rme^{- i \frac{\Gamma}{2 \hbar} \epsilon t } - (q - i) \rme^{-\frac{\Gamma}{2 \hbar}t}}{\epsilon + i} \frac{\bra{g} z \ket{\psi_E}}{\bra{g}  z \ket{\varphi}} \rmd E \right]
\end{multline}
En utilisant l'expression \ref{eq:q_en_fonction_des_c0},
\begin{multline}
<\tilde{d}(t)> = F_z \: c_g^{*} \: c_{\varphi}^{(0)} \bra{g}  z \ket{\varphi} \rme^{i (\Omega - \Omega_r) t} \\ \times \left[ \left( 1-\frac{i}{q} \right) \rme^{-\frac{\Gamma}{2 \hbar} t} + \frac{1}{(\pi q V)^2} \underbrace{\int \frac{(q+\epsilon ) \rme^{- i \frac{\Gamma}{2 \hbar} \epsilon t } - (q - i) \rme^{-\frac{\Gamma}{2 \hbar}t}}{\epsilon + i} \rmd E}_{I} \right]
\end{multline}
On rappelle ici le résultat $\int_{- \infty}^{+ \infty} \frac{1}{\epsilon + i} \rmd \epsilon = - \pi i$. Par souci de clarté l'intégrale est calculée ici séparément:
\begin{align}
I & = \int \frac{q + \epsilon + i - i }{\epsilon + i} \: \rme^{- i \frac{\Gamma}{2 \hbar} \epsilon t} \: \rmd E - (q-i) \rme^{-\frac{\Gamma}{2 \hbar}t} \int \frac{\rmd E}{\epsilon + i} \\
& = \frac{\Gamma}{2} \left[ (q-i) \int \frac{\rme^{- i \frac{\Gamma}{2 \hbar} \epsilon t}}{\epsilon + i} \: \rmd \epsilon +  \int \rme^{- i \frac{\Gamma}{2 \hbar} \epsilon t} \: \rmd \epsilon + (q-i) \rme^{-\frac{\Gamma}{2 \hbar}t} \int \frac{\rmd \epsilon}{\epsilon + i} \right]
\end{align}








