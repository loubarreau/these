\part{Délai de photoionisation résonante}
%Blablablablabla
% Police figures = Helvetica 17.5
\chapter{Délai de photoionisation attoseconde}
\label{chap:DelaiPI}
Dans son premier article de l'\textit{annus mirabilis} 1905, Albert Einstein pose les fondements de la mécanique quantique moderne en expliquant l'effet photoélectrique \mycite{Einstein1905}. Lorsqu'un système absorbe un photon d'énergie supérieure à son potentiel d'ionisation, un électron est émis. Pendant plus d'un siècle, l'émission du photoélectron est considérée comme un phénomène instantané. Cependant, le développement de sources de lumière ultra-brèves dans l'XUV a permis des premières mesures de délais de l'ordre de l'attoseconde entre l'émission d'électrons provenant de différents niveaux électroniques. L'interprétation des quantités mesurées est complexe car le temps n'est pas un opérateur de la mécanique quantique mais un paramètre \mycite{Pauli1933}. En réalité, elle fait appel à des notions développées par la théorie des collisions dans les années 1950. Lorsqu'une particule (i.e. un électron) est diffusée par un potentiel , elle est déphasée par rapport à une particule libre. La dérivée spectrale de cette phase est homogène à un temps et peut être interprétée comme un délai entre la particule diffusée et la particule libre.

Dans ce chapitre, nous présenterons les résultats de la théorie de la diffusion par un potentiel qui amènent à la définition du délai de Wigner. Ensuite, nous relierons ce délai aux quantités mesurées dans une expérience de photoionisation multicouleur. Enfin, nous détaillerons les principaux résultats expérimentaux de mesure de délais de photoionisation attoseconde dans les gaz.

\section{Diffusion par un potentiel central}
\label{sec:DiffusionPotentielCentral}
%V(r) à courte portée, équation de Schrodinger et de Schrodinger radiale...
%Pour r tend vers l'infini, les solutions doivent être proches des solutions de l'équation de Schrodinger radiale sans potentiel V.
\subsection{Position du problème}
On considère un paquet d'ondes électronique du continuum (d'énergie $E>0$), superposition d'états "monochromatiques" du continuum avec les amplitudes $a(E,t)$
\begin{equation}
\Psi(t, \bm{r}) = \int_{0}^{+ \infty} {a(E,t)\psi(E, \bm{r}) \rme^{-iEt/\hbar} \rmd E}
\label{eq:PaquetDonde}
\end{equation}
$\Psi$ est solution de l'équation de Schrödinger dépendante du temps 
\begin{equation}
i \hbar \frac{\partial \Psi}{\partial t} = [-\frac{\hbar^2}{2 \mu} \Delta + V(\bm{r})] \Psi(t,\bm{r})
\end{equation}
On suppose ici que le potentiel $V$ est indépendant du temps et correspond uniquement au potentiel d'interaction avec le centre diffusant. On suppose également que le potentiel $V$ est central, c'est-à-dire qu'il ne dépend que de la distance $r$ entre l'électron et le centre diffusant. Les états $\psi$ sont alors solutions de l'équation de Schrödinger non-dépendante du temps, exprimée en fonctions des coordonnées sphériques de l'électron $(r,\theta,\phi)$ dans le référentiel du centre de masse du système, qui s'identifie au c\oe ur ionique dans le cas de l'interaction entre un électron et un ion. $\mu$ correspond à la masse réduite du système qui s'identifie également dans ce cas à la masse de l'électron.\\
Pour étudier la diffusion du paquet d'onde électronique \ref{eq:PaquetDonde} sur le potentiel $V$, il faut connaître les états stationnaires $\psi$.
\begin{align}
[\hat{H_0} + V ] \psi(r,\theta,\phi) = E \psi(r,\theta,\phi)\\
[-\frac{\hbar^2}{2 \mu} \Delta + V(r)] \psi(r,\theta,\phi) = E \psi(r,\theta,\phi)
\end{align}
En coordonnées sphériques, l'opérateur Laplacien s'écrit:
\begin{equation}
\Delta \psi(r,\theta,\phi) = \frac{1}{r} \frac{\partial^2}{\partial r^2} (r \psi) + \frac{1}{r^2} \underbrace{(\frac{1}{\sin \theta} \frac{\partial}{\partial \theta} (\sin \theta \frac{\partial \psi}{\partial \theta}) +  \frac{1}{\sin^2 \theta } \frac{\partial^2 \psi}{\partial \phi^2})}_{-\frac{\hat{L}^2}{\hbar^2}\psi}
\end{equation}
En comparant l'expression précédente à celle de l'opérateur moment angulaire $\hat{L}^2$, on obtient une nouvelle expression pour l'équation aux valeurs propres, faisant apparaître un terme centrifuge:
\begin{equation}
[-\frac{\hbar^2}{2 \mu} \frac{1}{r} \frac{\partial^2}{\partial r^2} r + \frac{1}{2 \mu r^2} \hat{L^2} + V(r)] \psi(r,\theta,\phi) = E \psi(r,\theta,\phi)
\label{eq:Schrod_total}
\end{equation}
L'opérateur moment angulaire $\hat{L}$ n'agissant que sur les variables angulaires $\theta$ et $\phi$, la dépendance angulaire de l'hamiltonien est entièrement contenue dans le terme $\hat{L^2}$.  Cet hamiltonien commute avec les opérateurs $\hat{L^2}$ et $\hat{L_z}$. On peut alors chercher une base de l'espace des états de la particule constituée des fonctions propres communes aux trois observables $\hat{H}$, $\hat{L^2}$ et $\hat{L_z}$ avec les valeurs propres respectives $\hbar^2 k^2 / 2 \mu$, $\ell(\ell+1)\hbar^2$ et $m\hbar$. Les fonctions propres communes à $\hat{L^2}$ et $\hat{L_z}$ sont bien connues et sont les harmoniques sphériques $Y_{\ell}^{m}(\theta, \phi)$. Ainsi les solutions de \ref{eq:Schrod_total} sont à chercher sous la forme \[\psi(r, \theta, \phi) = R(r)Y_{\ell}^{m}(\theta, \phi) \] les fonctions d'onde correspondantes $\psi$ seront appelées ondes partielles. En reportant cette expression dans l'équation \ref{eq:Schrod_total}, on obtient l'équation radiale 
\begin{equation}
(-\frac{\hbar^2}{2 \mu} \frac{1}{r} \frac{\rmd^2}{\rmd r^2} r + \frac{\ell(\ell+1)\hbar^2}{2\mu r^2} + V(r)) R(r) = E R(r)
\label{eq:Schrod_radiale}
\end{equation}

Le cas particulier où $V$ est un potentiel coulombien $(\propto 1/r)$ nécessite un traitement particulier. Néanmoins l'étude des ondes partielles solutions de \ref{eq:Schrod_total} dans le cas d'un potentiel $V$ à courte portée, c'est-à-dire décroissant plus rapidement que $1/r$ à l'infini, permet une approche du concept de déphasage dû à la diffusion sur le potentiel $V$.

Dans la suite de ce paragraphe, nous allons d'abord étudier les solutions de \ref{eq:Schrod_total} dans le cas où $V$ est identiquement nul, les ondes sphériques libres. Nous étudierons ensuite les solutions dans le cas ou le potentiel $V$ est un potentiel à courte portée, et leur comportement asymptotique à longue distance $r$. Nous mettrons en évidence l'existence d'un déphasage $\delta_{E,\ell}$ entre les ondes sphériques libres et les ondes partielles introduit par la diffusion sur le potentiel $V$. Ce déphasage nous conduira à la définition des délais de Wigner.

\subsection{Les ondes sphériques libres}
\label{par:OndesSpheriquesLibres}
%Cohen p915 et 931
\`{A} longue distance $r$ du centre diffusant, on s'attend à ce que l'électron ne ressente quasiment pas les effets du potentiel à courte portée $V$. Les solutions de \ref{eq:Schrod_total} doivent donc avoir un comportement asymptotique similaire aux ondes sphériques libres $\psi^{(0)}(r,\theta,\phi) = R^{(0)}(r) Y_{\ell}^{m}(\theta, \phi)$, fonctions propres de l'hamiltonien $\hat{H_0}$. Il s'agit donc de résoudre l'équation radiale \ref{eq:Schrod_radiale} pour $V = 0$:
\begin{equation}
(-\frac{\hbar^2}{2 \mu} \frac{1}{r} \frac{\rmd^2}{\rmd r^2} r + \frac{\ell(\ell+1)\hbar^2}{2\mu r^2}) R^{(0)}(r) = E R^{(0)}(r)
\label{eq:Schrod_V0}
\end{equation}

On peut montrer, par récurrence (voir par exemple \mycite{CohenT2} Chapitre VIII) ou bien en remarquant que l'équation radiale se ramène à l'équation de Bessel sphérique, que les solutions de \ref{eq:Schrod_V0} sont de la forme
\begin{equation}
\psi^{(0)}_{k,\ell,m}(r,\theta,\phi) = \sqrt{\frac{2k^2}{\pi}} j_{\ell}(kr) Y_{\ell}^{m}(\theta, \phi)
\end{equation}
avec $k$,$\ell$ et $m$ paramétrant les valeurs propres de l'hamiltonien sans potentiel $\hat{H_0}$, $\hat{L^2}$ et $\hat{L_z}$:
\[ \hat{H_0} \psi^{(0)}_{k,\ell,m}(r,\theta,\phi) = \frac{\hbar^2 k^2}{2 \mu}\psi^{(0)}_{k,\ell,m}(r,\theta,\phi)\]
\[ \hat{L^2} \psi^{(0)}_{k,\ell,m}(r,\theta,\phi) = \ell(\ell+1)\hbar^2 \psi^{(0)}_{k,\ell,m}(r,\theta,\phi) \]
\[ \hat{L_z} \psi^{(0)}_{k,\ell,m}(r,\theta,\phi) = m \hbar \psi^{(0)}_{k,\ell,m}(r,\theta,\phi) \]
et $j_{\ell}$ une fonction de Bessel sphérique définie par
\[j_{\ell} (\rho) = (-1)^{\ell} \rho^{\ell} (\frac{1}{\rho} \frac{\rmd}{\rmd\rho})^{\ell} \frac{\sin \rho}{\rho}\]
Les trois premières fonctions de Bessel sphériques $j_0$,$j_1$ et $j_2$ sont représentées figure \ref{fig:Bessel_Spheriques}. Remarquons que la fonction $j_0$ s'identifie à la fonction sinus cardinal.

\begin{figure}
\centering
\def\svgwidth{\columnwidth}
\import{Figures/DelaisPI/}{Bessel_Spheriques.pdf_tex}
\caption{Fonctions de Bessel sphériques $j_\ell(\rho)$ (a) et $\rho^2 j_\ell^2(\rho)$, donnant la dépendance radiale de la probabilité de présence dans l'état $\ket{\psi^{(0)}_{k,\ell,m}}$ (b) pour $\ell = 0, 1, 2$.}
\label{fig:Bessel_Spheriques}
\end{figure}

La dépendance angulaire de l'onde sphérique libre est contenue dans l'harmonique sphérique $Y_{\ell}^{m}(\theta, \phi)$. Elle est donc fixée par les nombres quantiques $\ell$ et $m$ et non par l'énergie ($\propto k^2$). Si l'on se fixe une direction $(\theta_0,\phi_0)$, la probabilité de trouver la particule dans l'état $\ket{\psi^{(0)}_{k,\ell,m}}$ dans un angle solide $\rmd\Omega_0$ autour de $(\theta_0,\phi_0)$ et entre $r$ et $r+\rmd r$ est proportionnelle à  
$r^2 j_\ell^2 (kr) \mid Y_{\ell}^{m}(\theta_0, \phi_0) \mid ^2 \rmd r \rmd \Omega_0 $. La fonction $\rho^2 j_\ell^2(\rho)$ est représentée figure \ref{fig:Bessel_Spheriques}. Cette fonction prend des valeurs faibles pour $\rho < \sqrt{\ell(\ell+1)}$. La probabilité de présence de la particule dans l'état $\ket{\psi^{(0)}_{k,\ell,m}}$ est donc quasiment nulle pour $r < \frac{1}{k} \sqrt{\ell(\ell+1)}$. Cette distance critique peut être interprétée semi-classiquement comme un paramètre d'impact.

\paragraph*{Comportement asymptotique}
On cherche à déterminer le comportement de la fonction d'onde à longue distance du centre diffusant $\psi^{(0)}_{k,\ell,m} (r \rightarrow + \infty, \theta, \phi)$, c'est-à-dire le comportement asymptotique des fonctions de Bessel sphériques.\\
En appliquant une première fois l'opérateur $(\frac{1}{\rho} \frac{\rmd}{\rmd\rho})$ à la fonction $ \frac{\sin \rho}{\rho}$, $j_\ell (\rho)$ s'écrit
\[  j_{\ell} (\rho) = (-1)^{\ell} \rho^{\ell} (\frac{1}{\rho} \frac{\rmd}{\rmd\rho})^{\ell-1} [ \frac{\cos \rho}{\rho^2} - \frac{\sin \rho}{\rho^3}] \]
Pour $\rho \rightarrow + \infty$, $ \frac{\sin \rho}{\rho^3} \ll \frac{\cos \rho}{\rho^2}$. Si l'on applique une nouvelle fois l'opérateur $(\frac{1}{\rho} \frac{\rmd}{\rmd\rho})$, le terme dominant viendra encore de la dérivée du cosinus. Ainsi, 
\[ j_{\ell} (\rho \rightarrow + \infty) \sim (-1)^{\ell} \rho^{\ell} \frac{1}{\rho^\ell} \frac{1}{\rho} (\frac{\rmd}{\rmd\rho})^\ell \sin \rho \]
Avec $(\frac{\rmd}{\rmd\rho})^\ell \sin \rho = (-1)^\ell \sin (\rho - \ell \frac{\pi}{2})$, on obtient finalement
\[ j_{\ell} (\rho \rightarrow + \infty) \sim \frac{1}{\rho} \sin (\rho - \ell \frac{\pi}{2})  \]
Le comportement asymptotique de l'onde sphérique libre $\psi^{(0)}_{k,\ell,m} (r, \theta, \phi)$ est donc: 
%(1.8)
\begin{equation}
\setlength\fboxrule{0.5pt}
\boxed{
\psi^{(0)}_{k,\ell,m} (r \rightarrow + \infty, \theta, \phi) \sim \sqrt{\frac{2k^2}{\pi}} Y_{\ell}^{m}(\theta, \phi)  \frac{\sin(kr - \ell \pi / 2)}{kr}
}
\end{equation}
fonction réelle que l'on peut écrire sous forme complexe pour interpréter son expression sous forme d'ondes
%(1.9)
\begin{equation}
\setlength\fboxrule{0.5pt}
\boxed{
\psi^{(0)}_{k,\ell,m} (r \rightarrow + \infty, \theta, \phi) \sim - \sqrt{\frac{2k^2}{\pi}} Y_{\ell}^{m}(\theta, \phi)  \frac{\rme^{-ikr}\rme^{i\ell \frac{\pi}{2}} - \rme^{ikr}\rme^{-i\ell \frac{\pi}{2}}}{2ikr}
}
\label{eq:Onde_sph_libre}
\end{equation}
Pour $r \rightarrow + \infty$, $\psi^{(0)}_{k,\ell,m}$ est donc la superposition d'une onde sphérique entrante $e^{-ikr}/r$ et d'une onde sphérique sortante  $e^{+ikr}/r$ dont la phase relative est $\ell \pi$.


\subsection{Les ondes partielles}
\label{par:OndesPartielles}
%Cohen p919
On s'intéresse désormais à la résolution de l'équation \ref{eq:Schrod_radiale} dans le cas général d'un potentiel central $V(r)$ à courte portée, c'est-à-dire décroissant plus rapidement que $1/r$ pour $r \rightarrow + \infty$.

En posant $R(r) = \frac{1}{r} u(r)$, l'équation \ref{eq:Schrod_radiale} devient
\begin{equation}
[- \frac{\hbar^2}{2\mu} \frac{\rmd^2}{\rmd r^2} + \frac{\ell(\ell+1)}{2 \mu r^2} + V(r)] u(r)  = \frac{\hbar^2 k^2}{2\mu} u(r)
\label{eq:Scrod_radiale_u}
\end{equation}
à laquelle il faut ajouter la condition initiale $u(r=0) = 0$. 

\paragraph*{Comportement asymptotique} Pour $r \rightarrow + \infty$, le potentiel centrifuge et le potentiel $V(r)$ à courte portée sont négligeables et l'équation précédente devient
\begin{equation}
\frac{\rmd^2 u}{\rmd r^2} + k^2 u(r) \backsimeq 0
\end{equation}
dont la solution générale est de la forme
\[ u(r \rightarrow + \infty) \sim A \cos(kr) + B \sin(kr) \]
Si le potentiel $V$ est réel, on peut trouver des solutions $u$ réelles et donc choisir les constantes $A, B \in \mathbb{R}$. On peut alors réécrire 
\begin{equation}
u(r \rightarrow + \infty) \sim \sqrt{A^2+B^2} (\sin \beta_\ell \cos(kr) + \cos \beta_\ell \sin(kr) )
\end{equation}
Avec $\sin \beta_\ell = \frac{A}{\sqrt{A^2+B^2}}$ et $\cos \beta_\ell = \frac{B}{\sqrt{A^2+B^2}}$, soit
\begin{equation}
u(r \rightarrow + \infty) \sim C \sin(kr - \beta_\ell)
\end{equation}
La phase $\beta_\ell$ est déterminée par continuité de la solution de \ref{eq:Scrod_radiale_u} s'annulant en $r=0$. Dans le cas d'un potentiel $V$ identiquement nul, nous avons montré précédemment que la phase $\beta_\ell$ est égale à $\ell \pi/2$. On peut alors choisir cette valeur comme référence en définissant le déphasage $\delta_{E,\ell}$ tel que:
\[ u(r \rightarrow + \infty) \sim C \sin(kr -\ell \frac{\pi}{2} + \delta_{E,\ell}) \]
$\delta_{E,\ell}$ dépend de $\ell$ et de $k$, c'est-à-dire du moment angulaire et de l'énergie.

\paragraph*{Interprétation physique du déphasage} En injectant l'expression précédente de $u$ dans l'expression générale de la fonction d'onde, on obtient l'expression asymptotique de l'onde partielle $\psi_{k,\ell,m} (r \rightarrow + \infty, \theta, \phi)$:
\begin{equation}
\psi_{k,\ell,m} (r \rightarrow + \infty, \theta, \phi) \sim C Y_{\ell}^{m}(\theta, \phi)  \frac{\sin(kr - \ell \pi / 2 + \delta_{E,\ell})}{r}
\end{equation}
ou encore, en multipliant par un facteur de phase $\rme^{i \delta_{E,\ell}}$ et en choisissant la constante $C$ pour faciliter la comparaison avec l'expression asymptotique de l'onde sphérique libre \ref{eq:Onde_sph_libre}
\begin{equation}
\setlength\fboxrule{0.5pt}
\boxed{
\psi_{k,\ell,m} (r \rightarrow + \infty, \theta, \phi) \sim - Y_{\ell}^{m}(\theta, \phi)  \frac{\rme^{-ikr}\rme^{i\ell \frac{\pi}{2}} - \rme^{ikr}\rme^{-i\ell \frac{\pi}{2}} \rme^{2 i \delta_{E,\ell}}}{2ikr}
}
\label{eq:Onde_partielle}
\end{equation}

De la même manière que pour le cas de l'onde sphérique libre \ref{eq:Onde_sph_libre}, l'onde partielle est pour $r \rightarrow + \infty$ la superposition d'une onde sphérique entrante $e^{-ikr}/r$ et d'une onde sphérique sortante  $e^{+ikr}/r$ déphasée de $\ell \pi + 2 \delta_{E,\ell}$. Au facteur de normalisation près, on peut interpréter cette expression comme suit. L'onde entrante de départ est identique à celle du cas de la particule libre, et s'approche de la zone d'action du potentiel $V$ en étant de plus en plus perturbée par le potentiel. Après avoir rebroussé chemin et s'être transformée en onde sortante, elle a accumulé un déphasage $2 \delta_{E,\ell}$ par rapport à l'onde sortante libre qui aurait été obtenue dans le cas $V=0$. Ce déphasage $\delta_{E,\ell}$ est une quantité extrêmement importante. En effet, il caractérise tout l'effet du potentiel sur la particule de moment cinétique $\ell$ et d'énergie $E$. Par exemple, il est possible d'exprimer la section efficace de diffusion en fonction de $\delta_{E,\ell}$ \mycite{CohenT2}.

\begin{figure}
\centering
\def\svgwidth{\columnwidth}
\import{Figures/DelaisPI/}{Schema_delais_Wigner.pdf_tex}
\caption{Illustration de la diffusion d'un paquet d'onde sur un potentiel attractif. Adapté de \mycite{ArgentiPRA2017}}
\label{fig:Schema_Wigner}
\end{figure}

La théorie de la diffusion par un potentiel central exposée précédemment provient de la théorie des collisions: une onde se déplaçant dans le sens des $r$ positifs "de la gauche vers la droite", et diffusée par un potentiel situé en $r=0$ possède en $r \rightarrow + \infty$ un déphasage $2 \delta_{E,\ell}$ par rapport à la même onde qui se serait propagée sans potentiel. %Dans le contexte de la photoionisation, l'émission de l'électron est considérée comme une \textit{demi-collision}. Ainsi, la phase de diffusion en photoionisation est la phase du paquet d'onde électronique émis par rapport au paquet d'onde qui aurait été émis en l'absence de potentiel, soit uniquement $\delta_{E,\ell}$.

\paragraph*{Cas de la diffusion sur un potentiel Coulombien} Le potentiel Coulombien créé par un ion de charge $Z$ est proportionnel à $Z/r$, le traitement analytique précédent des potentiels à courte portée ne s'applique pas. Cependant, on peut montrer que la phase de diffusion sur un potentiel de Coulomb s'exprime comme: \mycite{Friedrich} %p.22
\begin{equation}
\setlength\fboxrule{0.5pt}
\boxed{
\Phi_{k,\ell}(r) = \frac{Z \: \ln 2kr}{k} - \ell \: \frac{\pi}{2} + \underbrace{\sigma_{k,\ell} + \delta_{k,\ell}}_{\eta_{k,\ell}}
}
\label{eq:dephasage_Coulomb}
\end{equation}
avec $\delta_{k,\ell}$ décrivant le déphasage asymptotique avec une onde libre de Coulomb et
\begin{equation}
\sigma_{k,\ell} = \arg [ \: \Gamma(\ell + 1 - \frac{i Z}{k} )]
\end{equation}
la fonction Gamma $\Gamma$ étant définie par
\begin{equation}
\Gamma (z+1) = \int_{0}^{+ \infty} {t^z \: \rme^{-t} \: \rmd t}
\label{eq:FonctionGamma}
\end{equation}
La phase $\Phi_{k,\ell}(r)$ dépend de la distance $r$ au centre diffusant. La longue portée du potentiel Coulombien déforme la fonction d'onde, même à grande distance de l'ion.

\subsection{Délai de Wigner}
%Définition
%Dalhstrom p27
%Cas du potentiel Coulombien (longue portée), Friedrich p21 ou calcul avec approx WKB Dahlstrom p26
D'après ce qui précède, on peut réécrire pour la partie radiale du paquet d'onde électronique diffusé \ref{eq:PaquetDonde}
\begin{equation}
r \Psi(t, r) \propto \int_{0}^{+ \infty} {\mid A(E) \mid \rme^{i(kr+2\delta_{E,\ell})} \rme^{-iEt/\hbar} \rmd E}
\label{eq:POEdiffusé}
\end{equation}
qui est déphasée de $\delta_{E,\ell}$ par rapport au même paquet d'onde électronique non diffusé par le potentiel
\begin{equation}
r \Psi_{V=0}(t, r) \propto \int_{0}^{+ \infty} {\mid A(E) \mid \rme^{ikr} \rme^{-iEt/\hbar} \rmd E}
\label{eq:POEnondiffusé}
\end{equation}
Ces deux intégrales contiennent des termes oscillant rapidement avec l'énergie. La plus importante contribution sera donc apportée par les points où la phase est stationnaire. En dérivant les phases de \ref{eq:POEdiffusé} et \ref{eq:POEnondiffusé}, on obtient les relations suvantes:
\begin{align*}
r \frac{\rmd k}{\rmd E} + 2 \frac{\rmd \delta_{E,\ell}}{\rmd E}- \frac{t}{\hbar} = 0 \\
r \frac{\rmd k}{\rmd E} - \frac{t}{\hbar} = 0
\end{align*}
soit 
\begin{align*}
t = \frac{r}{v} + 2 \hbar \frac{\rmd \delta_{E,\ell}}{\rmd E}\\
t = \frac{r}{v}
\end{align*}
Le paquet d'onde électronique diffusé est temporellement décalé d'une quantité
\begin{equation}
\setlength\fboxrule{1pt}
\boxed{
\tau_W = 2 \hbar \frac{\rmd \delta_{E,\ell}}{\rmd E}
}
\end{equation} 
par rapport au paquet d'onde électronique libre.\\
$\tau_W$ est appelé délai de Wigner (ou Eisenbud-Wigner-Smith), d'après les physiciens qui mirent en évidence cette relation entre les délais et les phases de diffusion \mycite{Eisenbud} \mycite{WignerPR1955} \mycite{SmithPR1960}. 

\paragraph{Exemple: Délai de Wigner pour un potentiel faible}
En utilisant l'approximation semi-classique de Brillouin-Kramers-Wentzel pour résoudre l'équation de Schrödinger pour un électron soumis à un potentiel $V$, on montre que le déphasage $\delta_{E,\ell}$ peut se mettre sous la forme \mycite{Friedrich} \mycite{DahlstromJPB2012}
\begin{equation}
\delta_{E,\ell} (E) = \frac{1}{\hbar} \lim_{x \to + \infty} \int_{- \infty}^{x} { [\sqrt{2m(E-V(x')}-\sqrt{2mE}] \rmd x'}
\end{equation}
Pour un potentiel faible ($V \ll E $), $\sqrt{2m(E-V(x')} \approx \sqrt{2mE}(1-\frac{V(x')}{2E})$, d'où
\begin{equation}
\delta_{E,\ell} (E) \approx - \frac{1}{\hbar} \sqrt{\frac{m}{2E}} \int_{- \infty}^{+ \infty} { V(x')\rmd x'} = - \frac{1}{\hbar} \sqrt{\frac{m}{2E}} I_V
\end{equation}
où l'on définit l'intégrale du potentiel $I_V$, indépendante de l'énergie.
On calcule alors le délai de Wigner correspondant
\begin{equation}
\tau_W = \hbar \frac{\rmd \delta_{E,\ell}}{\rmd E} = \sqrt{\frac{m}{8}} \frac{I_V}{E^{3/2}}
\end{equation}
Pour un potentiel attractif, $I_V < 0$ donc $\tau_W < 0$. L'électron diffusé par le potentiel est en avance sur l'électron libre. En effet, à énergie constante, l'énergie cinétique de l'électron est plus grande lorsqu'il passe au voisinage d'un potentiel attractif par rapport à un potentiel nul. Le délai introduit est d'autant plus grand que le potentiel est important.
On remarque également que le délai de Wigner est proportionnel à $E^{-3/2}$, c'est-à-dire qu'un électron sera plus affecté par le potentiel si son énergie est faible.


\section{Mesure de délais de photoionisation}
%Longtemps considéré comme accessible uniquement \textit{via} des expériences de pensée, le développement de sources XUV ultra-brèves a permis les premières mesures de délais de photoionisation durant les dix dernières années. Dans ce paragraphe, nous présentons les hypothèses sous-tendant la détermination expérimentale de délais de photoionisation attoseconde ainsi que les principaux résultats dans ce domaine.
\subsection{Délai de photoionisation et matrice de transition à un photon}
Lors de la photoionisation, l'électron est libéré dans le potentiel de l'ion et diffuse hors de celui-ci. La mesure du délai de photoionisation consiste à déterminer le temps mis par l'électron pour être ionisé, c'est-à-dire le délai entre l'absorption du photon et le départ de l'électron. Comment relier ce délai au délai de Wigner et à la phase de diffusion? \mycite{DahlstromJPB2012} \mycite{GuenotPRA2012}\mycite{ArgentiPRA2017}

Considérons le cas d'un atome dans son état fondamental $\ket{g}$ ionisé par une impulsion XUV $\mathcal{E}(t)$, centrée en $t_{XUV}=0$. $\mathcal{E}$ est polarisée selon $z$, longue et limitée par transformée de Fourier; son spectre est centré en $E_0 - E_g$, où $E_0$ et $E_g$ sont respectivement les énergies de l'état final et fondamental. La transformée de Fourier de $\mathcal{E}(t)$ est notée $\tilde{\mathcal{E}}(\Omega)$. On choisit comme fonctions propres du continuum (d'énergie $E = \hbar^2 k^2 / 2m >0$) en absence de champ électrique les fonctions $\ket{\psi_{\vec{k}}^{-}}$. Le paquet d'onde électronique ionisé par l'impulsion s'écrit alors
\begin{equation}
\ket{\Psi (t)} = - i \sqrt{2\pi} \int{ \ket{\psi_{\vec{k}}^{-}} \: \rme^{-iEt} \: M^{(1)}_{k} \rmd E}
\end{equation}
où $M^{(1)}_{k}$ est l'élément de matrice de transition dipolaire entre $\ket{g}$ et $\ket{\psi_{\vec{k}}}$. Au premier ordre de la théorie des perturbations et dans l'approximation dipolaire, on a 
\begin{equation}
 M^{(1)}_{k} = i \bra{\psi_{\vec{k}}} -e \: z \: \tilde{\mathcal{E}}(\Omega)\ket{g} = -i \: e \: \tilde{\mathcal{E}}(\Omega) \bra{\psi_{\vec{k}}} z \ket{g}
 \label{eq:M1phot}
\end{equation}
La fonction d'onde de l'état fondamental est séparée en une partie radiale et une partie angulaire: 
\begin{equation}
\psi_g(\vec{r}) = R_{n_g,\ell_g}(r)Y_{\ell_g}^{m_g}(\theta, \phi)
\label{eq:M1phot_g}
\end{equation}
Nous avons vu précédemment que les fonctions $\ket{\psi_{\vec{k}}}$ ne sont pas des ondes planes, mais des ondes de diffusion. On utilise leur décomposition en ondes partielles \mycite{LandauLifshitz} %p520
\begin{equation}
\psi_{\vec{k}}(\vec{r}) = (8 \pi)^{3/2} \: \sum_{L,M} i^L \rme^{-i \delta_{k,L}} \: Y_{L}^{M \: \ast} (\vec{k}) \: Y_{L}^{M} (\theta, \phi) R_{k,L}(r)
\label{eq:M1phot_k}
\end{equation}
En injectant \ref{eq:M1phot_g} et \ref{eq:M1phot_k} dans \ref{eq:M1phot}, en utilisant l'expression de l'opérateur en coordonnées sphériques $ z = r \cos \theta $ et en séparant les parties angulaire et radiale, il vient
\begin{multline}
 M^{(1)}_{k} = -i \: e \: \tilde{\mathcal{E}}(\Omega) (8 \pi)^{3/2} \sum_{L,M} i^{-L} \rme^{i \delta_{k,L}} \: Y_{L}^{M \: \ast} (\vec{k}) \bra{Y_{L}^{M} (\theta, \phi)} \cos \theta \ket{Y_{\ell_g}^{m_g}(\theta, \phi)} \times \\\bra{R_{k,L}(r)} r \ket{R_{n_g,\ell_g}(r)}
\end{multline}
Comme $\cos \theta = \sqrt{\frac{4 \pi}{3}} Y_1^0(\theta) $, on obtient en fonction des symboles $3j$ de Wigner \mycite{LandauLifshitz}%p412
\begin{multline}
\bra{Y_{L}^{M} (\theta, \phi)} Y_1^0(\theta) \ket{Y_{\ell_g}^{m_g}(\theta, \phi)} = (-1)^{-M} \: \sqrt{2 L +1} \sqrt {2 \ell_g +1} \times \\ \begin{pmatrix}
   L & 1 & \ell_g \\
   0 & 0 & 0 
\end{pmatrix} 
\begin{pmatrix}
   L & 1 & \ell_g \\
   -M & 0 & m_g 
\end{pmatrix}
\end{multline}
Les règles de sélection pour la transition à un photon imposent $L = \ell_g \pm 1$ et $M = m_g$. Le moment de transition $M^{(1)}_{k}$ s'écrit alors
\begin{multline}
 M^{(1)}_{k} \propto \sum_{L = \ell_g \pm 1; M = m_g} (-1)^{-M} i^{-L} \rme^{i \delta_{k,L}} \: Y_{L}^{M \: \ast} (\vec{k}) \sqrt{2 L +1} \sqrt {2 \ell_g +1} \\
\times
\begin{pmatrix}
   L & 1 & \ell_g \\
   0 & 0 & 0 
\end{pmatrix} 
\begin{pmatrix}
   L & 1 & \ell_g \\
   -M & 0 & m_g 
\end{pmatrix} \bra{R_{k,L}(r)} r \ket{R_{n_g,\ell_g}(r)}
\label{eq:M1phot_final}
\end{multline}
L'élément de matrice radial ainsi que les symboles $3j$ sont réels \mycite{CohenT2}. Si l'on ne considère qu'un unique canal d'ionisation, cet élément de matrice se simplifie. Cette situation se rencontre dans le cas de l'ionisation d'un électron $s$ ou bien si la transition vers $L = \ell_g + 1$ est favorisée par rapport à la transition vers $L = \ell_g - 1$ \mycite{FanoPRA1985}. On a alors la relation simple entre l'élément de matrice de transition et la phase de diffusion de l'état du continuum:
\begin{equation}
\delta_{k,L} = \arg M^{(1)}_{k} 
\end{equation}
Soit le délai de photoionisation \footnote{Notons que bien que le délai de photoionisation soit relié à l'élément de matrice de transition, il n'est pas correct d'interpréter ce dernier comme "le temps qu'il faut pour absorber un photon", ce phénomène étant instantané. Pour éviter cette confusion, un autre choix de fonctions du continuum, les fonctions d'onde \textit{sélectionnées par la transition}, permet d'avoir un élément de transition réel et la phase est alors uniquement contenue dans l'expression de la fonction d'onde \mycite{GaillacPRA2016}.}
\begin{equation}
\setlength\fboxrule{0.5pt}
\boxed{
\tau = \frac{\partial \arg M^{(1)}_{k}}{\partial \omega} = \frac{1}{2} \tau_W
}
\end{equation}
\`A un facteur 2 près, le délai de photoionisation à un photon coïncide avec le délai de Wigner. La photoionisation peut alors être interprétée comme une demie collision. Ce délai est bien plus faible que la limite de résolution des détecteurs électroniques utilisés expérimentalement. Il ne peut donc pas être mesuré directement, et a longtemps été considéré comme accessible uniquement \textit{via} des expériences de pensée.

\subsection{Matrice de transition à deux photons}
\label{sec:Matrice2photons}
Les premières mesures de délais de photoionisation ont été possibles grâce au développement de techniques d'interférométrie électronique attoseconde telles que le RABBIT \mycite{VeniardPRL1995}\mycite{PaulScience2001} ou le streaking attoseconde \mycite{KienbergerScience2004}\mycite{GoulielmakisScience2004}. Ces deux méthodes font intervenir un second champ électrique sonde dans le processus de photoionisation. Comment obtenir les informations sur l'ionisation à un photon à partir de mesures à deux photons? Par la suite nous nous limiterons au cas de l'interférométrie RABBIT, détaillée au chapitre \textbf{ref}, pour lequel le champ d'habillage peut être traité de manière perturbative. On utilise le formalisme et les développements de Marcus Dahlström et des théoriciens du LCPMR Paris \mycite{DahlstromChemPhys2013}.

\begin{figure}
\centering
\def\svgwidth{0.7\columnwidth}
\import{Figures/DelaisPI/}{Schema_2photons.pdf_tex}
\caption{Ionisation à deux photons XUV-IR depuis l'état initial d'énergie $E_g$ vers un état du continuum d'énergie $E_{\vec{k}}$. Deux chemins quantiques mènent à $\vec{k}$: le photon d'énergie $\Omega$ peut être absorbé en premier (a) ou en second (b), ce deuxième chemin étant de contribution négligeable par rapport au premier.}
\label{fig:Schema_2photons}
\end{figure}

On considère l'ionisation à deux couleurs décrite par le schéma \ref{fig:Schema_2photons}: le système d'énergie initiale $E_g$ est ionisé par l'absorption d'un photon d'énergie $\Omega$ et d'un photon d'énergie $\omega$. En général dans l'interférométrie RABBIT on a $\Omega > \omega$, le premier étant un photon XUV et le second un photon IR. Le photon IR peut être absorbé en premier ou en second. Cependant, il existe très peu d'états sous le seuil d'ionisation, ce qui rend l'absorption du photon IR en premier beaucoup moins probable qu'en second. On se concentrera donc par la suite sur les processus à deux photons décrits par le schéma \ref{fig:Schema_2photons} (a).\\
Les champs XUV et IR sont polarisés selon la même direction $z$ qui est choisie comme axe de quantification. Au second ordre de la théorie des perturbations, l'élément de transition à deux photons correspondant à l'absorption du photon XUV suivie du photon IR s'écrit comme la somme-intégrale sur tous les états intermédiaires d'énergie $E_\nu$ (discrets pour $E_\nu < 0$ et du continuum pour $E_\nu > 0$):
\begin{equation}
M_{\vec{k}}^{(2)} = \frac{1}{i} \: \mathcal{E}_\Omega \mathcal{E}_\omega \: \lim \limits_{\epsilon \rightarrow 0^+} \sum_{\nu} \! \! \! \! \! \! \! \! \! \int \frac{\bra{\vec{k}} z\ket{\nu} \bra{\nu} z \ket{g}} {E_g + \Omega - E_{\nu} + i \epsilon}
\label{eq:MatriceDeuxPhotons}
\end{equation}
De la même manière que précédemment, on utilise la décomposition en ondes partielles des fonctions d'onde du continuum \ref{eq:M1phot_k} et la séparation de la fonction d'onde de l'état fondamental en partie radiale et angulaire \ref{eq:M1phot_g}. 
\begin{multline}
M_{\vec{k}}^{(2)} = \frac{4 \pi}{3 i} (8 \pi)^{3/2} \: \mathcal{E}_\Omega \mathcal{E}_\omega \:  \sum_{L, M} (-i)^L \rme^{i \eta_L (\vec{k})} Y_{L}^{M}(\vec{k}) \sum_{\lambda, \mu} \bra{Y_{L}^{M}} Y_{1}^{0} \ket{Y_{\lambda}^{\mu}} \bra{Y_{\lambda}^{\mu}} Y_{1}^{0} \ket{Y_{\ell_g}^{m_g} } \times \\
\biggl[ \sum_{\nu, E_\nu <0} \frac{\bra{R_{k,L}} r \ket{R_{\nu,\lambda}} \bra{R_{\nu,\lambda}} r \ket{R_{n_g,\ell_g}}} {E_g + \Omega - E_{\nu}} \\
+ \lim \limits_{\epsilon \rightarrow 0^+} \int_{0}^{+\infty} \rmd E_\nu \frac{\bra{R_{k,L}} r \ket{R_{\nu,\lambda}} \bra{R_{\nu,\lambda}} r \ket{R_{n_g,\ell_g}}} {E_g + \Omega - E_{\nu} +i \epsilon} \biggr]
\label{eq:M2phot}
\end{multline}
Les états intermédiaires sont caractérisés par les nombres quantiques $\nu$, $\lambda$ et $\mu$. Les moments angulaires intermédiaire et final obéissent aux règles de sélection, rendant accessibles par transition dipolaire uniquement les états $\lambda = \ell_g \pm 1$; $L = \ell_g$ ou $\ell_g \pm 2$ et $M = \mu = m_g$. Le terme entre crochets dans \ref{eq:M2phot} sera noté par la suite $T_{L,\lambda,\ell_g}(k)$. Il correspond à la partie radiale de l'amplitude de transition, dans laquelle on a séparé la contribution des états discrets sous le seuil d'ionisation (dans la somme) des états du continuum (dans l'intégrale). Dans l'interférométrie RABBIT, l'énergie du photon XUV est supérieure au potentiel d'ionisation de l'atome $\Omega > |E_g|$, donc supérieure à l'énergie des états discrets sous le seuil. On a alors le dénominateur du premier terme $E_g + \Omega - E_\nu$ positif et grand, ce qui rend la contribution des états discrets dans $T_{L,\lambda,\ell_g}(k)$ négligeable devant celle des états du continuum. Dans l'intégrale, le dénominateur est imaginaire pur pour le moment $\kappa$ tel que $E_\kappa = \hbar^2 \kappa^2 / 2m = E_g + \Omega$. On obtient alors pour l'intégrale l'expression suivante, où $ \mathcal{P}$ représente la valeur principale de Cauchy:
\begin{multline}
 \lim \limits_{\epsilon \rightarrow 0^+} \int_{0}^{+\infty} \rmd E_\nu \frac{\bra{R_{k,L}} r \ket{R_{\nu,\lambda}} \bra{R_{\nu,\lambda}} r \ket{R_{n_g,\ell_g}}} {E_g + \Omega - E_{\nu} +i \epsilon} = \\
 \mathcal{P} \int_{0}^{+\infty} \rmd E_\nu \frac{\bra{R_{k,L}} r \ket{R_{\nu,\lambda}} \bra{R_{\nu,\lambda}} r \ket{R_{n_g,\ell_g}}} {E_g + \Omega - E_{\nu}} - i \pi \bra{R_{k,L}} r \ket{R_{\kappa,\lambda}} \bra{R_{\kappa,\lambda}} r \ket{R_{n_g,\ell_g}}
\end{multline}

Le calcul de $T_{L,\lambda,\ell_g}(k)$, et en particulier de sa phase, est complexe analytiquement et numériquement. M. Dahlström et collaborateurs \mycite{DahlstromChemPhys2013} ont donc développé une approximation pour calculer $T_{L,\lambda,\ell_g}(k)$, reposant sur l'analyse du comportement asymptotique des fonctions radiales, de manière similaire à l'analyse présentée aux paragraphes \ref{par:OndesSpheriquesLibres} et  \ref{par:OndesPartielles}.

\paragraph*{Comportement asymptotique} On réécrit $T_{L,\lambda,\ell_g}(k)$ sous la forme
\begin{equation}
T_{L,\lambda,\ell_g}(k) = \bra{R_{k,L}} r \ket{\rho_{\kappa,\lambda}}
\label{eq:ExpressionDeT}
\end{equation}
où $\rho_{\kappa,\lambda}(r)$ est une fonction d'onde perturbée dont l'expression est donnée par identification avec \ref{eq:M2phot}
\begin{multline}
\rho_{\kappa,\lambda}(r) = \sum_{\nu, E_\nu <0} \frac{R_{\nu,\lambda} (r)  \bra{R_{\nu,\lambda}} r \ket{R_{n_g,\ell_g}}} {E_\kappa - E_{\nu}} \\ + \mathcal{P} \int_{0}^{+\infty} \rmd E_\nu \frac{R_{\nu,\lambda}(r) \bra{R_{\nu,\lambda}} r \ket{R_{n_g,\ell_g}}} {E_\kappa - E_{\nu}} - i \pi R_{\kappa,\lambda}(r) \bra{R_{\kappa,\lambda}} r \ket{R_{n_g,\ell_g}}
\label{eq:ExpressionDeRho}
\end{multline}
Dans le cas de la diffusion sur un potentiel coulombien, on a montré (paragraphe \ref{sec:DiffusionPotentielCentral}) que pour $r \rightarrow + \infty$ la partie radiale de la fonction d'onde de l'état final s'écrit
\begin{equation}
R_{k,L}(r \rightarrow + \infty) \sim \frac{C_k}{r} \sin (kr + \Phi_{k,L}(r))
\label{eq:fonctionAsymptotiqueContinnum}
\end{equation}
où le déphasage $\Phi_{k,L}(r)$ est donné par l'équation \ref{eq:dephasage_Coulomb}.\\
D'après le paragraphe précédent, la contribution des états discrets à $\rho_{\kappa,\lambda}(r)$ est négligeable devant celle des états du continuum, ainsi 
\begin{equation}
\mathcal{R} \text{\textit{e}} \: [ \rho_{\kappa,\lambda} \:(r) ] \approx \mathcal{P} \int_{0}^{+\infty} \rmd E_\nu \frac{R_{\nu,\lambda}(r) \bra{R_{\nu,\lambda}} r \ket{R_{n_g,\ell_g}}} {E_\kappa - E_{\nu}}
\end{equation}
On étend l'intégration jusqu'à $- \infty$ et on remplace $R_{\nu,\lambda}$ par sa forme asymptotique
\begin{equation}
\mathcal{R} \text{\textit{e}} \: [ \rho_{\kappa,\lambda} (r \rightarrow + \infty)] \approx \mathcal{P} \int_{-\infty}^{+\infty} \rmd E_\nu \frac{C_k}{r} \sin (kr + \Phi_{k,\lambda}(r)) \frac{\bra{R_{\nu,\lambda}} r \ket{R_{n_g,\ell_g}}} {E_\kappa - E_{\nu}}
\end{equation}
\begin{equation}
\mathcal{R} \text{\textit{e}} \: [ \rho_{\kappa,\lambda} (r \rightarrow + \infty)] \approx - \frac{\pi C_\kappa}{r} \: \cos (\kappa r + \Phi_{\kappa,\lambda}(r)) \: \bra{R_{\kappa,\lambda}} r \ket{R_{n_g,\ell_g}}.
\label{eq:PartieReelleDeRho}
\end{equation}
La forme asymptotique de la partie imaginaire de $\rho_{\kappa,\lambda}$ s'obtient simplement en remplaçant \ref{eq:fonctionAsymptotiqueContinnum} dans \ref{eq:ExpressionDeRho}.
\begin{equation}
\mathcal{I} \text{\textit{m}} \: [ \rho_{\kappa,\lambda} (r \rightarrow + \infty)] = - \frac{\pi C_\kappa}{r} \: \sin (\kappa r + \Phi_{\kappa,\lambda}(r)) \: \bra{R_{\kappa,\lambda}} r \ket{R_{n_g,\ell_g}}
\end{equation}
Finalement, l'expression asymptotique de $\rho_{\kappa,\lambda}$ est 
\begin{equation}
\rho_{\kappa,\lambda} (r \rightarrow + \infty) \approx - \frac{\pi C_\kappa}{r} \: \rme^{i(\kappa r + \Phi_{\kappa,\lambda}(r))} \: \bra{R_{\kappa,\lambda}} r \ket{R_{n_g,\ell_g}}
\label{eq:ExpressionAsymptotiqueRho}
\end{equation}
Avec \ref{eq:ExpressionAsymptotiqueRho} et \ref{eq:fonctionAsymptotiqueContinnum} dans \ref{eq:ExpressionDeT}, il vient 
\begin{equation}
T_{L,\lambda,\ell_g}(k) \approx - \pi C_k C_\kappa \bra{R_{\kappa,\lambda}} r \ket{R_{n_g,\ell_g}} \int_{0}^{+ \infty} \frac{\sin (kr + \Phi_{k,L}(r))}{r} \: r \: \frac{\rme^{i(\kappa r + \Phi_{\kappa,\lambda}(r))}}{r} r^2 \rmd r
\end{equation}
En exprimant le sinus sous forme exponentielle, on obtient dans l'intégrale la somme de deux termes oscillants en $\rme^{i((\kappa + k) r + \Phi_{\kappa,\lambda}(r) + \Phi_{k,L}(r))}$ et $\rme^{i((\kappa - k) r + \Phi_{\kappa,\lambda}(r) - \Phi_{k,L}(r))}$. Dans l'interférométrie RABBIT, l'énergie $\omega$ du photon IR absorbé en second est égale à la différence d'énergie entre les états intermédiaire et final, et est beaucoup plus petite que l'énergie de l'état final $\hbar^2 k^2 / 2m - \hbar^2 \kappa^2 / 2m = \omega \ll \hbar^2 k^2 / 2m$. Ainsi le terme $\propto \rme^{i((\kappa + k)}$ oscille bien plus rapidement que le second terme $\propto \rme^{i((\kappa - k)}$, et l'intégrale se simplifie
\begin{equation}
T_{L,\lambda,\ell_g}(k) \approx - \pi C_k C_\kappa \bra{R_{\kappa,\lambda}} r \ket{R_{n_g,\ell_g}} \int_{0}^{+ \infty} -\frac{1}{2i} \rme^{i((\kappa - k) r + \Phi_{\kappa,\lambda}(r) - \Phi_{k,L}(r))} \: r \: \rmd r
\end{equation}
En remplaçant les déphasages par leurs expressions,
\begin{align}
\int_{0}^{+ \infty} \rme^{i((\kappa - k) r + \Phi_{\kappa,\lambda} - \Phi_{k,L})} \: r \: \rmd r & = \int_{0}^{+ \infty} r \: \rme^{i(\kappa-k)r} \rme^{i(\frac{Z \ln 2 \kappa r}{\kappa } - \frac{Z \ln 2 k r}{k})} \: \rme^{i(\eta_\lambda - \eta_L)} \rme^{i\frac{\pi}{2}(L - \lambda)} \rmd r \\
& = \frac{(2\kappa)^{iZ/\kappa}}{(2k)^{iZ/k}} \rme^{i(\eta_\lambda - \eta_L)} i^{L-\lambda} \int_{0}^{+ \infty} r^{1+iZ(\frac{1}{\kappa}-\frac{1}{k})} \rme^{i(\kappa-k)r} \rmd r \\
& = \frac{(2\kappa)^{iZ/\kappa}}{(2k)^{iZ/k}} \rme^{i(\eta_\lambda - \eta_L)} i^{L-\lambda} \left( \frac{i}{\kappa-k} \right)^{2+iZ(\frac{1}{\kappa}-\frac{1}{k})} \Gamma(2+iZ(\frac{1}{\kappa}-\frac{1}{k}))
\end{align}
Après changement de variable et en utilisant l'expression intégrale de la fonction Gamma définie par l'équation \ref{eq:FonctionGamma}. En remarquant que $i^{iZ(\frac{1}{\kappa}-\frac{1}{k})} = \rme^{-\frac{\pi}{2}Z(\frac{1}{\kappa}-\frac{1}{k})}$, on obtient l'expression asymptotique finale de l'élément de transition radial:
\begin{multline}
T_{L,\lambda,\ell_g}(k) \approx \frac{\pi}{2} C_k C_\kappa \bra{R_{\kappa,\lambda}} r \ket{R_{n_g,\ell_g}} \frac{\rme^{-\frac{\pi}{2}Z(\frac{1}{\kappa}-\frac{1}{k})}}{(\kappa - k)^2} \\ \times \rme^{i(\eta_\lambda - \eta_L)} i^{L-\lambda -1} \frac{(2\kappa)^{iZ/\kappa}}{(2k)^{iZ/k}} \Gamma(2+iZ(\frac{1}{\kappa}-\frac{1}{k}))
\end{multline}
Remarquons que la première ligne de cette expression est réelle et contient un terme exponentiel décrivant la transition entre deux états du continuum $\kappa$ et $k$. L'exponentielle décroit avec l'énergie du photon d'habillage $\omega = \hbar^2 k^2 / 2m - \hbar^2 \kappa^2$, et à énergie de photon $\omega$ fixée l'exponentielle augmente avec le moment final $k$. Expérimentalement, il sera alors plus facile d'habiller les photoélectrons avec un faisceau dans le moyen IR plutôt qu'avec 800nm, et d'habiller les photoélectrons de plus grande énergie.

\paragraph*{Phase} \`{A} partir de l'expression précédente, on déduit la phase de l'élément de transition radial
\begin{equation}
\arg T_{L,\lambda,\ell_g}(k) = \frac{\pi}{2} (L-\lambda -1) + \eta_\lambda - \eta_L + \underbrace{\arg \left(\frac{(2\kappa)^{iZ/\kappa}}{(2k)^{iZ/k}} \Gamma(2+iZ(\frac{1}{\kappa}-\frac{1}{k})) \right)}_{\phi_{cc}(k,\kappa)}
\end{equation}
où la phase $\phi_{cc}$ est associée uniquement à la transition entre deux états intermédiaire et final du continuum d'un potentiel coulombien. C'est-à-dire qu'elle ne dépend ni de l'état initial ni du champ XUV. Elle est considérée comme universelle \mycite{DahlstromChemPhys2013} et est induite par la mesure.
Il est désormais possible d'exprimer la phase de l'élément de matrice de transition à deux photons total \ref{eq:M2phot}:
\begin{equation}
\setlength\fboxrule{1pt}
\boxed{
\arg M_{\vec{k}}^{(2)} = \pi + \arg Y_{L}^{m_g}(k) + \phi_\Omega + \phi_\omega - \lambda \frac{\pi}{2} + \eta_\lambda (\kappa) + \phi_{cc}(k,\kappa)
}
\label{eq:PhaseMatrice2phot}
\end{equation}
où $\phi_\Omega$ et $\phi_\omega$ sont les phases des champs XUV et IR respectivement. Les phases de diffusion de l'état final $\eta_L$ s'annulent et n'interviennent pas dans l'expression finale \ref{eq:PhaseMatrice2phot}. Ainsi, de manière surprenante, exceptée la contribution de l'harmonique sphérique, les termes contenus dans \ref{eq:PhaseMatrice2phot} dépendent uniquement de l'état intermédiaire c'est-à-dire de la transition à un photon.

\subsection{Phase et délai mesurés par l'interférométrie RABBIT}
\label{subsec:PhaseRabbit}
\begin{figure}
\centering
\def\svgwidth{0.5\columnwidth}
\import{Figures/DelaisPI/}{Schema_Principe_Rabbit_simple.pdf_tex}
\caption{Chemins quantiques à deux photons interférant dans l'interférométrie RABBIT: l'absorption du photon XUV d'énergie $\Omega_q$ suivie de l'absorption du photon IR d'énergie $\omega$ ($a$) et l'absorption du photon XUV d'énergie $\Omega_{q+2}$ suivie de l'émsission stimulée du photon IR d'énergie $\omega$ ($e$) conduisent au même continuum final.}
\label{fig:PrincipeRabbitSimple}
\end{figure}
Comme vu chapitre \textbf{ref}, la technique RABBIT permet de mesurer une \textit{différence de phase} entre deux transitions à deux photons: l'absorption d'une harmonique suivie de l'absorption d'un photon d'habillage interférant avec l'absorption de l'harmonique suivante suivie de l'émission stimulée d'un photon d'habillage. Les chemins quantiques et les notations utilisées sont représentées figure \ref{fig:PrincipeRabbitSimple}. Le signal du pic satellite ('\textit{sideband}') s'exprime alors en fonction des éléments de transition à deux photons
\begin{equation}
S_{SB} \propto |M^{a}+M^{e}|^2 = |M^{a}|^2 + |M^{e}|^2 + 2 |M^{a}||M^{e}| \cos[\arg (M^{a \: *} M^{e})]
\end{equation}
En supposant qu'un seul moment angulaire intermédiaire contribue à la transition dans les deux chemins $a$ et $e$ \footnote{Dans le cas général pour l'ionisation d'un électron ne se trouvant pas en couche $s$, plusieurs moments angulaires intermédiaires contribuent. Le délai de Wigner \textit{apparent} correspond à la somme de toutes les contributions.}, grâce à \ref{eq:PhaseMatrice2phot} on a:
\begin{equation}
\setlength\fboxrule{0.5pt}
\boxed{
\arg (M^{a \: *} M^{e}) \approx 2 \omega \tau + \phi_{\Omega_{q+2}} - \phi_{\Omega_{q}} + \eta_{\lambda}(\kappa_{q+2}) - \eta_{\lambda}(\kappa_{q}) + \phi_{cc}(\kappa_{q+2}) - \phi_{cc}(\kappa_q)
}
\end{equation} 
avec $\phi_{\Omega_{q+2}} - \phi_{\Omega_{q}}$ la différence de phase spectrale entre deux harmoniques consécutives et $\tau$ le délai entre les impulsions IR et XUV. On retrouve l'expression présentée au chapitre \textbf{ref} permettant la caractérisation de trains d'impulsions attoseconde. Ainsi, la dérivée de la phase à $2 \omega$ mesurée par le RABBIT permet d'obtenir
\begin{equation}
\setlength\fboxrule{0.5pt}
\boxed{
\tau_{\mathrm{RABBIT}} \approx \tau_q + \tau_W + \tau_{cc}
}
\label{eq:DelaiRabbitTotal}
\end{equation}
$\tau_q$ est le délai de groupe des harmoniques, $\tau_W$ le délai de Wigner et $\tau_{cc}$ un délai introduit par la mesure dû à la transition entre deux états du continuum.

Ainsi, dans le cas d'un continuum non structuré, il existe une relation simple entre la phase de l'élément de matrice de transition à deux photons et le délai de photoionisation. Lors d'une mesure RABBIT, ce délai peut être directement mesuré à une constante près: un délai supplémentaire "continuum-continuum". Ces développements ont permis l'interprétation d'expériences de photoionisation attoseconde à deux couleurs en termes de délais de photoionisation. 

\section{Expériences de mesure de délais de photoionisation attoseconde dans les gaz}
L'équation \ref{eq:DelaiRabbitTotal} fait apparaître le délai de groupe de l'impulsion XUV en plus du délai de Wigner, dont il faut donc s'affranchir pour déterminer le délai de photoionisation. Ainsi, les expériences mesurent en réalité une \textit{différence de délais de photoionisation} soit entre deux niveaux électroniques d'un même système, soit entre deux systèmes différents, étudiés dans les exactes mêmes conditions.

\begin{figure}[h]
\centering
\def\svgwidth{0.6\columnwidth}
\import{Figures/DelaisPI/}{Schultze.pdf_tex}
\caption{Sprectrogramme de streaking attoseconde du néon. Les lignes verticales matérialisent les maxima du streaking des photoélectrons $2p$ et $2s$. La différence entre ces maxima est due au délai de photoionisation entre les électrons $s$ et $p$. Adapté de \mycite{SchultzeScience2010}.}
\label{fig:Schultze}
\end{figure}

Historiquement, la première expérience mesurant un délai de photoionisation attoseconde n'utilisait pas la technique RABBIT mais le streaking attoseconde \mycite{SchultzeScience2010}. Une impulsion attoseconde unique, centrée spectralement à 106 eV grâce à l'utilisation de miroirs multicouches et de filtres métalliques, photoionise simultanément les électrons $2s$ et $2p$ du néon. Une impulsion IR intense habille les électrons qui suivent alors les oscillations du champ électrique. La différence entre les traces des électrons $2s$ et $2p$ permet la mesure de la différence de délai de photoionisation entre les deux niveaux électroniques à cette énergie: $\tau_{2p} - \tau_{2s} =$21 $\pm$ 5 as (figure \ref{fig:Schultze}). Pour interpréter cette valeur, les auteurs ont également effectué des calculs Hartree-Fock et estimé le délai de Wigner entre les deux sous-couches du néon, qui dans ce modèle est de 6.5 as. Le désaccord entre l'expérience et la simulation a entraîné de nombreux travaux théoriques \mycite{MoorePRA2011}\mycite{DahlstromPRA2012}\mycite{KheifetsPRA2013} plus raffinés mais conduisant également à des délais plus faibles que le délai mesuré. 

\begin{figure}[h]
\centering
\def\svgwidth{0.6\columnwidth}
\import{Figures/DelaisPI/}{Klunder.pdf_tex}
\caption{Sprectrogrammes RABBIT de l'argon correspondant aux photoélectrons $3p$ (haut) et $3s$ (bas). Les lignes noire et grise représentent la variation de phase des pics satellites. \`A partir ce cette phase sont extraits les temps d'émission pour les électrons $s$ et $p$ aux trois énergies. Pour faciliter la comparaison, la courbe de délai correspondant aux électrons $p$ a été abaissée à l'énergie cinétique des électrons $s$ (pointillés bleus). La différence entre ces deux courbes donne la différence de quantité $\tau_W + \tau_{cc}$ entre les électrons $3s$ et $3p$ aux énergies d'excitation données. Adapté de \mycite{KlunderPRL2011}.}
\label{fig:Klunder}
\end{figure}

Par ailleurs, le délai de photoionisation entre les électrons $3p$ et $3s$ de l'argon a été mesuré avec l'interférométrie RABBIT par l'équipe d'Anne L'Huiller \mycite{KlunderPRL2011}. Une sélection spectrale par un filtre de chrome des harmoniques 21 à 27 du 800nm photoionise à la fois les électrons $3p$ et $3s$ de l'argon, qui sont alors bien séparés dans les spectre de photoélectrons (figure \ref{fig:Klunder}). \`A partir de la mesure de la phase des pics satellites et de calculs pour estimer les phases continuum-continuum $\phi_{cc}$, la différence de délai $\tau_{3s} - \tau_{3p}$ a été obtenue. Les valeurs mesurées ont été confirmées par une expérience ultérieure du même groupe \mycite{GuenotPRA2012}. Cependant la comparaison directe avec des calculs de délai de photoionisation à un photon pour l'argon est délicate dans cette gamme d'énergie proche du minimum de Cooper \mycite{CooperPR1962}\mycite{SamsonJESRP2002}. En effet, les fortes corrélations entre les électrons $3s$ et $3p$ dans le minimum de Cooper sont à prendre en compte dans la modélisation du délai à un photon, ainsi qu'un effet du potentiel ionique sur les transitions continuum-continuum pour les électrons $s$ de faible énergie cinétique, entre autres \mycite{DahlstromJPB2014}. Notons également que plusieurs voies d'ionisation vers des états de moment angulaire différents sont à prendre en compte.

En stabilisant activement l'interféromètre RABBIT, l'équipe de Lund a pu par la suite mesurer des délais relatifs entre les électrons de valence de plusieurs gaz rares entre 31 et 37 eV. La robustesse de la méthode expérimentale a été vérifiée en comparant la mesure directe de  $\tau_{Ne} - \tau_{He}$ avec le calcul de $\tau_{Ne} - \tau_{He}$ à partir de mesures croisées avec l'argon $\tau_{Ne} - \tau_{He} = (\tau_{Ar} - \tau_{He}) - (\tau_{Ar} - \tau_{Ne})$. La figure \ref{fig:Guenot} présente la comparaison des résultats expérimentaux avec les calculs de la différence de délais à deux photons $(\tau_W + \tau_{cc})^{\mathrm{gaz} 1} - (\tau_W + \tau_{cc})^{\mathrm{gaz} 2}$ à différents niveaux de théorie: RPAE "\textit{Random Phase Approximation with Exchange}" \mycite{KheifetsPRA2013} et MCHF "\textit{Multi-Configurational Hartree Fock}" \mycite{CarettePRA2013} plus le calcul du terme continuum-continuum CC. Le calcul RPAE inclut les corrélations entre les orbitales de valence. L'approche MCHF permet de prendre en compte les effets d'états doublement excités ou d'excitation des électrons $3s$ dans l'argon. Si l'accord théorie-expérience est bon dans le cas Ne-He, il est seulement qualitatif pour Ne-Ar et Ar-He. Ces résultats illustrent la difficulté d'une telle expérience ainsi que l'interprétation délicate des mesures en termes de délais de photoionisation à un photon à l'aide de calculs théoriques.
\begin{figure}
\centering
\def\svgwidth{\textwidth}
\import{Figures/DelaisPI/}{Guenot.pdf_tex}
\caption{Différences de délais entre couples de gaz rares (a) $\tau_{Ar} - \tau_{Ne}$ (b) $\tau_{Ar} - \tau_{He}$ (c) $\tau_{Ne} - \tau_{He}$. Les croix rouges correspondent à la mesure RABBIT, les lignes bleues à un calcul à un photon RPAE + CC, les carrés noirs à un calcul à deux photons RPAE et la ligne violette à un calcul MCHF + CC pour l'argon. Extrait de \mycite{GuenotJPB2014}.}
\label{fig:Guenot}
\end{figure}

Mentionnons également les travaux récents de \mycite{JordanPRA2017}, publiés pendant ma thèse, qui présentent la mesure de délai de photoémission entre les deux composantes spin-orbite du krypton et du xénon. Les différences d'énergie entre les deux états des ions sont de $\Delta E^{Kr} = 0.67$ eV et $\Delta E^{Xe} = 1.31$ eV, ce qui complexifie l'analyse du spectrogramme RABBIT. En effet, à 800 nm, l'écart énergétique entre une harmonique et un pic satellite est de 1.55 eV: pour le xénon les harmoniques $^2P_{1/2}$ et les pics satellites$^2P_{3/2}$ se recouvrent presque totalement. \`A chaque pas de temps de l'interféromètre les auteurs soustraient donc un spectre de photoélectrons mesuré sans habillage. Dans le krypton, un faible délai ($\approx$ 6 as) est mesuré entre 20 et 40 eV, tandis que des délais jusqu'à 30 as et variant beaucoup avec l'énergie sont mesurés dans le xénon. Ces mesures sont comparées à des calculs théoriques TDCIS "\textit{Time-Dependent Configuration-Interaction Singles}" et RRPA + CC "\textit{Relativistic Random Phase Approximation}" (figure \ref{fig:Jordan}). Dans le xénon, de fortes différences sont observés entre les deux calculs et l'expérience, en particulier au voisinage d'états doublement excités (lignes grises en figure \ref{fig:Jordan}) qui ne sont pas pris en compte dans les simulations.
\begin{figure}
\centering
\def\svgwidth{0.6\textwidth}
\import{Figures/DelaisPI/}{Jordan.pdf_tex}
\caption{Délais entre les photoélectrons associés aux états $^2P_{1/2}$ et $^2P_{3/2}$ de $Kr^+$ (a) et $Xe^+$ (b). Les mesures RABBIT (cercles noirs) sont comparées à deux types de calculs (symboles rouges et bleus). Les lignes vertes et grises représentent respectivement des états simplement et doublement excités de l'atome. Extrait de \mycite{JordanPRA2017}.}
\label{fig:Jordan}
\end{figure}

Le développement de la physique attoseconde a donc permis les premières mesures de délais associés au processus de photoionisation, intrinsèquement liés à la diffusion de l'électron sur le potentiel atomique. Cependant, les potentiels réels contiennent de nombreuses interactions multiélectroniques qui ne sont pas prises en compte dans les calculs théoriques, rendant alors difficile la comparaison directe entre la quantité mesurée et les délais de photoionisation à un photon. Les expériences, complexes à mettre en \oe{uvre}, permettent une mesure dans une gamme spectrale limitée et échantillonnée uniquement aux énergies harmoniques. Il apparaît donc ici la nécessité d'un meilleur échantillonnage, par exemple en utilisant de plus grandes longueurs d'onde pour la génération d'harmoniques, et d'une certaine accordabilité pour enrichir les mesures et améliorer la comparaison avec les modèles existants. En particulier en présence de résonances ou de structures dans le continuum.

\begin{figure}
\centering
\def\svgwidth{0.8\textwidth}
\import{Figures/DelaisPI/}{Swoboda.pdf_tex}
\caption{(a) Phases des pics satellites mesurées par RABBIT, corrigées de la phase spectrale harmonique et normalisées à 0 pour le pic satellite 18. Différentes couleurs indiquent différents écarts à la résonance, de -11 meV en rouge à +190 meV en jaune. (b) Phase du pic satellite 16 en fonction de l'écart à la résonance. Adapté de \mycite{SwobodaPRL2010}.}
\label{fig:Swoboda}
\end{figure}

En effet, des variations de phase au voisinage d'une résonance sous le seuil d'ionisation de l'hélium et d'une résonance de Hopfield dans $N_2$ ont été mesurées. L'harmonique H15 du laser titane:saphir est résonante avec l'état de Rydberg $1s3p$ de l'hélium à 23.09 eV, tout en permettant l'ionisation à deux photons et deux couleurs avec un photon IR à 800 nm. Ainsi, \mycite{SwobodaPRL2010} ont observé que le pic satellite 16 avait une phase bien différente des pics satellites supérieurs. Ils ont pu mesurer avec la technique RABBIT la phase de l'oscillation en fonction de l'écart entre l'énergie de H15 et la résonance. Un écart négatif à la résonance ne permettant pas l'ionisation à deux couleurs, seul un côté de la résonance a pu être exploré. La variation de phase correspondante, au voisinage de la résonance $1s3p$ est de 1.5 rad ($\approx \pi/2$).

\begin{figure}
\centering
\def\svgwidth{\textwidth}
\import{Figures/DelaisPI/}{Haessler.pdf_tex}
\caption{(a) Schéma des continua de $N_2^+$ et des états autoionisants. La série de Hopfield correspond aux états de Rydberg convergeant vers l'état B de $N_2^+$, couplés aux continua des états X et A de $N_2^+$. (b) Section efficace de photoionisation au voisinage de $n=3$ à 17.1 eV. Adapté de \mycite{RaoultJPB1983}. (c) Spectre de photoionisation de $N_2$ par un peigne d'harmoniques 11 à 15 du 800 nm et spectre différentiel avec et sans habillage. (d) Phase des pics satellites résolus en état électronique X et A et en niveaux vibrationnel $\nu^{'}$. Adapté de \mycite{HaesslerPRA2009}.}
\label{fig:Haessler}
\end{figure}

Finalement, notons que la technique RABBIT est également adaptée à la mesure de délais de photoionisation dans les molécules. \`A Saclay en 2009, \mycite{HaesslerPRA2009} ont effectué des mesures de phase dans $N_2$ au voisinage d'un état autoionisant. L'harmonique 11 du 800 nm est résonante avec l'état $n=3$ de la série de Hopfield de $N_2$ (figure \ref{fig:Haessler} (a-b)), état de Rydberg moléculaire convergeant vers l'état B de l'ion couplé fortement avec le continuum de l'état X et plus faiblement avec celui de l'état A. La photoionisation de $N_2$ avec un peigne d'harmoniques 11 à 15 donne un riche spectre de photoélectrons composé des états X et A de l'ion dans plusieurs niveaux vibrationnels (figure \ref{fig:Haessler} (c)). Ces deux états étant séparés de 1.11 eV, les pics satellites de l'un recouvrent partiellement les harmoniques de l'autre et il faut alors soustraire à chaque délai un spectre sans habillage pour analyser le spectrogramme RABBIT. Cette astuce a permis de déterminer la phase des pics satellites des deux états électroniques pour plusieurs niveaux vibrationnels (figure \ref{fig:Haessler} (d)). Pour le pic satellite 12 (résonant), une variation de phase de $\approx \pi$ rad est mesurée dans le canal X pour $\nu^{'} = 1$ et $\nu^{'} = 2$. Une variation de phase beaucoup plus faible pour ce pic satellite est mesurée dans le canal A, en accord avec le faible couplage entre ce continuum et l'état autoionisant résonant avec H11. Le laser n'était pas accordable dans cette expérience, ce qui n'a pas permis de déterminer la variation de phase complète au voisinage de la résonance. Des calculs TDSE ultérieurs \mycite{CaillatPRL2011} ont pu déterminer cette évolution complète.

Enfin par souci d'exhaustivité, mentionnons ici l'expérience de \mycite{KoturNatComm2016} mesurant la variation de phase au voisinage d'une résonance de Fano dans l'argon, publiée pendant ma thèse et étroitement liée à mes travaux. Ces résultats seront présentés en détail dans le chapitre \textbf{ref}.

\chapter{Résonances autoionisantes de Fano}
\label{chap:ResonancesFano}
\begin{figure}[h]
\centering
\def\svgwidth{0.3\textwidth}
\import{Figures/DelaisPI/}{PhotoFano.pdf_tex}
\caption{Ugo Fano (1912 - 2001). Extrait de \mycite{ClarkNature2001}.}
\label{fig:PhotoFano}
\end{figure}

En 1935, alors qu'Ugo Fano est en post-doctorat dans le laboratoire d'Enrico Fermi à Rome, un de ses collègues, Emilio Segrè, lui propose d'étudier des articles de spectroscopie de Herbert Beutler \mycite{Beutler1935}. Les spectres d'absorption du krypton mesurés présentent des raies d'absorption de forme inhabituelle (figure \ref{fig:Beutler} (a)). D'après Fano, cette asymétrie est caractéristique d'interférences entre plusieurs processus d'excitation. La formulation analytique de cette théorie est élaborée avec l'aide de Fermi et publiée en italien dans la revue \textit{Nuovo Cimento} \mycite{Fano1935}. Plusieurs années plus tard, Fano, qui a émigré aux \'{E}tats-Unis pendant la seconde guerre mondiale et travaille au \textit{National Burau of Standards}, observe un spectre de diffusion d'électrons par l'hélium dont les raies asymétriques lui rappellent les spectres de Beutler. Il formule alors une version modernisée de son modèle de 1935 \mycite{FanoPR1961}. L'article publié en 1961 demeure aujourd'hui l'un des plus cités de \textit{Physical Review}. Les développements rapides de la spectroscopie ont permis d'identifier les désormais "profils de Fano" dans une quantité d'atomes, molécules et nanostructures \mycite{MiroshnichenkoRevModPhys2010}. Sa formule prédisant la forme des spectres est utilisée dans de nombreux domaines de la physique atomique, nucléaire et de la matière condensée. La richesse du modèle de Fano permettant d'interpréter les profils de raie réside dans sa simplicité: de complexes phénomènes physiques sont contenus dans seulement quelques paramètres.

\begin{figure}
\centering
\def\svgwidth{\textwidth}
\import{Figures/DelaisPI/}{Beutler.pdf_tex}
\caption{(a) Spectre d'absorption du krypton entre 12.1 et 13.6 eV. Extrait de \mycite{Beutler1935}. (b) Spectre d'absorption de l'hélium entre 57.7 et 77.5 eV. Extrait de \mycite{MaddenCodling1965}.}
\label{fig:Beutler}
\end{figure}

Dans ce chapitre, nous rappellerons les développements et les principaux résultats du modèle de Fano pour l'autoionisation, phénomène qui se produit lorsqu'un système est excité dans un état d'énergie supérieure à son potentiel d'ionisation. Dans ces conditions, deux processus d'excitation interfèrent: l'ionisation directe et l'autoionisation après être resté transitoirement dans l'état excité. La durée de vie caractéristique des états autoionisants est traditionnellement déterminée en spectroscopie à partir de la largeur de raie, et est de l'ordre de la dizaine de femtosecondes pour les systèmes atomiques. Ensuite, nous présenterons les principales expériences récentes visant à mesurer directement (i.e. dans le domaine temporel) la durée de vie des résonances de Fano dans les gaz rares.

\section{Interaction entre un état discret et un continuum d'états}
\label{sec:UniqueContinuum}
Il s'agit de déterminer les profils de raie des résonances d'autoionisation, c'est-à-dire l'expression de la section efficace de transition. La suite de ce paragraphe utilise les notations de Fano \mycite{FanoPR1961}\mycite{Maquet2015}.
\subsection{Fonctions propres}
On considère un système atomique constitué d'un état du continuum $\ket{\psi_{E'}}$ et d'un état discret (pseudo-) lié $\ket{\varphi}$, couplés par $V_{E'}$.

\begin{figure}[h]
\centering
\def\svgwidth{0.5\textwidth}
\import{Figures/DelaisPI/}{Schema_Fano.pdf_tex}
\caption{Diagramme énergétique des états considérés dans le traitement de Fano de l'autoionisation.}
\label{fig:Schema_Fano}
\end{figure}

On a:
\begin{align}
\bra{\varphi}\hat{H}\ket{\varphi} & = E_\varphi \\
\bra{\psi_{E'}}\hat{H}\ket{\psi_{E''}} & = E' \delta(E''-E') \\
\bra{\psi_{E'}}\hat{H}\ket{\varphi} & = V_{E'}
\label{eq:Hamiltonien_Fano}
\end{align}
où $\delta$ est la distribution de Dirac. On cherche les fonctions propres du système sous la forme
\begin{equation}
\ket{\Psi_E} = a_E \ket{\varphi} + \int b_{E'} \ket{\psi_{E'}} \rmd E'
\label{eq:BaseDesConfigurations}
\end{equation}
avec $a_E$ et $b_E$ fonctions de l'énergie à déterminer, solutions du système d'équations
\begin{align}
\label{eq:SystUniqueContinuum1} a_E \: E_\varphi + \int b_{E'} V^*_{E'} \rmd E' & = E \: a_E \\
\label{eq:SystUniqueContinuum2} a_E \: V_{E'} + b_{E'} \: E' & = E \: b_{E'}
\end{align} 
Pour exprimer $b_{E'}$ en fonction de $a_E$ dans la seconde équation, il faut diviser par $(E-E')$ qui peut être nul. On utilise alors l'expression formelle de $b_{E'}$ avec $z$ une fonction à déterminer
\begin{equation}
b_{E'}=\left(\frac{1}{E-E'} + z(E) \delta(E-E') \right) \: V_{E'} \: a_E
\end{equation}
Pour déterminer $z$ on injecte l'expression précédente dans la première équation du système
\begin{equation}
a_E \: E_\varphi +  a_E \mathcal{P} \int \frac{1}{E-E'} \: V_{E'} V^*_{E'} \: \rmd E' +  a_E \: z(E) \int \delta(E-E') \: V_{E'} V^*_{E'} \: \rmd E'= E \: a_E
\end{equation}
\begin{equation}
z(E) = \frac{E - E_\varphi - \mathcal{P} \int \frac{|V_{E'}|^2}{E-E'} \rmd E'}{|V_{E}|^2}
\end{equation}
$\mathcal{P} \int \frac{|V_{E'}|^2}{E-E'} \rmd E'$, noté $F(E)$ par Fano, est l'écart en énergie entre la position de la résonance et $E_\varphi$. Si $V_{E'}$ est indépendant de l'énergie, $F(E) = 0$.
$a_E$ est déterminée en normalisant $\ket{\Psi_E}$, on a alors
\begin{equation}
|a_E|^2 = \frac{1}{|V_E|^2 \left( \pi^2 + z(E)^2 \right)} = \frac{|V_E|^2}{\left( E - E_\varphi - F(E) \right) ^2 + \pi^2 |V_E|^4}
\label{eq:Expression_aE}
\end{equation}
On reconnaît dans l'expression de $a_E$ une fonction lorentzienne de largeur à mi-hauteur
\begin{equation}
\setlength\fboxrule{0.5pt}
\boxed{
\Gamma = 2 \pi |V_E|^2 }
\end{equation}

\paragraph*{Comportement asymptotique} \`A large distance $r$, la fonction d'onde du continuum non modifiée $\psi_{E'}$ s'écrit, d'après le paragraphe \ref{sec:DiffusionPotentielCentral}
\begin{equation}
\psi_{E'}(r \rightarrow + \infty) \propto \sin \left( k(E') + \delta \right)
\end{equation}
On a alors
\begin{equation}
\int b_{E'} \ket{\psi_{E'}} \rmd E' \propto \int b_{E'} \sin \left( k(E')r + \delta \right) \rmd E'
\end{equation}
\begin{equation}
\propto \mathcal{P} \int V_{E'} a_E \frac{1}{E - E'} \sin \left( k(E')r + \delta \right) \rmd E' + \int V_{E'} a_E \sin \left( k(E')r + \delta \right) z(E) \delta(E-E') \rmd E'
\end{equation}
Si $V_{E'}$ s'annule rapidement pour $E' \neq E$, la valeur principale est donnée par l'équation \ref{eq:PartieReelleDeRho} et on a alors
\begin{equation}
\int b_{E'} \ket{\psi_{E'}} \rmd E' \propto V_E a_E \left[ - \pi \cos (k(E)r + \delta) + z(E) \sin(k(E)r + \delta) \right]
\end{equation}
En posant $\tan \Delta = - \frac{\pi}{z(E)} = - \frac{\pi |V_E|^2}{E - E_\varphi - F(E)} = - \frac{\Gamma / 2}{E - E_\varphi - F(E)}$ ;
\begin{equation}
\int b_{E'} \ket{\psi_{E'}} \rmd E' \propto V_E a_E \sin \left( k(E) r + \delta + \Delta \right)
\end{equation}
$\Delta$ est donc le déphasage introduit par l'interaction de configuration de $\ket{\psi_{E'}}$ avec l'état discret $\ket{\varphi}$. On peut exprimer les paramètres $a_E$ et $b_{E'}$ en fonction du déphasage $\Delta$:
\begin{equation}
|a_E|^2 = \frac{1}{|V_E|^2 \pi^2 \left( 1 + 1/\tan^2 \Delta \right)} 
\end{equation}
d'où
\begin{equation}
a_E = \frac{\sin \Delta}{\pi V_E} \: \: \: ; \: \: \: b_{E'} = \frac{V_{E'} \sin \Delta}{\pi V_E} \frac{1}{E - E'} - \cos \Delta \: \delta(E-E')
\label{eq:Expressions_aE_bE}
\end{equation}

\subsection{Section efficace}
Nous nous intéressons essentiellement aux transitions dipolaires électriques, mais les profils de Fano se rencontrent également dans d'autres types d'expériences comme la diffusion de particules. Par souci de généralité, l'opérateur de transition sera donc noté ici $\hat{T}$. La transition entre un état initial $\ket{g}$ et la résonance de Fano $\ket{\Psi_E}$ s'écrit, d'après \ref{eq:Expressions_aE_bE}
\begin{align}
\bra{\Psi_E} \hat{T} \ket{g} & = \frac{\sin \Delta}{\pi V^*_E} \bra{\varphi} \hat{T} \ket{g} + \frac{\sin \Delta}{\pi V^*_E} \: \mathcal{P} \int \rmd E' \frac{V_{E'}}{E-E'} \bra{\psi_{E'}} \hat{T} \ket{g} - \cos \Delta \bra{\psi_E} \hat{T} \ket{g} \\
& = \frac{\sin \Delta}{\pi V^*_E} \bra{\Phi} \hat{T} \ket{g} -  \cos \Delta \bra{\psi_E} \hat{T} \ket{g} 
\end{align}
avec
\begin{equation}
\ket{\Phi} = \ket{\varphi} + \mathcal{P} \int \rmd E' \frac{V_{E'}}{E-E'} \ket{\psi_{E'}}
\end{equation}
$\ket{\Phi}$ correspond donc à l'état lié $\ket{\varphi}$ modifié par un ajout d'états du continuum dû à l'interaction de configuration.

La section efficace $\sigma$ s'exprime alors
\begin{align}
\sigma & \propto \left| \bra{\Psi_E} \hat{T} \ket{g} \right| ^2 \\
& \propto \left| \frac{\sin \Delta}{\pi V^*_E} \bra{\Phi} \hat{T} \ket{g} -  \cos \Delta \bra{\psi_E} \hat{T} \ket{g} \right| ^2 \\
& \propto \left| \bra{\psi_E} \hat{T} \ket{g} \right| ^2 \: \times \: \left| \frac{\sin \Delta}{\pi V^*_E} \frac{\bra{\Phi} \hat{T} \ket{g}}{\bra{\psi_E} \hat{T} \ket{g}} -  \cos \Delta \right| ^2
\label{eq:SectionEfficaceFano}
\end{align}
En définissant la section efficace de fond $\sigma_0 \propto |\bra{\psi_E} \hat{T} \ket{g}|^2$ et en introduisant le paramètre d'asymétrie $q$ tel que
\begin{equation}
\setlength\fboxrule{0.5pt}
\boxed{
q = \frac{1}{\pi V^*_E} \frac{\bra{\Phi} \hat{T} \ket{g}}{\bra{\psi_E} \hat{T} \ket{g}}
}
\label{eq:Definition_q}
\end{equation}
l'équation \ref{eq:SectionEfficaceFano} devient
\begin{equation}
\sigma = \sigma_0 \: |q \: \sin \Delta - \cos \Delta|^2
\end{equation}
On définit également l'énergie réduite $\epsilon$ telle que
\begin{equation}
\setlength\fboxrule{0.5pt}
\boxed{
\epsilon = - \cot \Delta = \frac{E - E_\varphi - F(E)}{\pi |V_E|^2} = \frac{E - E_\varphi - F(E)}{\Gamma / 2}
}
\end{equation}
Et avec la relation $\sin^2 \Delta = 1/(1+\cot^2 \Delta)$, l'expression du profil de Fano est obtenue:
\begin{equation}
\setlength\fboxrule{1pt}
\boxed{
\sigma = \sigma_0 \: \frac{\left(q+\epsilon\right) ^2}{1+\epsilon^2}}
\label{eq:ProfilFano}
\end{equation}
Ainsi, la section efficace est définie par trois paramètres: $q$, l'énergie de la résonance $E_r = E_\varphi + F(E)$ et sa largeur $\Gamma$. D'après la définition de $q$ (\ref{eq:Definition_q}),
\begin{equation}
\frac{1}{2} \pi q^2 = \frac{|\bra{\Phi} \hat{T} \ket{g}|^2}{|\bra{\psi_E} \hat{T} \ket{g}|^2 \: \Gamma}
\end{equation}
$q$ caractérise donc le rapport entre les probabilités de transition vers l'état discret (modifié par le continuum) et vers le continuum (non perturbé), et détermine la forme du profil de raie.

\begin{figure}
\centering
\def\svgwidth{0.6\textwidth}
\import{Figures/DelaisPI/}{ProfilsFano.pdf_tex}
\caption{Allure de la section efficace d'absorption en fonction de l'énergie réduite $\epsilon$ pour plusieurs valeurs du paramètre $q$.}
\label{fig:ProfilsFano}
\end{figure}

La figure \ref{fig:ProfilsFano} présente l'allure de la section efficace d'absorption en fonction de l'énergie réduite $\epsilon$ pour plusieurs valeurs de $q$. Le profil de raie est d'autant plus asymétrique que $|q|$ est grand. Le sens de l'asymétrie est donné par le signe de $q$. D'après l'expression \ref{eq:ProfilFano}, la section efficace est minimale pour $\epsilon = - q$ et maximale pour $\epsilon = 1/q$. Remarquons ici que la position de la résonance ($\epsilon = 0$), ne correspond en général ni au maximum ni au minimum de la section efficace. Le cas particulier $q = 0$ est propre aux résonances de Fano et conduit à un trou dans le spectre d'absorption, parfois appelé "anti-résonance".

Une expression différente de la section efficace permet de rendre compte de l'interprétation physique du phénomène de résonance de Fano:
\begin{equation}
\sigma = \sigma_0 \: \frac{\left(q + \epsilon \right) ^2}{1+\epsilon^2} = \sigma_0 \left( \underbrace{1}_\textrm{Continuum} + \underbrace{\frac{q^2 - 1}{1 + \epsilon^2}}_\textrm{Lorentzienne: \'Etat discret} + \underbrace{\frac{2 q \epsilon}{1 + \epsilon^2}}_\textrm{Couplage} \right)
\label{eq:SectionEfficaceFano3termes}
\end{equation}

\section{Interaction entre un état discret et plusieurs continua}
\label{sec:PlusieursContinua}
Cette situation se rencontre lorsque plusieurs voies d'ionisation sont possibles, par exemple l'ionisation d'un électron $p$ peut conduire à l'émission d'un électron dans les continua $s$ ou $d$. Un état discret d'énergie supérieure au potentiel d'ionisation du système peut alors être couplé différemment aux deux continua.\\
Les fonctions d'onde des continua sont notées $\ket{\psi_{E'}}$ et $\ket{\chi_{E'}}$ et sont orthogonales. Le système est caractérisé par les équations suivantes:
\begin{align}
\bra{\varphi}\hat{H}\ket{\varphi} & = E_\varphi \\
\bra{\psi_{E'}}\hat{H}\ket{\psi_{E''}} & = E' \delta(E''-E') \\
\bra{\chi_{E'}}\hat{H}\ket{\chi_{E''}} & = E' \delta(E''-E') \\
\bra{\psi_{E'}}\hat{H}\ket{\varphi} & = V_{E'} \\
\bra{\chi_{E'}}\hat{H}\ket{\varphi} & = W_{E'} \\
\bra{\psi_{E'}}\hat{H}\ket{\chi_{E''}} & = 0
\end{align}
Les fonctions propres sont cherchées sous la forme
\begin{equation}
\ket{\Psi_E} = a_E \ket{\varphi} + \int b_{E'} \ket{\psi_{E'}} \rmd E' + \int c_{E'} \ket{\chi_{E'}} \rmd E'
\end{equation}
avec $a_E$, $b_{E'}$ et $c_{E'}$ solutions du système d'équations
\begin{align}
\label{eq:SystPlusieursContinua1} a_E \: E_\varphi + \int (b_{E'} V^*_{E'} + c_{E'} W^*_{E'}) \rmd E' & = E \: a_E \\
\label{eq:SystPlusieursContinua2} a_E \: V_{E'} + b_{E'} \: E' & = E \: b_{E'} \\
\label{eq:SystPlusieursContinua3} a_E \: W_{E'} + c_{E'} \: E' & = E \: c_{E'}
\end{align}
$V^*_{E'} \times$ \ref{eq:SystPlusieursContinua2} + $W^*_{E'} \times$ \ref{eq:SystPlusieursContinua3} donne
\begin{equation}
a_E \left( \: |V_{E'}|^2 + |W_{E'}|^2 \: \right) + E' \left( V^*_{E'} b_{E'} + W^*_{E'} c_{E'} \right) = E \left( V^*_{E'} b_{E'} + W^*_{E'} c_{E'} \right)
\label{eq:SystPlusieursContinua4}
\end{equation}
Tandis que $W_{E'} \times$ \ref{eq:SystPlusieursContinua2} - $V_{E'} \times$ \ref{eq:SystPlusieursContinua3} permet d'obtenir 
\begin{equation}
E' \left( W_{E'} b_{E'} - V_{E'} c_{E'} \right) = E \left( W_{E'} b_{E'} - V_{E'} c_{E'} \right)
\label{eq:SystPlusieursContinua5}
\end{equation}
Ces deux combinaisons linéaires étant orthogonales, \ref{eq:SystPlusieursContinua1}, \ref{eq:SystPlusieursContinua4} et \ref{eq:SystPlusieursContinua5} forment un système d'équations équivalent au précédent. On reconnaît dans \ref{eq:SystPlusieursContinua1} et \ref{eq:SystPlusieursContinua4} la forme des équations correspondant au cas à un unique continuum (équations \ref{eq:SystUniqueContinuum1} et \ref{eq:SystUniqueContinuum2}) avec $V_{E'} \leftrightarrow \sqrt{|V_{E'}|^2 + |W_{E'}|^2}$ et $V^*_{E'} b_{E'} \leftrightarrow V^*_{E'} b_{E'} + W^*_{E'} c_{E'}$.\\
\ref{eq:SystPlusieursContinua5} ne dépend pas de $a_E$, donc pas de l'état discret et sa solution est simplement
\begin{equation}
W_{E'} b_{E'} - V_{E'} c_{E'} = \delta (E-E')
\end{equation}
Ainsi, le cas de l'interaction d'un état discret avec plusieurs continua se ramène facilement au cas de l'interaction avec un continuum "interactif" (traité au paragraphe \ref{sec:UniqueContinuum}) et un continuum "non interactif" (contribuant seulement au fond), avec
\begin{align}
\ket{\psi^{int}_{E'}} & = \frac{V^*_{E'}}{\sqrt{|V_{E'}|^2 + |W_{E'}|^2}} \ket{\psi_{E'}} + \frac{W^*_{E'}}{\sqrt{|V_{E'}|^2 + |W_{E'}|^2}} \ket{\chi_{E'}} \\
\ket{\psi^{non \: int}_{E'}} & = \frac{W_{E'}}{\sqrt{|V_{E'}|^2 + |W_{E'}|^2}} \ket{\psi_{E'}} - \frac{V_{E'}}{\sqrt{|V_{E'}|^2 + |W_{E'}|^2}} \ket{\chi_{E'}} 
\end{align}

\section{Dynamique de l'autoionisation: \'Etat de l'art}
La dynamique de l'autoionisation est d'abord caractérisée par la mesure des largeurs de raie $\Gamma$ en spectroscopie statique, donnant accès à la durée de vie $\tau = \hbar / \Gamma$. Le tableau \ref{tab:ParamètresFano} résume les paramètres caractérisant quelques résonances de Fano dans les gaz rares qui seront étudiées dans ce manuscrit. Leur largeur typique est la dizaine de milliélectronvolts, soit une durée de vie de quelques dizaines de femtosecondes. Ainsi, la question de la mesure "en temps réel" de la dynamique d'autoionisation ne s'est posée qu'à partir de l'existence de sources de lumières ultra-brèves dans cette gamme d'énergie, c'est-à-dire avec les débuts de la physique attoseconde.

\begin{table}[h]
\begin{center}
\begin{tabular}{|c|c|c|c|c|c|}
\hline
Résonance & $E_r$ (eV) & q & $\Gamma$ (meV) & $\tau$ (fs) & Référence \\
\hline
He $2s2p$ & 60.147 & -2.77 & 36 & 17 & \mycite{DomkePRA1996} \\
\hline
He $sp3^+$ & 63.66 & -2.58 & 8 & 82 & \mycite{DomkePRA1996} \\
\hline
Ne $2s2p^63p$ & 45.54 & -1.58 & 16 & 41 & \mycite{SchulzPRA1996} \\
\hline
Ar $3s3p^64p$ & 26.606 & -0.28 & 80 & 8 & \mycite{BerrahJPB1996} \\
\hline
\end{tabular}
\end{center}
\caption{Energie $E_r$, paramètre d'asymétrie $q$, largeur spectrale $\Gamma$ et durée de vie $\tau$ de plusieurs résonances de Fano étudiées dans ce manuscrit.}
\label{tab:ParamètresFano}
\end{table}

En 2002, \mycite{DrescherNature2002} s'intéressent à la dynamique d'un autre processus de désexcitation électronique: la relaxation Auger. Une impulsion attoseconde unique centrée à 97 eV excite le krypton qui émet des électrons Auger. Sous l'influence d'une seconde impulsion IR brève et intense, les électrons Auger sont redistribués dans des bandes latérales. L'évolution temporelle du signal des bandes latérales permet d'obtenir une estimation de la durée de vie de la lacune, tout à fait comparable à la valeur obtenue grâce à la largeur de raie. Par la suite, \mycite{ZhaoPRA2005} étudient théoriquement la possibilité de mesurer directement dans le domaine temporel les durées de vie d'états autoionisants par la même méthode. La première mesure directe de la durée de vie de la résonance de Fano $2s2p$ de l'hélium avec cette technique est effectuée quelques années plus tard \mycite{GilbertsonPRL2010}. La figure \ref{fig:Gilbertson} montre le spectre d'électrons en fonction du délai entre une impulsion attoseconde unique et une impulsion IR de 9 fs. Conformément aux prédictions de \mycite{ZhaoPRA2005}, des bandes latérales sont observées de part et d'autre de la résonance lorsque les deux impulsions sont simultanées. Le signal total d'électrons au voisinage de la résonance est asymétrique autour de $\Delta t = 0$. Son évolution pour $\Delta t > 0$ (IR après XUV) est en très bon accord avec l'ajustement exponentiel de durée de vie $\tau = 17$ fs (voir tableau \ref{tab:ParamètresFano}).

\begin{figure}
\centering
\def\svgwidth{0.9\textwidth}
\import{Figures/DelaisPI/}{Gilbertson.pdf_tex}
\caption{Détermination directe de la durée de vie de la résonance $2s2p$ de l'hélium par streaking attoseconde. (a) Spectre d'électrons en fonction du délai XUV-IR. La résonance d'autoionisation (AI) ainsi que les bandes latérales dues à l'émission/absorption d'un photon IR (SB1, SB2) sont indiquées par des flèches. (b) Signal d'électrons à l'énergie de la résonance en fonction du temps et ajustement exponentiel avec $\tau = 17$ fs. Extrait de \mycite{GilbertsonPRL2010}.}
\label{fig:Gilbertson}
\end{figure}

Le groupe de Zenghu Chang a également mesuré la durée de vie de la résonance d'autoionisation $3s3p^64p$ de l'argon grâce à la méthode d'absorption transitoire attoseconde \mycite{WangPRL2010}. Les auteurs mesurent le spectre d'absorption de l'argon en fonction du délai entre une impulsion XUV attoseconde unique et une impulsion IR de 6-8 fs. L'impulsion IR couple les résonances avec le continuum de l'argon, ce qui modifie le spectre d'absorption. La durée de vie est obtenue en effectuant un ajustement exponentiel de l'évolution temporelle du spectre au voisinage de la résonance. La valeur obtenue ($\tau = 8,2$ fs, figure \ref{fig:Wang}) est en excellent accord avec les données spectroscopiques du tableau \ref{tab:ParamètresFano}.

\begin{figure}
\centering
\def\svgwidth{0.9\textwidth}
\import{Figures/DelaisPI/}{Wang.pdf_tex}
\caption{Détermination directe de la durée de vie de la résonance $3s3p^64p$ de l'argon par absorption transitoire attoseconde. (a) Spectre XUV transmis à travers une cellule d'argon en fonction du délai entre l'impulsion attoseconde unique et une impulsion IR ultra-brève. (b) Signal transmis au voisinage des résonances d'autoionisation (trait plein) et ajustement avec la convolution d'une gaussienne et d'une décroissance exponentielle de durée de vie indiquée (pointillés). Adapté de \mycite{WangPRL2010}.}
\label{fig:Wang}
\end{figure}

Cependant, la seule mesure de la durée de vie de la résonance ne permet pas de caractériser entièrement la dynamique électronique qui a lieu lors de l'autoionisation. En effet, la transformée de Fourier d'une décroissance exponentielle dans le temps est un profil de raie Lorentzien, très différent de l'asymétrie caractéristique des résonances de Fano. En 2005, \mycite{WickenhauserPRL2005} s'intéressent théoriquement à la dynamique complète de la construction de la résonance de Fano. L'interaction d'une résonance de Fano (décrite par le modèle présenté au paragraphe \ref{sec:UniqueContinuum}) avec une impulsion XUV de durée inférieure à la durée de vie de la résonance ($\tau_X < \tau_r$) et un champ laser est étudiée par la résolution de l'équation de Schrödinger dépendante du temps. Les auteurs calculent l'expression de la probabilité d'ionisation $P(E, t) = |\bra{E}\ket{\Psi (t)}|^2$ dépendant de l'énergie et du temps. L'expression analytique obtenue pour l'amplitude d'ionisation$\bra{E}\ket{\Psi (t)}$ (équation (11) de \mycite{WickenhauserPRL2005}) est à rapprocher de l'expression de la section efficace \ref{eq:SectionEfficaceFano3termes}. En effet, elle contient trois termes: l'excitation directe de l'état fondamental vers le continuum en l'absence de résonance, la transition indirecte vers le continuum \textit{via} la résonance, et un terme de couplage. $P(E, t)$ représente la construction du profil de raie au cours du temps, et est reproduit figure \ref{fig:Wickenhauser}. Pour $t \leq \tau_r$, le spectre large reflète la durée ultra-brève de l'impulsion de pompe $\tau_X$, seule l'excitation directe contribuant au spectre. Lorsque $t$ augmente, les interférences avec le chemin résonant ont lieu et modifient le spectre d'absorption, qui converge pour $t \rightarrow + \infty$ vers le profil de raie de la résonance. 

\begin{figure}[h]
\centering
\def\svgwidth{0.6\textwidth}
\import{Figures/DelaisPI/}{Wickenhauser.pdf_tex}
\caption{Représentation temporo-spectrale de la probabilité d'ionisation $P(E, t)$: calcul de la construction du profil de Fano au cours du temps. (1) Spectre pour $t \leq \tau_r$ répliquant l'impulsion de pompe. (2) Spectre pour $t \rightarrow + \infty$ correspondant au profil de raie ($q = 1$). Extrait de \mycite{WickenhauserPRL2005}.}
\label{fig:Wickenhauser}
\end{figure}

L'observation de la dynamique complète d'autoionisation à l'échelle attoseconde requiert donc la mesure de l'amplitude \textbf{et} de la phase de la transition, ce qui n'était pas accessible jusqu'alors.

\chapter{Transitions résonantes à deux photons}
\label{chap:2photons_et_Fano}
Dans la mesure de la section efficace d'absorption au voisinage d'une résonance de Fano (équation \ref{eq:ProfilFano}), l'information sur la phase de la transition est perdue. Nous avons vu précédemment que l'interférométrie RABBIT est une technique utilisée pour mesurer les phases d'éléments de transition à deux photons, que l'on peut interpréter en termes de phases d'éléments de transition à un photon. Dans quelle mesure cette méthode s'applique-t-elle lorsque l'on souhaite mesurer la phase d'une transition vers une résonance de Fano? Dans ce chapitre nous nous appuierons sur les travaux du groupe de Fernando Mart\'{i}n \mycite{JimenezGalanPRL2014}, \mycite{JimenezGalanPRA2016}, \mycite{ArgentiPRA2017}.

\section{Phase de l'amplitude de transition vers une résonance de Fano}
En utilisant une normalisation quelque peu différente de celle utilisée pour obtenir l'équation \ref{eq:Expression_aE}, la fonction d'onde caractérisant la résonance de Fano peut s'écrire, avec les notations du chapitre \ref{chap:ResonancesFano}
\begin{equation}
\ket{\Psi_E} = \frac{\epsilon}{\epsilon - i} \ket{\psi_E} + \frac{1}{\pi V_E} \frac{1}{\epsilon - i} \ket{\Phi}
\end{equation} 
Ainsi, l'amplitude de transition de l'état fondamental $\ket{g}$ vers $\ket{\Psi_E}$ est
\begin{align}
M^{(1)}_E = \bra{\Psi_E} \hat{T} \ket{g} & = \frac{\epsilon}{\epsilon + i} \bra{\psi_E} \hat{T} \ket{g} + \frac{1}{\pi V_E^*} \frac{1}{\epsilon + i} \bra{\Phi} \hat{T} \ket{g} \\
\label{eq:TransitionEnFonctiondeR_e}
& = \bra{\psi_E} \hat{T} \ket{g} \left( \frac{\epsilon + q}{\epsilon + i} \right) \\
& = \bra{\psi_E} \hat{T} \ket{g} \times \mathcal{R}(\epsilon)
\end{align}
Le module carré de l'expression précédente est évidemment identique à la section efficace de Fano (équation \ref{eq:ProfilFano}). Le facteur résonant $\mathcal{R}(\epsilon)$ introduit une phase dans l'amplitude de transition:
\begin{equation}
\arg M^{(1)}_E = \arg \mathcal{R}(\epsilon) = \arctan \epsilon - \pi \: \Theta (\epsilon + q) + \frac{\pi}{2}
\label{eq:Arg1photonFano}
\end{equation}
où $\Theta$ désigne la fonction de Heaviside. La phase de l'amplitude de transition présente donc un saut de $\pi$ en $\epsilon = -q$ (figure \ref{fig:FanoPlanComplexeFinal}(b)). Ce saut de phase n'est pas une discontinuité physique de l'amplitude de transition $M^{(1)}_E$ (qui est une fonction continue de $\epsilon$), mais reflète simplement la discontinuité de la fonction $\arg$ à l'origine. En effet, le facteur résonant $\mathcal{R}(\epsilon)$ peut être représenté sous la forme d'un cercle dans le plan complexe:
\begin{equation}
\mathcal{R}(\epsilon) = \frac{1-iq}{2} + \frac{1+iq}{2} \: \frac{\epsilon - i}{\epsilon + i}
\end{equation}
$\mathcal{R}(\epsilon)$ est un cercle de centre $\left(\frac{1}{2};- \frac{q}{2} \right)$ et de rayon $\frac{\sqrt{1+q^2}}{2}$ (figure \ref{fig:FanoPlanComplexeFinal}(a)).

\begin{figure}
\centering
\def\svgwidth{\textwidth}
\import{Figures/DelaisPI/}{Fano_plan_complexe_final.pdf_tex}
\caption{(a) Trajectoire de $\mathcal{R}(\epsilon)$ dans le plan complexe ($\epsilon$ variant de -15 à +15). (b) Phase de $\mathcal{R}(\epsilon)$ pour différentes valeurs du paramètre $q$.}
\label{fig:FanoPlanComplexeFinal}
\end{figure}

\section{Amplitude de transition à deux photons}
Il s'agit de combiner l'approche du chapitre \ref{chap:DelaiPI} avec le modèle de Fano présenté au chapitre \ref{chap:ResonancesFano}. Ainsi, on réécrit l'équation \ref{eq:MatriceDeuxPhotons}:
\begin{equation}
M_{\vec{k}}^{(2)} = \frac{1}{i} \: \mathcal{E}_\Omega \mathcal{E}_\omega \: \lim \limits_{\alpha \rightarrow 0^+} \sum_{\nu} \! \! \! \! \! \! \! \! \! \int \frac{\bra{\vec{k}} z\ket{\nu} \bra{\nu} z \ket{g}} {E_g + \Omega - E_{\nu} + i \alpha}
\end{equation}
Ici l'état intermédiaire $\ket{\nu}$ n'est pas une fonction du continuum mais un état autoionisant
\begin{equation}
\ket{\nu} = \ket{\Psi_E} = \frac{\epsilon}{\epsilon - i} \ket{\psi_E} + \frac{1}{\pi V_E} \frac{1}{\epsilon - i} \ket{\Phi}
\end{equation}

\begin{figure}
\centering
\def\svgwidth{0.5\textwidth}
\import{Figures/DelaisPI/}{Schema_2photons_Fano.pdf_tex}
\caption{Ionisation à deux photons XUV-IR depuis l'état initial vers un état du continuum \textit{via} une résonance d'autoionisation.}
\end{figure}

De la même manière qu'au chapitre \ref{chap:DelaiPI}, on ne s'intéresse qu'à la contribution des états intermédiaires du continuum, soit 
\begin{equation}
M_{\vec{k}, \text{Fano}}^{(2)} \approx \frac{1}{i} \: \mathcal{E}_\Omega \mathcal{E}_\omega \: \lim \limits_{\alpha \rightarrow 0^+} \int \rmd E \frac{\bra{\vec{k}} z\ket{\Psi_E} \bra{\Psi_E} z \ket{g}} {E_g + \Omega - E + i \alpha}
\end{equation}
D'après l'équation \ref{eq:TransitionEnFonctiondeR_e},
\begin{equation}
M_{\vec{k}, \text{Fano}}^{(2)} \approx \frac{1}{i} \: \mathcal{E}_\Omega \mathcal{E}_\omega \: \lim \limits_{\alpha \rightarrow 0^+} \int \rmd E \frac{\bra{\vec{k}} z\ket{\Psi_E}} {E_g + \Omega - E + i \alpha} \times  \left( \frac{\epsilon + q}{\epsilon + i} \right) \: \bra{\psi_E} z \ket{g}
\end{equation}
En insérant l'expression de $\ket{\Psi_E}$ puis en effectuant les intégrations sur les contours adéquats \mycite{JimenezGalanPRA2016}, on montre que pour la transition où le photon $\Omega$ est absorbé en premier (cas (a) de la figure \ref{fig:Schema_2photons}), si la durée des impulsions correspondantes est plus longue que la durée de vie de la résonance, on a:
\begin{equation}
M_{\vec{k}, \text{Fano}}^{(2)} \approx - \frac{\bra{\vec{k}} z \ket{\psi_E} \bra{\psi_E} z \ket{g}}{\omega} \times \frac{\epsilon + q(1-\gamma) + i \gamma}{\epsilon + i}
\label{eq:M2phot_Fano}
\end{equation}
où l'on introduit le paramètre $\gamma$ tel que:
\begin{equation}
\gamma = \frac{\bra{\vec{k}} z \ket{\varphi}}{\bra{\vec{k}} z \ket{\psi_E} V_E / \omega}
\end{equation}
$\gamma$ mesure le rapport des deux transitions entre la résonance et le continuum final: la transition directe entre l'état lié et le continuum $\bra{\vec{k}} z \ket{\varphi}$, et la transition indirecte passant par le continuum intermédiaire faisant intervenir le couplage entre l'état lié $\ket{\varphi}$ et $\ket{\psi_E}$ puis la transition radiative entre $\ket{\psi_E}$ et $\ket{\vec{k}}$, $\bra{\vec{k}} z \ket{\psi_E} V_E / \omega$.

L'expression \ref{eq:M2phot_Fano} est très similaire à l'équation \ref{eq:TransitionEnFonctiondeR_e}. On peut introduire un paramètre $q$ effectif conmplexe
\begin{equation}
q_{\text{eff}} = q(1-\gamma) + i \gamma
\end{equation}
ainsi qu'un facteur résonant effectif
\begin{align}
\mathcal{R}_{\text{eff}}(\epsilon) & = \frac{\epsilon + q_{\text{eff}}}{\epsilon + i}\\
& = \gamma + (1 - \gamma) \frac{\epsilon + q}{\epsilon + i}\\
& = \gamma + (1 - \gamma) \times \mathcal{R}(\epsilon) 
\end{align}
$\mathcal{R}_{\text{eff}}(\epsilon)$ décrit toujours un cercle dans le plan complexe, mais multiplié par $1 - \gamma$ et décalé de $\gamma$ sur l'axe des réels (figure \ref{fig:FanoComplexeEffectif}(a)). Ainsi, en général $\mathcal{R}_{\text{eff}}(\epsilon)$ n'intercepte plus l'origine et la variation de phase correspondante est différente. On retrouve le cas de $\mathcal{R}(\epsilon)$ uniquement pour $\gamma = 0$, c'est-à-dire si l'état discret n'est pas couplé radiativement au continuum final. Pour $\gamma > 0$, le cercle est contracté et la variation de phase est adoucie par rapport à la transition à un photon. Pour $\gamma < 0$, le cercle est au contraire agrandi et la phase varie alors sur $2 \pi$. Enfin, remarquons que $\gamma$ est proportionnel à $\omega$ donc d'une part il est de signe opposé pour l'absorption et l'émission du second photon, les variations de phase correspondantes sont alors différentes dans les deux cas. D'autre part la phase est a priori différente si le photon d'habillage est à 800 nm ou dans l'infrarouge moyen.

\begin{figure}
\centering
\def\svgwidth{\textwidth}
\import{Figures/DelaisPI/}{Fano_plan_complexe_eff_final.pdf_tex}
\caption{(a) Trajectoire de $\mathcal{R}_{\text{eff}}(\epsilon)$ dans le plan complexe ($\epsilon$ variant de -15 à +15). (b) Phase de $\mathcal{R}_{\text{eff}}(\epsilon)$ pour différentes valeurs du paramètre $\gamma$, $q$ étant constant et $q = 1$.}
\label{fig:FanoComplexeEffectif}
\end{figure}

Ainsi,
\begin{equation}
M_{\vec{k}, \text{Fano}}^{(2)} \approx - \frac{\bra{\vec{k}} z \ket{\psi_E} \bra{\psi_E} z \ket{g}}{\omega} \times \mathcal{R}_{\text{eff}}(\epsilon)
\end{equation}
et la phase
\begin{equation}
\arg M_{\vec{k}, \text{Fano}}^{(2)} \approx \arg M_{\vec{k}}^{(2)} + \arg [\mathcal{R}_{\text{eff}}(\epsilon)]
\label{eq:Arg2photonsFano}
\end{equation}
où $M_{\vec{k}}^{(2)}$ est un élément de transition à deux photons impliquant uniquement des états du continuum, calculé au chapitre \ref{chap:DelaiPI}.

L'interférométrie RABBIT mesurant une différence de phase entre deux chemins quantiques, si l'on choisit un chemin de référence ne passant par aucune résonance intermédiaire, le RABBIT donne accès à $\arg [\mathcal{R}_{\text{eff}}(\epsilon)]$. Si de plus l'état lié est faiblement couplé au continuum final ($\gamma \approx 0$), alors $\mathcal{R}_{\text{eff}}(\epsilon) \approx \mathcal{R}(\epsilon)$ et on obtient la mesure de la phase de la transition à un photon vers la résonance d'autoionisation que l'on cherche. 

\mycite{ArgentiPRA2017} ont montré que l'interprétation en termes de délais de Wigner n'est plus valable dans ce cas, mais la donnée des variations spectrales de la phase de l'élément de transition est bien plus riche qu'une simple valeur de délai, et permet d'accéder à toute la dynamique de l'autoionisation.

\paragraph*{Cas de plusieurs continua intermédiaires} Nous avons vu au paragraphe \ref{sec:PlusieursContinua} que lorsque la résonance de Fano est couplée à plusieurs continua, le système se ramène finalement à l'interaction avec un continuum "interactif" et un continuum "non interactif". Ainsi, pour la transition à deux photons, on montre que
\begin{equation}
M_{\vec{k}, \text{Fano, multicanaux}}^{(2)} \propto r + \frac{\epsilon + q_{\text{eff}}}{\epsilon + i}
\end{equation}
avec 
\begin{equation}
r = \frac{\bra{\vec{k}} z \ket{\Psi_E^{\text{non int}}} \bra{\Psi_E^{\text{non int}}} z \ket{g}}{\bra{\vec{k}} z \ket{\Psi_E^{\text{int}}} \bra{\Psi_E^{\text{int}}} z \ket{g}}
\end{equation}
Ce qui revient à déplacer $\mathcal{R}_{\text{eff}}(\epsilon)$ dans le plan complexe d'une quantité $r$ ($r \in \mathbb{C}$). La phase de $M_{\vec{k}, \text{Fano, multicanaux}}^{(2)}$ dépend alors des couplages avec les différents continua et est quelconque \textit{a priori}.


