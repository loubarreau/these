\chapternonumtoc{Introduction}
\subsubsection*{La photoionisation, une sonde de la structure électronique}
L'effet photoélectrique est l'émission d'un électron par un matériau sous l'action d'un rayonnement lumineux. Mis en évidence expérimentalement par H. Hertz en 1887 \mycite{Hertz1887}, ses principales caractéristiques sont déterminées par P. Lenard en 1902 \mycite{Lenard1902}: l'énergie des électrons émis (appelés \textit{photoélectrons}) ne dépend pas de l'intensité mais de la longueur d'onde de la lumière; seule la quantité d'électrons émis est proportionnelle à l'intensité lumineuse \mycite{BonzelProgSurfSci1995}. Pour expliquer ces observations, A. Einstein proposa en 1905 le concept de photon, ou quantum de lumière, qui révolutionna la physique moderne et contribua au développement de la mécanique quantique \mycite{Einstein1905}. L'effet photoélectrique a d'abord été étudié sur des solides (en particulier des métaux) avant d'être étendu aux vapeurs d'alcalins puis aux gaz de l'air et aux gaz rares \mycite{EncyclopediaOfPhysicsVolXXI}. Dans le contexte de cette thèse, l'effet photoélectrique correspond à l'ionisation simple d'un atome ou d'une molécule en phase gazeuse, à l'aide d'un ou deux photons. L'énergie de photon nécessaire se situe dans la gamme de l'eXtrême Ultra-Violet (XUV, 10 - 100 eV). Si l'absorption du photon est une transition quantique instantanée, l'électron doit ensuite s'extraire du potentiel ionique. Au cours de sa \textit{diffusion} sur le potentiel, les interactions avec le noyau et le reste du nuage électronique \textit{déphasent} la fonction d'onde électronique sortante. Grâce aux travaux issus de la théorie des collisions dans les années 1950, E. Wigner interprète les variations spectrales de ce déphasage comme un \textit{délai} \mycite{WignerPR1955}\mycite{SmithPR1960}\mycite{Eisenbud}. Pour la photoionisation, ces délais infimes sont restés théoriques jusqu'à leur observation expérimentale dans les années 2010 grâce au développement de la physique attoseconde \mycite{SchultzeScience2010}\mycite{KlunderPRL2011}. 

Au cours du XX$^{\text{ème}}$ siècle, la spectroscopie de photoélectrons se développa considérablement et s'affirma comme une méthode d'étude privilégiée de la structure (et de la dynamique) électronique des atomes, molécules et solides au même titre que la spectroscopie d'absorption ou d'émission de photons \mycite{SiegbahnRevModPhys1982}. En effet, en première approximation on peut considérer que la structure électronique de l'atome (ou molécule) ionisé n'est pas modifiée lors de la photoionisation. En régime de champ faible, un seul photon est absorbé et l'énergie du photoélectron obéit à la conservation de l'énergie. Dans ces conditions, l'énergie de liaison du photoélectron (la différence entre l'énergie du photon incident et l'énergie de l'électron) est simplement égale à l'énergie de l'orbitale atomique (ou moléculaire) mise en jeu. En réalité, les interactions entre les électrons sont responsables de processus plus complexes:
\begin{itemize}
\item l'absorption d'un photon peut conduire à l'émission d'un photoélectron et à l'excitation concomitante d'un électron lié. Dans ce cas l'énergie de liaison du photoélectron émis est supérieure à l'énergie de l'orbitale. Sur le spectre de photoélectrons, le pic correspondant apparaît comme un satellite du pic de photoionisation directe. C'est le processus de \textit{shake up} \mycite{AabergPR1967}.
\item lorsqu'un électron de c\oe ur est ionisé, le trou peut être comblé par un électron de valence. Pour assurer la conservation de l'énergie le système émet alors un second électron, c'est l'effet Auger \mycite{Auger1925}.
\item l'absorption du photon peut porter le système dans un état excité d'énergie supérieure à l'énergie d'ionisation. L'ionisation est alors indirecte après une excitation transitoire. Ce processus est appelé autoionisation \mycite{FanoPR1961}.
\end{itemize}
Les durées caractéristiques de ces différents phénomènes peuvent être estimées à partir des largeurs spectrales des pics associés en exploitant la conjugaison temps-énergie. Ces dernières sont typiquement de l'ordre de la centaine de milli-électron-volts (1 eV = 1.602 $\times 10^{-19}$ J) , ce qui correspond à des durées de vie de la dizaine de femtosecondes (1 fs = 1 $\times 10^{-15}$ s). Cependant, la durée de vie représente la durée du processus complet conduisant à l'émission d'un électron mais ne donne aucune information sur les \textit{dynamiques} électroniques qui ont lieu pendant les quelques femtosecondes qui suivent l'interaction avec le photon. Dans ce travail, nous nous sommes intéressés en particulier aux dynamiques électroniques ayant lieu lors de l'autoionisation.

\subsubsection*{Les harmoniques d'ordre élevé pour étudier les dynamiques des électrons à leur échelle de temps naturelle}
Les durées extrêmement courtes des dynamiques électroniques requièrent la production d'impulsions ultra-brèves pour leur étude résolue en temps. Une méthode de production en laboratoire d'impulsions de durée attoseconde (1 as = 1 $\times 10^{-18}$ s) repose sur la génération d'harmoniques d'ordre élevé (GHOE). Il s’agit d’un processus extrêmement non-linéaire qui permet de produire, lors de l'interaction laser-matière en régime de champ fort, un grand nombre d’harmoniques du fondamental. Dans ce processus, un laser infrarouge impulsionnel et énergétique est focalisé dans un jet de gaz atomique ou moléculaire. Si le milieu est centrosymétrique et si l'impulsion présente plusieurs cycles optiques, on assiste à l'émission d’un rayonnement cohérent composé des harmoniques impaires de la fréquence du laser de génération. De manière remarquable, l’intensité de ces harmoniques ne suit pas un comportement perturbatif. Au contraire, leur intensité est quasiment constante sur une large gamme spectrale. Ce phénomène a été observé pour la première fois à Chicago \mycite{McphersonJOSAB1987} et à Saclay \mycite{FerrayJPhysB1988} à la fin des années 1980. L'idée de produire des impulsions attosecondes par cette méthode émerge dans les années 1990 \mycite{FarkasPhysLettA1992}\mycite{HarrisOptComm1993}: si toutes les harmoniques émises sur une large gamme spectrale sont en phase, alors elles donnent lieu dans le domaine temporel à l'émission d'un train d'impulsions ultra-brèves. Il faudra attendre le début des années 2000 pour obtenir la démonstration expérimentale de la production d'un train d'impulsions attosecondes \mycite{PaulScience2001} et d'une impulsion attoseconde isolée \mycite{HentschelNature2001}.

Pour étudier une dynamique attoseconde par une méthode pompe-sonde dans le domaine temporel, l'idéal est de diposer d'impulsions de pompe et de sonde toutes deux de durée sub-femtoseconde \mycite{TzallasNatPhys2011}\mycite{TakahashiNatComm2013}. Le faible flux de photons produit par GHOE avec les systèmes lasers actuels rend difficile la séparation d'une impulsion attoseconde en deux faisceaux de pompe et de sonde d'intensité suffisante pour réaliser les expériences. La plupart des expériences pompe-sonde avec des harmoniques utilisent alors le laser fondamental IR comme deuxième impulsion et mettent à profit soit la variation rapide du champ électrique à l'échelle du cycle optique ("streaking attoseconde") soit la variation de l'intensité laser d'impulsions dont la durée a été réduite à quelques cycles optiques (un cycle optique a une durée de 2.6 fs à 800 nm) grâce à l'élargissement du spectre dans un gaz puis recompression par réflexion sur des miroirs multicouches \mycite{NisoliAPL1996}. Cette méthode a permis d'étudier des dynamiques ultra-rapides dans les atomes \mycite{GoulielmakisNature2010}, les molécules \mycite{SansoneNature2010}\mycite{CalegariScience2014} et les solides \mycite{SchultzeScience2014}. En utilisant le streaking attoseconde, la première mesure de délais entre les photoélectrons issus de différentes orbitales atomiques a été effectuée \mycite{SchultzeScience2010}. Plus récemment, le streaking attoseconde a été utilisé pour mettre en évidence un délai supplémentaire au voisinage d'états doublement excités \mycite{SabbarPRL2015}, ou de satellites de \textit{shake-up} \mycite{OssianderNatPhys2017}. Cependant, ce type d'expériences présente plusieurs inconvénients: il nécessite un dispositif expérimental complexe et exigeant; les expériences utilisant les variations rapides du champ infra-cycle requièrent des intensités élevées qui peuvent perturber la dynamique à étudier; et pour les expériences utilisant l'enveloppe, la résolution temporelle est intrinsèquement limitée par la durée du cycle optique laser.

Une approche équivalente, reposant sur la cohérence des harmoniques d'ordre élevé \mycite{AntoinePRL1996}\mycite{Salieres1999}, consiste à effectuer les mesures dans le domaine spectral. En effet, de manière similaire à l'interférométrie spectrale utilisée pour caractériser les impulsions optiques ultrabrèves (technique SPIDER)\mycite{IaconisOL1998}, il est possible de faire de l'\textit{interférométrie électronique} entre paquets d'ondes produits par le rayonnement harmonique afin d'étudier leurs dynamiques. Les harmoniques étant séparées dans le domaine spectral de l'énergie de deux photons laser, la photoionisation à deux photons et deux couleurs permet de faire interférer les photoélectrons issus de l'absorption d'une harmonique et d'un photon laser avec ceux provenant de  l'absorption de l'harmonique voisine et l'émission stimulée d'un photon laser. La technique expérimentale utilisée, appelée RABBIT (de l'anglais \textit{Reconstruction of Attosecond Beating By Interference of two-photon Transitions} \mycite{MullerAPB2002}), a été initialement développée pour mesurer la phase spectrale de l'émission harmonique \mycite{PaulScience2001} puis a été étendue à la caractérisation de paquets d'ondes électroniques produits lors de la photoionisation à deux photons XUV-laser d'un système atomique ou moléculaire en phase gazeuse. Avec cette méthode, il a été possible de mettre en évidence les délais attosecondes de photoionisation prédits par Wigner entre orbitales différentes d'un même atome \mycite{KlunderPRL2011} ou d'une même molécule \mycite{HuppertPRL2016}, et entre orbitales de valence d'atomes différents \mycite{GuenotJPB2014}\mycite{PalatchiJPB2014}. Il s'agit donc d'une technique de choix qui permet de révéler des détails fins du potentiel atomique. Si l'un des "bras" de l'interféromètre quantique fait intervenir une transition vers, par exemple, un état autoionisant, est-il possible de révéler les interactions électroniques se produisant lors de l'autoionisation? \'{E}tant donnée la largeur spectrale des résonances d'autoionisation ($\sim 100$ meV), on imagine qu'il est nécessaire d'accorder l'énergie des harmoniques pour couvrir toute la largeur de la résonance. Avant ma thèse, très peu d'expériences ont étudié l'influence d'un état résonant dans l'interférométrie électronique \mycite{HaesslerPRA2009}\mycite{SwobodaPRL2010}\mycite{TheseChirla}, sans toutefois en extraire des informations directes sur les dynamiques électroniques dans le domaine temporel. Au cours de ma thèse, l'équipe d'A. L'Huillier à Lund (Suède) a étudié les déphasages introduits par une résonance d'autoionisation dans l'interférométrie RABBIT \mycite{KoturNatComm2016}, indépendamment et parallèlement aux expériences que nous avons effectuées.

\subsubsection*{Vers une caractérisation complète du rayonnement harmonique}
Dans les paragraphes précédents, nous avons évoqué la richesse de la photoionisation comme sonde des dynamiques électroniques (également nucléaires aux échelles de temps plus longues) et le potentiel des harmoniques d'ordre élevé pour les expériences de photoionisation résolues temporellement. Une des caractéristiques essentielles du rayonnement n'a pas été discutée: sa \textit{polarisation}. En effet, l'interaction avec un rayonnement polarisé circulairement est une sonde de la nature chirale de la matière \mycite{Pasteur1848}. En particulier dans le domaine de l'XUV et des rayons X mous, la polarisation circulaire a permis de mettre en évidence plusieurs phénomènes: dans les molécules, la photoionisation d'une molécule chirale par un rayonnement polarisé circulairement crée une asymétrie avant-arrière dans la distribution angulaire des photoélectrons, un phénomène connu sous le nom de dichroïsme circulaire de photoélectrons \mycite{PowisJPCA2000}\mycite{BoweringPRL2001} et extrêmement sensible aux détails fins du potentiel moléculaire. L'interaction de molécules achirales avec un rayonnement synchrotron polarisé circulairement a également été utilisée comme sonde de dynamiques induites par l'ionisation d'électrons de c\oe ur \mycite{TravnikovaPRL2010}, ou au voisinage de résonances de forme \mycite{JahnkePRL2002} ou d'autoionisation \mycite{DowekPRL2010}. Dans les solides, l'interaction de rayons X polarisés circulairement avec un matériau magnétique donne lieu au dichroïsme circulaire magnétique \mycite{StohrJElecSpec1995}, qui est dépendant des éléments constitutifs du matériau et permet donc l'imagerie holographique de domaines magnétiques \mycite{EisebittNature2004}. Par conséquent, la production d'impulsions ultra-brèves dans l'XUV et les rayons X mous polarisées circulairement permettrait par exemple l'étude résolue en temps de la reconnaissance chirale \mycite{CombyJPCL2016} ou bien des dynamiques de spin ultra-rapides dans les matériaux magnétiques ("femtomagnétisme") \mycite{BeaurepairePRL1996}.

Ainsi, plusieurs méthodes de génération d'harmoniques d'ordre élevé polarisées circulairement ont été proposées ces dernières années \mycite{EichmannPRA1995}\mycite{ZhouPRL2009}\mycite{MairessePRL2010}\mycite{FerreNatPhot2015}. Cependant, aucune \textit{caractérisation complète} de l'état de polarisation des harmoniques n'avait été effectuée au début de ma thèse. En effet, la polarimétrie optique nécessite un polariseur, un analyseur et un élément déphaseur (lame quart d'onde) \mycite{BornWolf}. La faible transmission des optiques dans le domaine XUV, en particulier le manque d'éléments déphaseurs, rend la transposition des techniques expérimentales du visible dans l'XUV très difficile. La plupart des études de polarimétrie effectuées de manière optique sur les harmoniques jusqu'à présent sont donc "incomplètes", c'est-à-dire qu'elles ne permettent pas de mesurer le signe de l'ellipticité (qui caractérise le sens de rotation du vecteur champ électrique) ni de distinguer la partie polarisée du rayonnement d'une éventuelle partie dépolarisée.

La \textit{photoionisation}, cette fois étudiée dans le \textit{référentiel moléculaire}, peut agir comme un \textit{polarimètre}. En effet, lors de la photoionisation dissociative d'une molécule, la distribution angulaire des ions fragments et des électrons émis dépend de l'état de polarisation du rayonnement incident \mycite{LebechJCP2003}\mycite{DowekBook2012}. Ainsi si la réaction de photoionisation dissociative a été étudiée grâce à un rayonnement de polarisation connue (par exemple produit au synchrotron \mycite{NahonJSR2012}) permettant de la "calibrer", la mesure de la distribution des ions et des électrons peut servir à déterminer la polarisation de n'importe quelle source de lumière à la même fréquence. Cette méthode, appelée \textit{polarimétrie moléculaire}, a été développée par l'équipe de D. Dowek à Orsay. Nous l'avons utilisée au cours de ce travail pour déterminer complètement l'état de polarisation d'harmoniques d'ordre élevé générées dans des conditions particulières.

\subsubsection*{Objectifs et plan de la thèse}
Les deux principaux objectifs de ce travail sont:
\begin{enumerate}
\item étudier les dynamiques électroniques lors de l'autoionisation en utilisant les harmoniques d'ordre élevé et l'interférométrie électronique.
\item déterminer l'état de polarisation complet des harmoniques d'ordre élevé grâce à la photoionisation dissociative dans le référentiel moléculaire.
\end{enumerate}

Dans la \hyperref[part:GHOE]{première} partie, nous introduirons les bases théoriques et expérimentales de la génération d'harmoniques d'ordre élevé. Nous présenterons un modèle simple rendant compte des principales caractéristiques du rayonnement harmonique que nous détaillerons. Les techniques expérimentales de production des harmoniques, d'accordabilité du laser fondamental et d'interférométrie électronique RABBIT seront également explicitées.

Ensuite, la partie \ref{part:Delais} exposera les concepts fondamentaux issus de la théorie des collisions nécessaires à l'interprétation des délais de photoionisation de Wigner, ainsi que la théorie de Fano de l'autoionisation. Nous détaillerons en particulier les amplitudes de transition à deux photons \textit{via} un état autoionisant. Ces éléments de transition sont accessibles expérimentalement grâce à l'interférométrie RABBIT.

La \hyperref[part:Helium]{troisième} partie traitera des résultats expérimentaux obtenus lors de l'étude de l'autoionisation dans l'hélium. Les premières expériences effectuées à Saclay au voisinage de l'état doublement excité $2s2p$ ont permis de développer une nouvelle méthode d'interférométrie électronique résolue spectralement appelée \textit{Rainbow} RABBIT. Les mesures d'amplitude et de phase de transition effectuées par Rainbow RABBIT permettent de reconstruire la dynamique du paquet d'onde électronique issu de la résonance de Fano dans le domaine temporel. Ces études ont été complétées par d'autres mesures effectuées en collaboration avec l'équipe d'A. L'Huillier à l'université de Lund (Suède) qui seront également présentées. Enfin, les observables du Rainbow RABBIT seront comparées avec les observables d'expériences d'absorption transitoire attoseconde, un autre outil de la physique attoseconde qui a été utilisée pour sonder la résonance $2s2p$ de l'hélium parallèlement à nos travaux dans un autre laboratoire (groupe de T. Pfeifer au MPK-Heidelberg).

La partie \ref{part:Argon} traitera des résultats d'expériences RABBIT et Rainbow RABBIT au voisinage de résonances de Fano dans l'argon et le néon. Ces expériences ont été effectuées en collaboration avec l'équipe d'A. L'Huillier à Lund d'une part, et en collaboration avec le groupe de L. DiMauro et P. Agostini à l'université d'état de l'Ohio (Columbus, \'{E}tats-Unis) d'autre part. Les mesures sont comparées à des calculs effectués par l'équipe de F. Mart\'{i}n à Madrid (Espagne).

Finalement, dans la partie \ref{part:Polarimétrie} nous étudierons l'état de polarisation des harmoniques d'ordre élevé générées par un champ à deux couleurs polarisées circulairement en sens opposé. Nous présenterons d'abord différentes méthodes proposées pour obtenir un rayonnement harmonique polarisé circulairement et montrerons les avantages du schéma à deux couleurs. L'état de polarisation complet des harmoniques sera ensuite étudié théoriquement par la résolution de l'équation de Schrödinger dépendante du temps, et expérimentalement grâce à la méthode de polarimétrie moléculaire. Les expériences correspondantes ont été effectuées en collaboration avec l'équipe de D. Dowek à Orsay.


